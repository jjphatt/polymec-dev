%%%%%%%%%%%%%%%%%%%%%%%%%%%%%%%%%%%%
%                                  %
% Titre  : p.tex                   %
% Sujet  : Manuel de l'utilisateur %
%          de PT-Scotch 6.0        %
%          Corps du document       %
% Auteur : Francois Pellegrini     %
%                                  %
%%%%%%%%%%%%%%%%%%%%%%%%%%%%%%%%%%%%

% pdflatex -sPAPERSIZE=a4 p.tex
% dvips -sPAPERSIZE=a4 p.dvi -o ptscotch_user6.0.ps
% ps2pdf -sPAPERSIZE=a4 ptscotch_user6.0.ps ptscotch_user6.0.pdf

%% Formatage et pagination.

\documentclass{article}
\usepackage{a4}
\usepackage{url}
\usepackage[dvips]{graphicx}
%\documentstyle[11pt,a4,fullpage,epsf]{article}
%\textwidth      16.0cm
%\oddsidemargin   -0.5cm
%\evensidemargin  -0.5cm
%\marginparwidth  0.0cm
%\marginparsep    0.0cm
%\marginparpush   0.0cm
%\topmargin        0.5cm
%\headheight      0.0cm
%\headsep         0.0cm
%\textheight     25.0cm
%\footheight      0.0cm
%\footskip        0.0cm

\sloppy                                          % Gestion des overfull hbox
\renewcommand{\baselinestretch}{1.05}            % Hauteur lignes x 1.05

\setcounter{secnumdepth}{3}                      % Sous-sous-sections numerotees
\setcounter{tocdepth}{3}                         % Sous-sous-sections dans la table

%% Macros et commandes utiles.

\makeatletter
\@definecounter{enumv}                           % 8 niveaux d'itemizations
\@definecounter{enumvi}
\@definecounter{enumvii}
\@definecounter{enumviii}
\def\itemize{\ifnum \@itemdepth >8 \@toodeep\else \advance\@itemdepth \@ne
\edef\@itemitem{labelitem\romannumeral\the\@itemdepth}%
\list{\csname\@itemitem\endcsname}{\def\makelabel##1{\hss\llap{##1}}}\fi}
\let\enditemize =\endlist

\def\@iteme[#1]{\if@noparitem \@donoparitem      % Item long pour options
  \else \if@inlabel \indent \par \fi
         \ifhmode \unskip\unskip \par \fi
         \if@newlist \if@nobreak \@nbitem \else
                        \addpenalty\@beginparpenalty
                        \addvspace\@topsep \addvspace{-\parskip}\fi
           \else \addpenalty\@itempenalty \addvspace\itemsep
          \fi
    \global\@inlabeltrue
\fi
\everypar{\global\@minipagefalse\global\@newlistfalse
          \if@inlabel\global\@inlabelfalse
             \setbox\@tempboxa\hbox{#1}\relax
             \hskip \itemindent \hskip -\parindent
             \hskip -\labelwidth \hskip -\labelsep
             \ifdim \wd\@tempboxa > \labelwidth
               \box\@tempboxa\hfil\break
             \else
               \hbox to\labelwidth{\box\@tempboxa\hfil}\relax
               \hskip \labelsep
             \fi
             \penalty\z@ \fi
          \everypar{}}\global\@nobreakfalse
\if@noitemarg \@noitemargfalse \if@nmbrlist \refstepcounter{\@listctr}\fi \fi
\ignorespaces}
\def\iteme{\@ifnextchar [{\@iteme}{\@noitemargtrue \@iteme[\@itemlabel]}}

\let\@Hxfloat\@xfloat
\def\@xfloat#1[{\@ifnextchar{H}{\@HHfloat{#1}[}{\@Hxfloat{#1}[}}
\def\@HHfloat#1[H]{%
\expandafter\let\csname end#1\endcsname\end@Hfloat
\vskip\intextsep\def\@captype{#1}\parindent\z@
\ignorespaces}
\def\end@Hfloat{\vskip \intextsep}
\makeatother

\def\progsyn{\item[{\makebox[1.5em][l]{\bf Synopsis}}]\ ~\linebreak[0]\\*[1em]}
\def\progdes{\item[{\makebox[1.5em][l]{\bf Description}}]\ ~\linebreak[0]\\*[1em]}
\def\progopt{\item[{\makebox[1.5em][l]{\bf Options}}]~\linebreak[0]}
\def\progret{\item[{\makebox[1.5em][l]{\bf Return values}}]~\linebreak[0]}

\newcommand{\bn}{\begin{displaymath}}            % Equations non-numerotees
\newcommand{\en}{\end{displaymath}}
\newcommand{\bq}{\begin{equation}}               % Equations numerotees
\newcommand{\eq}{\end{equation}}

\newcommand{\lbo}{\linebreak[0]}
\newcommand{\lbt}{\linebreak[2]}
\newcommand{\noi}{{\noindent}}                   % Pas d'indentation
\newcommand{\spa}{{\protect \vspace{\bigskipamount}}} % Espace vertical

\newcommand{\eg}{{\it e\@.g\@.\/\ }}             % e.g.
\newcommand{\ie}{{\it i\@.e\@.\/\ }}             % i.e.

\newcommand{\chaco}{{\sc Chaco}}                 % "Chaco"
\newcommand{\metis}{\mbox{\sc Me$\!$T$\!$iS}}    % "MeTiS"
\newcommand{\parmetis}{\mbox{\sc ParMe$\!$T$\!$iS}}
\newcommand{\scotch}{{\sc Scotch}}               % "Scotch"
\newcommand{\libscotch}{{\sc libScotch}}         % "libScotch"
\newcommand{\ptscotch}{{\sc PT-Scotch}}          % "PT-Scotch"
\newcommand{\libptscotch}{{\sc libPTScotch}}     % "libPTScotch"

\newcommand{\eqdef}{\stackrel{\scriptscriptstyle \rm def}{=}}       % = as definition
\newcommand{\isapprox}{\mathop{\approx}\limits}

\newcommand{\lefta}{\longleftarrow}
\newcommand{\rghta}{\longrightarrow}
\newcommand{\botha}{\longleftrightarrow}
\newcommand{\Lefta}{\Longleftarrow}
\newcommand{\Rghta}{\Longrightarrow}
\newcommand{\Botha}{\Longleftrightarrow}

\newcommand{\HY}{{\rm H}}                        % H
\newcommand{\KP}{{\rm K}}                        % K
\newcommand{\MK}[1]{{\rm M}_{#1}}                % Mk
\newcommand{\MD}{\MK{2}}                         % M2
\newcommand{\PA}{{\rm P}}                        % P
\newcommand{\UB}{{\rm UB}}                       % UB
\newcommand{\SE}{{\rm SE}}                       % SE
\newcommand{\FFT}{{\rm FFT}}                     % FFT
\newcommand{\BF}{{\rm BF}}                       % BF
\newcommand{\BFB}{{\overline{\rm BF}}}           % BF bar
\newcommand{\CCC}{{\rm CCC}}                     % CCC
\newcommand{\CCCB}{{\overline{\rm CCC}}}         % CCC bar

\newcommand{\roo}[1]{{\rho_{\scriptscriptstyle {#1}}}} % Rho avec petit argument
\newcommand{\too}[1]{{\tau_{\scriptscriptstyle {#1}}}} % Tau avec petit argument
\newcommand{\xio}[1]{{\xi_{\scriptscriptstyle {#1}}}} % Xi avec petit argument

\newcommand{\SB}[1]{{\cal C}'_S\left({#1}\right)} % Comportement en espace
\newcommand{\TB}[1]{{\cal C}'_T\left({#1}\right)} % Comportement en temps
\newcommand{\SC}[1]{{\cal C}_S\left({#1}\right)} % Complexite en espace
\newcommand{\TC}[1]{{\cal C}_T\left({#1}\right)} % Complexite en temps

\newcommand{\dmap}{\mbox{$\delta_{map}$}}
\newcommand{\dexp}{\mbox{$\delta_{exp}$}}
\newcommand{\mmap}{\mbox{$\mu_{map}$}}
\newcommand{\mdil}{\mbox{$\mu_{dil}$}}
\newcommand{\mcom}{\mbox{$\mu_{com}$}}
\newcommand{\mexp}{\mbox{$\mu_{exp}$}}

\newcommand{\NNZ}{\mbox{NNZ}}
\newcommand{\OPC}{\mbox{OPC}}
\newcommand{\hnbr}{\mbox{$h_{\rm nbr}$}}
\newcommand{\hmin}{\mbox{$h_{\rm min}$}}
\newcommand{\hmax}{\mbox{$h_{\rm max}$}}
\newcommand{\havg}{\mbox{$h_{\rm avg}$}}
\newcommand{\hdlt}{\mbox{$h_{\rm dlt}$}}

%% Version du document.

\newcommand{\scotchver}{6.0}
\newcommand{\scotchversub}{6.0.5}
\newcommand{\scotchcitepuser}{\protect\cite{pell08c}}
\newcommand{\scotchcitesuser}{\protect\cite{pell08b}}

%% Page de garde.

\begin{document}

%\date{\today}
\date{\today}

\title{\includegraphics[scale=0.8]{p_f_logo.ps}\\[1em]
       {\LARGE\bf \ptscotch\ and \libptscotch\ {\sc \scotchver} \mbox{User's Guide}}\\[1em]%
       {\normalsize (version \scotchversub)}
}

\author{Fran\c cois Pellegrini\\
Universit\'e Bordeaux~1 \& LaBRI, UMR CNRS 5800\\
Bacchus team, INRIA Bordeaux Sud-Ouest\\
351 cours de la Lib\'eration, 33405 TALENCE, FRANCE\\
{\tt pelegrin@labri.fr}}

\maketitle

\begin{abstract}

This document describes the capabilities and operations of
\ptscotch\ and \libscotch, a software package and a software library
which compute parallel static mappings and parallel sparse matrix
block orderings of distributed graphs. It gives brief descriptions of
the algorithms, details the input/output formats, instructions for
use, installation procedures, and provides a number of examples.

\ptscotch\ is distributed as free/libre software, and has been
designed such that new partitioning or ordering methods can be added
in a straightforward manner. It can therefore be used as a testbed for
the easy and quick coding and testing of such new methods, and may
also be redistributed, as a library, along with third-party software
that makes use of it, either in its original or in updated forms.
\end{abstract}

\clearpage

%% Table des matieres.

\tableofcontents

%% Corps du document.

%%%%%%%%%%%%%%%%%%%%%%%%%%%%%%%%%%%%
%                                  %
% Titre  : p_i.tex                 %
% Sujet  : Manuel de l'utilisateur %
%          du projet 'PT-Scotch'   %
%          Introductions           %
% Auteur : Francois Pellegrini     %
%                                  %
%%%%%%%%%%%%%%%%%%%%%%%%%%%%%%%%%%%%

\section{Introduction}

\subsection{Static mapping}

The efficient execution of a parallel program on a parallel machine
requires that the communicating processes of the program be assigned
to the processors of the machine so as to minimize its overall running
time.
When processes have a limited duration and their logical dependencies
are accounted for, this optimization problem is referred to as
scheduling.
When processes are assumed to coexist simultaneously for the entire
duration of the program, it is referred to as mapping. It
amounts to balancing the computational weight of the processes among the
processors of the machine, while reducing the cost of communication by
keeping intensively inter-communicating processes on nearby
processors.

In most cases, the underlying computational structure of the parallel
programs to map can be conveniently modeled as a graph in which
vertices correspond to processes that handle distributed pieces of
data, and edges reflect data dependencies. The mapping problem can
then be addressed by assigning processor labels to the vertices of the
graph, so that all processes assigned to some processor are loaded and
run on it.
In a SPMD context, this is equivalent to the distribution
across processors of the data structures of parallel programs; in this
case, all pieces of data assigned to some processor are handled by a
single process located on this processor.

A mapping is called static if it is computed prior to the
execution of the program. Static mapping is NP-complete in the general
case~\cite{gajo79}. Therefore, many studies have been carried out in
order to find sub-optimal solutions in reasonable time, including
the development of specific algorithms for common topologies such
as the hypercube~\cite{errasa90,hamm92}.
When the target machine is assumed to have a communication network in
the shape of a complete graph, the static mapping problem turns into
the partitioning problem, which has also been intensely
studied~\cite{basi94,hele93a,kaku95a,kaku95c,posili90}.
However, when mapping onto parallel machines the communication network
of which is not a bus, not accounting for the topology of the target
machine usually leads to worse running times, because simple cut
minimization can induce more expensive long-distance
communication~\cite{hamm92,wacrevjo95}; the static mapping problem is
gaining popularity as most of the newer massively parallel machines
have a strongly NUMA architecture

\subsection{Sparse matrix ordering}

Many scientific and engineering problems can be modeled by sparse
linear systems, which are solved either by iterative or direct
methods.  To achieve efficiency with direct methods, one must minimize
the fill-in induced by factorization. This fill-in is a direct
consequence of the order in which the unknowns of the linear system
are numbered, and its effects are critical both in terms of memory and
of computation costs.
\\

Because there always exist large problem graphs which cannot fit in
the memory of sequential computers and cost too much to partition,
it is necessary to resort to parallel graph ordering tools.
\ptscotch\ provides such features.

\subsection{Contents of this document}

This document describes the capabilities and operations of \ptscotch,
a software package devoted to parallel static mapping and sparse
matrix block ordering.
It is the parallel extension of \scotch, a sequential software package
devoted to static mapping, graph and mesh partitioning, and sparse
matrix block ordering. While both packages share a significant amount
of code, because \ptscotch\ transfers control to the sequential
routines of the \libscotch\ library when the subgraphs on which it
operates are located on a single processor, the two sets of routines
have a distinct user's manual. Readers interested in the sequential
features of \scotch\ should refer to the {\it\scotch\ User's
Guide}~\scotchcitesuser.

The rest of this manual is organized as follows.
Section~\ref{sec-project} presents the goals of the \scotch\ project, and
section~\ref{sec-algo} outlines the most important aspects of the
parallel partitioning and ordering algorithms that it implements.
Section~\ref{sec-file} defines the formats of the files used in \ptscotch,
section~\ref{sec-prog} describes the programs of the
\ptscotch\ distribution, and section~\ref{sec-lib} defines the interface
and operations of the parallel routines of the \libscotch\ library.
Section~\ref{sec-install} explains how to obtain and install the
\scotch\ distribution.
Finally, some practical examples are given in
section~\ref{sec-examples}.
%, and instructions on how to implement new methods in the
%\libscotch\ library are provided in section~\ref{sec-coding}.

\section{The \scotch\ project}
\label{sec-project}

\subsection{Description}

\scotch\ is a project carried out at the {\it Laboratoire Bordelais de
Recherche en Informatique\/} (LaBRI) of the Universit\'e Bordeaux I,
and now within the Bacchus project of INRIA Bordeaux Sud-Ouest. Its goal
is to study the applications of graph theory to scientific computing,
using a ``divide and conquer'' approach.

It focused first on static mapping, and has resulted in the
development of the Dual Recursive Bipartitioning (or DRB) mapping
algorithm and in the study of several graph bipartitioning
heuristics~\cite{pell94a}, all of which have been implemented in the
\scotch\ software package~\cite{pero96a}. Then, it focused on the
computation of high-quality vertex separators for the ordering of
sparse matrices by nested dissection, by extending the work that has
been done on graph partitioning in the context of static
mapping~\cite{pero97a,peroam00a}. More recently, the ordering
capabilities of \scotch\ have been extended to native mesh structures,
thanks to hypergraph partitioning algorithms. New graph partitioning
methods have also been recently added~\cite{chpe06a,pell07b}.
Version {\sc 5.0} of \scotch\ was the first one to comprise parallel
graph ordering routines~\cite{chpe08}, and version {\sc 5.1} started
offering parallel graph partitioning features, while parallel static
mapping will be available in the next release.

\subsection{Availability}

Starting from version {\sc 4.0}, which has been developed at INRIA
within the ScAlApplix project, \scotch\ is available under a dual
licensing basis. On the one hand, it is downloadable from the
\scotch\ web page as free/libre software, to all interested parties
willing to use it as a library or to contribute to it as a testbed for
new partitioning and ordering methods. On the other hand, it can also
be distributed, under other types of licenses and conditions, to
parties willing to embed it tightly into closed, proprietary software.
\\

The free/libre software license under which \scotch\ {\sc\scotchver} is
distributed is the CeCILL-C license~\cite{cecill}, which has basically
the same features as the GNU LGPL (``{\it Lesser General Public
License}'')~\cite{lgpl}: ability to link the code as a library to any
free/libre or even proprietary software, ability to modify the code
and to redistribute these modifications. Version {\sc 4.0} of
\scotch\ was distributed under the LGPL itself. This version did not
comprise any parallel features.
\\

Please refer to section~\ref{sec-install} to see how to obtain the
free/libre distribution of \scotch.

\section{Algorithms}
\label{sec-algo}

\subsection{Parallel static mapping by Dual Recursive Bipartitioning}
\label{sec-drb}

For a detailed description of the sequential implementation of this
mapping algorithm and an extensive analysis of its performance, please
refer to~\cite{pell94a,pero96b}.
In the next sections, we will only outline the most important aspects
of the algorithm.

\subsubsection{Static mapping}

The parallel program to be mapped onto the target architecture is modeled
by a valuated unoriented graph $S$ called source graph or
process graph, the vertices of which represent the processes of the
parallel program, and the edges of which the communication channels between
communicating processes.
Vertex- and edge- valuations associate with every vertex $v_S$ and every
edge $e_S$ of $S$ integer numbers $w_S(v_S)$ and $w_S(e_S)$ which
estimate the computation weight of the corresponding process
and the amount of communication to be transmitted on the channel,
respectively.

The target machine onto which is mapped the parallel program is also
modeled by a valuated unoriented graph $T$ called target graph
or architecture graph.
Vertices $v_T$ and edges $e_T$ of $T$ are assigned integer weights
$w_T(v_T)$ and $w_T(e_T)$, which estimate the computational power of the
corresponding processor and the cost of traversal of the inter-processor
link, respectively.

A mapping from $S$ to $T$ consists of two applications
$\too{S,T} : V(S) \rghta V(T)$ and
$\roo{S,T} : E(S) \rghta {\cal P}(E(T))$,
where ${\cal P}(E(T))$ denotes the set of all simple loopless paths which
can be built from $E(T)$.
$\too{S,T}(v_S) = v_T$ if process $v_S$ of $S$ is mapped onto processor
$v_T$ of $T$, and $\roo{S,T}(e_S) = \{ e^1_T, e^2_T, \ldots, e^n_T \}$ if
communication channel $e_S$ of $S$ is routed through communication links
$e^1_T$, $e^2_T$, \ldots, $e^n_T$ of $T$.
$|\roo{S,T}(e_S)|$ denotes the dilation of edge $e_S$, that is, the number of
edges of $E(T)$ used to route $e_S$.

\subsubsection{Cost function and performance criteria}
\label{sec-algo-cost}

The computation of efficient static mappings requires an {\it a priori\/}
knowledge of the dynamic behavior of the target machine with respect to
the programs which are run on it.
This knowledge is synthesized in a cost function, the nature of which
determines the characteristics of the desired optimal mappings.
The goal of our mapping algorithm is to minimize some communication cost
function, while keeping the load balance within a specified tolerance.
The communication cost function $f_C$ that we have chosen is the sum,
for all edges, of their dilation multiplied by their weight:
\bn
f_C(\too{S,T},\roo{S,T})
\eqdef \hspace*{-0.25cm}\sum\limits_{e_S\in E(S)}\hspace*{-0.25cm}
w_S(e_S)\,|\roo{S,T}(e_S)|\enspace.
\en
This function, which has already been considered by several authors for
hypercube target topologies~\cite{errasa90,hamm92,hele94b}, has several
interesting properties:
it is easy to compute, allows incremental updates performed by
iterative algorithms, and
its minimization favors the mapping of intensively intercommunicating
processes onto nearby processors;
regardless of the type of routage implemented on the target machine
(store-and-forward or cut-through), it models the traffic on the
interconnection network and thus the risk of congestion.

The strong positive correlation between values of this function and
effective execution times has been experimentally verified by
Hammond~\cite{hamm92} on the CM-2, and by Hendrickson and
Leland~\cite{hele94a} on the nCUBE~2.
\\

The quality of mappings is evaluated with respect to the criteria for
quality that we have chosen: the balance of the computation load across
processors, and the minimization of the interprocessor communication cost
modeled by function~$f_C$. These criteria lead to the definition of
several parameters, which are described below.

For load balance, one can define $\mmap$, the average load per
computational power unit (which does not depend on the mapping), and
$\dmap$, the load imbalance ratio, as\\[-0.5em]
\bn
\mmap \eqdef
{\sum\limits_{v_S \in V(S)} w_S(v_S) \over
 \sum\limits_{v_T \in V(T)} w_T(v_T)}
\hspace*{2.5em}\mbox{~and~}
\en
\bn
\dmap \eqdef
{\sum\limits_{v_T \in V(T)}
   \left|\left(\!\!{1 \over w_T(v_T)}\hspace*{-0.3em}
         \sum\limits_{\scriptsize
                      \shortstack{$v_S \in V(S)$\\[-0.2em]
                                  $\too{S,T}(v_S) = v_T$}}
         \hspace*{-0.2em} w_S(v_S)\!\!\right)\:-\:\mmap\right| \over
\sum\limits_{v_S \in V(S)} w_S(v_S)}\enspace.
\en
However, since the maximum load imbalance ratio is provided by the user in
input of the mapping, the information given by these parameters is of little
interest, since what matters is the minimization of the communication cost
function under this load balance constraint.

For communication, the straightforward parameter to consider is $f_C$.
It can be normalized as $\mexp$, the average edge expansion, which can
be compared to $\mdil$, the average edge dilation; these are defined
as\\[-1.3em]
\bn
\mexp \eqdef {f_C \over \sum\limits_{e_S \in E(S)} w_S(e_S)}
\hspace*{2.5em}\mbox{~and~}\hspace*{2.5em}
\mdil \eqdef {\sum\limits_{e_S \in E(S)}|\roo{S,T}(e_S)| \over |E(S)|}
\enspace.
\en
$\dexp \eqdef {\mexp \over \mdil}$ is smaller than $1$ when the mapper
succeeds in putting heavily intercommunicating processes closer to each other
than it does for lightly communicating processes; they are equal if all edges
have same weight.

\subsubsection{The Dual Recursive Bipartitioning algorithm}
\label{sec-algo-drb}

Our mapping algorithm uses a divide and conquer approach to
recursively allocate subsets of processes to subsets of
processors~\cite{pell94a}.

It starts by considering a set of processors, also called domain,
containing all the processors of the target machine, and with which is
associated the set of all the processes to map.  At each step, the
algorithm bipartitions a yet unprocessed domain into two disjoint
subdomains, and calls a graph bipartitioning algorithm to split the
subset of processes associated with the domain across the two
subdomains, as sketched in the following.

\noi
{\renewcommand{\baselinestretch}{0.95}\footnotesize\tt {%
\begin{verbatim}
mapping (D, P)
Set_Of_Processors  D;
Set_Of_Processes   P;
{
  Set_Of_Processors  D0, D1;
  Set_Of_Processes   P0, P1;

  if (|P| == 0) return;  /* If nothing to do.     */
  if (|D| == 1) {        /* If one processor in D */
    result (D, P);       /* P is mapped onto it.  */
    return;
  }

  (D0, D1) = processor_bipartition (D);
  (P0, P1) = process_bipartition   (P, D0, D1);
  mapping (D0, P0);      /* Perform recursion. */
  mapping (D1, P1);
}
\end{verbatim}}}

\noi
The association of a subdomain with every process defines a partial
mapping of the process graph. As bipartitionings are performed,
the subdomain sizes decrease, up to give a complete mapping when all
subdomains are of size~one.
\\

The above algorithm lies on the ability to define five main objects:
\begin{itemize}
\item
a domain structure, which represents a set of processors in the target
architecture;
\item
a domain bipartitioning function, which, given a domain, bipartitions
it in two disjoint subdomains;
\item
a domain distance function, which gives, in the target graph, a measure
of the distance between two disjoint domains. Since domains may not be convex
nor connected, this distance may be estimated.
However, it must respect certain homogeneity properties, such as
giving more accurate results as domain sizes decrease.
The domain distance function is used by the graph bipartitioning algorithms
to compute the communication function to minimize, since it allows the mapper
to estimate the dilation of the edges that link vertices which belong to
different domains.
Using such a distance function amounts to considering that all routings
will use shortest paths on the target architecture, which is how most
parallel machines actually do.
We have thus chosen that our program would not provide routings for the
communication channels, leaving their handling to the communication system of
the target machine;
\item
a process subgraph structure, which represents the subgraph induced by a
subset of the vertex set of the original source graph;
\item
a process subgraph bipartitioning function, which bipartitions subgraphs
in two disjoint pieces to be mapped onto the two subdomains computed by
the domain bipartitioning function.
\end{itemize}
All these routines are seen as black boxes by the mapping program, which can
thus accept any kind of target architecture and process bipartitioning
functions.

\subsubsection{Partial cost function}

The production of efficient complete mappings requires that all graph
bipartitionings favor the criteria that we have chosen.
Therefore, the bipartitioning of a subgraph~$S'$ of $S$ should maintain
load balance within the user-specified tolerance, and minimize the
partial communication cost function $f'_C$, defined as
\bn
f'_C(\too{S,T},\roo{S,T}) \eqdef
\hspace*{-0.45cm}\sum\limits_{\mbox{\scriptsize
             \shortstack{$v\in V(S')$\\
                         $\{v,v'\}\in E(S)$}}}\hspace*{-0.45cm}
w_S(\{v,v'\})\,|\roo{S,T}(\{v,v'\})|\enspace,
\en
which accounts for the dilation of edges internal to subgraph~$S'$ as well as
for the one of edges which belong to the cocycle of $S'$, as shown in
Figure~\ref{fig-bipcost}.
Taking into account the partial mapping results issued by previous
bipartitionings makes it possible to avoid local choices that
might prove globally bad, as explained below.
This amounts to incorporating additional constraints to the standard graph
bipartitioning problem, turning it into a more general optimization problem
termed as skewed graph partitioning by some authors~\cite{heledr97}.

\begin{figure}[hbt]
\hfill
\parbox[b]{4.9cm}{
\hfill
\includegraphics[scale=0.40]{s_f_rua.eps}
\hfill\\
{\bf a.} Initial position.
}\ \hfill\
\parbox[b]{4.9cm}{
\hfill
\includegraphics[scale=0.40]{s_f_rub.eps}
\hfill\\
{\bf b.} After one vertex is moved.
}\hfill\
\caption%
{Edges accounted for in the partial communication cost function when
 bipartitioning the subgraph associated with domain~$D$ between
 the two subdomains $D_0$ and $D_1$ of~$D$.
 Dotted edges are of dilation zero, their two ends being mapped onto the
 same subdomain. Thin edges are cocycle edges.}
\label{fig-bipcost}
\end{figure}

%% \subsubsection{Execution scheme}

%% From an algorithmic point of view, our mapper behaves as a greedy algorithm,
%% since the mapping of a process to a subdomain is never reconsidered, and
%% at each step of which iterative algorithms can be applied.
%% The double recursive call performed at each step induces a recursion scheme
%% in the shape of a binary tree, each vertex of which corresponds to a
%% bipartitioning job, that is, the bipartitioning of both a domain and
%% its associated subgraph.

%% In the case of depth-first sequencing, as written in the above sketch,
%% bipartitioning jobs run in the left branches of the tree have no information
%% on the distance between the vertices they handle and neighbor vertices to be
%% processed in the right branches.
%% On the contrary, sequencing the jobs according to a by-level (breadth-first)
%% travel of the tree allows any bipartitioning job of a given level to
%% have information on the subdomains to which all the processes have been
%% assigned at the previous level.
%% Thus, when deciding in which subdomain to put a given process, a
%% bipartitioning job can account for the communication costs induced by
%% its neighbor processes, whether they are handled by the job itself or not,
%% since it can estimate in $f'_C$ the dilation of the corresponding edges.
%% This results in an interesting feedback effect: once an edge has been kept
%% in a cut between two subdomains, the distance between its end vertices will
%% be accounted for in the partial communication cost function to be minimized,
%% and following jobs will be more likely to keep these vertices close to
%% each other, as illustrated in Figure~\ref{fig-biprub}.
%% \begin{figure}[hbt]
%% \hfill
%% \parbox[b]{5.2cm}{
%% \hfill
%% \includegraphics[scale=0.40]{s_f_run.eps}
%% \hfill\\
%% {\bf a.} Depth-first sequencing.
%% }\ \hfill\
%% \parbox[b]{5.2cm}{
%% \hfill
%% \includegraphics[scale=0.40]{s_f_ruy.eps}
%% \hfill\\
%% {\bf b.} Breadth-first sequencing.
%% }\hfill\ %
%% \caption%
%% {Influence of depth-first and breadth-first sequencings on the
%%  bipartitioning of a domain~$D$ belonging to the leftmost branch of
%%  the bipartitioning tree.
%%  With breadth-first sequencing, the partial mapping data regarding vertices
%%  belonging to the right branches of the bipartitioning tree are more
%%  accurate (C.L. stands for ``Cut Level'').}
%% \label{fig-biprub}
%% \end{figure}
%% The relative efficiency of depth-first and breadth-first sequencing schemes
%% with respect to the structure of the source and target graphs is discussed
%% in~\cite{pero96b}.

\subsubsection{Parallel graph bipartitioning methods}
\label{sec-algo-bipart}

The core of our parallel recursive mapping algorithm uses process
graph parallel bipartitioning
methods as black boxes. It allows the mapper to run any type of graph
bipartitioning method compatible with our criteria for quality.
Bipartitioning jobs maintain an internal image of the current bipartition,
indicating for every vertex of the job whether it is currently assigned to the
first or to the second subdomain.
It is therefore possible to apply several different methods in sequence,
each one starting from the result of the previous one,
and to select the methods with respect to the job characteristics, thus
enabling us to define mapping strategies.
The currently implemented graph bipartitioning methods are listed below.
\begin{itemize}
\iteme[{\bf Band}]
Like the multi-level method which will be described below, the band
method is a meta-algorithm, in the sense that it does not itself
compute partitions, but rather helps other partitioning algorithms
perform better. It is a refinement algorithm which, from a given
initial partition, extracts a band graph of given width (which only
contains graph vertices that are at most at this distance from the
separator), calls a partitioning strategy on this band graph, and
prolongs\footnote{While a \emph{projection} is an application to a
space of lower dimension, a \emph{prolongation} refers to an
application to a space of higher dimension. Yet, the term projection
is also commonly used to refer to such a propagation, most often in
the context of a multilevel framework.} back the refined partition
on the original graph. This method was designed to be able to use
expensive partitioning heuristics, such as genetic algorithms, on
large graphs, as it dramatically reduces the problem space by several
orders of magnitude. However, it was found that, in a multi-level
context, it also improves partition quality, by coercing partitions in
a problem space that derives from the one which was globally defined
at the coarsest level, thus preventing local optimization refinement
algorithms to be trapped in local optima of the finer
graphs~\cite{chpe06a}.
\iteme[{\bf Diffusion}]
This global optimization method, the sequential formulation of which
is presented in~\cite{pell07b}, flows two kinds of antagonistic
liquids, scotch and anti-scotch, from two source vertices, and sets
the new frontier as the limit between vertices which contain scotch
and the ones which contain anti-scotch. In order to add
load-balancing constraints to the algorithm, a constant amount of
liquid disappears from every vertex per unit of time, so that no
domain can spread across more than half of the vertices. Because
selecting the source vertices is essential to the obtainment of
useful results, this method has been hard-coded so that the two
source vertices are the two vertices of highest indices, since in the
band method these are the anchor vertices which represent all of the
removed vertices of each part. Therefore, this method must be used on
band graphs only, or on specifically crafted graphs.
\iteme[{\bf Multi-level}]\label{sec-algo-mle}
This algorithm, which has been studied by several
authors~\cite{basi94,hele93b,kaku95a} and should be considered as a strategy
rather than as a method since it uses other methods as parameters, repeatedly
reduces the size of the graph to bipartition by finding matchings that
collapse vertices and edges, computes a partition for the coarsest
graph obtained, and prolongs the result back to the original graph,
as shown in Figure~\ref{fig-multiproc}.
\begin{figure}[hbt]
~\hfill\includegraphics[scale=0.50]{s_f_mult.eps}\hfill\ ~
\caption%
{The multi-level partitioning process. In the uncoarsening phase, the light
and bold lines represent for each level the prolonged partition obtained
from the coarser graph, and the partition obtained after refinement,
respectively.}
\label{fig-multiproc}
\end{figure}
The multi-level method, when used in conjunction with the banded
diffusion method to refine the prolonged partitions at every level,
usually stabilizes quality irrespective of the number of processors
which run the parallel static mapper.
\end{itemize}

\subsubsection{Mapping onto variable-sized architectures}
\label{sec-algo-variable}

Several constrained graph partitioning problems can be modeled as
mapping the problem graph onto a target architecture, the number of
vertices and topology of which depend dynamically on the structure of
the subgraphs to bipartition at each step.

Variable-sized architectures are supported by the DRB algorithm in the
following way: at the end of each bipartitioning step, if any of the
variable subdomains is empty (that is, all vertices of the subgraph
are mapped only to one of the subdomains), then the DRB process stops
for both subdomains, and all of the vertices are assigned to their
parent subdomain; else, if a variable subdomain has only one vertex
mapped onto it, the DRB process stops for this subdomain, and the
vertex is assigned to it.

The moment when to stop the DRB process for a specific subgraph can be
controlled by defining a bipartitioning strategy that tests for the
validity of a criterion at each bipartitioning step, and maps all of
the subgraph vertices to one of the subdomains when it becomes false.

\subsection{Parallel sparse matrix ordering by hybrid incomplete nested dissection}

When solving large sparse linear systems of the form $Ax=b$, it is
common to precede the numerical factorization by a symmetric
reordering. This reordering is chosen in such a way that pivoting down
the diagonal in order on the resulting permuted matrix $PAP^T$
produces much less fill-in and work than computing the factors of $A$
by pivoting down the diagonal in the original order (the fill-in is
the set of zero entries in $A$ that become non-zero in the factored
matrix).

\subsubsection{Hybrid incomplete nested dissection}
\label{sec-algo-nested}

The minimum degree and nested dissection algorithms are the two most
popular reordering schemes used to reduce fill-in and operation count
when factoring and solving sparse matrices.
\\

The minimum degree algorithm~\cite{tiwa67} is a local heuristic that
performs its pivot selection by iteratively selecting from the graph a
node of minimum degree. It is known to be a very fast and general
purpose algorithm, and has received much attention over the last three
decades (see for example~\cite{amdadu96,geli89,liu-85}). However, the
algorithm is intrinsically sequential, and very little can be
theoretically proved about its efficiency.
\\

The nested dissection algorithm~\cite{geli81} is a global, recursive
heuristic algorithm which computes a vertex set~$S$ that separates the
graph into two parts~$A$ and~$B$, ordering $S$ with the highest
remaining indices. It then proceeds recursively on parts~$A$ and~$B$
until their sizes become smaller than some threshold value. This
ordering guarantees that, at each step, no non zero term can appear
in the factorization process between unknowns of~$A$ and unknowns
of~$B$.

Many theoretical results have been obtained on nested dissection
ordering~\cite{chro89,lirota79}, and its divide and conquer nature
makes it easily parallelizable. The main issue of the nested
dissection ordering algorithm is thus to find small vertex separators
that balance the remaining subgraphs as evenly as possible.
Provided that good vertex separators are found, the nested dissection
algorithm produces orderings which, both in terms of fill-in and
operation count, compare favorably~\cite{gukaku96,kaku95a,pero97a} to
the ones obtained with the minimum degree algorithm~\cite{liu-85}.
Moreover, the elimination trees induced by nested dissection are
broader, shorter, and better balanced, and therefore
exhibit much more concurrency in the context of parallel Cholesky
factorization~\cite[and included
references]{aseilish91,geng89,geheling88,gukaku96,pero97a,shre92}.
\\

Due to their complementary nature, several schemes have been proposed
to hybridize the two methods~\cite{hero98,kaku98a,pero97a}.
Our implementation is based on a tight coupling of the nested dissection
and minimum degree algorithms, that allows each of them to take
advantage of the information computed by the other~\cite{peroam00a}.

However, because we do not provide a parallel implementation of the
minimum degree algorithm, this hybridization scheme can only take
place after enough steps of parallel nested dissection have been
performed, such that the subgraphs to be ordered by minimum degree
are centralized on individual processors.

\subsubsection{Parallel ordering}
\label{sec-algo-parallel}

The parallel computation of orderings in \ptscotch\ involves three
different levels of concurrency, corresponding to three key steps of
the nested dissection process: the nested dissection algorithm itself,
the multi-level coarsening algorithm used to compute separators at
each step of the nested dissection process, and the refinement of the
obtained separators. Each of these steps is described below.

\paragraph{Nested dissection}

As said above, the first level of concurrency relates to the
parallelization of the nested dissection method itself, which is
straightforward thanks to the intrinsically concurrent nature of the
algorithm. Starting from the initial graph, arbitrarily distributed
across $p$ processors but preferably balanced in terms of vertices,
the algorithm proceeds as illustrated in Figure~\ref{fig-nedi}~: once
a separator has been computed in parallel, by means of a method described
below, each of the $p$ processors participates in the building of the
distributed induced subgraph corresponding to the first separated part
(even if some processors do not have any vertex of it). This induced
subgraph is then folded onto the first $\lceil\frac{p}{2}\rceil$
processors, such that the average number of vertices per processor,
which guarantees efficiency as it allows the shadowing of
communications by a subsequent amount of computation, remains
constant. During the folding process, vertices and adjacency lists
owned by the $\lfloor\frac{p}{2}\rfloor$ sender processors are
redistributed to the $\lceil\frac{p}{2}\rceil$ receiver processors so
as to evenly balance their loads.

The same procedure is used to build, on the
$\lfloor\frac{p}{2}\rfloor$ remaining processors, the folded induced
subgraph corresponding to the second part. These two constructions
being completely independent, the computations of the two induced
subgraphs and their folding can be performed in parallel, thanks to the
temporary creation of an extra thread per processor. When the vertices
of the separated graph are evenly distributed across the processors,
this feature favors load balancing in the subgraph building phase,
because processors which do not have many vertices of one part will
have the rest of their vertices in the other part, thus yielding the
same overall workload to create both graphs in the same time. This
feature can be disabled when the communication system of the target
machine is not thread-safe.

At the end of the folding process, every processor has a folded
subgraph fragment of one of the two folded subgraphs, and the nested
dissection process car recursively proceed independently on each
subgroup of $\frac{p}{2}$ (then $\frac{p}{4}$, $\frac{p}{8}$,
etc\@.) processors, until each subgroup is reduced to a single
processor. From then on, the nested dissection process will go on
sequentially on every processor, using the nested dissection routines
of the \scotch\ library, eventually ending in a coupling with minimum
degree methods~\cite{peroam00a}, as described in the previous section.

\begin{figure}
~\hfill%
\includegraphics[scale=0.38]{p_f_nedi.eps}
\hfill~\\*[-1em]
\caption{Diagram of a nested dissection step for a (sub-)graph
  distributed across four processors. Once the separator is known, the
  two induced subgraphs are built and folded (this can be done in
  parallel for both subgraphs), yielding two subgraphs, each of them
  distributed across two processors.}
\label{fig-nedi}
\end{figure}

\paragraph{Graph coarsening}
\label{secalgocoarsen}

The second level of concurrency concerns the computation of
separators. The approach we have chosen is the now classical
multi-level one~\cite{basi94,hele95,kaku98a}. It consists in
repeatedly computing a set of increasingly coarser albeit
topologically similar versions of the
graph to separate, by finding matchings which collapse vertices and
edges, until the coarsest graph obtained is no larger than a few
hundreds of vertices, then computing a separator on this coarsest
graph, and prolonging back this separator, from coarser to finer
graphs, up to the original graph.
Most often, a local optimization algorithm, such as
Kernighan-Lin~\cite{keli70} or Fiduccia-Mattheyses~\cite{fima82} (FM), is
used in the uncoarsening phase to refine the partition that is
prolonged back at every level, such that the granularity of the
solution is the one of the original graph and not the one of the
coarsest graph.

The main features of our implementation are outlined in
Figure~\ref{fig-sepa}. Once the matching phase is complete, the
coarsened subgraph building phase takes place. It can be
parametrized so as to allow one to choose between two options. Either
all coarsened vertices are kept on their local processors (that is,
processors that hold at least one of the ends of the coarsened edges),
as shown in the first steps of Figure~\ref{fig-sepa}, which decreases
the number of vertices owned by every processor and speeds-up future
computations, or else coarsened graphs are folded and duplicated, as
shown in the next steps of Figure~\ref{fig-sepa}, which increases the
number of working copies of the graph and can thus reduce
communication and increase the final quality of the separators.

As a matter of fact, separator computation algorithms, which are local
heuristics, heavily depend on the quality of the coarsened graphs,
and we have observed with the sequential version of \scotch\ that
taking every time the best partition among two ones, obtained from
two fully independent multi-level runs, usually improved overall ordering
quality. By enabling the folding-with-duplication routine (which will
be referred to as ``fold-dup'' in the following) in the first
coarsening levels, one can implement this approach in parallel, every
subgroup of processors that hold a working copy of the graph being
able to perform an almost-complete independent multi-level
computation, save for the very first level which is shared by all
subgroups, for the second one which is shared by half of the subgroups,
and so on.

The problem with the fold-dup approach is that it consumes a lot of
memory. Consequently, a good strategy can be to resort to folding only
when the number of vertices of the graph to be considered reaches some
minimum threshold. This threshold allows one to set a trade off
between the level of completeness of the independent multi-level runs
which result from the early stages of the fold-dup process, which
impact partitioning quality, and the amount of memory to be used in
the process.

Once all working copies of the coarsened graphs are folded on
individual processors, the algorithm enters a multi-sequential phase,
illustrated at the bottom of Figure~\ref{fig-sepa}: the routines of
the sequential \scotch\ library are used on every processor to
complete the coarsening process, compute an initial partition, and
prolong it back up to the largest centralized coarsened graph stored
on the processor. Then, the partitions are prolonged back in parallel
to the finer distributed graphs, selecting the best partition between
the two available when prolonging to a level where fold-dup had been
performed. This distributed prolongation process is repeated until we
obtain a partition of the original graph.

\begin{figure}
~\hfill%
\includegraphics[scale=0.30]{p_f_sepa.eps}
\hfill~\\*[-1em]
\caption{Diagram of the parallel computation of the separator of a
  graph distributed across four processors, by parallel coarsening
  with folding-with-duplication in the last stages, multi-sequential
  computation of initial partitions that are locally prolonged back
  and refined on every processor, and then parallel uncoarsening of
  the best partition encountered.}
\label{fig-sepa}
\end{figure}

\paragraph{Band refinement}

The third level of concurrency concerns the refinement heuristics
which are used to improve the prolonged separators. At the coarsest
levels of the multi-level algorithm, when computations are restricted to
individual processors, the sequential FM algorithm of \scotch\ is
used, but this class of algorithms does not parallelize well.

This problem can be solved in two ways: either by developing
scalable and efficient local optimization algorithms, or by being able
to use the existing sequential FM algorithm on very large graphs.
In~\cite{chpe06a} has been proposed a solution
which enables both approaches, and is based on the following
reasoning. Since every refinement is performed by means of a local
algorithm, which perturbs only in a limited way the position of the
prolonged separator, local refinement algorithms need only to be
passed a subgraph that contains the vertices that are very close to
the prolonged separator.

The computation and use of distributed band graphs is outlined in
Figure~\ref{fig-band}. Given a distributed graph and an initial
separator, which can be spread across several processors, vertices
that are closer to separator vertices than some small user-defined
distance are selected by spreading distance information from all of
the separator vertices, using our halo exchange routine. Then, the
distributed band graph is created, by adding on every processor two
anchor vertices, which are connected to the last layers of vertices of
each of the parts. The vertex weight of the anchor vertices is equal
to the sum of the vertex weights of all of the vertices they replace,
to preserve the balance of the two band parts. Once the separator of
the band graph has been refined using some local optimization
algorithm, the new separator is prolonged back to the original
distributed graph.

\begin{figure}
~\hfill%
\includegraphics[scale=0.37]{p_f_band.eps}
\hfill~\\*[-1em]
\caption{Creation of a distributed band graph. Only vertices closest
  to the separator are kept. Other vertices are replaced by anchor
  vertices of equivalent total weight, linked to band vertices of the
  last layer. There are two anchor vertices per processor, to
  reduce communication. Once the separator has been refined on the
  band graph using some local optimization algorithm, the new
  separator is prolonged back to the original distributed graph.}
\label{fig-band}
\end{figure}

Basing on these band graphs, we have implemented a multi-sequential
refinement algorithm, outlined in Figure~\ref{fig-multi}. At every
distributed uncoarsening step, a distributed band graph is
created. Centralized copies of this band graph are then gathered on
every participating processor, which serve to run fully independent
instances of our sequential FM algorithm. The perturbation of the
initial state of the sequential FM algorithm on every processor allows
us to explore slightly different solution spaces, and thus to improve
refinement quality. Finally, the best refined band separator is
prolonged back to the distributed graph, and the uncoarsening process
goes on.

\begin{figure}
~\hfill%
\includegraphics[scale=0.28]{p_f_multi.eps}
\hfill~\\*[-1em]
\caption{Diagram of the multi-sequential refinement of a separator
  prolonged back from a coarser graph distributed across four processors
  to its finer distributed graph. Once the distributed band graph is
  built from the finer graph, a centralized version of it is gathered
  on every participating processor. A sequential FM optimization can
  then be run independently on every copy, and the best improved
  separator is then distributed back to the finer graph.}
\label{fig-multi}
\end{figure}

\subsubsection{Performance criteria}
\label{sec-order-perf}

The quality of orderings is evaluated with respect to several
criteria. The first one, \NNZ, is the number of non-zero terms in the
factored reordered matrix. The second one, \OPC, is the operation
count, that is the number of arithmetic operations required to factor
the matrix. The operation count that we have considered takes into
consideration all operations (additions, subtractions,
multiplications, divisions) required by Cholesky factorization, except
square roots; it is equal to $\sum_c n_c^2$, where $n_c$ is the number
of non-zeros of column $c$ of the factored matrix, diagonal included.

A third criterion for quality is the shape of the elimination tree;
concurrency in parallel solving is all the higher as the elimination tree is
broad and short. To measure its quality, several parameters can be defined:
\hmin, \hmax, and \havg\ denote the minimum, maximum, and average heights
of the tree\footnote%
{We do not consider as leaves the disconnected vertices that are present in
some meshes, since they do not participate in the solving process.},
respectively, and \hdlt\ is the variance, expressed as a percentage of \havg.
Since small separators result in small chains in the elimination tree,
\havg\ should also indirectly reflect the quality of separators.
                                  % Introductions
%%%%%%%%%%%%%%%%%%%%%%%%%%%%%%%%%%%%
%                                  %
% Titre  : p_c.tex                 %
% Sujet  : Manuel de l'utilisateur %
%          du projet 'PT-Scotch'   %
%          Changes                 %
% Auteur : Francois Pellegrini     %
%                                  %
%%%%%%%%%%%%%%%%%%%%%%%%%%%%%%%%%%%%

\section{Updates}
\label{sec-changes}

\subsection{Changes from version 5.0}

\ptscotch\ now provides routines to compute in
parallel partitions of distributed graphs.

A new integer index type has been created in the Fortran interface, to
address array indices larger than the maximum value which can be
stored in a regular integer. Please refer to
Section~\ref{sec-install-inttypesize} for more information.

A new set of routines has been designed, to ease the use of the
\libscotch\ as a dynamic library. The {\tt SCOTCH\_\lbt version}
routine returns the version, release and patchlevel numbers of the
library being used. The {\tt SCOTCH\_\lbt *Alloc} routines,
which are only available in the C interface at the time being,
dynamically allocate storage space for the opaque API
\scotch\ structures, which frees application programs from the need
to be systematically recompiled because of possible changes of
\scotch\ structure sizes.

\subsection{Changes from version 5.1}

Unlike its sequential counterpart, version {\sc 6.0} of
\ptscotch\ does not bring major algorithmic improvements with respect
to the latest {\sc 5.1.12} release of the {\sc 5.1} branch.

In order to ease the work of people writing numerical solvers, it
exposes in its interface a new distributed graph handling routine,
{\tt SCOTCH\_\lbt dgraph\lbt Redist}, that builds a redistributed
graph from an existing distributed graph and partition data. See
Section~\ref{sec-lib-dgraphredist}.
                                  % Changes since previous versions
%%%%%%%%%%%%%%%%%%%%%%%%%%%%%%%%%%%%
%                                  %
% Titre  : s_f.tex                 %
% Sujet  : Manuel de l'utilisateur %
%          du projet 'Scotch'      %
%          Formats de fichiers 5.0 %
% Auteur : Francois Pellegrini     %
%                                  %
%%%%%%%%%%%%%%%%%%%%%%%%%%%%%%%%%%%%

\section{Files and data structures}
\label{sec-file}

For the sake of portability and readability, all the data files shared
by the different programs of the \scotch\ project are coded in plain
ASCII text exclusively.  Although we may speak of ``lines'' when
describing file formats, text-formatting characters such as newlines
or tabulations are not mandatory, and are not taken into account when
files are read.  They are only used to provide better readability and
understanding.  Whenever numbers are used to label objects, and unless
explicitely stated, {\bf numberings always start from zero}, not one.

\subsection{Distributed graph files}
\label{sec-file-dsgraph}

Because even very large graphs are most often stored in the form of
centralized files, the distributed graph loading routine of the
\ptscotch\ package, as well as all parallel programs which handle
distributed graphs, are able to read centralized graph files in the
\scotch\ format and to scatter them on the fly across the available
processors (the format of centralized \scotch\ graph files is
described in the {\it\scotch\ User's Guide}~\scotchcitesuser).
However, in order to reduce loading time, a distributed graph format
has been designed, so that the different file fragments which comprise
distributed graph files can be read in parallel and be stored on
local disks on the nodes of a parallel or grid cluster.
\\

Distributed graph files, which usually end in ``{\tt \@.dgr}'',
describe fragments of valuated graphs, which can be valuated process
graphs to be mapped onto target architectures, or graphs representing
the adjacency structures of matrices to order.

In \scotch, graphs are represented by means of adjacency lists: the
definition of each vertex is accompanied by the list of all of its
neighbors, i.e.  all of its adjacent arcs. Therefore, the overall
number of edge data is twice the number of edges. Distributed graphs
are stored as a set of files which contain each a subset of graph vertices
and their adjacencies. The purpose of this format is to speed-up the
loading and saving of large graphs when working for some time with the
same number of processors: the distributed graph loading routine will
allow each of the processors to read in parallel from a different file.
Consequently, the number of files must be equal to the number of
processors involved in the parallel loading phase.
\\

The first line of a distributed graph file holds the distributed graph
file version number, which is currently {\tt 2}. The second line holds
the number of files across which the graph data is distributed
(referred to as {\tt proc\lbo glb\lbo nbr} in \libscotch; see for
instance Figure~\ref{fig-lib-dgraf-one},
page~\pageref{fig-lib-dgraf-one}, for a detailed example), followed by
the number of this file in the sequence (ranging from $0$ to $({\tt
proc\lbo glb\lbo nbr} - 1)$, and analogous to {\tt proc\lbo loc\lbo
num} in Figure~\ref{fig-lib-dgraf-one}).
The third line holds the global number of graph vertices
(referred to as {\tt vert\lbo glb\lbo nbr}), followed by the global
number of arcs (inappropriately called {\tt edge\lbo glb\lbo nbr}, as
it is in fact equal to twice the actual number of edges).
The fourth line holds the number of vertices contained in
this graph fragment (analogous to {\tt vert\lbo loc\lbo nbr}),
followed by its local number of arcs (analogous to
{\tt edge\lbo loc\lbo nbr}).
The fifth line holds two figures: the graph base index value ({\tt
baseval}) and a numeric flag.

The graph base index value records the value of the starting index
used to describe the graph; it is usually $0$ when the graph has been
output by C programs, and $1$ for Fortran programs. Its purpose is to
ease the manipulation of graphs within each of these two environments,
while providing compatibility between them.

The numeric flag, similar to the one used by the \chaco\ graph
format~\cite{hele93c}, is made of three decimal digits.
A non-zero value in the units indicates that vertex weights are provided.
A non-zero value in the tenths indicates that edge weights are provided.
A non-zero value in the hundredths indicates that vertex labels are provided;
if it is the case, vertices can be stored in any order in the file; else,
natural order is assumed, starting from the starting global index of
each fragment.

This header data is then followed by as many lines as there are
vertices in the graph fragment, that is, {\tt vert\lbo loc\lbo nbr}
lines. Each of these lines begins with the vertex label, if necessary,
the vertex load, if necessary, and the vertex degree, followed by the
description of the arcs. An arc is defined by the load of the edge, if
necessary, and by the label of its other end vertex.
The arcs of a given vertex can be provided in any order in its
neighbor list. If vertex labels are provided, vertices can also be
stored in any order in the file.

Figure~\ref{fig-file-dsgraph} shows the contents of two complementary
distributed graph files modeling a cube with unity vertex and edge
weights and base $0$, distributed across two processors.

\begin{figure}[hbt]
\begin{center}
\begin{minipage}{4.0cm}
{\renewcommand{\baselinestretch}{1.05}
 \footnotesize \tt
\begin{verbatim}
2
2       0
8       24
4       12
0       000
3       4       2       1
3       5       3       0
3       6       0       3
3       7       1       2
\end{verbatim}}
\end{minipage}
\hfil~\hfil
\begin{minipage}{4.0cm}
{\renewcommand{\baselinestretch}{1.05}
 \footnotesize \tt
\begin{verbatim}
2
2       1
8       24
4       12
0       000
3       0       6       5
3       1       7       4
3       2       4       7
3       3       5       6
\end{verbatim}}
\end{minipage}
\end{center}
\caption{Two complementary distributed graph files representing
a cube distributed across two processors.}
\label{fig-file-dsgraph}
\end{figure}
                                  % Formats de fichiers
%%%%%%%%%%%%%%%%%%%%%%%%%%%%%%%%%%%%
%                                  %
% Titre  : p_p.tex                 %
% Sujet  : Manuel de l'utilisateur %
%          du projet 'PT-Scotch'   %
%          Programmes 6.0          %
% Auteur : Francois Pellegrini     %
%                                  %
%%%%%%%%%%%%%%%%%%%%%%%%%%%%%%%%%%%%

\section{Programs}
\label{sec-prog}

\subsection{Invocation}

All of the programs comprised in the \scotch\ and
\ptscotch\ distributions have been designed to run in
command-line mode without any interactive prompting,
so that they can be called easily from other programs by means of
``\mbox{\tt system$\,$()}'' or ``\mbox{\tt popen$\,$()}'' system calls, or be
piped together on a single shell command line. In order to facilitate this,
whenever a stream name is asked for (either on input or output), the user may
put a single ``{\tt -}'' to indicate standard input or output.
Moreover, programs read their input in the same order as stream names are
given in the command line. It allows them to read all their data from a
single stream (usually the standard input), provided that these data are
ordered properly.

A brief on-line help is provided with all the programs. To get this
help, use the ``{\tt -h}'' option after the program name. The case of
option letters is not significant, except when both the lower and
upper cases of a letter have different meanings. When passing
parameters to the programs, only the order of file names is
significant; options can be put anywhere in the command line, in any
order. Examples of use of the different programs
of the \ptscotch\ project are provided in section~\ref{sec-examples}.

Error messages are standardized, but may not be fully explanatory.
However, most of the errors you may run into should be related to file
formats, and located in ``\mbox{\tt \ldots Load}'' routines.
In this case, compare your data formats with the definitions
given in section~\ref{sec-file}, and use the {\tt dgtst}
% and {\tt dmtst}
program of the \ptscotch\ distribution to check the consistency of
your distributed source graphs.
% and meshes.
\\

According to your MPI environment, you may either run the programs
directly, or else have to invoke them by means of a command such as
{\tt mpirun}. Check your local MPI documentation to see how to
specify the number of processors on which to run them.

\subsection{File names}
\label{sec-prog-filename}

\subsubsection{Sequential and parallel file opening}

The programs of the \ptscotch\ distribution can handle either
the classical centralized \scotch\ graph files, or the
distributed \ptscotch\ graph files described in
section~\ref{sec-file-dsgraph}.

In order to tell whether programs should read from, or write to, a single
file located on only one processor, or to multiple instances of the same
file on all of the processors, or else to distinct files on each of the
processors, a special grammar has been designed, which is based on the
``{\tt \%}'' escape character. Four such escape sequences are defined,
which are interpreted independently on every processor, prior to
file opening. By default, when a filename is provided, it is assumed
that the file is to be opened on only one of the processors, called
the root processor, which is usually process $0$ of the communicator
within which the program is run. Using any of the first three
escape sequences below will instruct programs to open in parallel a file
of name equal to the interpreted filename, on every processor on which they
are run. 
\begin{itemize}
\iteme[{\tt \%p}]
Replaced by the number of processes in the global communicator in
which the program is run. Leads to parallel opening.
\iteme[{\tt \%r}]
Replaced on each process running the program by the rank of this
process in the global communicator. Leads to parallel opening.
\iteme[{\tt \%-}]
Discarded, but leads to parallel opening. This sequence is mainly used to
instruct programs to open on every processor a file of identical
name. The opened files can be, according whether the given path leads
to a shared directory or to directories that are local to each
processor, either to the opening of multiple instances of the same
file, or to the opening of distinct files which may each have a
different content, respectively (but in this latter case it is much
recommended to identify files by means of the ``{\tt \%r}'' sequence).
\iteme[{\tt \%\%}]
Replaced by a single ``{\tt \%}'' character. File names using this
escape sequence are not considered for parallel opening, unless one or
several of the three other escape sequences are also present.
\end{itemize}
For instance, filename ``{\tt brol}'' will lead to the
opening of file ``{\tt brol}'' on the root processor only, filename
``{\tt \%-brol}'' (or even ``{\tt br\%-ol}'') will lead to the
parallel opening of files called ``{\tt brol}'' on every processor,
and filename ``{\tt brol\%p-\%r}'' will lead to the opening of files
``{\tt brol2-0}'' and ``{\tt brol2-1}'', respectively, on each of the
two processors on which which would run a program of the
\ptscotch\ distribution.

\subsubsection{Using compressed files}
\label{sec-prog-compressed}

Starting from version 5.0.6, \scotch\ allows users to provide and
retrieve data in compressed form. Since this feature requires that
the compression and decompression tasks run in the same time as data
is read or written, it can only be done on systems which support
multi-threading (Posix threads) or multi-processing (by means of
{\tt fork} system calls).

To determine if a stream has to be handled in compressed form,
\scotch\ checks its extension. If it is ``{\tt .gz}'' ({\tt gzip}
format), ``{\tt .bz2}'' ({\tt bzip2} format) or ``{\tt .lzma}''
({\tt lzma} format), the stream is assumed to be compressed according
to the corresponding format. A filter task will then be used to process
it accordingly if the format is implemented in \scotch\ and enabled on
your system.

To date, data can be read and written in {\tt bzip2} and {\tt gzip}
formats, and can also be read in the {\tt lzma} format. Since the
compression ratio of {\tt lzma} on \scotch\ graphs is $30\%$ better
than the one of {\tt gzip} and {\tt bzip2} (which are almost
equivalent in this case), the {\tt lzma} format is a very good choice
for handling very large graphs. To see how to enable compressed data
handling in \scotch, please refer to Section~\ref{sec-install}.
\\

When the compressed format allows it, several files can be provided on
the same stream, and be uncompressed on the fly. For instance, the
command ``{\tt cat brol.grf.gz brol.xyz.gz | gout -.gz -.gz -Mn -
brol.iv}'' concatenates the topology and geometry data of some graph
{\tt brol} and feed them as a single compressed stream to the standard
input of program {\tt gout}, hence the ''{\tt -.gz}'' to indicate a
compressed standard stream.

\subsection{Description}

\subsubsection{{\tt dgmap} / {\tt dgpart}}
\label{sec-prog-dgmap}

\begin{itemize}
\progsyn
{\tt dgmap} [{\it input\_graph\_file} [{\it input\_\lbt target\_\lbt file} [{\it output\_\lbt mapping\_\lbt file} [{\it output\_\lbt log\_\lbt file}]]]] {\it options}\\
~\\
{\tt dgpart} {\it number\_\lbt of\_\lbt parts} [{\it input\_graph\_file} [{\it output\_\lbt mapping\_\lbt file} [{\it output\_\lbt log\_\lbt file}]]] {\it options}

\progdes

The {\tt dgmap} program is the parallel static mapper. It uses a
static mapping strategy to compute a mapping of the given source graph
to the given target architecture. The implemented algorithms aim at
assigning source graph vertices to target vertices such that every
target vertex receives a set of source vertices of summed weight
proportional to the relative weight of the target vertex in the target
architecture, and such that the communication cost function $f_C$ is
minimized (see Section~\ref{sec-algo-cost} for the definition and
rationale of this cost function).

Since its main purpose is to provide mappings that exhibit high
concurrency for communication minimization in the mapped application,
it comprises a parallel implementation of the dual recursive
bipartitioning algorithm~\cite{pell94a}, as well as all of the
sequential static mapping methods used by its sequential counterpart
{\tt gmap}, to be used on subgraphs located on single processors.

{\tt dgpart} is a simplified interface to {\tt dgmap}, which performs
graph partitioning instead of static mapping. Consequently, the
desired number of parts has to be provided, in lieu of the target
architecture.

The {\tt -b} and {\tt -c} options allow the user to set preferences on
the behavior of the mapping strategy which is used by default. The
{\tt -m} option allows the user to define a custom mapping strategy.

The {\it input\_graph\_file} filename can refer either to a
centralized or to a distributed graph, according to the semantics
defined in Section~\ref{sec-prog-filename}. The mapping file must be
a centralized file.

\progopt\\*
Since the program is devoted to experimental studies, it has many
optional parameters, used to test various execution modes. Values
set by default will give best results in most cases.
\begin{itemize}
\iteme[{\tt -b}{\it rat}]
Set the maximum load imbalance ratio to \textit{rat}, which should
be a value comprised between $0$ and $1$. This option can be used in
conjunction with option \texttt{-c}, but is incompatible with option
\texttt{-m}.
\iteme[{\tt -c}{\it flags}]
Tune the default mapping strategy according to the given preference
flags. Some of these flags are antagonistic, while others can be
combined. See Section~\ref{sec-lib-format-strat-default} for more
information. The currently available flags are the following.
\begin{itemize}
\iteme[{\tt b}]
Enforce load balance as much as possible.
\iteme[{\tt q}]
Privilege quality over speed. This is the default behavior.
\iteme[{\tt s}]
Privilege speed over quality.
\iteme[{\tt t}]
Use only safe methods in the strategy.
\iteme[{\tt x}]
Favor scalability.
\end{itemize}
This option can be used in conjunction with option \texttt{-b}, but is
incompatible with option \texttt{-m}.
The resulting strategy string can be displayed by means
of the {\tt -vs} option.
\iteme[{\tt -h}]
Display the program synopsis.
\iteme[{{\tt -m}{\it strat}}]
Apply parallel static mapping strategy {\it strat}. The format of parallel
mapping strategies is defined in section~\ref{sec-lib-format-pmap}.
This option is incompatible with options \texttt{-b} and
\texttt{-c}.
\iteme[{\tt -r}{\it num}]
Set the number of the root process which will be used for centralized
file accesses. Set to $0$ by default.
\iteme[{\tt -s}{\it obj}]
Mask source edge and vertex weights. This option allows the user to
``unweight'' weighted source graphs by removing weights from edges and
vertices at loading time. {\it obj\/} may contain several of the following
switches.
\begin{itemize}
\iteme[{\tt e}]
Remove edge weights, if any.
\iteme[{\tt v}]
Remove vertex weights, if any.
\end{itemize}
\iteme[{\tt -V}]
Print the program version and copyright.
\iteme[{\tt -v}{\it verb}]
Set verbose mode to {\it verb}, which may contain several of the following
switches.
%For a detailed description of the data displayed, please
%refer to the manual page of {\tt dgmtst} below.
\begin{itemize}
\iteme[{\tt a}]
Memory allocation information.
\iteme[{\tt m}]
Mapping information, similar to the one displayed by the {\tt gmtst}
program of the sequential \scotch\ distribution.
\iteme[{\tt s}]
Strategy information. This parameter displays the default mapping
strategy used by {\tt gmap}.
\iteme[{\tt t}]
Timing information.
\end{itemize}
\end{itemize}
\end{itemize}

\subsubsection{{\tt dgord}}

\begin{itemize}
\progsyn
{\tt dgord} [{\it input\_graph\_file} [{\it output\_ordering\_file} [{\it output\_log\_file}]]] {\it options}

\progdes

The {\tt dgord} program is the parallel sparse matrix block
orderer. It uses an ordering strategy to compute block orderings of
sparse matrices represented as source graphs, whose vertex weights
indicate the number of DOFs per node (if this number is non
homogeneous) and whose edges are unweighted, in order to minimize
fill-in and operation count.

Since its main purpose is to provide orderings that exhibit high
concurrency for parallel block factorization, it comprises a parallel
nested dissection method~\cite{geli81}, but sequential
classical~\cite{liu-85} and state-of-the-art~\cite{peroam00a}
minimum degree algorithms are implemented as well, to be used on
subgraphs located on single processors.

Ordering methods can be combined by means of selection, grouping, and
condition operators, so as to define ordering strategies, which can be
passed to the program by means of the {\tt -o} option. The {\tt -c}
option allows the user to set preferences on the behavior of the
ordering strategy which is used by default.

The {\it input\_graph\_file} filename can refer either to a
centralized or to a distributed graph, according to the semantics
defined in Section~\ref{sec-prog-filename}. The ordering file must be
a centralized file.

\progopt\\*
Since the program is devoted to experimental studies, it has many
optional parameters, used to test various execution modes. Values
set by default will give best results in most cases.
\begin{itemize}
\iteme[{\tt -c}{\it flags}]
Tune the default ordering strategy according to the given preference
flags. Some of these flags are antagonistic, while others can be
combined. See Section~\ref{sec-lib-format-strat-default} for more
information. The resulting strategy string can be displayed by means
of the {\tt -vs} option.
\begin{itemize}
\iteme[{\tt b}]
Enforce load balance as much as possible.
\iteme[{\tt q}]
Privilege quality over speed. This is the default behavior.
\iteme[{\tt s}]
Privilege speed over quality.
\iteme[{\tt t}]
Use only safe methods in the strategy.
\iteme[{\tt x}]
Favor scalability.
\end{itemize}
\iteme[{\tt -h}]
Display the program synopsis.
\iteme[{\tt -m}{\it output\_mapping\_file}]
Write to {\it output\_mapping\_file\/} the mapping of graph vertices to
column blocks. All of the separators and leaves produced by the nested
dissection method are considered as distinct column blocks, which may
be in turn split by the ordering methods that are applied to them.
Distinct integer numbers are associated with each of the column blocks,
such that the number of a block is always greater than the ones of its
predecessors in the elimination process, that is, its descendants in
the elimination tree.
The structure of mapping files is described in detail in the relevant
section of the {\it\scotch\ User's Guide}~\scotchcitesuser.

When the geometry of the graph is available, this mapping file may be
processed by program {\tt gout} to display the vertex separators and
supervariable amalgamations that have been computed.
\iteme[{{\tt -o}{\it strat}}]
Apply parallel ordering strategy {\it strat}. The format of parallel
ordering strategies is defined in section~\ref{sec-lib-format-pord}.
\iteme[{\tt -r}{\it num}]
Set the number of the root process which will be used for centralized
file accesses. Set to $0$ by default.
\iteme[{\tt -t}{\it output\_tree\_file}]
Write to {\it output\_tree\_file\/} the structure of the separator
tree. The data that is written resembles much the one of a mapping
file: after a first line that contains the number of lines to follow,
there are that many lines of mapping pairs, which associate an integer
number with every graph vertex index. This integer number is the
number of the column block which is the parent of the column block to
which the vertex belongs, or $-1$ if the column block to which the
vertex belongs is a root of the separator tree (there can be several
roots, if the graph is disconnected).

Combined to the column block mapping data produced by option {\tt -m},
the tree structure allows one to rebuild the separator tree.
\iteme[{\tt -V}]
Print the program version and copyright.
\iteme[{\tt -v}{\it verb}]
Set verbose mode to {\it verb}, which may contain several of the following
switches.
%For a detailed description of the data displayed, please
%refer to the manual page of {\tt gotst}.
\begin{itemize}
\iteme[{\tt a}]
Memory allocation information.
\iteme[{\tt s}]
Strategy information. This parameter displays the default parallel
ordering strategy used by {\tt dgord}.
\iteme[{\tt t}]
Timing information.
\end{itemize}
\end{itemize}
\end{itemize}

\subsubsection{{\tt dgpart}}

\begin{itemize}
\progsyn
{\tt dgpart} [{\it number\_of\_parts} [{\it input\_\lbt graph\_\lbt file} [{\it output\_\lbt mapping\_\lbt file} [{\it output\_\lbt log\_\lbt file}]]]] {\it options}

\progdes

The {\tt dgpart} program is the parallel graph partitioner. It is
in fact a shortcut for the {\tt dgmap} program, where the number of
parts is turned into a complete graph with same number of vertices
which is passed to the static mapping routine.

Save for the {\it number\_of\_parts} parameter which replaces the {\it
input\_target\_file}, the parameters of {\tt dgpart} are identical to
the ones of {\tt dgmap}. Please refer to its manual page, in
Section~\ref{sec-prog-dgmap}, for a description of all of the
available options.
\end{itemize}

\subsubsection{{\tt dgscat}}

\begin{itemize}
\progsyn
{\tt dgscat} [{\it input\_graph\_file} [{\it output\_graph\_file}]] {\it options}

\progdes

The {\tt dgscat} program creates a distributed source graph, in the
\scotch\ distributed graph format, from the given centralized source
graph file.

The {\it input\_graph\_file} filename should therefore refer to a
centralized graph, while {\it output\_graph\_file} must refer to a
distributed graph, according to the semantics defined in
Section~\ref{sec-prog-filename}.

\progopt\\[-1em]
\begin{itemize}
\iteme[{\tt -c}]
Check the consistency of the distributed graph at the end of the
graph loading phase.
\iteme[{\tt -h}]
Display the program synopsis.
\iteme[{\tt -r}{\it num}]
Set the number of the root process which will be used for centralized
file accesses. Set to $0$ by default.
\iteme[{\tt -V}]
Print the program version and copyright.
\end{itemize}
\end{itemize}

\subsubsection{{\tt dgtst}}

\begin{itemize}
\progsyn
{\tt dgtst} [{\it input\_graph\_file} [{\it output\_data\_file}]] {\it options}

\progdes

The program {\tt dgtst} is the source graph tester. It checks the
consistency of the input source graph structure (matching of arcs,
number of vertices and edges, etc\@.), and gives some statistics
regarding edge weights, vertex weights, and vertex degrees.

It produces the same results as the {\tt gtst} program of the
\scotch\ sequential distribution.

\progopt
\begin{itemize}
\iteme[{\tt -h}]
Display the program synopsis.
\iteme[{\tt -r}{\it num}]
Set the number of the root process which will be used for centralized
file accesses. Set to $0$ by default.
\iteme[{\tt -V}]
Print the program version and copyright.
\end{itemize}
\end{itemize}
                                  % Programmes
%%%%%%%%%%%%%%%%%%%%%%%%%%%%%%%%%%%%
%                                  %
% Titre  : p_l.tex                 %
% Sujet  : Manuel de l'utilisateur %
%          du projet 'PT-Scotch'   %
%          Bibliotheque            %
% Auteur : Francois Pellegrini     %
%                                  %
%%%%%%%%%%%%%%%%%%%%%%%%%%%%%%%%%%%%

\section{Library}
\label{sec-lib}

All of the features provided by the programs of the
\ptscotch\ distribution may be directly accessed by calling
the appropriate functions of the \libscotch\ library, archived
in files {\tt libptscotch.a} and {\tt libptscotcherr.a}.
All of the existing parallel routines belong to four distinct classes:
\begin{itemize}
\item
distributed source graph handling routines, which serve to declare,
build, load, save, and check the consistency of distributed source
graphs;
%, along with their geometry data;
\item
strategy handling routines, which allow the user to declare and build
parallel mapping and ordering strategies;
\item
parallel graph partitioning and static mapping routines, which allow
the user to declare, compute, and save distributed static mappings of
distributed source graphs;
\item
parallel ordering routines, which allow the user to declare, compute,
and save distributed orderings of distributed source graphs.
\end{itemize}
Error handling is performed using the existing sequential routines of
the \scotch\ distribution, which are described in the
{\it\scotch\ User's Guide}~\scotchcitesuser. Their use is recalled in
Section~\ref{sec-lib-error}.

A \parmetis\ compatibility library, called {\tt lib\lbo ptscotch\lbo
parmetis.a}, is also available. It allows users who were previously
using \parmetis\ in their software to take advantage of the efficieny
of \ptscotch\ without having to modify their code. The services
provided by this library are described in
Section~\ref{sec-lib-parmetis}.

\subsection{Running at proper thread level}
\label{sec-lib-thread}

Since \ptscotch\ is based on the MPI API, all processes must call some
flavor of \texttt{MPI\_\lbt Init} before using any routine of the
library that performs communication. The thread support level of MPI
has to be set in accordance to the level required by the library.

If \ptscotch\ has been compiled without the \texttt{-DSCOTCH\_\lbt
PTHREAD} flag, a call to the simple \texttt{MPI\_\lbt Init} routine
will suffice, because no concurrent MPI calls will be performed by
library routines. Else, the extended \texttt{MPI\_\lbt Init\_\lbt
thread} initialization routine has to be used, to request the
\texttt{MPI\_\lbt THREAD\_\lbt MULTIPLE} level, and the provided
thread support level value returned by the routine must be checked
carefully. If your MPI implementation does not provide the
\texttt{MPI\_\lbt THREAD\_\lbt MULTIPLE} level, you will have to
recompile \ptscotch\ without the \texttt{-DSCOTCH\_\lbt PTHREAD}
flag. Else, library calls may cause random bugs in the communication
subsystem, resulting in program crashes.

\subsection{Calling the routines of {\sc libScotch}}

\subsubsection{Calling from C}

All of the C routines of the \libscotch\ library are prefixed with
``{\tt SCOTCH\_}''. The remainder of the function names is made of the
name of the type of object to which the functions apply (e\@.g\@.
``{\tt dgraph}'', ``{\tt dorder}'', etc.),
followed by the type of action performed on this object: ``{\tt
Init}'' for the initialization of the object, ``{\tt Exit}'' for the
freeing of its internal structures, ``{\tt Load}'' for loading the
object from one or several streams, and so on.

Typically, functions that return an error code return zero if the
function succeeds, and a non-zero value in case of error.

For instance, the {\tt SCOTCH\_\lbt dgraph\lbt Init} and {\tt
SCOTCH\_\lbt dgraph\lbt Load} routines, described in
section~\ref{sec-lib-dgraph}, can be called from C by using the
following code.
{\tt
\begin{verbatim}
#include <stdio.h>
#include <mpi.h>
#include "ptscotch.h"
  ...
  SCOTCH_Dgraph     grafdat;
  FILE *            fileptr;

  if (SCOTCH_dgraphInit (&grafdat) != 0) {
    ... /* Error handling */
  }
  if ((fileptr = fopen ("brol.grf", "r")) == NULL) {
  ... /* Error handling */
  }
  if (SCOTCH_dgraphLoad (&grafdat, fileptr, -1, 0) != 0) {
    ... /* Error handling */
  }
  ...
\end{verbatim}
}

Since ``{\tt ptscotch.h}'' uses several system and communication
objects which are declared in ``{\tt stdio.h}'' and ``{\tt mpi.h}'',
respectively, these latter files must be included beforehand
in your application code.

Although the ``{\tt scotch.h}'' and ``{\tt ptscotch.h}'' files may
look very similar on your system, never mistake them, and always use
the ``{\tt ptscotch.h}'' file as the right include file for compiling
a program which uses the parallel routines of the \libscotch\ library,
whether it also calls sequential routines or not.

\subsubsection{Calling from Fortran}

The routines of the \libscotch\ library can also be called from
Fortran. For any C function named {\tt SCOTCH\_\lbt {\it type\lbt
Action}()} which is documented in this manual, there exists a {\tt
SCOTCHF\lbt {\it TYPE\lbt ACTION\/}()} Fortran counterpart, in which
the separating underscore character is replaced by an ``{\tt
F}''. In most cases, the Fortran routines have exactly the same
parameters as the C functions, save for an added trailing {\tt
INTEGER} argument to store the return value yielded by the function
when the return type of the C function is not {\tt void}.
\\

Since all the data structures used in \libscotch\ are
opaque, equivalent declarations for these structures must
be provided in Fortran. These structures must therefore
be defined as arrays of {\tt DOUBLEPRECISION}s, of sizes
given in file {\tt ptscotchf.h}, which must be included whenever
necessary.

For routines that read or write data using a {\tt FILE~*} stream
in C, the Fortran counterpart uses an {\tt INTEGER} parameter which
is the numer of the Unix file descriptor corresponding to the logical
unit from which to read or write. In most Unix implementations of
Fortran, standard descriptors 0 for standard input (logical unit 5),
1 for standard output (logical unit 6) and 2 for standard error are
opened by default. However, for files that are opened using
{\tt OPEN} statements, an additional function must be used to obtain
the number of the Unix file descriptor from the number of the logical
unit. This function is called \texttt{PXFFILENO} in the normalized
POSIX Fortran API, and files which use it should include the
\texttt{USE IFPOSIX} directive whenever necessary. An alternate, non
normalized, function also exists in most Unix implementations of
Fortran, and is called {\tt FNUM}.

For instance, the {\tt SCOTCH\_\lbt dgraph\lbt Init} and
{\tt SCOTCH\_\lbt dgraph\lbt Load} routines, described in
sections~\ref{sec-lib-dgraphinit} and~\ref{sec-lib-dgraphload},
respectively, can be called from Fortran by using the following code.
{\tt
\begin{verbatim}
        INCLUDE "ptscotchf.h"
        DOUBLEPRECISION GRAFDAT(SCOTCH_DGRAPHDIM)
        INTEGER RETVAL
        ...
        CALL SCOTCHFDGRAPHINIT (GRAFDAT (1), RETVAL)
        IF (RETVAL .NE. 0) THEN
        ...
        OPEN (10, FILE='brol.grf')
        CALL SCOTCHFDGRAPHLOAD (GRAFDAT (1), PXFFILENO (10), 1, 0, RETVAL)
        CLOSE (10)
        IF (RETVAL .NE. 0) THEN
        ...
\end{verbatim}
}

Although the ``{\tt scotchf.h}'' and ``{\tt ptscotchf.h}'' files may
look very similar on your system, never mistake them, and always use
the ``{\tt ptscotchf.h}'' file as the include file for compiling a
Fortran program that uses the parallel routines of the
\libscotch\ library, whether it also calls sequential routines or not.

All of the Fortran routines of the \libscotch\ library are stubs which
call their C counterpart. While this poses no problem for the usual
integer and double precision data types, some conflicts may occur at
compile or run time if your MPI implementation does not represent the
{\tt MPI\_Comm} type in the same way in C and in Fortran. Please check
on your platform to see in the {\tt mpi.h} include file if the {\tt
MPI\_Comm} data type is represented as an {\tt int}. If it is the
case, there should be no problem in using the Fortran routines of
the \ptscotch\ library.

\subsubsection{Compiling and linking}

The compilation of C or Fortran routines which use parallel
routines of the \libscotch\ library requires that either {\tt
ptscotch.h} or {\tt ptscotchf.h} be included, respectively. Since some
of the parallel routines of the \libscotch\ library must be passed MPI
communicators, it is necessary to include MPI files {\tt mpi.h} or
{\tt mpif.h}, respectively, before the relevant \ptscotch\ include
files, such that prototypes of the parallel \libscotch\ routines are
properly defined.

The parallel routines of the \libscotch\ library, along with
taylored versions of the sequential routines, are grouped in a
library file called {\tt libptscotch.a}. Default error routines that
print an error message and exit are provided in the classical
\scotch\ library file {\tt libptscotcherr.a}.

Therefore, the linking of applications that make use of the
\libscotch\ library with standard error handling is carried out by
using the following options: ``{\tt -lptscotch -lptscotcherr -lmpi -lm}''.
The ``{\tt -lmpi}'' option is most often not necessary, as the MPI
library is automatically considered when compiling with commands such
as {\tt mpicc}.

If you want to handle errors by yourself, you should not link with
library file {\tt libptscotcherr.a}, but rather provide a
{\tt SCOTCH\_\lbt error\lbt Print()} routine. Please refer to
Section~\ref{sec-lib-error} for more information on error handling.
Section~\ref{sec-lib-error} for more information on error handling.

Programs that use both sequential and parallel routines of
\scotch\ need only be linked against the parallel version of the
library, as it also contains an adapted version of the sequential
routines. The reason why the sequential routines are duplicated in the
parallel \ptscotch\ library is because they are slightly modified so
as to keep track of the parallel environment. This allows one to
create ``multi-sequential'' routines that can exchange data with other
processes, for instance. Because the \libscotch\ data structures
contain extra parameters, never mix the \texttt{scotch.h} sequential
include file and the \texttt{libptscotch.a} parallel library, as the
latter expects \scotch\ data structures to be larger than the ones
defined in the sequential include file. Consequently, when using only
sequential routines in a sequential program, include the
\texttt{scotch.h} file only and link the program against the sequential
\texttt{libscotch.a} library only. When using only parallel routines,
or when using a mix of sequential and parallel routines, include the
\texttt{ptscotch.h} file only and link the program against the parallel
\texttt{libptscotch.a} library only. When using only sequential
routines in a parallel program, both options can be used.

\subsubsection{Machine word size issues}
\label{sec-lib-inttypesize}

Graph indices are represented in \scotch\ as integer values of type
{\tt SCOTCH\_\lbt Num}. By default, this type equates to the {\tt int}
C type, that is, an integer type of size equal to the one of 
the machine word. However, it can represent any other integer
type. Indeed, the size of the {\tt SCOTCH\_\lbt Num} integer type can
be coerced to 32 or 64 bits by using the ``{\tt -DINTSIZE32}'' or
``{\tt -DINTSIZE64}'' compilation flags, respectively, or else by
using the ``{\tt -DINT=}'' definition (see
Section~\ref{sec-install-inttypesize} for more information on the
setting of these compilation flags).

This may, however, pose a problem with MPI, the interface of which is
based on the regular {\tt int} type. \ptscotch\ has been coded so
as to avoid typecast bugs, but overflow errors may result from the
conversion of values of a larger integer type into {\tt int}s when
handling communication buffer indices.

Consequently, the C interface of \scotch\ uses two types of integers.
Graph-related quantities are passed as {\tt SCOTCH\_\lbt Num}s,
while system-related values such as file handles, as well as
return values of \libscotch\ routines, are always passed as
{\tt int}s.

Because of the variability of library integer type sizes, one must be
careful when using the Fortran interface of \scotch, as it does not
provide any prototyping information, and consequently cannot produce
any warning at link time. In the manual pages of the
\libscotch\ routines, Fortran prototypes are written using three types
of {\tt INTEGER}s. As for the C interface, the regular {\tt INTEGER}
type is used for system-based values, such as file handles and MPI
communicators, as well as for return values of the
\libscotch\ routines, while the {\tt INTEGER*}{\it num} type
should be used for all graph-related values, in accordance to the size
of the {\tt SCOTCH\_\lbt Num} type, as set by the
``{\tt -DINTSIZE}{\it x}'' compilation flags. Also, the
{\tt INTEGER*}{\it idx} type represents an integer type of a size
equivalent to the one of a {\tt SCOTCH\_\lbt Idx}, as set by the
``{\tt -DIDXSIZE}{\it x}'' compilation flags. Values of this type are
used in the Fortran interface to represent arbitrary array indices
which can span across the whole address space, and consequently
deserve special treatment.

In practice, when \scotch\ is compiled on a 32-bit architecture so as
to use 64-bit {\tt SCOTCH\_\lbt Num}s, graph indices should be
declared as {\tt INTEGER*8}, while error return values
should still be declared as plain {\tt INTEGER} (that is,
{\tt INTEGER*4}) values. On a 32\_64-bit architecture, irrespective of
whether {\tt SCOTCH\_\lbt Num}s are defined as {\tt INTEGER*4}
or {\tt INTEGER*8} quantities, the {\tt SCOTCH\_\lbt Idx} type
should always be defined as a 64-bit quantity, that is, an
{\tt INTEGER*8}, because it stores differences between memory
addresses, which are represented by 64-bit values.
The above is no longer a problem if \scotch\ is compiled such that
{\tt int}s equate 64-bit integers. In this case, there is no need to
use any type coercing definition.
\\

The \metis\ v3 compatibility library provided by \scotch\ can also
run on a 64-bit architecture. Yet, if you are willing to use it this
way, you will have to replace all {\tt int}'s that are passed to the
\metis\ routines by 64-bit integer {\tt SCOTCH\_\lbt Num} values (even
the option configuration values). However, in this case, you will no
longer be able to link against the service routines of the genuine
\metis/\parmetis\ v3 library, as they are only available as a 32-bit
implementation.

\subsection{Data formats}

All of the data used in the \libscotch\ interface are of integer type
{\tt SCOTCH\_Num}. To hide the internals of \ptscotch\ to callers, all
of the data structures are opaque, that is, declared within {\tt
ptscotch.h} as dummy arrays of double precision values, for the sake of
data alignment. Accessor routines, the names of which end in ``{\tt
Size}'' and ``{\tt Data}'', allow callers to retrieve information from
opaque structures.
\\

In all of the following, whenever arrays are defined, passed, and
accessed, it is assumed that the first element of these arrays is
always labeled as {\tt baseval}, whether {\tt baseval} is set to $0$
(for C-style arrays) or $1$ (for Fortran-style arrays). \ptscotch\
internally manages with base values and array pointers so as to
process these arrays accordingly.

\subsubsection{Distributed graph format}
\label{sec-lib-format-dgraph}

In \ptscotch, distributed source graphs are represented so as to
distribute graph data without any information duplication which could
hinder scalability. The only data which are replicated on every
process are of a size linear in the number of processes and
small. Apart from these, the sum across all processes of all of the
vertex data is in $O(v+p)$, where $v$ is the overall number of
vertices in the distributed graph and $p$ the number of processes, and
the sum of all of the edge data is in $O(e)$, where $e$ is the overall
number of arcs (that is, twice the number of edges) in the distributed
graph. When graphs are ill-distributed, the overall halo vertex
information may also be in $o(e)$ at worst, which makes the distributed
graph structure fully scalable.

Distributed source graphs are described by means of adjacency
lists. The description of a distributed graph requires several {\tt
SCOTCH\_Num} scalars and arrays, as shown for instance in
Figures~\ref{fig-lib-dgraf-one} and~\ref{fig-lib-dgraf-two}.
Some of these data are said to be global, and are duplicated on every
process that holds part of the distributed graph; their names contain
the ``{\tt glb}'' infix. Others are local, that is, their value may
differ for each process; their names contain the ``{\tt loc}'' or
``{\tt gst}'' infix. Global data have the following meaning:
\begin{itemize}
\iteme[{\tt baseval}]
Base value for all array indexings.
\iteme[{\tt vertglbnbr}]
Overall number of vertices in the distributed graph.
\iteme[{\tt edgeglbnbr}]
Overall number of arcs in the distributed graph. Since edges are
represented by both of their ends, the number of edge data in
the graph is twice the number of edges.
\iteme[{\tt procglbnbr}]
Overall number of processes that share distributed graph data.
\iteme[{\tt proccnttab}]
Array holding the current number of local vertices borne by every process.
\iteme[{\tt procvrttab}]
Array holding the global indices from which the vertices of every
process are numbered. For optimization purposes, this array has an
extra slot which stores a number which must be greater than all of the
assigned global indices. For each process $p$, it must be ensured that
$\mbox{\tt proc\lbt vrt\lbt tab[}p + 1\mbox{\tt]} \geq
(\mbox{\tt proc\lbt vrt\lbt tab[}p\mbox{\tt]} +
\mbox{\tt proc\lbt cnt\lbt tab[}p\mbox{\tt]})$, that is, that no process
can have more local vertices than allowed by its range of global indices.
When the global numbering of vertices is continuous, for each process $p$,
$\mbox{\tt proc\lbt vrt\lbt tab[}p + 1\mbox{\tt]} =
(\mbox{\tt proc\lbt vrt\lbt tab[}p\mbox{\tt]} +
\mbox{\tt proc\lbt cnt\lbt tab[}p\mbox{\tt]})$.
\end{itemize}
Local data have the following meaning:
\begin{itemize}
\iteme[{\tt vertlocnbr}]
Number of local vertices borne by the given process. In
fact, on every process $p$, {\tt vert\lbt loc\lbt nbr} is equal to
${\tt proc\lbt cnt\lbt tab}\mbox{\tt [}p\mbox{\tt]}$.
\iteme[{\tt vertgstnbr}]
Number of both local and ghost vertices borne by the given process.
Ghost vertices are local images of neighboring vertices located on
distant processes.
\iteme[{\tt vertloctab}]
Array of start indices in {\tt edgeloctab} and {\tt edgegsttab} of
vertex adjacency sub-arrays.
\iteme[{\tt vendloctab}]
Array of after-last indices in {\tt edgeloctab} and {\tt edgegsttab}
of vertex adjacency sub-arrays.
For any local vertex $i$, with $\mbox{\tt baseval} \leq i < (\mbox{\tt
baseval} + \mbox{\tt vertlocnbr})$,
$\mbox{\tt vend\lbt loc\lbt tab[}i\mbox{\tt ]} -
\mbox{\tt vert\lbt loc\lbt tab[}i\mbox{\tt ]}$
is the degree of vertex $i$.

When all vertex adjacency lists are stored in order in
{\tt edge\lbt loc\lbt tab} without any empty space between them,
it is possible to save memory by not allocating the physical memory
for {\tt vend\lbt loc\lbt tab}. In this case, illustrated in
Figure~\ref{fig-lib-dgraf-one}, {\tt vert\lbt loc\lbt tab} is of size
$\mbox{\tt vert\lbt loc\lbt nbr} + 1$ and {\tt vend\lbt loc\lbt tab}
points to $\mbox{\tt vert\lbt loc\lbt tab} + 1$.
For these graphs , called ``compact edge array graphs'', or ``compact
graphs'' for short, {\tt vert\lbt loc\lbt tab} is sorted in
ascending order, $\mbox{\tt vert\lbt loc\lbt tab[}\lbt\mbox{\tt
baseval]} = \mbox{\tt baseval}$ and $\mbox{\tt vert\lbt loc\lbt
tab[}\lbt\mbox{\tt baseval} + \mbox{\tt vert\lbt loc\lbt nbr]} =
(\mbox{\tt baseval} + \mbox{\tt edge\lbt loc\lbt nbr})$.

Since {\tt vertloctab} and {\tt vendloctab} only account for local
vertices and not for ghost vertices, the sum across all processes of the
sizes of these arrays does not depend on the number of ghost vertices;
it is equal to $(v+p)$ for compact graphs and to $2v$ else.
\iteme[{\tt veloloctab}]
Optional array, of size {\tt vert\lbt loc\lbt nbr}, holding the
integer load associated with every vertex.
\iteme[{\tt edgeloctab}]
Array, of a size equal at least to $\left(\max_{i}(\mbox{\tt vend\lbt
loc\lbt tab[} i \mbox{\tt ]}) - \mbox{\tt baseval}\right)$, holding
the adjacency array of every local vertex.
For any local vertex $i$, with $\mbox{\tt baseval} \leq i < (\mbox{\tt
baseval} + \mbox{\tt vertlocnbr})$, the global indices of the neighbors
of $i$ are stored in {\tt edge\lbt loc\lbt tab}
from {\tt edge\lbt loc\lbt tab\lbt [vert\lbt loc\lbt tab[$i$]]} to
$\mbox{\tt edge\lbt loc\lbt tab[vend\lbt loc\lbt tab[}i\mbox{\tt ]} -
1\mbox{\tt ]}$, inclusive.

Since ghost vertices do not have adjacency arrays, because only arcs
from local vertices to ghost vertices are recorded and not the
opposite, the overall sum of the sizes of all {\tt edge\lbt loc\lbt
tab} arrays is $e$.
\iteme[{\tt edgegsttab}]
Optional array holding the local and ghost indices of neighbors of
local vertices.
For any local vertex $i$, with $\mbox{\tt baseval} \leq i < (\mbox{\tt
baseval} + \mbox{\tt vertlocnbr})$, the local and ghost indices
of the neighbors of $i$ are stored in {\tt edge\lbt gst\lbt tab}
from {\tt edge\lbt gst\lbt tab\lbt [vert\lbt loc\lbt tab[$i$]]} to
{\tt edge\lbt gst\lbt tab[vend\lbt loc\lbt tab[$i$]\lbt $- 1$]},
inclusive.

Local vertices are numbered in global vertex order,
starting from {\tt baseval} to $(\mbox{\tt baseval} +
\mbox{\tt vertlocnbr} - 1)$, inclusive. Ghost vertices are also
numbered in global vertex order, from $(\mbox{\tt baseval} +
\mbox{\tt vertlocnbr})$ to $(\mbox{\tt baseval} +
\mbox{\tt vertgstnbr} - 1)$, inclusive.

Only {\tt edgeloctab} has to be provided by the user. {\tt edge\lbt
gst\lbt tab} is internally computed by \ptscotch\ whenever needed,
or can be explicitey asked for by the user by calling function
{\tt SCOTCH\_\lbt dgraph\lbt Ghst}.
This array can serve to index user-defined arrays of quantities borne
by graph vertices, which can be exchanged between neighboring
processes thanks to the {\tt SCOTCH\_\lbt dgraph\lbt Halo}
routine documented in Section~\ref{sec-lib-dgraphhalo}.
\iteme[{\tt edloloctab}]
Optional array, of a size equal at least to $\left(\max_{i}(\mbox{\tt
vend\lbt loc\lbt tab[} i \mbox{\tt ]}) - \mbox{\tt baseval}\right)$,
holding the integer load associated with every arc. Matching arcs
should always have identical loads.
\end{itemize}

\begin{figure}
\centering\includegraphics[scale=0.47]{p_f_gr1.eps}
\caption{Sample distributed graph and its description by
\libscotch\ arrays using a continuous numbering and compact edge
arrays. Numbers within vertices are vertex indices. Top graph is a
global view of the distributed graph, labeled with global, continuous,
indices. Bottom graphs are local views labeled with local and ghost
indices, where ghost vertices are drawn in black.  Since the edge
array is compact, all {\tt vertloctab} arrays are of size $\mbox{\tt
vertlocnbr} + 1$, and {\tt vendloctab} points to $\mbox{\tt
vertloctab} + 1$. {\tt edgeloctab} edge arrays hold global indices
of end vertices, while optional {\tt edgegsttab} edge arrays hold
local and ghost indices. {\tt edgelocnbr} is the local number of arcs
(that is, twice the number of edges), including arcs to local vertices
as well as to ghost vertices {\tt veloloctab} and {\tt edloloctab} are
not represented.}
\label{fig-lib-dgraf-one}
\end{figure}

\begin{figure}
\centering\includegraphics[scale=0.47]{p_f_gr2.eps}
\caption{Adjacency structure of the sample graph of
Figure~\protect\ref{fig-lib-dgraf-one} with a disjoint edge array and
a discontinuous ordering. Both {\tt vertloctab} and {\tt vendloctab}
are of size {\tt vertlocnbr}. This allows for the handling of dynamic
graphs, the structure of which can evolve with time.}
\label{fig-lib-dgraf-two}
\end{figure}

Dynamic graphs can be handled elegantly by using the
{\tt vend\lbt loc\lbt tab} and {\tt proc\lbt vrt\lbt tab} arrays.
In order to dynamically manage distributed graphs, one just has to
reserve index ranges large enough to create new vertices on each
process, and to allocate
{\tt vert\lbt loc\lbt tab}, {\tt vend\lbt loc\lbt tab} and
{\tt edge\lbt loc\lbt tab} arrays that are large
enough to contain all of the expected new vertex and edge
data. This can be done by passing {\tt SCOTCH\_\lbt graph\lbo Build} a
maximum number of local vertices, {\tt vert\lbt loc\lbt max}, greater
than the current number of local vertices, {\tt vert\lbt loc\lbt nbr}.

On every process $p$, vertices are globally labeled starting from
${\tt proc\lbt vrt\lbt tab}\mbox{\tt [}p\mbox{\tt ]}$, and locally
labeled from {\tt baseval}, leaving free space at the end of the local
arrays. To remove some vertex of local index $i$, one just has to
replace $\mbox{\tt vert\lbt loc\lbt tab}\mbox{\tt [}i\mbox{\tt ]}$ and
${\tt vend\lbt loc\lbt tab}\mbox{\tt [}i\mbox{\tt ]}$ with the values
of ${\tt vert\lbt loc\lbt tab}\lbt \mbox{\tt [vert\lbt loc\lbt
nbr}-1\mbox{\tt ]}$ and ${\tt vend\lbt loc\lbt tab}\lbt \mbox{\tt
[vert\lbt loc\lbt nbr}-1\mbox{\tt ]}$, respectively, and browse the
adjacencies of all neighbors of former vertex $(\mbox{\tt
vert}\lbt\mbox{\tt loc}\lbt\mbox{\tt nbr}-1)$ such that all
$(\mbox{\tt vert}\lbt\mbox{\tt loc}\lbt\mbox{\tt nbr}-1)$ indices are
turned into $i$s. Then, {\tt vert\lbt loc\lbt nbr} must be
decremented, and {\tt SCOTCH\_\lbt dgraph\lbt Build()} must be called
to account for the change of topology. If a graph building routine
such as {\tt SCOTCH\_\lbt dgraph\lbt Load()} or {\tt SCOTCH\_\lbt
dgraph\lbt Build()} had already been called on the {\tt SCOTCH\_\lbt
Dgraph} structure, {\tt SCOTCH\_\lbt dgraph\lbt Free()} has to be
called first in order to free the internal structures associated with
the older version of the graph, else these data would be lost, which
would result in memory leakage.

To add a new vertex, one has to fill {\tt vert\lbt loc\lbt tab\lbt
[vertnbr\lbt -1]} and {\tt vend\lbt loc\lbt tab\lbt [vertnbr\lbt -1]}
with the starting and end indices of the adjacency sub-array of the
new vertex. Then, the adjacencies of its neighbor vertices must also
be updated to account for it. If free space had been reserved at the
end of each of the neighbors, one just has to increment the
${\tt vend\lbt loc\lbt tab}\mbox{\tt [}i\mbox{\tt ]}$ values of every
neighbor $i$, and add the index of the new vertex at the end of the
adjacency sub-array. If the sub-array cannot be extended, then it has
to be copied elsewhere in the edge array, and both ${\tt vert\lbt
loc\lbt tab}\mbox{\tt [}i\mbox{\tt ]}$ and ${\tt vend\lbt loc\lbt
tab}\mbox{\tt [}i\mbox{\tt ]}$ must be updated accordingly. With
simple housekeeping of free areas of the edge array, dynamic arrays
can be updated with as little data movement as possible.

\subsubsection{Block ordering format}
\label{sec-lib-format-order}

Block orderings associated with distributed graphs are described by
means of block and permutation arrays, made of {\tt SCOTCH\_Num}s.  In
order for all orderings to have the same structure, irrespective of
whether they are centralized or distributed, or whether they are
created from graphs or meshes, all ordering data indices start from
{\tt baseval}. Consequently, row indices are related to vertex
indices in memory in the following way: row $i$ is associated with
vertex $i$ of the {\tt SCOTCH\_\lbt Dgraph} structure as if the vertex
numbering used for the graph was continuous.

Block orderings are made of the following data:
\begin{itemize}
\iteme[{\tt permtab}]
Array holding the permutation of the reordered matrix. Thus, if $k =
\mbox{\tt permtab}\mbox{\tt [}i\mbox{\tt ]}$, then row $i$ of the
original matrix is now row $k$ of the reordered matrix, that is, row
$i$ is the $k^{\mbox{th}}$ pivot.
\iteme[{\tt peritab}]
Inverse permutation of the reordered matrix. Thus, if $i = \mbox{\tt
peritab[$k$]}$, then row $k$ of the reordered matrix was row $i$ of
the original matrix.
\iteme[{\tt cblknbr}]
Number of column blocks (that is, supervariables) in the block ordering.
\iteme[{\tt rangtab}]
Array of ranges for the column blocks. Column block $c$, with
$\mbox{\tt baseval} \leq c < (\mbox{\tt cblknbr} + \mbox{\tt
baseval})$, contains columns with indices ranging from {\tt
rangtab[$i$]} to {\tt rangtab[$i + 1$]}, exclusive, in the reordered
matrix. Therefore,
{\tt rangtab[baseval]} is always equal to {\tt baseval}, and
{\tt rangtab[cblknbr + baseval]} is always equal to
$\mbox{\tt vert\lbt glb\lbt nbr} + \mbox{\tt baseval}$.
% for graphs
% and to $\mbox{\tt vnod\lbt glb\lbt nbr} + \mbox{\tt baseval}$ for meshes.
In order to avoid memory errors when column blocks are all single
columns, the size of {\tt rangtab} must always be one more than the
number of columns, that is, $\mbox{\tt vert\lbt glb\lbt nbr} + 1$.
% for graphs and $\mbox{\tt vnod\lbt glb\lbt nbr} + 1$ for meshes.
\iteme[{\tt treetab}]
Array of ascendants of permuted column blocks in the separators tree.
{\tt treetab[i]} is the index of the father of column block $i$ in the
separators tree, or $-1$ if column block $i$ is the root of the
separators tree. Whenever separators or leaves of the separators tree
are split into subblocks, as the block splitting, minimum fill or minimum
degree methods do, all subblocks of the same level are linked to the
column block of higher index belonging to the closest separator
ancestor. Indices in {\tt treetab} are based, in the same way as for
the other blocking structures. See Figure~\ref{fig-lib-ord-block} for
a complete example.
\end{itemize}

\begin{figure}
\centering\includegraphics[scale=0.47]{p_f_orb.eps}
\caption{Arrays resulting from the ordering by complete nested
dissection of a 4 by 3 grid based from $1$. Leftmost grid is the
original grid, and righmost grid is the reordered grid, with
separators shown and column block indices written in bold.}
% using strategy  n{sep=hf{bal=0},ole=s,ose=s}
\label{fig-lib-ord-block}
\end{figure}

\subsection{Strategy strings}

The behavior of the static mapping and block ordering routines of the
\libscotch\ library is parametrized by means of strategy strings,
which describe how and when given partitioning or ordering methods
should be applied to graphs and subgraphs
% , or to meshes and submeshes.

\subsubsection{Using default strategy strings}
\label{sec-lib-format-strat-default}

While strategy strings can be built by hand, according to the syntax
given in the next sections, users who do not have specific needs can
take advantage of default strategies already implemented in the
\libscotch, which will yield very good results in most cases. By
doing so, they will spare themselves the hassle of updating their
strategies to comply to subsequent syntactic changes, and they will
benefit from the availability of new partitioning or ordering methods
as soon as they are made available.

The simplest way to use default strategy strings is to avoid
specifying any. By initializing a strategy object, by means of the
{\tt SCOTCH\_\lbt stratInit} routine, and by using the initialized
strategy object as is, without further parametrization, this object
will be filled with a default strategy when passing it as a parameter
to the next partitioning or ordering routine to be called. On return,
the strategy object will contain a fully specified strategy, tailored
for the type of operation which has been requested. Consequently, a
fresh strategy object that was used to partition a graph cannot be
used afterward as a default strategy for calling an ordering routine,
for instance, as partitioning and ordering strategies are incompatible.

The \libscotch\ also provides helper routines which allow users to
express their preferences on the kind of strategy that they
need. These helper routines, which are of the form
{\tt SCOTCH\_\lbt strat\lbt *\lbt Build}, tune default strategy strings
according to parameters provided by the user, such as the requested
number of parts (used as a hint to select the most efficient
partitioning routines), the desired maximum load imbalance ratio,
and a set of preference flags. While some of these flags are
antagonistic, most of them can be combined, by means of addition or
``binary or'' operators. These flags are the following.
They are grouped by application class.

\paragraph{Global flags}

\begin{itemize}
\iteme[{\tt SCOTCH\_STRATDEFAULT}]
Default behavior. No flags are set.
\iteme[{\tt SCOTCH\_STRATBALANCE}]
Enforce load balance as much as possible.
\iteme[{\tt SCOTCH\_STRATQUALITY}]
Privilege quality over speed.
\iteme[{\tt SCOTCH\_STRATSAFETY}]
Do not use methods that can lead to the occurrence of problematic
events, such as floating point exceptions, which could not be properly
handled by the calling software.
\iteme[{\tt SCOTCH\_STRATSPEED}]
Privilege speed over quality.
\end{itemize}

%% \paragraph{Mapping and partitioning flags}

%% \begin{itemize}
%% \iteme[{\tt SCOTCH\_STRATRECURSIVE}]
%% Use only recursive bipartitioning methods, and not direct k-way
%% methods. When this flag is not set, any combination of methods can be
%% used, so as to achieve the best result according to other user
%% preferences.
%% \iteme[{\tt SCOTCH\_STRATREMAP}]
%% Use the strategy for remapping an existing partition.
%% \end{itemize}

\paragraph{Ordering flags}

\begin{itemize}
\iteme[{\tt SCOTCH\_STRATLEVELMAX}]
Create at most the prescribed levels of nested dissection
separators. If the number of levels is less than the logarithm of the
number of processing elements used, distributed pieces of the
separated subgraph may have to be centralized so that the leaves can
be ordered, which may result in memory shortage.
\iteme[{\tt SCOTCH\_STRATLEVELMIN}]
Create at least the prescribed levels of nested dissection separators.
When used in conjunction with {\tt SCOTCH\_\lbt STRAT\lbt LEVEL\lbt
MAX}, the exact number of nested dissection levels will be performed,
unless the graph to order is too small.
\iteme[{\tt SCOTCH\_STRATLEAFSIMPLE}]
Order nested dissection leaves as cheaply as possible.
\iteme[{\tt SCOTCH\_STRATSEPASIMPLE}]
Order nested dissection separators as cheaply as possible.
\end{itemize}

\subsubsection{Parallel mapping strategy strings}
\label{sec-lib-format-pmap}

A parallel mapping strategy is made of one or several parallel
mapping methods, which can be combined by means of strategy
operators. The strategy operators that can be used in mapping
strategies are listed below, by increasing precedence.
\begin{itemize}
\iteme[{\tt (}{\it strat\/}{\tt )}]
Grouping operator.
The strategy enclosed within the parentheses is treated as a single
mapping method.
\iteme[{\tt /}{\it cond\/}{\tt ?}{\it strat1\/}{[{\tt :}{\it strat2}]{\tt ;}}]
Condition operator. According to the result of the evaluation of
condition {\it cond}, either {\it strat1\/} or {\it strat2\/} (if it
is present) is applied. The condition applies to the characteristics
of the current mapping task, and can be built from logical and
relational operators. Conditional operators are listed below, by
increasing precedence.
\begin{itemize}
\iteme[{\it cond1\/}{\tt |}{\it cond2}]
Logical or operator. The result of the condition is true if {\it cond1\/}
or {\it cond2\/} are true, or both.
\iteme[{\it cond1\/}{\tt \&}{\it cond2}]
Logical and operator. The result of the condition is true only if both
{\it cond1\/} and {\it cond2\/} are true.
\iteme[{\tt !}{\it cond}]
Logical not operator. The result of the condition is true only if
{\it cond\/} is false.
\iteme[{\it var} {\it relop} {\it val}]
Relational operator, where {\it var\/} is a node variable, {\it val\/} is
either a node variable or a constant of the type of variable {\it var}, and
{\it relop\/} is one of '{\tt\verb+<+}', '{\tt\verb+=+}', and '{\tt\verb+>+}'.
The node variables are listed below, along with their types.
\begin{itemize}
\iteme[{\tt edge}]
The global number of arcs of the current subgraph.
Integer.
\iteme[{\tt levl}]
The level of the subgraph in the recursion tree, starting from zero
for the initial graph at the root of the tree.
Integer.
\iteme[{\tt load}]
The overall sum of the vertex loads of the subgraph. It is equal to
{\tt vert} if the graph has no vertex loads.
Integer.
\iteme[{\tt mdeg}]
The maximum degree of the subgraph.
Integer.
\iteme[{\tt proc}]
The number of processes on which the current subgraph is distributed
at this level of the separators tree.
Integer.
\iteme[{\tt rank}]
The rank of the current process among the group of processes on
which the current subgraph is distributed at this level of the
separators tree.
Integer.
\iteme[{\tt vert}]
The global number of vertices of the current subgraph.
Integer.
\end{itemize}
\end{itemize}
\iteme[{\it method\/}{[{\tt \{}{\it parameters\/}{\tt \}}]}]
Parallel graph mapping method. Available parallel mapping methods
are listed below.
\end{itemize}
The currently available parallel mapping methods are the following.
\begin{itemize}
\iteme[{\tt r}]
Dual recursive bipartitioning method. The parameters of the dual recursive
bipartitioning method are given below.
\begin{itemize}
\iteme[{\tt seq=}{\it strat}]
Set the sequential mapping strategy that is used on every centralized
subgraph of the recursion tree, once the dual recursive bipartitioning
process has gone far enough such that the number of processes handling
some subgraph is restricted to one.
\iteme[{\tt sep=}{\it strat}]
Set the parallel graph bipartitioning strategy that is used on every
current job of the recursion tree. Parallel graph bipartitioning
strategies are described below, in section~\ref{sec-lib-format-pbipart}.
\end{itemize}
\end{itemize}

\subsubsection{Parallel graph bipartitioning strategy strings}
\label{sec-lib-format-pbipart}

A parallel graph bipartitioning strategy is made of one or several
parallel graph bipartitioning methods, which can be combined by means
of strategy operators. Strategy operators are listed below, by
increasing precedence.

\begin{itemize}
\iteme[{\it strat1\/}{\tt |}{\it strat2}]
Selection operator. The result of the selection is the best bipartition of
the two that are obtained by the distinct application of {\it strat1\/} and
{\it strat2\/} to the current bipartition.
\iteme[{\it strat1$\:$}{\it strat2}]
Combination operator. Strategy {\it strat2\/} is applied to the bipartition
resulting from the application of strategy {\it strat1\/} to the current
bipartition. Typically, the first method used should compute an initial
bipartition from scratch, and every following method should use the
result of the previous one at its starting point.
\iteme[{\tt (}{\it strat\/}{\tt )}]
Grouping operator.
The strategy enclosed within the parentheses is treated as a single
bipartitioning method.
\iteme[{\tt /}{\it cond\/}{\tt ?}{\it strat1\/}{[{\tt :}{\it strat2}]{\tt ;}}]
Condition operator. According to the result of the evaluation of condition
{\it cond}, either {\it strat1\/} or {\it strat2\/} (if it is present) is
applied. The condition applies to the characteristics of the current active
graph, and can be built from logical and relational operators. Conditional
operators are listed below, by increasing precedence.
\begin{itemize}
\iteme[{\it cond1\/}{\tt |}{\it cond2}]
Logical or operator. The result of the condition is true if {\it cond1\/}
or {\it cond2\/} are true, or both.
\iteme[{\it cond1\/}{\tt \&}{\it cond2}]
Logical and operator. The result of the condition is true only if both
{\it cond1\/} and {\it cond2\/} are true.
\iteme[{\tt !}{\it cond}]
Logical not operator. The result of the condition is true only if
{\it cond\/} is false.
\iteme[{\it var} {\it relop} {\it val}]
Relational operator, where {\it var\/} is a graph or node variable,
{\it val\/} is either a graph or node variable or a constant of the type of
variable {\it var\/}, and {\it relop\/} is one of
'{\tt\verb+<+}', '{\tt\verb+=+}', and '{\tt\verb+>+}'.
The graph and node variables are listed below, along with their types.
\begin{itemize}
\iteme[{\tt edge}]
The global number of edges of the current subgraph.
Integer.
\iteme[{\tt levl}]
The level of the subgraph in the bipartition or multi-level tree, starting from zero
for the initial graph at the root of the tree.
Integer.
\iteme[{\tt load}]
The overall sum of the vertex loads of the subgraph. It is equal to
{\tt vert} if the graph has no vertex loads.
Integer.
\iteme[{\tt load0}]
The vertex load of the first subset of the current bipartition of the current
graph.
Integer.
\iteme[{\tt proc}]
The number of processes on which the current subgraph is distributed
at this level of the nested dissection process.
Integer.
\iteme[{\tt rank}]
The rank of the current process among the group of processes on
which the current subgraph is distributed at this level of the
nested dissection process.
Integer.
\iteme[{\tt vert}]
The number of vertices of the current subgraph.
Integer.
\end{itemize}
\end{itemize}
The currently available parallel vertex separation methods are the
following.
\begin{itemize}
\iteme[{\tt b}]
Band method. Basing on the current distributed graph and on its
partition, this method creates a new distributed graph reduced to the
vertices which are at most at a given distance from the current
frontier, runs a parallel graph bipartitioning strategy on this graph,
and prolongs back the new bipartition to the current graph. This method
is primarily used to run bipartition refinement methods during the
uncoarsening phase of the multi-level parallel graph bipartitioning
method. The parameters of the band method are listed below.
\begin{itemize}
\iteme[{\tt bnd=}{\it strat}]
Set the parallel graph bipartitioning strategy to be applied to the band
graph.
\iteme[{\tt org=}{\it strat}]
Set the parallel graph bipartitioning strategy to be applied to the
full distributed graph if the band graph could not be extracted.
\iteme[{\tt width=}{\it val}]
Set the maximum distance from the current frontier of vertices to be
kept in the band graph. $0$ means that only frontier vertices
themselves are kept, $1$ that immediate neighboring vertices are kept
too, and so on.
\end{itemize}
\iteme[{\tt d}]
Parallel diffusion method. This method, presented in its sequential
formulation in~\cite{pell07b}, flows two kinds of antagonistic
liquids, scotch and anti-scotch, from two source vertices, and sets
the new frontier as the limit between vertices which contain scotch
and the ones which contain anti-scotch. Because selecting the source
vertices is essential to the obtainment of useful results, this method
has been hard-coded so that the two source vertices are the two
vertices of highest indices, since in the band method these are the
anchor vertices which represent all of the removed vertices of each
part. Therefore, this method must be used on band graphs only, or on
specifically crafted graphs. Applying it to any other graphs is very
likely to lead to extremely poor results.  The parameters of the
diffusion bipartitioning method are listed below.
\begin{itemize}
\iteme[{\tt dif=}{\it rat}]
Fraction of liquid which is diffused to neighbor vertices at each
pass. To achieve convergence, the sum of the {\tt dif} and {\tt rem}
parameters must be equal to $1$, but in order to speed-up the diffusion
process, other combinations of higher sum can be tried. In this case,
the number of passes must be kept low, to avoid numerical overflows
which would make the results useless.
\iteme[{\tt pass=}{\it nbr}]
Set the number of diffusion sweeps performed by the algorithm. This
number depends on the width of the band graph to which the diffusion
method is applied. Useful values range from $30$ to $500$ according
to chosen {\tt dif} and {\tt rem} coefficients.
\iteme[{\tt rem=}{\it rat}]
Fraction of liquid which remains on vertices at each pass. See above.
\end{itemize}
\iteme[{\tt m}]
Parallel multi-level method. The parameters of the multi-level method
are listed below.
\begin{itemize}
\iteme[{\tt asc=}{\it strat}]
Set the strategy that is used to refine the distributed bipartition
obtained at ascending levels of the uncoarsening phase by
prolongation of the bipartition computed for coarser graphs.
% or meshes.
This strategy is not applied to the coarsest graph,
% or mesh
for which only the {\tt low} strategy is used.
\iteme[{\tt dlevl=}{\it nbr}]
Set the minimum level after which duplication is allowed in the
folding process. A value of $-1$ results in duplication being always
performed when folding.
\iteme[{\tt dvert=}{\it nbr}]
Set the average number of vertices per process under which
the folding process is performed during the coarsening phase.
\iteme[{\tt low=}{\it strat}]
Set the strategy that is used to compute the bipartition of the
coarsest distributed graph,
% or mesh
at the lowest level of the coarsening process.
\iteme[{\tt rat=}{\it rat}]
Set the threshold maximum coarsening ratio over which graphs
% or meshes
are no longer coarsened. The ratio of any given coarsening cannot be
less that $0.5$ (case of a perfect matching), and cannot be greater
than $1.0$. Coarsening stops when either the coarsening ratio is
above the maximum coarsening ratio, or the graph
% or mesh
has fewer node vertices than the minimum number of vertices allowed.
\iteme[{\tt vert=}{\it nbr}]
Set the threshold minimum size under which graphs
% or meshes
are no longer coarsened. Coarsening stops when either the coarsening
ratio is above the maximum coarsening ratio, or the graph
% or mesh
has fewer node vertices than the minimum number of vertices allowed.
\end{itemize}
\iteme[{\tt q}]
Multi-sequential method. The current distributed graph and its
separator are centralized on every process that holds a part of it, and
a sequential graph bipartitioning method is applied independently to each
of them. Then, the best bipartition found is prolonged back to the
distributed graph. This method is primarily designed to operate on
band graphs, which are orders of magnitude smaller than their parent
graph. Else, memory bottlenecks are very likely to occur.
The parameters of the multi-sequential method are listed below.
\begin{itemize}
\iteme[{\tt strat=}{\it strat}]
Set the sequential edge separation strategy that is used to refine
the bipartition of the centralized graph. For a description of all of
the available sequential bipartitioning methods, please refer to the
{\it\scotch\ User's Guide}~\scotchcitesuser.
\end{itemize}
\iteme[{\tt x}]
Load balance enforcement method. This method moves as many vertices
from the heaviest part to the lightest one so as to reduce load
imbalance as much as possible, without impacting communication load
too negatively. The only parameter of this method is listed below.
\begin{itemize}
\iteme[{\tt sbbt=}{\it nbr}]
Number of sub-buckets to sort communication gains. $5$ is a common
value.
\end{itemize}
\iteme[{\tt z}]
Zero method. This method moves all of the vertices to the first
part, resulting in an empty frontier. Its main use is to stop the
bipartitioning process whenever some condition is true.
\end{itemize}
\end{itemize}

\subsubsection{Parallel ordering strategy strings}
\label{sec-lib-format-pord}

A parallel ordering strategy is made of one or several parallel
ordering methods, which can be combined by means of strategy
operators. The strategy operators that can be used in ordering
strategies are listed below, by increasing precedence.
\begin{itemize}
\iteme[{\tt (}{\it strat\/}{\tt )}]
Grouping operator.
The strategy enclosed within the parentheses is treated as a single
ordering method.
\iteme[{\tt /}{\it cond\/}{\tt ?}{\it strat1\/}{[{\tt :}{\it strat2}]{\tt ;}}]
Condition operator. According to the result of the evaluation of
condition {\it cond}, either {\it strat1\/} or {\it strat2\/} (if it
is present) is applied. The condition applies to the characteristics
of the current node of the separators tree, and can be built from
logical and relational operators. Conditional operators are listed
below, by increasing precedence.
\begin{itemize}
\iteme[{\it cond1\/}{\tt |}{\it cond2}]
Logical or operator. The result of the condition is true if {\it cond1\/}
or {\it cond2\/} are true, or both.
\iteme[{\it cond1\/}{\tt \&}{\it cond2}]
Logical and operator. The result of the condition is true only if both
{\it cond1\/} and {\it cond2\/} are true.
\iteme[{\tt !}{\it cond}]
Logical not operator. The result of the condition is true only if
{\it cond\/} is false.
\iteme[{\it var} {\it relop} {\it val}]
Relational operator, where {\it var\/} is a node variable, {\it val\/} is
either a node variable or a constant of the type of variable {\it var}, and
{\it relop\/} is one of '{\tt\verb+<+}', '{\tt\verb+=+}', and '{\tt\verb+>+}'.
The node variables are listed below, along with their types.
\begin{itemize}
\iteme[{\tt edge}]
The global number of arcs of the current subgraph.
Integer.
\iteme[{\tt levl}]
The level of the subgraph in the separators tree, starting from zero
for the initial graph at the root of the tree.
Integer.
\iteme[{\tt load}]
The overall sum of the vertex loads of the subgraph. It is equal to
{\tt vert} if the graph has no vertex loads.
Integer.
\iteme[{\tt mdeg}]
The maximum degree of the subgraph.
Integer.
\iteme[{\tt proc}]
The number of processes on which the current subgraph is distributed
at this level of the separators tree.
Integer.
\iteme[{\tt rank}]
The rank of the current process among the group of processes on
which the current subgraph is distributed at this level of the
separators tree.
Integer.
\iteme[{\tt vert}]
The global number of vertices of the current subgraph.
Integer.
\end{itemize}
\end{itemize}
\iteme[{\it method\/}{[{\tt \{}{\it parameters\/}{\tt \}}]}]
Parallel graph ordering method. Available parallel ordering methods
are listed below.
\end{itemize}
The currently available parallel ordering methods are the following.
\begin{itemize}
\iteme[{\tt n}]
Nested dissection method. The parameters of the nested dissection
method are given below.
\begin{itemize}
\iteme[{\tt ole=}{\it strat}]
Set the parallel ordering strategy that is used on every distributed
leaf of the parallel separators tree if the node separation strategy
{\tt sep} has failed to separate it further.
\iteme[{\tt ose=}{\it strat}]
Set the parallel ordering strategy that is used on every distributed
separator of the separators tree.
\iteme[{\tt osq=}{\it strat}]
Set the sequential ordering strategy that is used on every centralized
subgraph of the separators tree, once the nested dissection process has
gone far enough such that the number of processes handling some
subgraph is restricted to one.
\iteme[{\tt sep=}{\it strat}]
Set the parallel node separation strategy that is used on every
current leaf of the separators tree to make it grow. Parallel
node separation strategies are described below, in
section~\ref{sec-lib-format-pnsep}.
\end{itemize}
\iteme[{\tt q}]
Sequential ordering method. The distributed graph is gathered onto a
single process which runs a sequential ordering strategy. The only
parameter of the sequential method is given below.
\begin{itemize}
\iteme[{\tt strat=}{\it strat}]
Set the sequential ordering strategy that is applied to the
centralized graph. For a description of all of the available
sequential ordering methods, please refer to the
{\it\scotch\ User's Guide}~\scotchcitesuser.
\end{itemize}
\iteme[{\tt s}]
Simple method. Vertices are ordered in their natural order. This
method is fast, and should be used to order separators if the number
of extra-diagonal blocks is not relevant
\end{itemize}

\subsubsection{Parallel node separation strategy strings}
\label{sec-lib-format-pnsep}

A parallel node separation strategy is made of one or several parallel
node separation methods, which can be combined by means of strategy
operators. Strategy operators are listed below, by increasing
precedence.

\begin{itemize}
\iteme[{\it strat1\/}{\tt |}{\it strat2}]
Selection operator. The result of the selection is the best vertex separator of
the two that are obtained by the distinct application of {\it strat1\/} and
{\it strat2\/} to the current separator.
\iteme[{\it strat1$\:$}{\it strat2}]
Combination operator. Strategy {\it strat2\/} is applied to the vertex
separator resulting from the application of strategy {\it strat1\/} to the
current separator. Typically, the first method used should compute an initial
separation from scratch, and every following method should use the
result of the previous one as a starting point.
\iteme[{\tt (}{\it strat\/}{\tt )}]
Grouping operator.
The strategy enclosed within the parentheses is treated as a single
separation method.
\iteme[{\tt /}{\it cond\/}{\tt ?}{\it strat1\/}{[{\tt :}{\it strat2}]{\tt ;}}]
Condition operator. According to the result of the evaluation of condition
{\it cond}, either {\it strat1\/} or {\it strat2\/} (if it is present) is
applied. The condition applies to the characteristics of the current subgraph,
and can be built from logical and relational operators. Conditional
operators are listed below, by increasing precedence.
\begin{itemize}
\iteme[{\it cond1\/}{\tt |}{\it cond2}]
Logical or operator. The result of the condition is true if {\it cond1\/}
or {\it cond2\/} are true, or both.
\iteme[{\it cond1\/}{\tt \&}{\it cond2}]
Logical and operator. The result of the condition is true only if both
{\it cond1\/} and {\it cond2\/} are true.
\iteme[{\tt !}{\it cond}]
Logical not operator. The result of the condition is true only if
{\it cond\/} is false.
\iteme[{\it var} {\it relop} {\it val}]
Relational operator, where {\it var\/} is a graph or node variable,
{\it val\/} is either a graph or node variable or a constant of the type of
variable {\it var\/}, and {\it relop\/} is one of
'{\tt\verb+<+}', '{\tt\verb+=+}', and '{\tt\verb+>+}'.
The graph and node variables are listed below, along with their types.
\begin{itemize}
\iteme[{\tt edge}]
The global number of edges of the current subgraph.
Integer.
\iteme[{\tt levl}]
The level of the subgraph in the separators tree, starting from zero
for the initial graph at the root of the tree.
Integer.
\iteme[{\tt load}]
The overall sum of the vertex loads of the subgraph. It is equal to
{\tt vert} if the graph has no vertex loads.
Integer.
\iteme[{\tt proc}]
The number of processes on which the current subgraph is distributed
at this level of the nested dissection process.
Integer.
\iteme[{\tt rank}]
The rank of the current process among the group of processes on
which the current subgraph is distributed at this level of the
nested dissection process.
Integer.
\iteme[{\tt vert}]
The number of vertices of the current subgraph.
Integer.
\end{itemize}
\end{itemize}
The currently available parallel vertex separation methods are the
following.
\begin{itemize}
\iteme[{\tt b}]
Band method. Basing on the current distributed graph and on its
partition, this method creates a new distributed graph reduced to the
vertices which are at most at a given distance from the current
separator, runs a parallel vertex separation strategy on this graph,
and prolongs back the new separator to the current graph. This method
is primarily used to run separator refinement methods during the
uncoarsening phase of the multi-level parallel graph separation
method. The parameters of the band method are listed below.
\begin{itemize}
\iteme[{\tt strat=}{\it strat}]
Set the parallel vertex separation strategy to be applied to the band
graph.
\iteme[{\tt width=}{\it val}]
Set the maximum distance from the current separator of vertices to be
kept in the band graph. $0$ means that only separator vertices
themselves are kept, $1$ that immediate neighboring vertices are kept
too, and so on.
\end{itemize}
\iteme[{\tt m}]
Parallel vertex multi-level method. The parameters of the vertex
multi-level method are listed below.
\begin{itemize}
\iteme[{\tt asc=}{\it strat}]
Set the strategy that is used to refine the distributed vertex
separators obtained at ascending levels of the uncoarsening phase by
prolongation of the separators computed for coarser graphs.
% or meshes.
This strategy is not applied to the coarsest graph,
% or mesh
for which only the {\tt low} strategy is used.
\iteme[{\tt dlevl=}{\it nbr}]
Set the minimum level after which duplication is allowed in the
folding process. A value of $-1$ results in duplication being always
performed when folding.
\iteme[{\tt dvert=}{\it nbr}]
Set the average number of vertices per process under which
the folding process is performed during the coarsening phase.
\iteme[{\tt low=}{\it strat}]
Set the strategy that is used to compute the vertex separator of the
coarsest distributed graph,
% or mesh
at the lowest level of the coarsening process.
\iteme[{\tt rat=}{\it rat}]
Set the threshold maximum coarsening ratio over which graphs
% or meshes
are no longer coarsened. The ratio of any given coarsening cannot be
less that $0.5$ (case of a perfect matching), and cannot be greater
than $1.0$. Coarsening stops when either the coarsening ratio is
above the maximum coarsening ratio, or the graph
% or mesh
has fewer node vertices than the minimum number of vertices allowed.
\iteme[{\tt vert=}{\it nbr}]
Set the threshold minimum size under which graphs
% or meshes
are no longer coarsened. Coarsening stops when either the coarsening
ratio is above the maximum coarsening ratio, or the graph
% or mesh
has fewer node vertices than the minimum number of vertices allowed.
\end{itemize}
\iteme[{\tt q}]
Multi-sequential method. The current distributed graph and its
separator are centralized on every process that holds a part of it, and
a sequential vertex separation method is applied independently to each
of them. Then, the best separator found is prolonged back to the
distributed graph. This method is primarily designed to operate on
band graphs, which are orders of magnitude smaller than their parent
graph. Else, memory bottlenecks are very likely to occur.
The parameters of the multi-sequential method are listed below.
\begin{itemize}
\iteme[{\tt strat=}{\it strat}]
Set the sequential vertex separation strategy that is used to refine
the separator of the centralized graph. For a description of all of
the available sequential methods, please refer to the
{\it\scotch\ User's Guide}~\scotchcitesuser.
\end{itemize}
\iteme[{\tt z}]
Zero method. This method moves all of the node vertices to the first
part, resulting in an empty separator. Its main use is to stop the
separation process whenever some condition is true.
\end{itemize}
\end{itemize}

\subsection{Distributed graph handling routines}
\label{sec-lib-dgraph}

\subsubsection{{\tt SCOTCH\_dgraphAlloc}}

\begin{itemize}
\progsyn

{\tt\begin{tabular}{l@{}l}
SCOTCH\_Dgraph * SCOTCH\_dgraphAlloc ( & void)
\end{tabular}}

\progdes

The {\tt SCOTCH\_dgraphAlloc} function allocates a memory area of a
size sufficient to store a {\tt SCOTCH\_\lbt Dgraph} structure. It is
the user's responsibility to free this memory when it is no longer
needed. The allocated space must be initialized before use, by means
of the {\tt SCOTCH\_\lbt dgraph\lbt Init} routine.

\progret

{\tt SCOTCH\_dgraphAlloc} returns the pointer to the memory area if it
has been successfully allocated, and {\tt NULL} else.
\end{itemize}

\subsubsection{{\tt SCOTCH\_dgraphBand}}

\begin{itemize}
\progsyn

{\tt\begin{tabular}{l@{}ll}
int SCOTCH\_dgraphBand ( & SCOTCH\_Dgraph * const    & orggrafptr, \\
                         & const SCOTCH\_Num         & fronlocnbr, \\
                         & const SCOTCH\_Num * const & fronloctab, \\
                         & const SCOTCH\_Num         & distval,    \\
                         & SCOTCH\_Dgraph * const    & bndgrafptr) \\
\end{tabular}}

{\tt\begin{tabular}{l@{}ll}
scotchfdgraphband ( & doubleprecision (*)   & orggrafdat, \\
                    & integer*{\it num}     & seedlocnbr, \\
                    & integer*{\it num} (*) & seedloctab, \\
                    & integer*{\it num}     & distval,    \\
                    & doubleprecision (*)   & bndgrafdat, \\
                    & integer               & ierr)
\end{tabular}}

\progdes

The {\tt SCOTCH\_dgraphBand} routine creates in the {\tt SCOTCH\_\lbt
Dgraph} structure pointed to by {\tt bndgrafptr} a distributed band
graph induced from the {\tt SCOTCH\_\lbt Dgraph} pointed to by {\tt
orggrafptr}. The distributed band graph will contain all the vertices
of the original graph located at a distance smaller than or equal to
{\tt distval} from any vertex provided in the {\tt seedloctab} lists
of seed vertices.

On each process, the {\tt seedloctab} array should contain the local
indices of the local vertices that will serve as seeds. The number of
such local vertices is passed to {\tt SCOTCH\_\lbt dgraph\lbt Band} in
the {\tt seedlocnbr} value. The size of the {\tt seedloctab} array
should be at least equal to the number of local vertices of the
original graph, as it is internally used as a queue array. Hence, no
user data should be placed immediately after the {\tt seedlocnbr}
values in the array, as they are most likely to be overwritten by
{\tt SCOTCH\_\lbt dgraph\lbt Band}.

{\tt bndgrafptr} must have been initialized with the
{\tt SCOTCH\_\lbt dgraph\lbt Init} routine before
{\tt SCOTCH\_dgraph\lbt Band} is called. The communicator that is
passed to it can either be the communicator used by the original graph
{\tt org\lbt graf\lbt ptr}, or any congruent communicator created by
using {\tt MPI\_\lbt Comm\_\lbt dup} on this communicator. Using a
distinct communicator for the induced band graph allows subsequent
library routines to be called in parallel on the two graphs after the
band graph is created.

Induced band graphs have vertex labels attached to each of their
vertices, in the {\tt vlbl\lbt loc\lbt tab} array. If the original
graph had vertex labels attached to it, band graph vertex labels are
the labels of the corresponding vertices in the original graph. Else,
band graph vertex labels are the global indices of corresponding
vertices in the original graph.

Depending on original graph vertex and seed distributions, the
distribution of induced band graph vertices may be highly
imbalanced. In order for further computations on this distributed
graph to scale well, a redistribution of its data may be necessary,
using the {\tt SCOTCH\_dgraph\lbt Redist} routine.

\progret

{\tt SCOTCH\_dgraphBand} returns $0$ if the band graph structure has
been successfully created, and $1$ else.
\end{itemize}

\subsubsection{{\tt SCOTCH\_dgraphBuild}}

\begin{itemize}
\progsyn

{\tt\begin{tabular}{l@{}ll}
int SCOTCH\_dgraphBuild ( & SCOTCH\_Dgraph *    & grafptr,    \\
                          & const SCOTCH\_Num   & baseval,    \\
                          & const SCOTCH\_Num   & vertlocnbr, \\
                          & const SCOTCH\_Num   & vertlocmax, \\
                          & const SCOTCH\_Num * & vertloctab, \\
                          & const SCOTCH\_Num * & vendloctab, \\
                          & const SCOTCH\_Num * & veloloctab, \\
                          & const SCOTCH\_Num * & vlblocltab, \\
                          & const SCOTCH\_Num   & edgelocnbr, \\
                          & const SCOTCH\_Num   & edgelocsiz, \\
                          & const SCOTCH\_Num * & edgeloctab, \\
                          & const SCOTCH\_Num * & edgegsttab, \\
                          & const SCOTCH\_Num * & edloloctab)
\end{tabular}}

{\tt\begin{tabular}{l@{}ll}
scotchfdgraphbuild ( & doubleprecision (*)   & grafdat,    \\
                     & integer*{\it num}     & baseval,    \\
                     & integer*{\it num}     & vertlocnbr, \\
                     & integer*{\it num}     & vertlocmax, \\
                     & integer*{\it num} (*) & vertloctab, \\
                     & integer*{\it num} (*) & vendloctab, \\
                     & integer*{\it num} (*) & veloloctab, \\
                     & integer*{\it num} (*) & vlblloctab, \\
                     & integer*{\it num}     & edgelocnbr, \\
                     & integer*{\it num}     & edgelocsiz, \\
                     & integer*{\it num} (*) & edgeloctab, \\
                     & integer*{\it num} (*) & edgegsttab, \\
                     & integer*{\it num} (*) & edloloctab, \\
                     & integer               & ierr)
\end{tabular}}

\progdes

The {\tt SCOTCH\_dgraphBuild} routine fills the distributed source
graph structure pointed to by {\tt grafptr} with all of the data that
are passed to it.

{\tt baseval} is the graph base value for index arrays (typically $0$ for
structures built from C and $1$ for structures built from Fortran).
{\tt vertlocnbr} is the number of local vertices on the calling
process, used to create the {\tt proccnttab} array.
{\tt vertlocmax} is the maximum number of local vertices to be created
on the calling process, used to create the {\tt proc\lbt vrt\lbt tab}
array of global indices, and which must be set to {\tt vert\lbt
loc\lbt nbr} for graphs wihout holes in their global numbering.
{\tt vertloctab} is the local adjacency index array, of size $({\tt
vertlocnbr} + 1)$ if the edge array is compact (that is, if {\tt
vendloctab} equals $\mbox{\tt vertloctab}+1$ or {\tt NULL}), or of
size {\tt vertlocnbr} else.
{\tt vendloctab} is the adjacency end index array, of size {\tt
vertlocnbr} if it is disjoint from {\tt vertloctab}.
{\tt veloloctab} is the local vertex load array, of size
{\tt vertlocnbr} if it exists.
{\tt vlblloctab} is the local vertex label array, of size
{\tt vertlocnbr} if it exists.
{\tt edgelocnbr} is the local number of arcs (that is, twice the
number of edges), including arcs to local vertices as well as to
ghost vertices.
{\tt edgelocsiz} is lower-bounded by the minimum size of the edge
array required to encompass all used adjacency values; it is therefore
at least equal to the maximum of the {\tt vendloctab} entries, over
all local vertices, minus {\tt baseval}; it can be set to {\tt
edgelocnbr} if the edge array is compact.
{\tt edgeloctab} is the local adjacency array, of size at least
{\tt edgelocsiz}, which stores the global indices of end vertices.
{\tt edgegsttab} is the adjacency array, of size at least
{\tt edgelocsiz}, if it exists; if {\tt edgegsttab} is given, it is
assumed to be a pointer to an empty array to be filled with ghost
vertex data computed by {\tt SCOTCH\_dgraph\lbt Ghst} whenever
needed by communication routines such as
{\tt SCOTCH\_dgraph\lbt Halo}.
{\tt edloloctab} is the arc load array, of size {\tt edgelocsiz}
if it exists.

The {\tt vendloctab}, {\tt veloloctab}, {\tt vlblloctab},
{\tt edloloctab} and {\tt edgegsttab} arrays are optional,
and a null pointer can be passed as argument whenever
they are not defined.

Note that, for \ptscotch\ to operate properly, either all the arrays
of a kind must be set to null on all processes, or else all of them
must be non null. This is mandatory because some algorithms require that
collective communication operations be performed when some kind of
data is present. If some processes considered that the arrays are
present, and start such communications, while others did not, a
deadlock would occur. In most cases, this situation will be
anticipated and an error message will be issued, stating that graph
data are inconsistent.

The situation above may accidentally arise when some processes don't
own any edge or vertex. In that case, depending on the implementation,
a user \texttt{malloc} of size zero may return a null pointer rather
than a non null pointer to an area of size zero, leading to the
aforementioned inconsistencies. In order to avoid this problem, it is
necessary to ensure that no null pointer will be returned, even in the
case when zero bytes are requested. A workaround can be to call
\texttt{malloc (\textit{x} | 4)} instead of \texttt{malloc
(\textit{x})}. The ``\texttt{| 4}'' will consume only $4$ extra bytes
at most, depending on the value of \texttt{\textit{x}}.

Since, in Fortran, there is no null reference, passing the
{\tt scotchf\lbt dgraph\lbt build} routine a reference equal to
{\tt vertloctab} in the {\tt veloloctab} or {\tt vlblloctab} fields
makes them be considered as missing arrays. The same holds for
{\tt edloloctab} and {\tt edgegsttab} when they are passed a
reference equal to {\tt edgeloctab}. Setting {\tt vendloctab}
to refer to one cell after {\tt vertloctab} yields the same result,
as it is the exact semantics of a compact vertex array.

To limit memory consumption, {\tt SCOTCH\_\lbt dgraph\lbo Build} does
not copy array data, but instead references them in the {\tt
SCOTCH\_\lbt Dgraph} structure. Therefore, great care should be taken
not to modify the contents of the arrays passed to {\tt SCOTCH\_\lbt
dgraph\lbo Build} as long as the graph structure is in use. Every
update of the arrays should be preceded by a call to {\tt SCOTCH\_\lbt
dgraph\lbo Free}, to free internal graph structures, and eventually
followed by a new call to {\tt SCOTCH\_\lbt dgraph\lbo Build} to
re-build these internal structures so as to be able to use the new
distributed graph.

To ensure that inconsistencies in user data do not result in an
erroneous behavior of the \libscotch\ routines, it is recommended, at
least in the development stage of your application code, to call the
{\tt SCOTCH\_\lbt dgraph\lbt Check} routine on the newly created
{\tt SCOTCH\_\lbt Dgraph} structure before calling any other
\libscotch\ routine.

\progret

{\tt SCOTCH\_dgraphBuild} returns $0$ if the graph structure has been
successfully set with all of the input data, and $1$ else.
\end{itemize}

\subsubsection{{\tt SCOTCH\_dgraphCheck}}

\begin{itemize}
\progsyn

{\tt\begin{tabular}{l@{}ll}
int SCOTCH\_dgraphCheck ( & const SCOTCH\_Dgraph * & grafptr)
\end{tabular}}

{\tt\begin{tabular}{l@{}ll}
scotchfdgraphcheck ( & doubleprecision (*) & grafdat, \\
                     & integer             & ierr)
\end{tabular}}

\progdes

The {\tt SCOTCH\_dgraphCheck} routine checks the consistency of the
given {\tt SCOTCH\_\lbt Dgraph} structure. It can be used in client
applications to determine if a graph which has been created from
user-generated data by means of the {\tt SCOTCH\_\lbt dgraph\lbt Build}
routine is consistent, prior to calling any other routines of the
\libscotch\ library which would otherwise return internal error
messages or crash the program.

\progret

{\tt SCOTCH\_dgraphCheck} returns $0$ if graph data are consistent, and
$1$ else.

\end{itemize}

\subsubsection{{\tt SCOTCH\_dgraphCoarsen}}

\begin{itemize}
\progsyn

{\tt\begin{tabular}{l@{}ll}
int SCOTCH\_dgraphCoarsen ( & SCOTCH\_Dgraph * const & finegrafptr, \\
                            & const SCOTCH\_Num      & coarnbr,     \\
                            & const double           & coarrat,     \\
                            & const SCOTCH\_Num      & flagval,     \\
                            & SCOTCH\_Dgraph * const & coargrafptr, \\
                            & SCOTCH\_Num * const    & multloctab)  \\
\end{tabular}}

{\tt\begin{tabular}{l@{}ll}
scotchfdgraphcoarsen ( & doubleprecision (*)   & finegrafdat, \\
                       & integer*{\it num}     & coarnbr,     \\
                       & doubleprecision       & coarrat,     \\
                       & integer*{\it num}     & flagval,     \\
                       & doubleprecision (*)   & coargrafdat, \\
                       & integer*{\it num} (*) & multloctab,  \\
                       & integer               & ierr)
\end{tabular}}

\progdes

The {\tt SCOTCH\_dgraphCoarsen} routine creates in the
{\tt SCOTCH\_\lbt Dgraph} structure pointed to by {\tt coargrafptr}
a distributed coarsened graph from the {\tt SCOTCH\_\lbt Dgraph}
pointed to by {\tt finegrafptr}. The coarsened graph is created only
if it is comprises more than {\tt coarnbr} vertices, or if the
coarsening ratio is lower than {\tt coarrat}. Valid coarsening ratio
values range from $0.5$ (in the case of a perfect matching) to $1.0$
(if no vertex could be coarsened). Classical threshold values range
from $0.7$ to $0.8$.

The {\tt flagval} flag specifies the type of coarsening. Several
groups of flags can be combined, by means of addition or
``binary or'' operators. When {\tt SCOTCH\_\lbt COARSEN\lbt
NO\lbt MERGE} is set, isolated vertices are never merged with other
vertices. This preserves the topology of the graph, at the expense of
a higher coarsening ratio. When {\tt SCOTCH\_\lbt COARSEN\lbt
FOLD} or {\tt SCOTCH\_\lbt COARSEN\lbt FOLD\lbt DUP} are set, if a
coarsened graph is created, it is folded onto half of the processes of
the initial communicator. In the case of {\tt SCOTCH\_\lbt COARSEN\lbt
FOLD\lbt DUP}, a second copy is created (duplicated) onto the other
half. The two copies may not be identical, if the number of processors
of the finer graph is odd.

The {\tt multloctab} array must be of a size that is big enough to
store multinode data for the resulting coarsened graph. This array
will contain pairs of consecutive {\tt SCOTCH\_\lbt Num} values,
representing the global indices of the two fine vertices that have
been coarsened into each of the local coarse vertices. In case of
plain coarsening, the size of the array must be at least twice the
maximum expected number of local coarse vertices, that is, on each
processor, twice the value of {\tt vert\lbt loc\lbt nbr} of the finer
graph, because in the worst case no coarsening may happen on some
processor. In case of folding, a redistribution of vertices is
performed. Hence, the maximum number of coarse vertices on some
processor is upper-bounded by the expected maximum global number of
coarse vertices, divided by the resulting number of processors, that
is, the integer floor value of half of the number of processors of the
finer graph.

{\tt coargrafptr} must have been initialized with the
{\tt SCOTCH\_\lbt dgraph\lbt Init} routine before
{\tt SCOTCH\_dgraph\lbt Coarsen} is called. The communicator that is
passed to it can either be the communicator used by the fine graph
{\tt fine\lbt graf\lbt ptr}, or any congruent communicator created by
using {\tt MPI\_\lbt Comm\_\lbt dup} on this communicator. Using a
distinct communicator for the coarsened subgraph allows subsequent
library routines to be called in parallel on the two graphs after the
coarse graph is created.

Depending on the way vertex mating is performed, the distribution of
coarsened graph vertices may be imbalanced. In order for further
computations on this distributed graph to scale well, a redistribution
of its data might be necessary, using the
{\tt SCOTCH\_dgraph\lbt Redist} routine.

\progret

{\tt SCOTCH\_dgraphCoarsen} returns $0$ if the coarse graph
structure has been successfully created, $1$ if the coarse graph was
not created because it did not enforce the threshold parameters, and
$2$ on error.
\end{itemize}

\subsubsection{{\tt SCOTCH\_dgraphData}}
\label{sec-lib-func-scotchdgraphdata}

\begin{itemize}
\progsyn

{\tt\begin{tabular}{l@{}ll}
void SCOTCH\_dgraphData ( & const SCOTCH\_Graph * & grafptr,    \\
                          & SCOTCH\_Num *         & baseptr,    \\
                          & SCOTCH\_Num *         & vertglbptr, \\
                          & SCOTCH\_Num *         & vertlocptr, \\
                          & SCOTCH\_Num *         & vertlocptz, \\
                          & SCOTCH\_Num *         & vertgstptr, \\
                          & SCOTCH\_Num **        & vertloctab, \\
                          & SCOTCH\_Num **        & vendloctab, \\
                          & SCOTCH\_Num **        & veloloctab, \\
                          & SCOTCH\_Num **        & vlblloctab, \\
                          & SCOTCH\_Num *         & edgeglbptr, \\
                          & SCOTCH\_Num *         & edgelocptr, \\
                          & SCOTCH\_Num *         & edgelocptz, \\
                          & SCOTCH\_Num **        & edgeloctab, \\
                          & SCOTCH\_Num **        & edgegsttab, \\
                          & SCOTCH\_Num **        & edloloctab, \\
                          & MPI\_Comm *           & comm)
\end{tabular}}

{\tt\begin{tabular}{l@{}ll}
scotchfdgraphdata ( & doubleprecision (*)   & grafdat,    \\
                    & integer*{\it num} (*) & indxtab,    \\
                    & integer*{\it num}     & baseval,    \\
                    & integer*{\it num}     & vertglbnbr, \\
                    & integer*{\it num}     & vertlocnbr, \\
                    & integer*{\it num}     & vertlocmax, \\
                    & integer*{\it num}     & vertgstnbr, \\
                    & integer*{\it idx}     & vertlocidx, \\
                    & integer*{\it idx}     & vendlocidx, \\
                    & integer*{\it idx}     & velolocidx, \\
                    & integer*{\it idx}     & vlbllocidx, \\
                    & integer*{\it num}     & edgeglbnbr, \\
                    & integer*{\it num}     & edgelocnbr, \\
                    & integer*{\it num}     & edgelocsiz, \\
                    & integer*{\it idx}     & edgelocidx, \\
                    & integer*{\it idx}     & edgegstidx, \\
                    & integer*{\it idx}     & edlolocidx, \\
                    & integer               & comm)
\end{tabular}}

\progdes

The {\tt SCOTCH\_dgraphData} routine is the dual of the
{\tt SCOTCH\_\lbt dgraph\lbo Build} routine. It is a multiple
accessor that returns scalar values and array references.

{\tt baseptr} is the pointer to a location that will hold the graph base
value for index arrays (typically $0$ for
structures built from C and $1$ for structures built from Fortran).
{\tt vertglbptr} is the pointer to a location that will hold the
global number of vertices.
{\tt vertlocptr} is the pointer to a location that will hold the
number of local vertices.
{\tt vertlocptz} is the pointer to a location that will hold the
maximum allowed number of local vertices, that is,
$(\mbox{\tt proc\lbt vrt\lbt tab[}p + 1\mbox{\tt]} -
\mbox{\tt proc\lbt vrt\lbt tab[}p\mbox{\tt]})$, where $p$ is the
rank of the local process.
{\tt vertgstptr} is the pointer to a location that will hold the
number of local and ghost vertices if it has already been computed
by a prior call to {\tt SCOTCH\_\lbt dgraph\lbo Ghst}, and $-1$ else.
{\tt vertloctab} is the pointer to a location that will hold the
reference to the adjacency index array, of size
$\mbox{\tt *vertlocptr} + 1$ if the adjacency array is compact,
or of size {\tt *vertlocptr} else.
{\tt vendloctab} is the pointer to a location that will hold the
reference to the adjacency end index array, and is equal to
$\mbox{\tt vertloctab} + 1$ if the adjacency array is compact.
{\tt veloloctab} is the pointer to a location that will hold the
reference to the vertex load array, of size {\tt *vertlocptr}.
{\tt vlblloctab} is the pointer to a location that will hold the
reference to the vertex label array, of size {\tt vertlocnbr}.
{\tt edgeglbptr} is the pointer to a location that will hold the
global number of arcs (that is, twice the number of global edges).
{\tt edgelocptr} is the pointer to a location that will hold the
number of local arcs (that is, twice the number of local edges).
{\tt edgelocptz} is the pointer to a location that will hold the
declared size of the local edge array, which must encompass all
used adjacency values; it is at least equal to {\tt *edgelocptr}.
{\tt edgeloctab} is the pointer to a location that will hold the
reference to the local adjacency array of global indices, of size
at least {\tt *edgelocptz}.
{\tt edgegsttab} is the pointer to a location that will hold the
reference to the ghost adjacency array, of size at least
{\tt *edgelocptz}; if it is non null, its data are valid if
{\tt vertgstnbr} is non-negative.
{\tt edloloctab} is the pointer to a location that will hold the
reference to the arc load array, of size {\tt *edgelocptz}.
{\tt comm} is the pointer to a location that will hold the MPI
communicator of the distributed graph.

Any of these pointers can be set to {\tt NULL} on input if the
corresponding information is not needed. Else, the reference to a
dummy area can be provided, where all unwanted data will be written.

Since there are no pointers in Fortran, a specific mechanism is used
to allow users to access graph arrays. The {\tt scotchf\lbt dgraph\lbt
data} routine is passed an integer array, the first element of which
is used as a base address from which all other array indices are
computed. Therefore, instead of returning references, the routine
returns integers, which represent the starting index of each of the
relevant arrays with respect to the base input array, or {\tt vert\lbt
loc\lbt idx}, the index of {\tt vert\lbt loc\lbt tab}, if they do not
exist. For instance, if some base array {\tt myarray\lbt (1)} is
passed as parameter {\tt indxtab}, then the first cell of array {\tt
vert\lbt loc\lbt tab} will be accessible as {\tt myarray\lbt
(vert\lbt loc\lbt idx)}.
In order for this feature to behave properly, the {\tt indxtab}
array must be word-aligned with the graph arrays. This is
automatically enforced on most systems, but some care should be
taken on systems that allow to access data that is not
word-aligned. On such systems, declaring the array after a
dummy {\tt double\lbt precision} array can coerce the compiler
into enforcing the proper alignment. The integer value returned in
{\tt comm} is the communicator itself, not its index with respect to
{\tt indxtab}. Also, on 32\_64 architectures,
such indices can be larger than the size of a regular
{\tt INTEGER}. This is why the indices to be returned are defined by
means of a specific integer type. See
Section~\ref{sec-lib-inttypesize} for more information on this
issue.
\end{itemize}

\subsubsection{{\tt SCOTCH\_dgraphExit}}

\begin{itemize}
\progsyn

{\tt\begin{tabular}{l@{}ll}
void SCOTCH\_dgraphExit ( & SCOTCH\_Dgraph * & grafptr)
\end{tabular}}

{\tt\begin{tabular}{l@{}ll}
scotchfdgraphexit ( & doubleprecision (*) & grafdat)
\end{tabular}}

\progdes

The {\tt SCOTCH\_dgraphExit} function frees the contents of a
{\tt SCOTCH\_\lbt Dgraph} structure previously initialized by
{\tt SCOTCH\_\lbt dgraphInit}. All subsequent calls to
{\tt SCOTCH\_\lbt dgraph} routines other than {\tt SCOTCH\_\lbt
dgraphInit}, using this structure as parameter, may yield
unpredictable results.

If {\tt SCOTCH\_\lbt dgraph\lbt Init} was called with a
communicator that is not a predefined MPI communicator, it is
the user's responsibility to free this communicator after
all graphs that use it have been freed by means of the
{\tt SCOTCH\_\lbt dgraph\lbt Exit} routine.

\end{itemize}

\subsubsection{{\tt SCOTCH\_dgraphFree}}

\begin{itemize}
\progsyn

{\tt\begin{tabular}{l@{}ll}
void SCOTCH\_dgraphFree ( & SCOTCH\_Dgraph * & grafptr)
\end{tabular}}

{\tt\begin{tabular}{l@{}ll}
scotchfdgraphfree ( & doubleprecision (*) & grafdat)
\end{tabular}}

\progdes

The {\tt SCOTCH\_dgraphFree} function frees the graph data of a
{\tt SCOTCH\_\lbt Dgraph} structure previously initialized by
{\tt SCOTCH\_\lbt dgraph\lbt Init}, but preserves its internal
communication data structures. This call is equivalent to
a call to {\tt SCOTCH\_\lbt dgraph\lbt Exit} immediately
followed by a call to {\tt SCOTCH\_\lbt dgraph\lbt Init} with the
same communicator as in the previous {\tt SCOTCH\_\lbt dgraph\lbt
Init} call. Consequently, the given {\tt SCOTCH\_\lbt Dgraph}
structure remains ready for subsequent calls to any distributed
graph handling routine of the \libscotch\ library.
\end{itemize}

\subsubsection{{\tt SCOTCH\_dgraphGather}}

\begin{itemize}
\progsyn

{\tt\begin{tabular}{l@{}ll}
int SCOTCH\_dgraphGather ( & SCOTCH\_Dgraph * const      & dgrfptr, \\
                           & const SCOTCH\_Graph * const & cgrfptr)
\end{tabular}}

{\tt\begin{tabular}{l@{}ll}
scotchfdgraphgather ( & doubleprecision (*) & dgrfdat, \\
                      & doubleprecision (*) & cgrfdat, \\
                      & integer             & ierr)
\end{tabular}}

\progdes

The {\tt SCOTCH\_dgraphGather} routine gathers the contents of the
distributed {\tt SCOTCH\_\lbt Dgraph} structure pointed to by {\tt
dgrfptr} to the centralized {\tt SCOTCH\_\lbt Graph} structure(s)
pointed to by {\tt cgrfptr}.

If only one of the processes has a non-null {\tt cgrfptr}
pointer, it is considered as the root process to which distributed
graph data is sent. Else, all of the processes must provide a valid
{\tt cgrfptr} pointer, and each of them will receive a copy of
the centralized graph.

\progret

{\tt SCOTCH\_dgraphGather} returns $0$ if the graph structure has
been successfully gathered, and $1$ else.
\end{itemize}

\subsubsection{{\tt SCOTCH\_dgraphInducePart}}

\begin{itemize}
\progsyn

{\tt\begin{tabular}{l@{}ll}
int SCOTCH\_dgraphInducePart ( & SCOTCH\_Dgraph * const    & orggrafptr,    \\
                               & const SCOTCH\_Num * const & orgpartloctab, \\
                               & const SCOTCH\_Num         & indpartval,    \\
                               & const SCOTCH\_Num         & indvertlocnbr, \\
                               & SCOTCH\_Dgraph * const    & indgrafptr)
\end{tabular}}

{\tt\begin{tabular}{l@{}ll}
scotchfdgraphinducepart ( & doubleprecision (*)   & orggrafdat,    \\
                          & integer*{\it num} (*) & orgpartloctab, \\
                          & integer*{\it num}     & indpartval,    \\
                          & integer*{\it num}     & indvertlocnbr, \\
                          & doubleprecision (*)   & indgrafdat,    \\
                          & integer               & ierr)
\end{tabular}}

\progdes

The {\tt SCOTCH\_dgraphInducePart} routine creates in the
{\tt SCOTCH\_\lbt Dgraph} structure pointed to by {\tt indgrafptr}
a distributed induced subgraph of the {\tt SCOTCH\_\lbt Dgraph}
pointed to by {\tt orggrafptr}. The local vertices of every processor
that are kept in the induced subgraph are the ones for which the
values contained in the {\tt orgpart\lbt loctab} array cells are equal
to {\tt indpartval}.

The {\tt orgpartloctab} array must be of a size at least equal to
the number of local vertices of the original graph. It may be larger,
e.g. equal to the number of local plus ghost vertices, if needed by
the user, but only the area corresponding to the local vertices will
be used by {\tt SCOTCH\_\lbt dgraph\lbt Induce\lbt Part}.

{\tt indvertlocnbr} is the number of local vertices in the induced
subgraph. It must therefore be equal to the number of local vertices
that have their associated {\tt org\lbt part\lbt loc\lbt tab} cell
value equal to {\tt indpartval}. This value is necessary to internal
array allocations. While it could have been easily computed by
\scotch, by traversing the {\tt orgpart\lbt gsttab} array, making it
used-provided spares such a traversal if the user already knows the
value. If it is not the case, setting this value to {\tt -1} will make
\scotch\ compute it automatically.

{\tt indgrafptr} must have been initialized with the {\tt SCOTCH\_\lbt
dgraph\lbt Init} routine before {\tt SCOTCH\_dgraph\lbt Induce\lbt
Part} is called. The communicator that is passed to it can either be
the communicator used by the original graph {\tt org\lbt graf\lbt
ptr}, or any congruent communicator created by using {\tt MPI\_\lbt
Comm\_\lbt dup} on this communicator. Using a distinct communicator
for the induced subgraph allows subsequent library routines to be
called in parallel on the two graphs after the induced graph is
created.

Induced band graphs have vertex labels attached to each of their
vertices, in the {\tt vlbl\lbt loc\lbt tab} array. If the original
graph had vertex labels attached to it, induced graph vertex labels
are the labels of the corresponding vertices in the original
graph. Else, induced graph vertex labels are the global indices of
corresponding vertices in the original graph.

Depending on the partition array, the distribution of induced graph
vertices may be highly imbalanced. In order for further computations
on this distributed graph to scale well, a redistribution of its data
may be necessary, using the {\tt SCOTCH\_dgraph\lbt Redist} routine.

\progret

{\tt SCOTCH\_dgraphInducePart} returns $0$ if the induced graph
structure has been successfully created, and $1$ else.
\end{itemize}

\subsubsection{{\tt SCOTCH\_dgraphInit}}
\label{sec-lib-dgraphinit}

\begin{itemize}
\progsyn

{\tt\begin{tabular}{l@{}ll}
int SCOTCH\_dgraphInit ( & SCOTCH\_Dgraph * & grafptr, \\
                         & MPI\_Comm        & comm)
\end{tabular}}

{\tt\begin{tabular}{l@{}ll}
scotchfdgraphinit ( & doubleprecision (*) & grafdat, \\
                    & integer             & comm, \\
                    & integer             & ierr)
\end{tabular}}

\progdes

The {\tt SCOTCH\_dgraphInit} function initializes a {\tt SCOTCH\_\lbt
Dgraph} structure so as to make it suitable for future parallel
operations. It should be the first function to be called upon a {\tt
SCOTCH\_\lbt Dgraph} structure. By accessing the communicator handle
which is passed to it, {\tt SCOTCH\_dgraphInit} can know how many
processes will be used to manage the distributed graph and can allocate
its private structures accordingly.

{\tt SCOTCH\_dgraphInit} does not make a duplicate of the communicator
which is passed to it, but instead keeps a reference to it, so that
all future communications needed by \libscotch\ to process this graph
will be performed using this communicator. Therefore, it is the user's
responsibility, whenever several \libscotch\ routines might be called
in parallel, to create appropriate duplicates of communicators so as
to avoid any potential interferences between concurrent
communications.

When the distributed graph is no longer of use, the {\tt
SCOTCH\_\lbt dgraph\lbt Exit} function must be called, to free
its internal data arrays.

If {\tt SCOTCH\_\lbt dgraph\lbt Init} was called with a
communicator that is not a predefined MPI communicator (such as
{\tt MPI\_\tt COMM\_\lbt WORLD} or {\tt MPI\_\tt COMM\_\lbt SELF}), it
is the user's responsibility to free this communicator after
all graphs that use it have been freed by means of the
{\tt SCOTCH\_\lbt dgraph\lbt Exit} routine.

\progret

{\tt SCOTCH\_dgraphInit} returns $0$ if the graph structure has been
successfully initialized, and $1$ else.
\end{itemize}

\subsubsection{{\tt SCOTCH\_dgraphRedist}}
\label{sec-lib-dgraphredist}

\begin{itemize}
\progsyn

{\tt\begin{tabular}{l@{}ll}

int SCOTCH\_dgraphRedist ( & SCOTCH\_Dgraph * const    & orggrafptr, \\
                           & const SCOTCH\_Num * const & partloctab, \\
                           & const SCOTCH\_Num * const & permgsttab, \\
                           & const SCOTCH\_Num         & vertlocdlt, \\
                           & const SCOTCH\_Num         & edgelocdlt, \\
                           & SCOTCH\_Dgraph * const    & redgrafptr)
\end{tabular}}

{\tt\begin{tabular}{l@{}ll}
scotchfdgraphredist ( & doubleprecision (*)   & orggrafdat, \\
                      & integer*{\it num} (*) & partloctab, \\
                      & integer*{\it num} (*) & permgsttab, \\
                      & integer*{\it num}     & vertlocdlt, \\
                      & integer*{\it num}     & edgelocdlt, \\
                      & doubleprecision (*)   & redgrafptr)
\end{tabular}}

\progdes

The {\tt SCOTCH\_dgraphRedist} routine initializes and fills the
redistributed graph structure pointed to by
{\tt red\lbt graf\lbt ptr} with a new distributed graph
made from data redistributed from the original
graph pointed to by {\tt org\lbt graf\lbt ptr}.

The partition array, {\tt part\lbt loc\lbt tab}, must always be
provided. It holds the part number associated with each local
vertex. Part indices are {\em not\/} based: target vertices are
numbered from $0$ to the number of parts minus $1$.

Whenever provided, the permutation array {\tt perm\lbt gst\lbt tab}
must be of a size equal to the number of local and ghost vertices of
the source graph (that is, {\tt vert\lbt gst\lbt nbr}, see
Section~\ref{sec-lib-format-dgraph}). Its contents must be based, that
is, permutation global indices start at {\tt baseval}. Both its local
and ghost contents must be valid. Consequently, it is the user's
responsibility to call {\tt SCOTCH\_dgraph\lbt Halo} whenever
necessary so as to propagate part values of the local vertices to
their ghost counterparts on other processes. {\tt SCOTCH\_\lbt
dgraph\lbt Redist} does not perform this halo exchange itself
because users may already have computed these values by themselves
when computing the new partition. If {\tt perm\lbt gst\lbt tab} is not
provided by the user, vertices in each part are reordered according to
their global indices in the source graph.

{\tt redgrafptr} must have been initialized with the {\tt SCOTCH\_\lbt
dgraph\lbt Init} routine before {\tt SCOTCH\_dgraph\lbt Redist} is
called. The communicator that is passed to it can either be the
communicator used by the original graph {\tt org\lbt graf\lbt ptr},
or any congruent communicator created by using {\tt MPI\_\lbt
Comm\_\lbt dup} on this communicator. Using a distinct communicator
for the redistributed graph allows subsequent library routines to be
called in parallel on the two graphs after the redistributed graph is
created.

Redistributed graphs have vertex labels attached to each of their
vertices, in the {\tt vlbl\lbt loc\lbt tab} array. If the original
graph had vertex labels attached to it, redistributed graph vertex
labels are the labels of the corresponding vertices in the original
graph. Else, redistributed graph vertex labels are the global indices
of corresponding vertices in the original graph.

\progret

{\tt SCOTCH\_dgraphRedist} returns $0$ if the redistributed graph has
been successfully created, and $1$ else.
\end{itemize}

\subsubsection{{\tt SCOTCH\_dgraphScatter}}

\begin{itemize}
\progsyn

{\tt\begin{tabular}{l@{}ll}
int SCOTCH\_dgraphScatter ( & SCOTCH\_Dgraph * const      & dgrfptr, \\
                            & const SCOTCH\_Graph * const & cgrfptr)
\end{tabular}}

{\tt\begin{tabular}{l@{}ll}
scotchfdgraphscatter ( & doubleprecision (*) & dgrfdat, \\
                       & doubleprecision (*) & cgrfdat, \\
                       & integer             & ierr)
\end{tabular}}

\progdes

The {\tt SCOTCH\_dgraphScatter} routine scatters the contents of the
centralized {\tt SCOTCH\_\lbt Graph} structure pointed to by
{\tt cgrfptr} across the processes of the distributed
{\tt SCOTCH\_\lbt Dgraph} structure pointed to by {\tt dgrfptr}.

Only one of the processes should provide a non-null {\tt cgrfptr}
parameter. This process is considered the root process for the
scattering operation. Since, in Fortran, there is no null reference,
processes which are not the root must indicate it by passing a pointer
to the distributed graph structure equal to the pointer to their
centralized graph structure.

The scattering is performed such that graph vertices are evenly
spread across the processes of the communicator associated with
the distributed graph, in ascending order. Every process receives
either
$\left\lceil\frac{\mbox{vertglbnbr}}{\mbox{procglbnbr}}\right\rceil$
or
$\left\lfloor\frac{\mbox{vertglbnbr}}{\mbox{procglbnbr}}\right\rfloor$
vertices, according to its rank: processes of lower ranks are filled
first, eventually with one more vertex than processes of higher ranks.

\progret

{\tt SCOTCH\_dgraphScatter} returns $0$ if the graph structure has
been successfully scattered, and $1$ else.
\end{itemize}

\subsubsection{{\tt SCOTCH\_dgraphSize}}

\begin{itemize}
\progsyn

{\tt\begin{tabular}{l@{}ll}
void SCOTCH\_dgraphSize ( & const SCOTCH\_Dgraph * & grafptr,    \\
                          & SCOTCH\_Num *          & vertglbptr, \\
                          & SCOTCH\_Num *          & vertlocptr, \\
                          & SCOTCH\_Num *          & edgeglbptr, \\
                          & SCOTCH\_Num *          & edgelocptr)
\end{tabular}}

{\tt\begin{tabular}{l@{}ll}
scotchfdgraphsize ( & doubleprecision (*) & grafdat,    \\
                    & integer*{\it num}   & vertglbnbr, \\
                    & integer*{\it num}   & vertlocnbr, \\
                    & integer*{\it num}   & edgeglbnbr, \\
                    & integer*{\it num}   & edgelocnbr)
\end{tabular}}

\progdes

The {\tt SCOTCH\_dgraphSize} routine fills the four areas of type
{\tt SCOTCH\_\lbt Num} pointed to by {\tt vertglbptr},
{\tt vertlocptr}, {\tt edgeglbptr} and {\tt edgelocptr}
with the number of global vertices and arcs (that is, twice the number
of edges) of the given graph pointed to by {\tt grafptr}, as well as
with the number of local vertices and arcs borne by each of the
calling processes.

Any of these pointers can be set to {\tt NULL} on input if the
corresponding information is not needed. Else, the reference to a
dummy area can be provided, where all unwanted data will be written.

This routine is useful to get the size of a graph read by means
of the {\tt SCOTCH\_\lbt dgraph\lbo Load} routine, in order to allocate
auxiliary arrays of proper sizes. If the whole structure of the
graph is wanted, function {\tt SCOTCH\_dgraph\lbo Data} should be
preferred.

\end{itemize}

\subsection{Distributed graph I/O routines}

\subsubsection{{\tt SCOTCH\_dgraphLoad}}
\label{sec-lib-dgraphload}

\begin{itemize}
\progsyn

{\tt\begin{tabular}{l@{}ll}
int SCOTCH\_dgraphLoad ( & SCOTCH\_Dgraph * & grafptr, \\
                         & FILE *           & stream, \\
                         & SCOTCH\_Num      & baseval, \\
                         & SCOTCH\_Num      & flagval)
\end{tabular}}

{\tt\begin{tabular}{l@{}ll}
scotchfdgraphload ( & doubleprecision (*) & grafdat, \\
                    & integer             & fildes, \\
                    & integer*{\it num}   & baseval, \\
                    & integer*{\it num}   & flagval, \\
                    & integer             & ierr)
\end{tabular}}

\progdes

The {\tt SCOTCH\_dgraphLoad} routine fills the {\tt
SCOTCH\_\lbt Dgraph} structure pointed to by {\tt grafptr} with the
centralized or distributed source graph description available from
one or several streams {\tt stream} in the \scotch\ graph formats
(please refer to section~\ref{sec-file-dsgraph} for a description
of the distributed graph format, and to the {\it\scotch\ User's
Guide}~\scotchcitesuser\ for the centralized graph format).

When only one stream pointer is not null, the associated source graph
file must be a centralized one, the contents of which are spread
across all of the processes. When all stream pointers are non null,
they can either refer to multiple instances of the same centralized
graph, or to the distinct fragments of a distributed graph. In the
first case, all processes read all of the contents of the centralized
graph files but keep only the relevant part. In the second case, every
process reads its fragment in parallel.

To ease the handling of source graph files by programs written in C as
well as in Fortran, the base value of the graph to read can be set
to {\tt 0} or {\tt 1}, by setting the {\tt baseval} parameter to the
proper value. A value of {\tt -1} indicates that the graph base should
be the same as the one provided in the graph description that is read
from {\tt stream}.

The {\tt flagval} value is a combination of the following integer values,
that may be added or bitwise-ored:
\begin{itemize}
\iteme[{\tt 0}]
Keep vertex and edge weights if they are present in the {\tt stream} data.
\iteme[{\tt 1}]
Remove vertex weights. The graph read will have all of its vertex weights
set to one, regardless of what is specified in the {\tt stream} data.
\iteme[{\tt 2}]
Remove edge weights. The graph read will have all of its edge weights
set to one, regardless of what is specified in the {\tt stream} data.
\end{itemize}

Fortran users must use the {\tt PXFFILENO} or {\tt FNUM} functions to
obtain the number of the Unix file descriptor {\tt fildes} associated
with the logical unit of the graph file. Processes which would pass a
{\tt NULL} stream pointer in C must pass descriptor number {\tt -1} in
Fortran.

\progret

{\tt SCOTCH\_dgraphLoad} returns $0$ if the distributed graph
structure has been successfully allocated and filled with the data
read, and $1$ else.
\end{itemize}

\subsubsection{{\tt SCOTCH\_dgraphSave}}

\begin{itemize}
\progsyn

{\tt\begin{tabular}{l@{}ll}
int SCOTCH\_dgraphSave ( & const SCOTCH\_Dgraph * & grafptr, \\
                         & FILE *                 & stream)
\end{tabular}}

{\tt\begin{tabular}{l@{}ll}
scotchfdgraphsave ( & doubleprecision (*) & grafdat, \\
                    & integer             & fildes, \\
                    & integer             & ierr)
\end{tabular}}

\progdes

The {\tt SCOTCH\_dgraphSave} routine saves the contents of the {\tt
SCOTCH\_\lbt Dgraph} structure pointed to by {\tt grafptr} to streams
{\tt stream}, in the \scotch\ distributed graph format (see
section~\ref{sec-file-dsgraph}).

Fortran users must use the {\tt PXFFILENO} or {\tt FNUM} functions to
obtain the number of the Unix file descriptor {\tt fildes} associated
with the logical unit of the graph file.

\progret

{\tt SCOTCH\_dgraphSave} returns $0$ if the graph structure has been
successfully written to {\tt stream}, and $1$ else.
\end{itemize}

\subsection{Data handling and exchange routines}

\subsubsection{{\tt SCOTCH\_dgraphGhst}}
\label{sec-lib-dgraphghst}

\begin{itemize}
\progsyn

{\tt\begin{tabular}{l@{}ll}

int SCOTCH\_dgraphGhst ( & SCOTCH\_Dgraph * const & grafptr)
\end{tabular}}

{\tt\begin{tabular}{l@{}ll}
scotchfdgraphghst ( & doubleprecision (*) & grafdat, \\
                    & integer             & ierr)
\end{tabular}}

\progdes

The {\tt SCOTCH\_dgraphGhst} routine fills the {\tt edge\lbt gst\lbt
tab} arrays of the distributed graph structure pointed to by {\tt
grafptr} with the local and ghost vertex indices corresponding to the
global vertex indices contained in its {\tt edge\lbt loc\lbt tab}
arrays, according to the semantics described in
Section~\ref{sec-lib-format-dgraph}.

If memory areas had not been previously reserved by the user for the
{\tt edge\lbt gst\lbt tab} arrays and linked to the distributed graph
structure through a call to {\tt SCOTCH\_\lbt dgraph\lbt Build}, they
are allocated. Their references can be retrieved on every process by
means of a call to {\tt SCOTCH\_\lbt dgraph\lbt Data}, which will also
return the number of local and ghost vertices, suitable for allocating
vertex data arrays for {\tt SCOTCH\_\lbt dgraph\lbt Halo}.

\progret

{\tt SCOTCH\_dgraphGhst} returns $0$ if ghost vertex data has been
successfully computed, and $1$ else.
\end{itemize}

\subsubsection{{\tt SCOTCH\_dgraphHalo}}
\label{sec-lib-dgraphhalo}

\begin{itemize}
\progsyn

{\tt\begin{tabular}{l@{}ll}

int SCOTCH\_dgraphHalo ( & SCOTCH\_Dgraph * const & grafptr, \\
                         & void *                 & datatab, \\
                         & MPI\_Datatype          & typeval)
\end{tabular}}

{\tt\begin{tabular}{l@{}ll}
scotchfdgraphhalo ( & doubleprecision (*) & grafdat, \\
                    & doubleprecision (*) & datatab, \\
                    & integer             & typeval, \\
                    & integer             & ierr)
\end{tabular}}

\progdes

The {\tt SCOTCH\_dgraphHalo} routine propagates the data borne by
local vertices to all of the corresponding halo vertices located on
neighboring processes, in a synchronous way. On every process,
{\tt datatab} should point to a data array of a size sufficient to
hold {\tt vert\lbt gst\lbt nbr} elements of the data type to be
exchanged, the first {\tt vertlocnbr} slots of which must already be
filled with the information associated with the local vertices. On
completion, the $({\tt vert\lbt gst\lbt nbr} - {\tt vert\lbt loc\lbt
nbr})$ remaining slots are filled with copies of the corresponding
remote data obtained from the local parts of the data arrays of
neighboring processes.

When the MPI data type to be used is not a collection of contiguous
entries, great care should be taken in the definition of the upper
bound of the type (by using the {\tt MPI\_\lbo UB} pseudo-datatype),
such that when asking MPI to send a certain number of elements of the
said type located at some address, contiguous areas in memory will be
considered. Please refer to the MPI documentation regarding the
creation of derived datatypes~\cite[Section 3.12.3]{mpi11} for more
information.

To perform its data exchanges, the {\tt SCOTCH\_dgraph\lbt Halo}
routine requires ghost vertex management data provided by the {\tt
SCOTCH\_\lbt dgraph\lbt Ghst} routine. Therefore, the {\tt edge\lbt
gst\lbt tab} array returned by the {\tt SCOTCH\_dgraph\lbt Data}
routine will always be valid after a call to {\tt SCOTCH\_dgraph\lbt
Halo}, if it was not already.

In case useful computation can be carried out during the halo
exchange, an asynchronous version of this routine is available, called
{\tt SCOTCH\_\lbt dgraph\lbt Halo\lbt Async}.

\progret

{\tt SCOTCH\_dgraphHalo} returns $0$ if halo data has been
successfully exchanged, and $1$ else.
\end{itemize}

\subsubsection{{\tt SCOTCH\_dgraphHaloAsync}}
\label{sec-lib-dgraphhaloasync}

\begin{itemize}
\progsyn

{\tt\begin{tabular}{l@{}ll}

int SCOTCH\_dgraphHaloAsync ( & SCOTCH\_Dgraph * const        & grafptr, \\
                              & void *                        & datatab, \\
                              & MPI\_Datatype                 & typeval, \\
                              & SCOTCH\_DgraphHaloReq * const & requptr)
\end{tabular}}

{\tt\begin{tabular}{l@{}ll}
scotchfdgraphhaloasync ( & doubleprecision (*) & grafdat, \\
                         & doubleprecision (*) & datatab, \\
                         & integer             & typeval, \\
                         & doubleprecision (*) & requptr, \\
                         & integer             & ierr)
\end{tabular}}

\progdes

The {\tt SCOTCH\_dgraphHaloAsync} routine propagates the data borne by
local vertices to all of the corresponding halo vertices located on
neighboring processes, in an asynchronous way. On every process, {\tt
datatab} should point to a data array of a size sufficient to hold
{\tt vert\lbt gst\lbt nbr} elements of the data type to be exchanged,
the first {\tt vertlocnbr} slots of which must already be filled with
the information associated with the local vertices. On completion, the
$({\tt vert\lbt gst\lbt nbr} - {\tt vert\lbt loc\lbt nbr})$ remaining
slots are filled with copies of the corresponding remote data obtained
from the local parts of the data arrays of neighboring processes.

The semantics of {\tt SCOTCH\_dgraphHaloAsync} is similar to the one
of {\tt SCOTCH\_dgraph\lbt Halo}, except that it returns as soon as
possible, while effective communication may not have started nor
completed. Also, it possesses an additional parameter, {\tt requptr},
which must point to a {\tt SCOTCH\_\lbt Dgraph\lbt Halo\lbt Req} data
structure. Similarly to asynchronous MPI calls, users can wait for the
completion of a {\tt SCOTCH\_dgraph\lbt Halo\lbt Async} routine by
calling the {\tt SCOTCH\_dgraph\lbt Halo\lbt Wait} routine, passing it
a pointer to this request structure. In Fortran, the request structure
must be defined as an array of {\tt DOUBLEPRECISION}s, of size
{\tt SCOTCH\_\lbt DGRAPH\lbt HALO\lbt REQDIM}. This constant is
defined in file {\tt ptscotchf.h}, which must be included whenever
necessary.

The effective means for {\tt SCOTCH\_dgraph\lbt Halo\lbt Async} to
perform its task may vary at compile time, depending on the presence
of a thread-safe MPI library or on the existence of asynchronous
collective communication routines such as {\tt MPE\_\lbt Ialltoallv}.
In case no method for performing asynchronous collective communication
is available, {\tt SCOTCH\_dgraph\lbt Halo\lbt Async} will internally
call {\tt SCOTCH\_dgraph\lbt Halo} to perform synchronous communication.

Because of possible limitations in the implementation of third-party
communication routines, it is not recommended to perform simultaneous
{\tt SCOTCH\_dgraph\lbt Halo\lbt Async} calls on the same communicator.

\progret

{\tt SCOTCH\_dgraphHaloAsync} returns $0$ if the halo data exchange
has been successfully started, and $1$ else.
\end{itemize}

\subsubsection{{\tt SCOTCH\_dgraphHaloWait}}
\label{sec-lib-dgraphhalowait}

\begin{itemize}
\progsyn

{\tt\begin{tabular}{l@{}ll}

int SCOTCH\_dgraphHaloWait ( & SCOTCH\_DgraphHaloReq * const & requptr)
\end{tabular}}

{\tt\begin{tabular}{l@{}ll}
scotchfdgraphhalowait ( & doubleprecision (*) & requptr, \\
                        & integer             & ierr)
\end{tabular}}

\progdes

The {\tt SCOTCH\_dgraphHaloWait} routine waits for the termination of
an asynchronous halo exchange process, started by a call to {\tt
SCOTCH\_dgraph\lbt Halo\lbt Async}, and represented by its request,
pointed to by {\tt requptr}.

In Fortran, the request structure must be defined as an array of
{\tt DOUBLEPRECISION}s, of size {\tt SCOTCH\_\lbt DGRAPH\lbt HALO\lbt
REQDIM}. This constant is defined in file {\tt ptscotchf.h}, which
must be included whenever necessary.

\progret

{\tt SCOTCH\_dgraphHaloWait} returns $0$ if halo data has been
successfully exchanged, and $1$ else.
\end{itemize}

\subsection{Distributed graph mapping and partitioning routines}

{\tt SCOTCH\_dgraphMap} and {\tt SCOTCH\_dgraphPart} provide
high-level functionalities and free the user from the burden of
calling in sequence several of the low-level routines also described
in this section.

\subsubsection{{\tt SCOTCH\_dgraphMap}}

\begin{itemize}
\progsyn

{\tt\begin{tabular}{l@{}ll}
int SCOTCH\_dgraphMap ( & const SCOTCH\_Dgraph * & grafptr, \\
                        & const SCOTCH\_Arch *   & archptr, \\
                        & const SCOTCH\_Strat *  & straptr, \\
                        & SCOTCH\_Num *          & partloctab)
\end{tabular}}

{\tt\begin{tabular}{l@{}ll}
scotchfdgraphmap ( & doubleprecision (*)   & grafdat,    \\
                   & doubleprecision (*)   & archdat,    \\
                   & doubleprecision (*)   & stradat,    \\
                   & integer*{\it num} (*) & partloctab, \\
                   & integer               & ierr)
\end{tabular}}

\progdes

The {\tt SCOTCH\_dgraphMap} routine computes a mapping of the
distributed source graph structure pointed to by {\tt grafptr} onto
the target architecture pointed to by {\tt archptr}, using the mapping
strategy pointed to by {\tt straptr}, and returns distributed
fragments of the partition data in the array pointed to by {\tt
partloctab}.

The {\tt partloctab} array should have been previously allocated, of a
size sufficient to hold as many {\tt SCOTCH\_\lbt Num} integers as
there are local vertices of the source graph on each of the processes.

On return, every cell of the mapping array holds the number of the
target vertex to which the corresponding source vertex is mapped.
The numbering of target values is {\em not\/} based: target vertices
are numbered from $0$ to the number of target vertices minus $1$.

{\bf Attention}: version {\sc 6.0} of \scotch\ does not allow yet to
map distributed graphs onto target architectures which are not
complete graphs. This restriction will be removed in the next release.

\progret

{\tt SCOTCH\_dgraphMap} returns $0$ if the partition of the graph has
been successfully computed, and $1$ else. In this last case, the
{\tt partloctab} arrays may however have been partially or completely
filled, but their contents is not significant.
\end{itemize}

\subsubsection{{\tt SCOTCH\_dgraphMapCompute}}

\begin{itemize}
\progsyn

{\tt\begin{tabular}{l@{}ll}
int SCOTCH\_dgraphMapCompute ( & const SCOTCH\_Dgraph * & grafptr, \\
                               & SCOTCH\_Dmapping *     & mappptr, \\
                               & const SCOTCH\_Strat *  & straptr)
\end{tabular}}

{\tt\begin{tabular}{l@{}ll}
scotchfdgraphmapcompute ( & doubleprecision (*) & grafdat, \\
                          & doubleprecision (*) & mappdat, \\
                          & doubleprecision (*) & stradat, \\
                          & integer             & ierr)
\end{tabular}}

\progdes

The {\tt SCOTCH\_dgraphMapCompute} routine computes a mapping
on the given {\tt SCOTCH\_\lbt Dmapping} structure pointed
to by {\tt mappptr} using the parallel mapping strategy pointed to
by {\tt stratptr}.

On return, every cell of the distributed mapping array (see
section~\ref{sec-lib-dgraph-map-init}) holds the number of the target
vertex to which the corresponding source vertex is mapped. The
numbering of target values is {\em not\/} based: target vertices are
numbered from $0$ to the number of target vertices, minus $1$.

{\bf Attention}: version {\sc 6.0} of \scotch\ does not allow yet to
map distributed graphs onto target architectures which are not
complete graphs. This restriction will be removed in the next release.

\progret

{\tt SCOTCH\_dgraphMapCompute} returns $0$ if the mapping has been
successfully computed, and $1$ else. In this latter case, the local
mapping arrays may however have been partially or completely filled,
but their contents is not significant.
\end{itemize}

\subsubsection{{\tt SCOTCH\_dgraphMapExit}}

\begin{itemize}
\progsyn

{\tt\begin{tabular}{l@{}ll}
void SCOTCH\_dgraphMapExit ( & const SCOTCH\_Dgraph * & grafptr, \\
                             & SCOTCH\_Dmapping *     & mappptr)
\end{tabular}}

{\tt\begin{tabular}{l@{}ll}
scotchfdgraphmapexit ( & doubleprecision (*) & grafdat, \\
                       & doubleprecision (*) & mappdat)
\end{tabular}}

\progdes

The {\tt SCOTCH\_dgraphMapExit} function frees the contents of a
{\tt SCOTCH\_\lbt Dmapping} structure previously initialized by
{\tt SCOTCH\_\lbt dgraph\lbt Map\lbt Init}. All subsequent calls to
{\tt SCOTCH\_\lbt dgraph\lbt Map*} routines other than
{\tt SCOTCH\_\lbt dgraph\lbt Map\lbt Init}, using this structure
as parameter, may yield unpredictable results.
\end{itemize}

\subsubsection{{\tt SCOTCH\_dgraphMapInit}}
\label{sec-lib-dgraph-map-init}

\begin{itemize}
\progsyn

{\tt\begin{tabular}{l@{}ll}
int SCOTCH\_dgraphMapInit ( & const SCOTCH\_Dgraph * & grafptr, \\
                            & SCOTCH\_Dmapping *     & mappptr, \\
                            & const SCOTCH\_Arch *   & archptr, \\
                            & SCOTCH\_Num *          & partloctab)
\end{tabular}}

{\tt\begin{tabular}{l@{}ll}
scotchfdgraphmapinit ( & doubleprecision (*)   & grafdat,    \\
                       & doubleprecision (*)   & mappdat,    \\
                       & doubleprecision (*)   & archdat,    \\
                       & integer*{\it num} (*) & partloctab, \\
                       & integer               & ierr)
\end{tabular}}

\progdes

The {\tt SCOTCH\_dgraphMapInit} routine fills the distributed mapping
structure pointed to by {\tt mappptr} with all of the data that is
passed to it. Thus, all subsequent calls to ordering routines such as
{\tt SCOTCH\_\lbt dgraph\lbt Map\lbt Compute}, using this mapping
structure as parameter, will place mapping results in field {\tt
part\lbt loc\lbt tab}.

{\tt partloctab} is the pointer to an array of as many {\tt
SCOTCH\_\lbt Num}s as there are local vertices in each local fragment
of the distributed graph pointed to by {\tt grafptr}, and which will
receive the indices of the vertices of the target architecture pointed
to by {\tt archptr}.

It should be the first function to be called upon a {\tt SCOTCH\_\lbt
Dmapping} structure. When the distributed mapping structure is no
longer of use, call function {\tt SCOTCH\_dgraph\lbt \lbt Map\lbt
Exit} to free its internal structures.

\progret

{\tt SCOTCH\_dgraphMapInit} returns $0$ if the distributed mapping
structure has been successfully initialized, and $1$ else.
\end{itemize}

\subsubsection{{\tt SCOTCH\_dgraphMapSave}}

\begin{itemize}
\progsyn

{\tt\begin{tabular}{l@{}ll}
int SCOTCH\_dgraphMapSave ( & const SCOTCH\_Dgraph *   & grafptr, \\
                            & const SCOTCH\_Dmapping * & mappptr, \\
                            & FILE *                   & stream)
\end{tabular}}

{\tt\begin{tabular}{l@{}ll}
scotchfdgraphmapsave ( & doubleprecision (*) & grafdat, \\
                       & doubleprecision (*) & mappdat, \\
                       & integer             & fildes,  \\
                       & integer             & ierr)
\end{tabular}}

\progdes

The {\tt SCOTCH\_dgraphMapSave} routine saves the contents of the {\tt
SCOTCH\_\lbt Dmapping} structure pointed to by {\tt mappptr} to stream
{\tt stream}, in the \scotch\ mapping format. Please refer to the
{\it\scotch\ User's Guide}~\scotchcitesuser\ for more information about
this format.

Since the mapping format is centralized, only one process should
provide a valid output stream; other processes must pass a null
pointer.

Fortran users must use the {\tt PXFFILENO} or {\tt FNUM} functions to
obtain the number of the Unix file descriptor {\tt fildes} associated
with the logical unit of the mapping file.

\progret

{\tt SCOTCH\_dgraphMapSave} returns $0$ if the mapping structure
has been successfully written to {\tt stream}, and $1$ else.
\end{itemize}

\subsubsection{{\tt SCOTCH\_dgraphPart}}

\begin{itemize}
\progsyn

{\tt\begin{tabular}{l@{}ll}
int SCOTCH\_dgraphPart ( & const SCOTCH\_Dgraph * & grafptr, \\
                         & const SCOTCH\_Num      & partnbr, \\
                         & const SCOTCH\_Strat *  & straptr, \\
                         & SCOTCH\_Num *          & partloctab)
\end{tabular}}

{\tt\begin{tabular}{l@{}ll}
scotchfdgraphpart ( & doubleprecision (*)   & grafdat,    \\
                    & integer*{\it num}     & partnbr,    \\
                    & doubleprecision (*)   & stradat,    \\
                    & integer*{\it num} (*) & partloctab, \\
                    & integer               & ierr)
\end{tabular}}

\progdes

The {\tt SCOTCH\_dgraphPart} routine computes a partition into {\tt
partnbr} parts of the distributed source graph structure pointed to
by {\tt grafptr}, using the graph partitioning strategy pointed to by
{\tt stratptr}, and returns distributed fragments of the partition
data in the array pointed to by {\tt partloctab}.

The {\tt partloctab} array should have been previously allocated, of a
size sufficient to hold as many {\tt SCOTCH\_\lbt Num} integers as
there are local vertices of the source graph on each of the processes.

On return, every array cell holds the number of the part to which the
corresponding vertex is mapped. Parts are numbered from $0$ to
$\mbox{\tt partnbr} - 1$.

\progret

{\tt SCOTCH\_dgraphPart} returns $0$ if the partition of the graph has
been successfully computed, and $1$ else. In this latter case, the
{\tt partloctab} array may however have been partially or completely
filled, but its content is not significant.
\end{itemize}

\subsection{Distributed graph ordering routines}

\subsubsection{{\tt SCOTCH\_dgraphOrderCblkDist}}

\begin{itemize}
\progsyn

{\tt\begin{tabular}{l@{}ll}
SCOTCH\_Num SCOTCH\_dgraphOrderCblkDist ( & const SCOTCH\_Dgraph * & grafptr, \\
                                          & SCOTCH\_Dordering *    & ordeptr)
\end{tabular}}

{\tt\begin{tabular}{l@{}ll}
scotchfdgraphordercblkdist ( & doubleprecision (*) & grafdat, \\
                             & doubleprecision (*) & ordedat, \\
                             & integer*{\it num}   & cblkglbnbr)
\end{tabular}}

\progdes

The {\tt SCOTCH\_dgraphOrderCblkDist} routine returns on all processes
the global number of distributed elimination tree (super-)nodes
possessed by the given distributed ordering. Distributed elimination
tree nodes are produced for instance by parallel nested dissection,
before the ordering process goes sequential. Subsequent sequential
nodes generated locally afterwards on individual processes are not
accounted for in this figure.

This routine is used to allocate space for the tree structure arrays
to be filled by the {\tt SCOTCH\_\lbt dgraph\lbt Order\lbt Tree\lbt
Dist} routine.

\progret

{\tt SCOTCH\_dgraphOrderCblkDist} returns a positive number if the
number of distributed elimination tree nodes has been successfully
computed, and a negative value else.
\end{itemize}

\subsubsection{{\tt SCOTCH\_dgraphOrderCompute}}

\begin{itemize}
\progsyn

{\tt\begin{tabular}{l@{}ll}
int SCOTCH\_dgraphOrderCompute ( & const SCOTCH\_Dgraph * & grafptr, \\
                                 & SCOTCH\_Dordering *    & ordeptr, \\
                                 & const SCOTCH\_Strat *  & straptr)
\end{tabular}}

{\tt\begin{tabular}{l@{}ll}
scotchfdgraphordercompute ( & doubleprecision (*) & grafdat, \\
                            & doubleprecision (*) & ordedat, \\
                            & doubleprecision (*) & stradat, \\
                            & integer             & ierr)
\end{tabular}}

\progdes

The {\tt SCOTCH\_dgraphOrderCompute} routine computes in parallel a
distributed block ordering of the distributed graph structure pointed
to by {\tt grafptr}, using the distributed ordering strategy pointed to
by {\tt stratptr}, and stores its result in the distributed ordering
structure pointed to by {\tt ordeptr}.

\progret

{\tt SCOTCH\_dgraphOrderCompute} returns $0$ if the ordering has been
successfully computed, and $1$ else. In this latter case, the ordering
arrays may however have been partially or completely filled, but their
contents are not significant.
\end{itemize}

\subsubsection{{\tt SCOTCH\_dgraphOrderExit}}

\begin{itemize}
\progsyn

{\tt\begin{tabular}{l@{}ll}
void SCOTCH\_dgraphOrderExit ( & const SCOTCH\_Dgraph * & grafptr, \\
                               & SCOTCH\_Dordering *    & ordeptr)
\end{tabular}}

{\tt\begin{tabular}{l@{}ll}
scotchfgraphdorderexit ( & doubleprecision (*) & grafdat, \\
                         & doubleprecision (*) & ordedat)
\end{tabular}}

\progdes

The {\tt SCOTCH\_dgraphOrderExit} function frees the contents of a
{\tt SCOTCH\_\lbt Dordering} structure previously initialized by
{\tt SCOTCH\_\lbt dgraph\lbt Order\lbt Init}. All subsequent calls to
{\tt SCOTCH\_\lbt dgraph\lbt Order*} routines other than
{\tt SCOTCH\_\lbt dgraph\lbt Order\lbt Init}, using this structure
as parameter, may yield unpredictable results.
\end{itemize}

\subsubsection{{\tt SCOTCH\_dgraphOrderInit}}
\label{sec-lib-dgraphorderinit}

\begin{itemize}
\progsyn

{\tt\begin{tabular}{l@{}ll}
int SCOTCH\_dgraphOrderInit ( & const SCOTCH\_Dgraph * & grafptr, \\
                              & SCOTCH\_Dordering *    & ordeptr)
\end{tabular}}

{\tt\begin{tabular}{l@{}ll}
scotchfdgraphorderinit ( & doubleprecision (*) & grafdat, \\
                         & doubleprecision (*) & ordedat, \\
                         & integer             & ierr)
\end{tabular}}

\progdes

The {\tt SCOTCH\_dgraph\lbt Order\lbt Init} routine initializes the
distributed ordering structure pointed to by {\tt ordeptr} so that it
can be used to store the results of the parallel ordering of the
associated distributed graph, to be computed by means of the
{\tt SCOTCH\_\lbt dgraph\lbt Order\lbt Compute} routine.

The {\tt SCOTCH\_\lbt dgraph\lbt Order\lbt Init} routine should be the
first function to be called upon a {\tt SCOTCH\_\lbt Dordering}
structure for ordering distributed graphs. When the ordering structure
is no longer of use, the {\tt SCOTCH\_\lbt dgraph\lbt Order\lbt Exit}
function must be called, in order to to free its internal structures.

\progret

{\tt SCOTCH\_dgraphOrderInit} returns $0$ if the distributed ordering
structure has been successfully initialized, and $1$ else.
\end{itemize}

\subsubsection{{\tt SCOTCH\_dgraphOrderSave}}

\begin{itemize}
\progsyn

{\tt\begin{tabular}{l@{}ll}
int SCOTCH\_dgraphOrderSave ( & const SCOTCH\_Dgraph *    & grafptr, \\
                              & const SCOTCH\_Dordering * & ordeptr, \\
                              & FILE *                    & stream)
\end{tabular}}

{\tt\begin{tabular}{l@{}ll}
scotchfdgraphordersave ( & doubleprecision (*) & grafdat, \\
                         & doubleprecision (*) & ordedat, \\
                         & integer             & fildes,  \\
                         & integer             & ierr)
\end{tabular}}

\progdes

The {\tt SCOTCH\_dgraphOrderSave} routine saves the contents of the {\tt
SCOTCH\_\lbt Dordering} structure pointed to by {\tt ordeptr} to stream
{\tt stream}, in the \scotch\ ordering format. Please refer to the
{\it\scotch\ User's Guide}~\scotchcitesuser\ for more information about
this format.

Since the ordering format is centralized, only one process should
provide a valid output stream; other processes must pass a null
pointer.

Fortran users must use the {\tt PXFFILENO} or {\tt FNUM} functions to
obtain the number of the Unix file descriptor {\tt fildes} associated
with the logical unit of the ordering file. Processes which would pass
a {\tt NULL} stream pointer in C must pass descriptor number {\tt -1}
in Fortran.

\progret

{\tt SCOTCH\_dgraphOrderSave} returns $0$ if the ordering structure
has been successfully written to {\tt stream}, and $1$ else.
\end{itemize}

\subsubsection{{\tt SCOTCH\_dgraphOrderSaveMap}}

\begin{itemize}
\progsyn

{\tt\begin{tabular}{l@{}ll}
int SCOTCH\_dgraphOrderSaveMap ( & const SCOTCH\_Dgraph *    & grafptr, \\
                                 & const SCOTCH\_Dordering * & ordeptr, \\
                                 & FILE *                    & stream)
\end{tabular}}

{\tt\begin{tabular}{l@{}ll}
scotchfgraphdordersavemap ( & doubleprecision (*) & grafdat, \\
                            & doubleprecision (*) & ordedat, \\
                            & integer             & fildes,  \\
                            & integer             & ierr)
\end{tabular}}

\progdes

The {\tt SCOTCH\_dgraphOrderSaveMap} routine saves the block
partitioning data associated with the {\tt SCOTCH\_\lbt Dordering}
structure pointed to by {\tt ordeptr} to stream {\tt stream}, in the
\scotch\ mapping format. A target domain number is associated with
every block, such that all node vertices belonging to the same block
are shown as belonging to the same target vertex.
The resulting mapping file can be used by the {\tt gout} program
to produce pictures showing the different separators and blocks.
Please refer to the {\it\scotch\ User's Guide} for more information
on the \scotch\ mapping format and on {\tt gout}.

Since the block partitioning format is centralized, only one process
should provide a valid output stream; other processes must pass a
null pointer.

Fortran users must use the {\tt PXFFILENO} or {\tt FNUM} functions to
obtain the number of the Unix file descriptor {\tt fildes} associated
with the logical unit of the ordering file. Processes which would pass
a {\tt NULL} stream pointer in C must pass descriptor number {\tt -1}
in Fortran.

\progret

{\tt SCOTCH\_dgraphOrderSaveMap} returns $0$ if the ordering structure
has been successfully written to {\tt stream}, and $1$ else.
\end{itemize}

\subsubsection{{\tt SCOTCH\_dgraphOrderSaveTree}}

\begin{itemize}
\progsyn

{\tt\begin{tabular}{l@{}ll}
int SCOTCH\_dgraphOrderSaveTree ( & const SCOTCH\_Dgraph *    & grafptr, \\
                                  & const SCOTCH\_Dordering * & ordeptr, \\
                                  & FILE *                    & stream)
\end{tabular}}

{\tt\begin{tabular}{l@{}ll}
scotchfdgraphordersavetree ( & doubleprecision (*) & grafdat, \\
                             & doubleprecision (*) & ordedat, \\
                             & integer             & fildes,  \\
                             & integer             & ierr)
\end{tabular}}

\progdes

The {\tt SCOTCH\_dgraphOrderSaveTree} routine saves the tree
hierarchy information associated with the {\tt SCOTCH\_\lbt Dordering}
structure pointed to by {\tt ordeptr} to stream {\tt stream}.

The format of the tree output file resembles the one of a mapping or
ordering file: it is made up of as many lines as there are vertices in
the ordering. Each of these lines holds two integer numbers. The first
one is the index or the label of the vertex, and the second one is the
index of its parent node in the separators tree, or $-1$ if the vertex
belongs to a root node.

Since the tree hierarchy format is centralized, only one process
should provide a valid output stream; other processes must pass a
null pointer.

Fortran users must use the {\tt PXFFILENO} or {\tt FNUM} functions to
obtain the number of the Unix file descriptor {\tt fildes} associated
with the logical unit of the ordering file. Processes which would pass
a {\tt NULL} stream pointer in C must pass descriptor number {\tt -1}
in Fortran.

\progret

{\tt SCOTCH\_dgraphOrderSaveTree} returns $0$ if the ordering structure
has been successfully written to {\tt stream}, and $1$ else.
\end{itemize}

\subsubsection{{\tt SCOTCH\_dgraphOrderPerm}}

\begin{itemize}
\progsyn

{\tt\begin{tabular}{l@{}ll}
int SCOTCH\_dgraphOrderPerm ( & const SCOTCH\_Dgraph * & grafptr, \\
                              & SCOTCH\_Dordering *    & ordeptr, \\
                              & SCOTCH\_Num *          & permloctab)
\end{tabular}}

{\tt\begin{tabular}{l@{}ll}
scotchfdgraphorderperm ( & doubleprecision (*)   & grafdat,    \\
                         & doubleprecision (*)   & ordedat,    \\
                         & integer*{\it num} (*) & permloctab, \\
                         & integer               & ierr)
\end{tabular}}

\progdes

The {\tt SCOTCH\_dgraphOrderPerm} routine fills the distributed direct
permutation array {\tt permloctab} according to the ordering provided
by the given distributed ordering pointed to by {\tt ordeptr}. Each
{\tt permloctab} local array must be of size {\tt vertlocnbr}.

\progret

{\tt SCOTCH\_dgraphOrderPerm} returns $0$ if the distributed
permutation has been successfully computed, and $1$ else.
\end{itemize}

\subsubsection{{\tt SCOTCH\_dgraphOrderTreeDist}}

\begin{itemize}
\progsyn

{\tt\begin{tabular}{l@{}ll}
int SCOTCH\_dgraphOrderTreeDist ( & const SCOTCH\_Dgraph * & grafptr,   \\
                                  & SCOTCH\_Dordering *    & ordeptr,   \\
                                  & SCOTCH\_Num *          & treeglbtab \\
                                  & SCOTCH\_Num *          & sizeglbtab)
\end{tabular}}

{\tt\begin{tabular}{l@{}ll}
scotchfdgraphordertreedist ( & doubleprecision (*)   & grafdat,    \\
                             & doubleprecision (*)   & ordedat,    \\
                             & integer*{\it num} (*) & treeglbtab, \\
                             & integer*{\it num} (*) & sizeglbtab, \\
                             & integer               & ierr)
\end{tabular}}

\progdes

The {\tt SCOTCH\_dgraphOrderTreeDist} routine fills on all processes
the arrays representing the distributed part of the elimination tree
structure associated with the given distributed ordering. This structure
describes the sizes and relations between all distributed elimination
tree (super-)nodes. These nodes are mainly the result of parallel
nested dissection, before the ordering process goes sequential.
Sequential nodes generated locally on individual processes are not
represented in this structure.

A node can either be a leaf column block, which has no
descendants, or a nested dissection node, which has most often
three sons: its two separated sub-parts and the separator. A
nested dissection node may have two sons only if the separator is
empty; it cannot have only one son. Sons are indexed such that the
separator of a block, if any, is always the son of highest index.
Hence, the order of the indices of the two sub-parts matches the one
of the direct permutation of the unknowns.

For any column block $i$, {\tt treeglbtab[}$i${\tt ]} holds the index
of the father of node $i$ in the elimination tree, or $-1$ if $i$ is
the root of the tree. All node indices start from
{\tt baseval}. {\tt size\lbt glb\lbt tab[}$i${\tt ]} holds the number
of graph vertices possessed by node $i$, plus the ones of all
of its descendants if it is not a leaf of the tree. Therefore, the
{\tt size\lbt glb\lbt tab} value of the root vertex is always equal to
the number of vertices in the distributed graph.

Each of the {\tt treeglbtab} and {\tt size\lbt glb\lbt tab} arrays
must be large enough to hold a number of {\tt SCOTCH\_\lbt Num}s equal
to the number of distributed elimination tree nodes and column blocks,
as returned by the {\tt SCOTCH\_\lbt dgraph\lbt Order\lbt Cblk\lbt Dist}
routine.

\progret

{\tt SCOTCH\_dgraphOrderTreeDist} returns $0$ if the arrays describing
the distributed part of the distributed tree structure have been
successfully filled, and $1$ else.
\end{itemize}

\subsection{Centralized ordering handling routines}
\label{sec-lib-corder}

Since distributed ordering structures maintain scattered information
which cannot be easily collated, the only practical way to access this
information is to centralize it in a sequential {\tt SCOTCH\_\lbt
Ordering} structure. Several routines are provided to
create and destroy sequential orderings attached to a distributed
graph, and to gather the information contained in a distributed
ordering on such a sequential ordering structure.

Since the arrays which represent centralized ordering must be of a
size equal to the global number of vertices, these routines are not
scalable and may require much memory for very large graphs.

\subsubsection{{\tt SCOTCH\_dgraphCorderExit}}

\begin{itemize}
\progsyn

{\tt\begin{tabular}{l@{}ll}
void SCOTCH\_dgraphCorderExit ( & const SCOTCH\_Dgraph * & grafptr, \\
                                & SCOTCH\_Ordering *     & cordptr)
\end{tabular}}

{\tt\begin{tabular}{l@{}ll}
scotchfdgraphcorderexit ( & doubleprecision (*) & grafdat, \\
                          & doubleprecision (*) & corddat)
\end{tabular}}

\progdes

The {\tt SCOTCH\_dgraphCorderExit} function frees the contents of a
centralized {\tt SCOTCH\_\lbt Ordering} structure previously
initialized by {\tt SCOTCH\_\lbt dgraph\lbt Corder\lbt Init}.
\end{itemize}


\subsubsection{{\tt SCOTCH\_dgraphCorderInit}}
\label{sec-lib-graph-corder-init}

\begin{itemize}
\progsyn

{\tt\begin{tabular}{l@{}ll}
int SCOTCH\_dgraphCorderInit ( & const SCOTCH\_Dgraph * & grafptr, \\
                               & SCOTCH\_Ordering *     & cordptr, \\
                               & SCOTCH\_Num *          & permtab, \\
                               & SCOTCH\_Num *          & peritab, \\
                               & SCOTCH\_Num *          & cblkptr, \\
                               & SCOTCH\_Num *          & rangtab, \\
                               & SCOTCH\_Num *          & treetab)
\end{tabular}}

{\tt\begin{tabular}{l@{}ll}
scotchfdgraphcorderinit ( & doubleprecision (*)   & grafdat, \\
                          & doubleprecision (*)   & corddat, \\
                          & integer*{\it num} (*) & permtab, \\
                          & integer*{\it num} (*) & peritab, \\
                          & integer*{\it num}     & cblknbr, \\
                          & integer*{\it num} (*) & rangtab, \\
                          & integer*{\it num} (*) & treetab, \\
                          & integer               & ierr)
\end{tabular}}

\progdes

The {\tt SCOTCH\_dgraph\lbt Corder\lbt Init} routine fills the
centralized ordering structure pointed to by {\tt cordptr} with all of
the data that are passed to it. This routine is the equivalent of the
{\tt SCOTCH\_\lbt graph\lbt Order\lbt Init} routine of the
\scotch\ sequential library, except that it takes a distributed graph
as input. It is used to initialize a centralized ordering structure on
which a distributed ordering will be centralized by means of the
{\tt SCOTCH\_\lbt dgraph\lbt Order\lbt Gather} routine. Only the
process on which distributed ordering data is to be centralized has
to handle a centralized ordering structure.

{\tt permtab} is the ordering permutation array, of size ${\tt
vert\lbt glb\lbt nbr}$, {\tt peritab} is the inverse ordering permutation array,
of size ${\tt vert\lbt glb\lbt nbr}$, {\tt cblkptr} is the pointer to a
{\tt SCOTCH\_\lbt Num} that will receive the number of produced
column blocks, {\tt rangtab} is the array that holds the column block
span information, of size $\mbox{\tt vert\lbt glb\lbt nbr} + 1$, and {\tt treetab}
is the array holding the structure of the separators tree, of size
${\tt vert\lbt glb\lbt nbr}$. Please refer to
Section~\ref{sec-lib-format-order} for an explanation of their semantics.
Any of these five output fields can be set to {\tt NULL} if the
corresponding information is not needed. Since, in Fortran, there is
no null reference, passing a reference to {\tt grafptr} will have the
same effect.

The {\tt SCOTCH\_\lbt dgraph\lbt Corder\lbt Init} routine should be
the first function to be called upon a {\tt SCOTCH\_\lbt Ordering}
structure to be used for gathering distributed ordering data. When the
centralized ordering structure is no longer of use, the {\tt
SCOTCH\_\lbt dgraph\lbt Corder\lbt Exit} function must be called, in
order to to free its internal structures.

\progret

{\tt SCOTCH\_dgraphCorderInit} returns $0$ if the ordering structure has
been successfully initialized, and $1$ else.
\end{itemize}

\subsubsection{{\tt SCOTCH\_dgraphOrderGather}}
\label{sec-lib-dgraph-order-gather}

\begin{itemize}
\progsyn

{\tt\begin{tabular}{l@{}ll}
int SCOTCH\_dgraphOrderGather ( & const SCOTCH\_Dgraph * & grafptr, \\
                                & SCOTCH\_Dordering *    & cordptr, \\
                                & SCOTCH\_Ordering *     & cordptr)
\end{tabular}}

{\tt\begin{tabular}{l@{}ll}
scotchfdgraphordergather ( & doubleprecision (*) & grafdat, \\
                           & doubleprecision (*) & dorddat, \\
                           & doubleprecision (*) & corddat, \\
                           & integer             & ierr)
\end{tabular}}

\progdes

The {\tt SCOTCH\_dgraph\lbt Order\lbt Gather} routine gathers the
distributed ordering data borne by {\tt dordptr} to the
centralized ordering structure pointed to by {\tt cordptr}.

\progret

{\tt SCOTCH\_dgraphOrderGather} returns $0$ if the centralized
ordering structure has been successfully updated, and $1$ else.
\end{itemize}

\subsection{Strategy handling routines}
\label{sec-lib-strat}

This section presents basic strategy handling routines which are also
described in the {\it\scotch\ User's Guide} but which are duplicated
here for the sake of readability, as well as a strategy declaration
routine which is specific to the \ptscotch\ library.

\subsubsection{{\tt SCOTCH\_stratExit}}

\begin{itemize}
\progsyn

{\tt\begin{tabular}{l@{}ll}
void SCOTCH\_stratExit ( & SCOTCH\_Strat * & archptr)
\end{tabular}}

{\tt\begin{tabular}{l@{}ll}
scotchfstratexit ( & doubleprecision (*) & stradat)
\end{tabular}}

\progdes

The {\tt SCOTCH\_stratExit} function frees the contents of a
{\tt SCOTCH\_\lbt Strat} structure previously initialized by
{\tt SCOTCH\_\lbt strat\lbt Init}. All subsequent calls to
{\tt SCOTCH\_\lbt strat} routines other than {\tt SCOTCH\_\lbt
strat\lbt Init}, using this structure as parameter, may yield
unpredictable results.
\end{itemize}

\subsubsection{{\tt SCOTCH\_stratInit}}

\begin{itemize}
\progsyn

{\tt\begin{tabular}{l@{}ll}
int SCOTCH\_stratInit ( & SCOTCH\_Strat * & straptr)
\end{tabular}}

{\tt\begin{tabular}{l@{}ll}
scotchfstratinit ( & doubleprecision (*) & stradat, \\
                   & integer             & ierr)
\end{tabular}}

\progdes

The {\tt SCOTCH\_stratInit} function initializes a {\tt SCOTCH\_\lbt
Strat} structure so as to make it suitable for future operations. It
should be the first function to be called upon a {\tt SCOTCH\_\lbt
Strat} structure. When the strategy data is no longer of use, call
function {\tt SCOTCH\_\lbt strat\lbt Exit} to free its internal
structures.

\progret

{\tt SCOTCH\_stratInit} returns $0$ if the strategy structure has been
successfully initialized, and $1$ else.
\end{itemize}

\subsubsection{{\tt SCOTCH\_stratSave}}

\begin{itemize}
\progsyn

{\tt\begin{tabular}{l@{}ll}
int SCOTCH\_stratSave ( & const SCOTCH\_Strat * & straptr, \\
                        & FILE *                & stream)
\end{tabular}}

{\tt\begin{tabular}{l@{}ll}
scotchfstratsave ( & doubleprecision (*) & stradat, \\
                   & integer             & fildes,  \\
                   & integer             & ierr)
\end{tabular}}

\progdes

The {\tt SCOTCH\_stratSave} routine saves the contents of the {\tt
SCOTCH\_\lbt Strat} structure pointed to by {\tt straptr} to stream
{\tt stream}, in the form of a text string. The methods and
parameters of the strategy string depend on the type of the strategy,
that is, whether it is a bipartitioning, mapping, or ordering
strategy, and to which structure it applies, that is, graphs or
meshes.

Fortran users must use the {\tt PXFFILENO} or {\tt FNUM} functions to
obtain the number of the Unix file descriptor {\tt fildes} associated
with the logical unit of the output file.

\progret

{\tt SCOTCH\_stratSave} returns $0$ if the strategy string has been
successfully written to {\tt stream}, and $1$ else.
\end{itemize}

\subsection{Strategy creation routines}
\label{sec-lib-strat-creation}

Strategy creation routines parse the user-provided strategy string and
populate the given opaque strategy object with a tree-shaped structure
that represents the parsed expression. It is this structure that will
be later traversed by the generic routines for partitioning, mapping or
ordering, so as to determine which specific partitioning, mapping or
ordering method to be called on a subgraph being considered.

Because strategy creation routines call third-party lexical analyzers
that may have been implemented in a non-reentrant way, no guarantee is
given on the reentrance of these routines. Consequently, strategy
creation routines that might be called simultaneously by multiple
threads should be protected by a mutex.

\subsubsection{{\tt SCOTCH\_stratDgraphMap}}

\begin{itemize}
\progsyn

{\tt\begin{tabular}{l@{}ll}
int SCOTCH\_stratDgraphMap ( & SCOTCH\_Strat * & straptr, \\
                             & const char *    & string)
\end{tabular}}

{\tt\begin{tabular}{l@{}ll}
scotchfstratdgraphmap ( & doubleprecision (*) & stradat, \\
                        & character (*)       & string,  \\
                        & integer             & ierr)
\end{tabular}}

\progdes

The {\tt SCOTCH\_stratDgraphMap} routine fills the strategy
structure pointed to by {\tt straptr} with the distributed graph
mapping strategy string pointed to by {\tt string}. The format of this
strategy string is described in Section~\ref{sec-lib-format-pmap}.
From this point, strategy {\tt strat} can only be used as a
distributed graph mapping strategy, to be used by functions {\tt
SCOTCH\_\lbt dgraph\lbt Part}, {\tt SCOTCH\_\lbt dgraph\lbt Map} or
{\tt SCOTCH\_\lbt dgraph\lbt Map\lbt Compute}. This routine must be
called on every process with the same strategy string.

When using the C interface, the array of characters pointed to by
{\tt string} must be null-terminated.

\progret

{\tt SCOTCH\_stratDgraphMap} returns $0$ if the strategy string
has been successfully set, and $1$ else.
\end{itemize}

\subsubsection{{\tt SCOTCH\_stratDgraphMapBuild}}

\begin{itemize}
\progsyn

{\tt\begin{tabular}{l@{}ll}
int SCOTCH\_stratDgraphMapBuild ( & SCOTCH\_Strat *   & straptr, \\
                                  & const SCOTCH\_Num & flagval, \\
                                  & const SCOTCH\_Num & procnbr, \\
                                  & const SCOTCH\_Num & partnbr, \\
                                  & const double      & balrat)
\end{tabular}}

{\tt\begin{tabular}{l@{}ll}
scotchfstratdgraphmapbuild ( & doubleprecision (*) & stradat, \\
                             & integer*{\it num}   & flagval, \\
                             & integer*{\it num}   & procnbr, \\
                             & integer*{\it num}   & partnbr, \\
                             & doubleprecision     & balrat,  \\
                             & integer             & ierr)
\end{tabular}}

\progdes

The {\tt SCOTCH\_stratDgraphMapBuild} routine fills the strategy
structure pointed to by {\tt straptr} with a default mapping strategy
tuned according to the preference flags passed as {\tt flagval} and to
the desired number of parts {\tt partnbr} and imbalance ratio {\tt
balrat}, to be used on {\tt procnbr} processes. From this point, the
strategy structure can only be used as a parallel mapping strategy, to
be used by function {\tt SCOTCH\_\lbt dgraph\lbt Map}, for
instance. See Section~\ref{sec-lib-format-strat-default} for a
description of the available flags.

\progret

{\tt SCOTCH\_stratDgraphMapBuild} returns $0$ if the strategy string
has been successfully set, and $1$ else.
\end{itemize}

\subsubsection{{\tt SCOTCH\_stratDgraphOrder}}

\begin{itemize}
\progsyn

{\tt\begin{tabular}{l@{}ll}
int SCOTCH\_stratDgraphOrder ( & SCOTCH\_Strat * & straptr, \\
                               & const char *    & string)
\end{tabular}}

{\tt\begin{tabular}{l@{}ll}
scotchfstratdgraphorder ( & doubleprecision (*) & stradat, \\
                          & character (*)       & string,  \\
                          & integer             & ierr)
\end{tabular}}

\progdes

The {\tt SCOTCH\_stratDgraphOrder} routine fills the strategy
structure pointed to by {\tt straptr} with the distributed graph
ordering strategy string pointed to by {\tt string}. The format of
this strategy string is described in Section~\ref{sec-lib-format-pord}.
From this point, strategy {\tt strat} can only be used as a
distributed graph ordering strategy, to be used by function {\tt
SCOTCH\_\lbt dgraph\lbt Order\lbt Compute}. This routine must be
called on every process with the same strategy string.

When using the C interface, the array of characters pointed to by
{\tt string} must be null-terminated.

\progret

{\tt SCOTCH\_stratDgraphOrder} returns $0$ if the strategy string
has been successfully set, and $1$ else.
\end{itemize}

\subsubsection{{\tt SCOTCH\_stratDgraphOrderBuild}}

\begin{itemize}
\progsyn

{\tt\begin{tabular}{l@{}ll}
int SCOTCH\_stratDgraphOrderBuild ( & SCOTCH\_Strat *   & straptr, \\
                                    & const SCOTCH\_Num & flagval, \\
                                    & const SCOTCH\_Num & procnbr, \\
                                    & const SCOTCH\_Num & levlnbr, \\
                                    & const double      & balrat)
\end{tabular}}

{\tt\begin{tabular}{l@{}ll}
scotchfstratdgraphorderbuild ( & doubleprecision (*) & stradat, \\
                               & integer*{\it num}   & flagval, \\
                               & integer*{\it num}   & procnbr, \\
                               & integer*{\it num}   & levlnbr, \\
                               & doubleprecision     & balrat,  \\
                               & integer             & ierr)
\end{tabular}}

\progdes

The {\tt SCOTCH\_stratDgraphOrderBuild} routine fills the strategy
structure pointed to by {\tt straptr} with a default parallel ordering
strategy tuned according to the preference flags passed as {\tt
flagval} and to the desired nested dissection imbalance ratio {\tt
balrat}, to be used on {\tt procnbr} processes. From this point, the
strategy structure can only be used as a parallel ordering strategy, to
be used by function {\tt SCOTCH\_\lbt dgraph\lbt Order}, for
instance.

See Section~\ref{sec-lib-format-strat-default} for a description of
the available flags. When any of the {\tt SCOTCH\_\lbt STRAT\lbt
LEVEL\lbt MIN} or {\tt SCOTCH\_\lbt STRAT\lbt LEVEL\lbt MAX} flags is
set, the {\tt levlnbr} parameter is taken into account.

\progret

{\tt SCOTCH\_stratDgraphOrderBuild} returns $0$ if the strategy string
has been successfully set, and $1$ else.
\end{itemize}

\subsection{Other data structure routines}
\label{sec-lib-other}

\subsubsection{{\tt SCOTCH\_dmapAlloc}}

\begin{itemize}
\progsyn

{\tt\begin{tabular}{l@{}l}
SCOTCH\_Dmapping * SCOTCH\_dmapAlloc ( & void)
\end{tabular}}

\progdes

The {\tt SCOTCH\_dmapAlloc} function allocates a memory area of a
size sufficient to store a {\tt SCOTCH\_\lbt Dmapping} structure. It is
the user's responsibility to free this memory when it is no longer
needed.

\progret

{\tt SCOTCH\_dmapAlloc} returns the pointer to the memory area if it
has been successfully allocated, and {\tt NULL} else.
\end{itemize}

\subsubsection{{\tt SCOTCH\_dorderAlloc}}

\begin{itemize}
\progsyn

{\tt\begin{tabular}{l@{}l}
SCOTCH\_Dordering * SCOTCH\_dorderAlloc ( & void)
\end{tabular}}

\progdes

The {\tt SCOTCH\_dorderAlloc} function allocates a memory area of a
size sufficient to store a {\tt SCOTCH\_\lbt Dordering} structure. It is
the user's responsibility to free this memory when it is no longer
needed.

\progret

{\tt SCOTCH\_dorderAlloc} returns the pointer to the memory area if it
has been successfully allocated, and {\tt NULL} else.
\end{itemize}

\subsection{Error handling routines}
\label{sec-lib-error}

The handling of errors that occur within library routines is
often difficult, because library routines should be able to issue
error messages that help the application programmer to find the
error, while being compatible with the way the application handles
its own errors.

To match these two requirements, all the error and warning messages
produced by the routines of the \libscotch\ library are issued using
the user-definable variable-length argument routines {\tt SCOTCH\_\lbt
error\lbt Print} and {\tt SCOTCH\_\lbt error\lbt PrintW}. Thus, one
can redirect these error messages to his own error handling routines,
and can choose if he wants his program to terminate on error or to
resume execution after the erroneous function has returned.

In order to free the user from the burden of writing a basic error
handler from scratch, the {\tt libptscotcherr.a} library provides error
routines that print error messages on the standard error stream
{\tt stderr} and return control to the application. Application
programmers who want to take advantage of them have to add
{\tt -lptscotcherr} to the list of arguments of the linker, after
the {\tt -lptscotch} argument.

\subsubsection{{\tt SCOTCH\_errorPrint}}
\label{sec-lib-errorprint}

\begin{itemize}
\progsyn

{\tt\begin{tabular}{l@{}ll}
void SCOTCH\_errorPrint ( & const char * const & errstr, ... )
\end{tabular}}

\progdes

The {\tt SCOTCH\_errorPrint} function is designed to output a
variable-length argument error string to some stream.
\end{itemize}

\subsubsection{{\tt SCOTCH\_errorPrintW}}

\begin{itemize}
\progsyn

{\tt\begin{tabular}{l@{}ll}
void SCOTCH\_errorPrintW ( & const char * const & errstr, ...)
\end{tabular}}

\progdes

The {\tt SCOTCH\_errorPrintW} function is designed to output a
variable-length argument warning string to some stream.
\end{itemize}

\subsubsection{{\tt SCOTCH\_errorProg}}

\begin{itemize}
\progsyn

{\tt\begin{tabular}{l@{}ll}
void SCOTCH\_errorProg ( & const char * & progstr)
\end{tabular}}

\progdes

The {\tt SCOTCH\_errorProg} function is designed to be called at
the beginning of a program or of a portion of code to identify the
place where subsequent errors take place.
This routine is not reentrant, as it is only a minor help function. It
is defined in {\tt lib\lbt scotch\lbt err.a} and is used by the
standalone programs of the \scotch\ distribution.
\end{itemize}

\subsection{Miscellaneous routines}
\label{sec-lib-misc}

\subsubsection{{\tt SCOTCH\_memCur}}

\begin{itemize}
\progsyn

{\tt\begin{tabular}{l@{}l}
SCOTCH\_Idx SCOTCH\_memCur ( & void)
\end{tabular}}

{\tt\begin{tabular}{l@{}ll}
scotchfmemcur ( & integer*{\it idx} & memcur) \\

\end{tabular}}

\progdes

When \scotch\ is compiled with the {\tt COMMON\_\lbt MEMORY\_\lbt
TRACE} flag set, the {\tt SCOTCH\_memCur} routine returns the amount
of memory, in bytes, that is currently allocated by \scotch\ on the
current processing element, either by itself or on the behalf of the
user. Else, the routine returns {\tt -1}.

The returned figure does not account for the memory that has been
allocated by the user and made visible to \scotch\ by means of
routines such as {\tt SCOTCH\_\lbt dgraph\lbt Build} calls. This
memory is not under the control of \scotch, and it is the user's
responsibility to free it after calling the relevant
{\tt SCOTCH\_\lbt *\lbt Exit} routines.

Some third-party software used by \scotch, such as the strategy string
parser, may allocate some memory for internal use and never free it.
Consequently, there may be small discrepancies between memory
occupation figures returned by \scotch\ and those returned by
third-party tools. However, these discrepancies should not exceed a
few kilobytes.

While memory occupation is internally recorded in a variable of type
{\tt intptr\_\lbt t}, it is output as a {\tt SCOTCH\_\lbt Idx} for the
sake of interface homogeneity, especially for Fortran. It is therefore
the installer's responsibility to make sure that the support integer
type of {\tt SCOTCH\_\lbt Idx} is large enough to not overflow. See
section~\ref{sec-lib-inttypesize} for more information.
\end{itemize}

\subsubsection{{\tt SCOTCH\_memMax}}

\begin{itemize}
\progsyn

{\tt\begin{tabular}{l@{}l}
SCOTCH\_Idx SCOTCH\_memMax ( & void)
\end{tabular}}

{\tt\begin{tabular}{l@{}ll}
scotchfmemmax ( & integer*{\it idx} & memcur) \\

\end{tabular}}

\progdes

When \scotch\ is compiled with the {\tt COMMON\_\lbt MEMORY\_\lbt
TRACE} flag set, the {\tt SCOTCH\_memMax} routine returns the maximum
amount of memory, in bytes, ever allocated by \scotch\ on the current
processing element, either by itself or on the behalf of the
user. Else, the routine returns {\tt -1}.

The returned figure does not account for the memory that has been
allocated by the user and made visible to \scotch\ by means of
routines such as {\tt SCOTCH\_\lbt dgraph\lbt Build} calls. This
memory is not under the control of \scotch, and it is the user's
responsibility to free it after calling the relevant
{\tt SCOTCH\_\lbt *\lbt Exit} routines.

Some third-party software used by \scotch, such as the strategy string
parser, may allocate some memory for internal use and never free it.
Consequently, there may be small discrepancies between memory
occupation figures returned by \scotch\ and those returned by
third-party tools. However, these discrepancies should not exceed a
few kilobytes.

While memory occupation is internally recorded in a variable of type
{\tt intptr\_\lbt t}, it is output as a {\tt SCOTCH\_\lbt Idx} for the
sake of interface homogeneity, especially for Fortran. It is therefore
the installer's responsibility to make sure that the support integer
type of {\tt SCOTCH\_\lbt Idx} is large enough to not overflow. See
section~\ref{sec-lib-inttypesize} for more information.
\end{itemize}

\subsubsection{{\tt SCOTCH\_randomProc}}

\begin{itemize}
\progsyn

{\tt\begin{tabular}{l@{}ll}
void SCOTCH\_randomProc ( & int & procnum)
\end{tabular}}

{\tt\begin{tabular}{l@{}ll}
scotchfrandomproc ( & integer & procnum )
\end{tabular}}

\progdes

The {\tt SCOTCH\_randomProc} routine records internally the provided
number, which contributes to the initialization of the pseudo-random
generator. Hence, providing different values for each process,
e.g. process rank, will result in different random seeds across
processors. This allows processes to compute concurrently different
initial partitions, in the course of the parallel multilevel framework
with folding and duplication.

In order for the provided number to be taken into account,
{\tt SCOTCH\_\lbt random\lbt Proc} must be called before any other
routine of the \libscotch\ that is likely to use (and consequently
to initialize) the pseudo-random number generator. By default, it is
set to zero.
\end{itemize}

\subsubsection{{\tt SCOTCH\_randomReset}}

\begin{itemize}
\progsyn

{\tt\begin{tabular}{l@{}l}
void SCOTCH\_randomReset ( & void)
\end{tabular}}

{\tt\begin{tabular}{l@{}l}
scotchfrandomreset ( & )
\end{tabular}}

\progdes

The {\tt SCOTCH\_randomReset} routine resets the seed of the
pseudo-random generator used by the graph partitioning routines
of the \libscotch\ library. Two consecutive calls to
the same \libscotch\ partitioning or ordering routines, separated
by a call to {\tt SCOTCH\_\lbt random\lbt Reset}, will always yield
the same results.
\end{itemize}

\subsubsection{{\tt SCOTCH\_randomSeed}}

\begin{itemize}
\progsyn

{\tt\begin{tabular}{l@{}ll}
void SCOTCH\_randomSeed ( & SCOTCH\_Num & seedval)
\end{tabular}}

{\tt\begin{tabular}{l@{}ll}
scotchfrandomseed ( & integer*{\it num} & seedval )
\end{tabular}}

\progdes

The {\tt SCOTCH\_randomSeed} routine sets to {\tt seedval} the
seed of the pseudo-random generator used internally by several
algorithms of \scotch. All subsequent calls to {\tt SCOTCH\_\lbt
random\lbt Reset} will use this value to reset the pseudo-random
generator.

This routine needs only to be used by users willing to evaluate
the robustness and quality of partitioning algorithms with respect to
the variability of random seeds. Else, depending whether \scotch\ has
been compiled with any of the flags {\tt COMMON\_\lbt RANDOM\_\lbt
FIXED\_\lbt SEED} or {\tt SCOTCH\_\lbt DETERMINISTIC} set or not,
either the same pseudo-random seed will be always used, or a
process-dependent seed will be used, respectively.
\end{itemize}

\subsection{\parmetis\ compatibility library}
\label{sec-lib-parmetis}

The \parmetis\ compatibility library provides stubs which redirect some
calls to \parmetis\ routines to the corresponding \ptscotch\ counterparts.
In order to use this feature, the only thing to do is to re-link the
existing software with the {\tt lib\lbo ptscotch\lbo parmetis} library, and
eventually with the original \parmetis\ library if the software uses
\parmetis\ routines which do not need to have \ptscotch\ equivalents, such
as graph transformation routines.
In that latter case, the ``{\tt -lptscotch\lbt parmetis}'' argument must be
placed before the ``{\tt -lparmetis}'' one (and of course before the
``{\tt -lptscotch}'' one too), so that routines that are redefined by
\ptscotch\ are chosen instead of their \parmetis\ counterpart. Routines
of \parmetis\ which are not redefined by \ptscotch\ may also require
that the sequential \metis\ library be linked too. When no other
\parmetis\ routines than the ones redefined by \ptscotch\ are used,
the ``{\tt -lparmetis}'' argument can be omitted. See
Section~\ref{sec-examples} for an example.

\subsubsection{{\tt ParMETIS\_V3\_NodeND}}

\begin{itemize}
\progsyn

{\tt\begin{tabular}{l@{}ll}
void ParMETIS\_V3\_NodeND ( & const SCOTCH\_Num * const & vtxdist, \\
                            & const SCOTCH\_Num * const & xadj, \\
                            & const SCOTCH\_Num * const & adjncy, \\
                            & const SCOTCH\_Num * const & numflag, \\
                            & const SCOTCH\_Num * const & options, \\
                            & SCOTCH\_Num * const       & order, \\
                            & SCOTCH\_Num * const       & sizes, \\
                            & MPI\_Comm *               & comm)
\end{tabular}}

{\tt\begin{tabular}{l@{}ll}
parmetis\_v3\_nodend ( & integer*{\it num} (*) & vtxdist, \\
                       & integer*{\it num} (*) & xadj, \\
                       & integer*{\it num} (*) & adjncy, \\
                       & integer*{\it num}     & numflag, \\
                       & integer*{\it num} (*) & options, \\
                       & integer*{\it num} (*) & order, \\
                       & integer*{\it num} (*) & sizes, \\
                       & integer               & comm)
\end{tabular}}

\progdes

The {\tt ParMETIS\_V3\_NodeND} function performs a nested dissection
ordering of the distributed graph passed as arrays {\tt vtxdist},
{\tt xadj} and {\tt adjncy}, using the default \ptscotch\ ordering
strategy. Unlike for \parmetis, this routine will compute an ordering
even when the number of processors on which it is run is not a power
of two. The {\tt options} array is not used. When the number of
processors is a power of two, the contents of the {\tt sizes} array is
equivalent to the one returned by the original {\tt ParMETIS\_V3\_NodeND}
routine, else it is filled with $-1$ values.

Users willing to get the tree structure of orderings computed on
numbers of processors which are not power of two should use the native
\ptscotch\ ordering routine, and extract the relevant information from
the distributed ordering with the {\tt SCOTCH\_\lbt dgraph\lbt
Order\lbt Cblk\lbt Dist} and {\tt SCOTCH\_\lbt dgraph\lbt Order\lbt
Tree\lbt Dist} routines.

Similarly, as there is no {\tt ParMETIS\_V3\_NodeWND} routine in
\parmetis, users willing to order distributed graphs with node weights
should directly call the \ptscotch\ routines.
\end{itemize}

\subsubsection{{\tt ParMETIS\_V3\_PartGeomKway}}

\begin{itemize}
\progsyn

{\tt\begin{tabular}{l@{}ll}
void ParMETIS\_V3\_PartGeomKway ( & const SCOTCH\_Num * const & vtxdist, \\
                                  & const SCOTCH\_Num * const & xadj, \\
                                  & const SCOTCH\_Num * const & adjncy, \\
                                  & const SCOTCH\_Num * const & vwgt, \\
                                  & const SCOTCH\_Num * const & adjwgt, \\
                                  & const SCOTCH\_Num * const & wgtflag, \\
                                  & const SCOTCH\_Num * const & numflag, \\
                                  & const SCOTCH\_Num * const & ndims, \\
                                  & const float * const       & xyz, \\
                                  & const SCOTCH\_Num * const & ncon, \\
                                  & const SCOTCH\_Num * const & nparts, \\
                                  & const float * const       & tpwgts, \\
                                  & const float * const       & ubvec, \\
                                  & const SCOTCH\_Num * const & options, \\
                                  & SCOTCH\_Num * const       & edgecut, \\
                                  & SCOTCH\_Num * const       & part, \\
                                  & MPI\_Comm *               & comm)
\end{tabular}}

{\tt\begin{tabular}{l@{}ll}
parmetis\_v3\_partgeomkway ( & integer*{\it num} (*) & vtxdist, \\
                             & integer*{\it num} (*) & xadj, \\
                             & integer*{\it num} (*) & adjncy, \\
                             & integer*{\it num} (*) & vwgt, \\
                             & integer*{\it num} (*) & adjwgt, \\
                             & integer*{\it num}     & wgtflag, \\
                             & integer*{\it num}     & numflag, \\
                             & integer*{\it num}     & ndims, \\
                             & float (*)             & xyz, \\
                             & integer*{\it num}     & ncon, \\
                             & integer*{\it num}     & nparts, \\
                             & float (*)             & tpwgts, \\
                             & float (*)             & ubvec, \\
                             & integer*{\it num} (*) & options, \\
                             & integer*{\it num}     & edgecut, \\
                             & integer*{\it num} (*) & part, \\
                             & integer               & comm)
\end{tabular}}

\progdes

The {\tt ParMETIS\_V3\_PartGeomKway} function computes a partition into
{\tt nparts} parts of the distributed graph passed as arrays {\tt
vtxdist}, {\tt xadj} and {\tt adjncy}, using the default
\ptscotch\ mapping strategy. The partition is returned in the form of
the distributed vector {\tt part}, which holds the indices of the
parts to which every vertex belongs, from $0$ to $(\mbox{\tt nparts} -
1)$.

Since \scotch\ does not handle geometry, the {\tt ndims} and {\tt xyz}
arrays are not used, and this routine directly calls the
{\tt ParMETIS\_\lbt V3\_\lbt Part\lbt Kway} stub.
\end{itemize}

\subsubsection{{\tt ParMETIS\_V3\_PartKway}}

\begin{itemize}
\progsyn

{\tt\begin{tabular}{l@{}ll}
void ParMETIS\_V3\_PartKway ( & const SCOTCH\_Num * const & vtxdist, \\
                              & const SCOTCH\_Num * const & xadj, \\
                              & const SCOTCH\_Num * const & adjncy, \\
                              & const SCOTCH\_Num * const & vwgt, \\
                              & const SCOTCH\_Num * const & adjwgt, \\
                              & const SCOTCH\_Num * const & wgtflag, \\
                              & const SCOTCH\_Num * const & numflag, \\
                              & const SCOTCH\_Num * const & ncon, \\
                              & const SCOTCH\_Num * const & nparts, \\
                              & const float * const       & tpwgts, \\
                              & const float * const       & ubvec, \\
                              & const SCOTCH\_Num * const & options, \\
                              & SCOTCH\_Num * const       & edgecut, \\
                              & SCOTCH\_Num * const       & part, \\
                              & MPI\_Comm *               & comm)
\end{tabular}}

{\tt\begin{tabular}{l@{}ll}
parmetis\_v3\_partkway ( & integer*{\it num} (*) & vtxdist, \\
                         & integer*{\it num} (*) & xadj, \\
                         & integer*{\it num} (*) & adjncy, \\
                         & integer*{\it num} (*) & vwgt, \\
                         & integer*{\it num} (*) & adjwgt, \\
                         & integer*{\it num}     & wgtflag, \\
                         & integer*{\it num}     & numflag, \\
                         & integer*{\it num}     & ncon, \\
                         & integer*{\it num}     & nparts, \\
                         & float (*)             & tpwgts, \\
                         & float (*)             & ubvec, \\
                         & integer*{\it num} (*) & options, \\
                         & integer*{\it num}     & edgecut, \\
                         & integer*{\it num} (*) & part, \\
                         & integer               & comm)
\end{tabular}}

\progdes

The {\tt ParMETIS\_V3\_PartKway} function computes a partition into
{\tt nparts} parts of the distributed graph passed as arrays {\tt
vtxdist}, {\tt xadj} and {\tt adjncy}, using the default
\ptscotch\ mapping strategy. The partition is returned in the form of
the distributed vector {\tt part}, which holds the indices of the
parts to which every vertex belongs, from $0$ to $(\mbox{\tt nparts} -
1)$.

Since \scotch\ does not handle multiple constraints, only the first
constraint is taken into account to define the respective weights of
the parts. Consequently, only the first {\tt nparts} cells of the
{\tt tpwgts} array are considered. The {\tt ncon}, {\tt ubvec}
and {\tt options} parameters are not used.
\end{itemize}
                                  % Bibliotheque
%%%%%%%%%%%%%%%%%%%%%%%%%%%%%%%%%%%%
%                                  %
% Titre  : p_d.tex                 %
% Sujet  : Manuel de l'utilisateur %
%          du projet 'PT-Scotch'   %
%          Distribution programmes %
% Auteur : Francois Pellegrini     %
%                                  %
%%%%%%%%%%%%%%%%%%%%%%%%%%%%%%%%%%%%

\section{Installation}
\label{sec-install}

Version {\sc \scotchver} of the \scotch\ software package, which
contains the \ptscotch\ routines, is distributed as free/libre
software under the CeCILL-C free/libre software license~\cite{cecill},
which is very similar to the GNU LGPL license. Therefore, it is not
distributed as a set of binaries, but instead in the form of a source
distribution, which can be downloaded from the \scotch\ web page at
\url{http://www.labri.fr/~pelegrin/scotch/}~.
\\

All \scotch\ users are welcome to send an e-mail to the author so that
they can be added to the \scotch\ mailing list, and be automatically
informed of new releases and publications.
\\

The extraction process will create a {\tt scotch\_\scotchversub}
directory, containing several subdirectories and files. Please refer
to the files called {\tt LICENSE\_\lbt EN.txt} or
{\tt LICENCE\_\lbt FR.txt}, as well as file
{\tt INSTALL\_\lbt EN.txt}, to see under which conditions your
distribution of \scotch\ is licensed and how to install it.

\subsection{Thread issues}

To enable the use of POSIX threads in some routines, the {\tt
SCOTCH\_\lbt PTHREAD} flag must be set. If your MPI implementation is
not thread-safe, make sure this flag is not defined at compile time.
If the flag is defined, make sure to use the \texttt{MPI\_\lbt
Init\_\lbt thread} MPI routine to initialize the communication
subsystem, at the \texttt{MPI\_\lbt THREAD\_\lbt MULTIPLE} level
(see Section~\ref{sec-lib-thread}).

\subsection{File compression issues}

To enable on-the-fly compression and decompression of various formats,
the relevant flags must be defined. These flags are {\tt COMMON\_\lbt
FILE\_\lbt COMPRESS\_\lbt BZ2} for {\tt bzip2} (de)compression, {\tt
COMMON\_\lbt FILE\_\lbt COMPRESS\_\lbt GZ} for {\tt gzip}
(de)compression, and {\tt COMMON\_\lbt FILE\_\lbt COMPRESS\_\lbt LZMA}
for {\tt lzma} decompression. Note that the corresponding
development libraries must be installed on your system before compile
time, and that compressed file handling can take place only on systems
which support multi-threading or multi-processing. In the first case,
you must set the {\tt SCOTCH\_\lbt PTHREAD} flag in order to take
advantage of these features.

On Linux systems, the development libraries to install are {\tt
libbzip2\_1-\lbt devel} for the {\tt bzip2} format, {\tt zlib1-\lbt
devel} for the {\tt gzip} format, and {\tt liblzma0-\lbt devel} for
the {\tt lzma} format. The names of the libraries may vary according
to operating systems and library versions. Ask your system engineer in
case of trouble.

\subsection{Machine word size issues}
\label{sec-install-inttypesize}

The integer values handled by \scotch\ are based on the
{\tt SCOTCH\_\lbt Num} type, which equates by default to the {\tt int}
C type, corresponding to the {\tt INTEGER} Fortran type, both of which
being of machine word size. To coerce the length of the
{\tt SCOTCH\_\lbt Num} integer type to 32 or 64 bits, one can use the
``{\tt -DINTSIZE32}'' or ``{\tt -DINTSIZE64}'' flags, respectively, or
else use the ``{\tt -DINT=}'' definition, at compile time. For
instance, adding ``{\tt -DINT=long}'' to the {\tt CFLAGS} variable in
the {\tt Makefile.inc} file to be placed at the root of the source
tree will make all {\tt SCOTCH\_\lbt Num} integers become {\tt long} C
integers.

Whenever doing so, make sure to use integer types of equivalent length
to declare variables passed to \scotch\ routines from caller C and
Fortran procedures. Also, because of API conflicts, the
\metis\ compatibility library will not be usable. It is usually safer
and cleaner to tune your C and Fortran compilers to make them
interpret {\tt int} and {\tt INTEGER} types as 32 or 64 bit values,
than to use the aforementioned flags and coerce type lengths in your
own code.

Fortran users also have to take care of another size issue: since
there are no pointers in Fortran~77, the Fortran interface of some
routines converts pointers to be returned into integer indices with
respect to a given array (e.g. see
Section~\ref{sec-lib-func-scotchdgraphdata}).
For 32\_64 architectures, such indices can be larger than the size of
a regular {\tt INTEGER}. This is why the indices to be returned are
defined by means of a specific integer type, {\tt SCOTCH\_Idx}. To
coerce the length of this index type to 32 or 64 bits, one can use the
``{\tt -DIDXSIZE32}'' or ``{\tt -DIDXSIZE64}'' flags, respectively, or
else use the ``{\tt -DIDX=}'' definition, at compile time. For
instance, adding ``{\tt -DIDX="long~long"}'' to the {\tt CFLAGS}
variable in the {\tt Makefile.inc} file to be placed at the root of
the source tree will equate all {\tt SCOTCH\_\lbt Idx} integers to C
{\tt long long} integers. By default, when the size of
{\tt SCOTCH\_\lbt Idx} is not explicitly defined, it is assumed to be
the same as the size of {\tt SCOTCH\_\lbt Num}.
                                  % Distribution
%%%%%%%%%%%%%%%%%%%%%%%%%%%%%%%%%%%%
%                                  %
% Titre  : p_e.tex                 %
% Sujet  : Manuel de l'utilisateur %
%          du projet 'PT-Scotch'   %
%          Exemples d'utilisation  %
% Auteur : Francois Pellegrini     %
%                                  %
%%%%%%%%%%%%%%%%%%%%%%%%%%%%%%%%%%%%

\section{Examples}
\label{sec-examples}

This section contains chosen examples destined to show how the programs
of the \ptscotch\ project interoperate and can be combined.
It is assumed that parallel programs are launched by means of the
{\tt mpirun} command, which comprises a {\tt -np} option to set the
number of processes on which to run them.
Character ``{\tt\bf \%}'' in bold represents the shell prompt.
\begin{itemize}
\item
Create a distributed source graph file of $7$ fragments from the
centralized source graph file {\tt brol.grf} stored in the current
directory of process $0$ of the MPI environment, and stores the
resulting fragments in files labeled with the proper number of
processors and processor ranks.
\\

\noi
{\tt
{\bf\%} mpirun -np 7 dgscat brol.grf brol-\%p-\%r.dgr
}

\item
Compute on $3$ processors the ordering of graph {\tt brol.grf}, to be
saved in a file called {\tt brol.ord} written by process $0$ of the
MPI environment.
\\

\noi
{\tt
{\bf\%} mpirun -np 7 dgord brol.grf brol.ord
}

\item
Compute on $4$ processors the first three levels of nested dissection
of graph {\tt brol.grf}, and create an {\sc Open Inventor} file called
{\tt brol.iv} to show the resulting separators and leaves.
\\

\noi
{\tt
{\bf\%} mpirun -np 4 dgord brol.grf /dev/null '-On\{sep=\lbt /(levl\lbt <\lbt 3)\lbt ?\lbt m\{\lbt asc=\lbt b\{strat=\lbt q\{\lbt strat=\lbt f\}\},\lbt low=\lbt q\{\lbt strat=\lbt h\},\lbt seq=\lbt q\{\lbt strat=\lbt m\{low=\lbt h,asc=\lbt b\{\lbt strat=\lbt f\}\}\}\};,\lbt ole=\lbt s,\lbt ose=\lbt s,\lbt osq=\lbt n\{\lbt sep=\lbt /(levl\lbt <\lbt 3)\lbt ?\lbt m\{asc=\lbt b\{\lbt strat=\lbt f\},\lbt low=\lbt h\};\}\}' -mbrol.map
\\
{\bf\%} gout brol.grf brol.xyz brol.map brol.iv
}
\item
Compute on $4$ processors an ordering of the compressed graph {\tt
brol.\lbt grf.\lbt gz}, and output the resulting ordering on
compressed form.
\\

\noi
{\tt
{\bf\%} mpirun -np 4 dgord brol.grf.gz brol.ord.gz
}
\item
Recompile a program that used \parmetis\ so that it uses \ptscotch\ instead.
\\

\noi
{\tt
{\bf\%} mpicc brol.c -o brol -I\$\{parmetisdir\} -lptscotchparmetis -lptscotch -lptscotcherr -lparmetis -lmetis -lm}
\spa

\noi
Note that the ``{\tt -lptscotch\lbt parmetis}'' option must be placed before the
``{\tt -lparmetis}'' one, so that routines that are redefined by \ptscotch\ are
selected instead of their \parmetis\ counterpart. When no other
\parmetis\ routines than the ones redefined by \ptscotch\ are used, the
``{\tt -lparmetis -lmetis}'' options can be omitted. The ``{\tt -I\$\{parmetisdir\}}
option may be necessary to provide the path to the original {\tt parmetis.h}
include file, which contains the prototypes of all of the \parmetis\ routines.

\end{itemize}
                                  % Relevant examples
%%%%%%%%%%%%%%%%%%%%%%%%%%%%%%%%%%%%%
%                                  %
% Titre  : s_n.tex                 %
% Sujet  : Manuel de l'utilisateur %
%          du projet 'Scotch'      %
%          Codage de nouvelles     %
%          methodes                %
% Auteur : Francois Pellegrini     %
%                                  %
%%%%%%%%%%%%%%%%%%%%%%%%%%%%%%%%%%%%

\section{Adding new features to \scotch}
\label{sec-coding}

Since \scotch\ is libre/free software, users have the ability
to add new features to it. Moreover, as \scotch\ is intended to be a
testbed for new partitioning and ordering algorithms, it has been
developed in a very modular way, to ease the development and inclusion
of new partitioning and ordering methods to be called within
\scotch\ strategies.

All of the source code for partitioning and ordering methods for
graphs and meshes is located in the {\tt src/\lbt libscotch/} source
subdirectory. Source file names have a very regular pattern, based on
the internal data structures they handle.

\subsection{Graphs and meshes}

The basic structures in \scotch\ are the {\tt Graph} and {\tt Mesh}
structures, which model a simple symmetric graph the definition of
which is given in file {\tt graph.h}, and a simple mesh, in the form
of a bipartite graph, the definition of which is given in file {\tt
mesh.h}, respectively. From this structure are derived enriched graph
and mesh structures:
\begin{itemize}
\item
{\tt Bgraph}, in file {\tt bgraph.h}: graph with bipartition, that is,
edge separation, information attached to it;
\item
{\tt Kgraph}, in file {\tt kgraph.h}: graph with mapping information
attached to it;
\item
{\tt Hgraph}, in file {\tt hgraph.h}: graph with halo information
attached to it, for computing graph orderings;
\item
{\tt Vgraph}, in file {\tt vgraph.h}: graph with vertex bipartition
information attached to it;
\item
{\tt Hmesh}, in file {\tt hmesh.h}: mesh with halo information
attached to it, for computing graph orderings;
\item
{\tt Vmesh}, in file {\tt vmesh.h}: graph with vertex bipartition
information attached to it.
\end{itemize}
As version {\sc 4.0} of the \libscotch\ does not provide mesh
mapping capabilities, neither {\tt Bmesh} nor {\tt Kmesh} structures
have been defined to date, but they well may be in the future.

All of the structures are in fact defined as {\tt typedef}ed types.

\subsection{Methods}

\subsection{Adding a new method to \scotch}

We will assume in this section that the new method to add is a graph
separation method. The procedure explained below is exactly the same
for graph bipartitioning, graph mapping, graph ordering, mesh
separation, or mesh ordering methods.

Please proceed as explained below.
\begin{enumerate}
\item
Write the code of the method itself. First, choose a free two-letter
code to describe your method, say ``xy''. In the {\tt libscotch}
source directory, create files {\tt vgraph\_\lbt separate\_\lbt xy.c}
and {\tt vgraph\_\lbt separate\_\lbt xy.h}, basing on existing
files such as {\tt vgraph\_\lbt separate\_\lbt gg.c} and {\tt
vgraph\_\lbt separate\_\lbt gg.h}, for instance.

If the method is complex, it can be split across several other files,
which will be named {\tt vgraph\_\lbt separate\_\lbt xy\_\lbt first\lbt
module\lbt name.c}, {\tt vgraph\_\lbt separate\_\lbt xy\_\lbt second\lbt
module\lbt name.c}, eventually with matching header files.

If the method has parameters, create a structure called {\tt
Vgraph\lbt Separate\lbt Xy\lbt Param}, which contains types that
are handled by the strategy parser, such as {\tt INT} and
{\tt double}.

The execution of your method should result in the setting or in the
updating of the {\tt Vgraph} structure that is passed to it. See its
definition in {\tt vgraph.h} and read several simple graph separation
methods, such as {\tt vgraph\_\lbt separate\_\lbt zr.c}, to figure out
what all of its parameters mean.

At the end of your method, always call, when the {\tt SCOTCH\_\lbt
DEBUG\_\lbt VGRAPH2} debug flag is set, the {\tt vgraph\lbt Check}
routine, to avoid the spreading of eventual bugs to other parts of
the \libscotch\ library.
\item
Add the method to the parser tables. The files to update are
{\tt vgraph\_\lbt separate\_\lbt st.c} and {\tt vgraph\_\lbt
separate\_\lbt st.h}, where ``{\tt st}'' stands for ``strategy''.

First, edit {\tt vgraph\_\lbt separate\_\lbt st.h}. In the {\tt
Vgraph\lbt Separate\lbt St\lbt Method\lbt Type} enumeration,
add a line for your new method {\tt VGRAPH\lbt SEPA\lbt ST\lbt
METH\lbt XY}. Then, edit {\tt vgraph\_\lbt separate\_\lbt st.c},
where all of the remaining actions take place.

In the top of the file, add a {\tt \#include} directive to include
{\tt vgraph\_\lbt separate\_\lbt xy.h}.

If the method has parameters, create a {\tt vgraph\lbt separate\lbt
default\lbt xy} C union, basing on an existing one, and fill it with
the default values of your method parameters.

In the {\tt vgraph\lbt separate\lbt st\lbt meth\lbt tab} method array,
add a line for the new method. To do so, choose a free single-letter
code that will be used to designate the new method in strategy strings.
If the method has parameters, the last field should be a pointer to
the default structure, else it should be set to {\tt NULL}.

If the method has parameters, update the {\tt vgraph\lbt separate\lbt
st\lbt para\lbt tab} parameter array. Add one data block per
parameter. The first field is the name of the method to which the
parameter applies, that is, {\tt VGRAPH\lbt SEPA\lbt ST\lbt
METH\lbt XY}. The second field is the type of the parameter, which can
be:
\begin{itemize}
\item
{\tt STRATPARAMCASE}: the support type is an {\tt int}. It receives
the index in the case string, given as last field of the parameter
line, of the selected case character code;
\item
{\tt STRATPARAMDOUBLE}: the support type is a {\tt double} value;
\item
{\tt STRATPARAMINT}: the support type is an {\tt INT}, which
is the generic integer type handled internally by \scotch. This type
has variable extent, depending on compilation flags,
as described in Section~\ref{sec-lib-inttypesize};
\item
{\tt STRATPARAMSTRING}: a (small) character string.
\item
{\tt STRATPARAMSTRAT}: strategy. For instance, the graph ordering
method by nested dissection takes a vertex partitioning strategy as
one of its parameters, to compute the vertex separators.
\end{itemize}
The fourth and fifth fields are the address of the location of the
default structure and the address of the parameter within this default
structure, respectively. From these two values can be computed at run
time the offset of the parameter within any instance of the parameter
structure, which is used to fill the actual structures in the parsed
strategy evaluation tree.
The value of the sixth parameter depends on the type of the
parameter. It should be {\tt NULL} for {\tt STRAT\lbt PARAM\lbt
DOUBLE} and {\tt STRAT\lbt PARAM\lbt INT} parameters, points to the
string of available case letters for {\tt STRAT\lbt PARAM\lbt CASE}
parameters, points to the target string buffer for {\tt STRAT\lbt
PARAM\lbt STRING} parameters, and points to the relevant method
parsing table for for {\tt STRAT\lbt PARAM\lbt STRAT} parameters.
\item
Edit the makefile of the \libscotch\ source directory to enable the
compilation and linking of the method. Depending on \libscotch\
versions, this makefile is either called {\tt Makefile} or {\tt
make\_\lbt gen}.
\item
Compile in debug mode and experiment with your routine, by creating
strategies that contain its single-letter code.
\item
To change the default strategy string used by the \libscotch\ library,
update file {\tt library\_\lbt graph\_\lbt order.c}, since it is
the graph ordering routine which makes use of graph vertex separation
methods to compute separators for the nested dissection ordering method.
\end{enumerate}

\subsection{Licensing of new methods and of derived works}

According to the terms of the GNU Lesser General Public License
(LGPL)~\cite{lgpl}, under which the \scotch\ software package is
distributed, the works that are carried out to improve and
extend the \libscotch\ library must be licensed under the same
terms. Basically, it means that you will have to distribute the
sources of your new methods, along with the sources of \scotch, to any
recipient of your modified version of the \libscotch, and that you
grant these recipients the same rights of update and redistribution as
the ones that are given to you under the terms of the LGPL. Please
read it carefully to know what you can do and cannot do with the
\scotch\ distribution.
\\

You should have received a copy of the GNU Lesser General Public
License along with the \scotch\ distribution; if not, write to the
Free Software Foundation, Inc., 59 Temple Place, Suite 330, Boston,
MA 02111-1307, USA.
                                  % Addition of a new method

%% Remerciements.

\section*{Credits}

I wish to thank all of the following people:
\begin{itemize}
\item
C\'edric Chevalier, during his PhD at LaBRI, did research on efficient
parallel matching algorithms and coded the parallel multi-level
algorithm of \ptscotch. He also studied parallel genetic refinement
algorithms. Many thanks to him for the great job!
\item
Yves Secretan contributed to the MinGW32 port.
\end{itemize}

%% Bibliographie.

\bibliographystyle{plain}
\bibliography{p}

\end{document}
