%%%%%%%%%%%%%%%%%%%%%%%%%%%%%%%%%%%%
%                                  %
% Titre  : s.tex                   %
% Sujet  : Manuel de l'utilisateur %
%          de Scotch 6.0           %
%          Corps du document       %
% Auteur : Francois Pellegrini     %
%                                  %
%%%%%%%%%%%%%%%%%%%%%%%%%%%%%%%%%%%%

%% Formatage et pagination.

% pdflatex -sPAPERSIZE=a4 s.tex
% dvips -sPAPERSIZE=a4 s.dvi -o scotch_user6.0.ps
% ps2pdf -sPAPERSIZE=a4 scotch_user6.0.ps scotch_user6.0.pdf

\documentclass{article}
\usepackage{a4}
\usepackage{url}
\usepackage[dvips]{graphicx}
%\documentstyle[11pt,a4,fullpage,epsf]{article}
%\textwidth      16.0cm
%\oddsidemargin   -0.5cm
%\evensidemargin  -0.5cm
%\marginparwidth  0.0cm
%\marginparsep    0.0cm
%\marginparpush   0.0cm
%\topmargin        0.5cm
%\headheight      0.0cm
%\headsep         0.0cm
%\textheight     25.0cm
%\footheight      0.0cm
%\footskip        0.0cm

\sloppy                                          % Gestion des overfull hbox
\renewcommand{\baselinestretch}{1.05}            % Hauteur lignes x 1.05

\setcounter{secnumdepth}{3}                      % Sous-sous-sections numerotees
\setcounter{tocdepth}{3}                         % Sous-sous-sections dans la table

%% Macros et commandes utiles.

\makeatletter
\@definecounter{enumv}                           % 8 niveaux d'itemizations
\@definecounter{enumvi}
\@definecounter{enumvii}
\@definecounter{enumviii}
\def\itemize{\ifnum \@itemdepth >8 \@toodeep\else \advance\@itemdepth \@ne
\edef\@itemitem{labelitem\romannumeral\the\@itemdepth}%
\list{\csname\@itemitem\endcsname}{\def\makelabel##1{\hss\llap{##1}}}\fi}
\let\enditemize =\endlist

\def\@iteme[#1]{\if@noparitem \@donoparitem      % Item long pour options
  \else \if@inlabel \indent \par \fi
         \ifhmode \unskip\unskip \par \fi
         \if@newlist \if@nobreak \@nbitem \else
                        \addpenalty\@beginparpenalty
                        \addvspace\@topsep \addvspace{-\parskip}\fi
           \else \addpenalty\@itempenalty \addvspace\itemsep
          \fi
    \global\@inlabeltrue
\fi
\everypar{\global\@minipagefalse\global\@newlistfalse
          \if@inlabel\global\@inlabelfalse
             \setbox\@tempboxa\hbox{#1}\relax
             \hskip \itemindent \hskip -\parindent
             \hskip -\labelwidth \hskip -\labelsep
             \ifdim \wd\@tempboxa > \labelwidth
               \box\@tempboxa\hfil\break
             \else
               \hbox to\labelwidth{\box\@tempboxa\hfil}\relax
               \hskip \labelsep
             \fi
             \penalty\z@ \fi
          \everypar{}}\global\@nobreakfalse
\if@noitemarg \@noitemargfalse \if@nmbrlist \refstepcounter{\@listctr}\fi \fi
\ignorespaces}
\def\iteme{\@ifnextchar [{\@iteme}{\@noitemargtrue \@iteme[\@itemlabel]}}

\let\@Hxfloat\@xfloat
\def\@xfloat#1[{\@ifnextchar{H}{\@HHfloat{#1}[}{\@Hxfloat{#1}[}}
\def\@HHfloat#1[H]{%
\expandafter\let\csname end#1\endcsname\end@Hfloat
\vskip\intextsep\def\@captype{#1}\parindent\z@
\ignorespaces}
\def\end@Hfloat{\vskip \intextsep}
\makeatother

\def\progsyn{\item[{\makebox[1.5em][l]{\bf Synopsis}}]\ ~\linebreak[0]\\*[1em]}
\def\progdes{\item[{\makebox[1.5em][l]{\bf Description}}]\ ~\linebreak[0]\\*[1em]}
\def\progopt{\item[{\makebox[1.5em][l]{\bf Options}}]~\linebreak[0]}
\def\progret{\item[{\makebox[1.5em][l]{\bf Return values}}]~\linebreak[0]}

\newcommand{\bn}{\begin{displaymath}}            % Equations non-numerotees
\newcommand{\en}{\end{displaymath}}
\newcommand{\bq}{\begin{equation}}               % Equations numerotees
\newcommand{\eq}{\end{equation}}

\newcommand{\lbo}{\linebreak[0]}
\newcommand{\lbt}{\linebreak[2]}
\newcommand{\noi}{{\noindent}}                   % Pas d'indentation
\newcommand{\spa}{{\protect \vspace{\bigskipamount}}} % Espace vertical

\newcommand{\eg}{{\it e\@.g\@.\/\ }}             % e.g.
\newcommand{\ie}{{\it i\@.e\@.\/\ }}             % i.e.

\newcommand{\chaco}{{\sc Chaco}}                 % "chaco"
\newcommand{\scotch}{{\sc Scotch}}               % "scotch"
\newcommand{\libscotch}{{\sc libScotch}}         % "libscotch"
\newcommand{\ptscotch}{{\sc PT-Scotch}}          % "PT-Scotch"
\newcommand{\metis}{\mbox{\sc Me$\!$T$\!$iS}}    % "MeTiS"

\newcommand{\eqdef}{\stackrel{\scriptscriptstyle \rm def}{=}}       % = as definition
\newcommand{\isapprox}{\mathop{\approx}\limits}

\newcommand{\lefta}{\longleftarrow}
\newcommand{\rghta}{\longrightarrow}
\newcommand{\botha}{\longleftrightarrow}
\newcommand{\Lefta}{\Longleftarrow}
\newcommand{\Rghta}{\Longrightarrow}
\newcommand{\Botha}{\Longleftrightarrow}

\newcommand{\HY}{{\rm H}}                        % H
\newcommand{\KP}{{\rm K}}                        % K
\newcommand{\MK}[1]{{\rm M}_{#1}}                % Mk
\newcommand{\MD}{\MK{2}}                         % M2
\newcommand{\PA}{{\rm P}}                        % P
\newcommand{\UB}{{\rm UB}}                       % UB
\newcommand{\SE}{{\rm SE}}                       % SE
\newcommand{\FFT}{{\rm FFT}}                     % FFT
\newcommand{\BF}{{\rm BF}}                       % BF
\newcommand{\BFB}{{\overline{\rm BF}}}           % BF bar
\newcommand{\CCC}{{\rm CCC}}                     % CCC
\newcommand{\CCCB}{{\overline{\rm CCC}}}         % CCC bar

\newcommand{\roo}[1]{{\rho_{\scriptscriptstyle {#1}}}} % Rho avec petit argument
\newcommand{\too}[1]{{\tau_{\scriptscriptstyle {#1}}}} % Tau avec petit argument
\newcommand{\xio}[1]{{\xi_{\scriptscriptstyle {#1}}}} % Xi avec petit argument

\newcommand{\SB}[1]{{\cal C}'_S\left({#1}\right)} % Comportement en espace
\newcommand{\TB}[1]{{\cal C}'_T\left({#1}\right)} % Comportement en temps
\newcommand{\SC}[1]{{\cal C}_S\left({#1}\right)} % Complexite en espace
\newcommand{\TC}[1]{{\cal C}_T\left({#1}\right)} % Complexite en temps

\newcommand{\dmap}{\mbox{$\delta_{map}$}}
\newcommand{\dexp}{\mbox{$\delta_{exp}$}}
\newcommand{\mmap}{\mbox{$\mu_{map}$}}
\newcommand{\mdil}{\mbox{$\mu_{dil}$}}
\newcommand{\mcom}{\mbox{$\mu_{com}$}}
\newcommand{\mexp}{\mbox{$\mu_{exp}$}}

\newcommand{\NNZ}{\mbox{NNZ}}
\newcommand{\OPC}{\mbox{OPC}}
\newcommand{\hnbr}{\mbox{$h_{\rm nbr}$}}
\newcommand{\hmin}{\mbox{$h_{\rm min}$}}
\newcommand{\hmax}{\mbox{$h_{\rm max}$}}
\newcommand{\havg}{\mbox{$h_{\rm avg}$}}
\newcommand{\hdlt}{\mbox{$h_{\rm dlt}$}}

%% Version du document.

\newcommand{\scotchver}{6.0}
\newcommand{\scotchversub}{6.0.6}
\newcommand{\scotchcitepuser}{\protect\cite{pell08c}}
\newcommand{\scotchcitesuser}{\protect\cite{pell08b}}

%% Page de garde.

\begin{document}

\date{\today}
%\date{Revision 0.4 $\beta$\\\today}

\title{\includegraphics{s_f_logo.ps}\\[1em]
       {\LARGE\bf \scotch\ and \libscotch\ {\sc \scotchver} User's Guide}\\[1em]%
       {\normalsize (version \scotchversub)}
}

\author{Fran\c cois Pellegrini\\
Universit\'e de Bordeaux \& LaBRI, UMR CNRS 5800\\
TadAAM team, INRIA Bordeaux Sud-Ouest\\
351 cours de la Lib\'eration, 33405 TALENCE, FRANCE\\
{\tt francois.pellegrini@labri.fr}}

\maketitle

\begin{abstract}
This document describes the capabilities and operations of
\scotch\ and \libscotch, a software package and a software library
devoted to static mapping, edge- and vertex-based graph partitioning,
and sparse matrix block ordering of graphs and meshes/hypergraphs. It
gives brief descriptions of the algorithms, details the input/output
formats, instructions for use, installation procedures, and provides a
number of examples.

\scotch\ is distributed as free/libre software, and has been
designed such that new partitioning or ordering methods can be added
in a straightforward manner. It can therefore be used as a testbed for
the easy and quick coding and testing of such new methods, and may
also be redistributed, as a library, along with third-party software
that makes use of it, either in its original or in updated forms.
\end{abstract}

\clearpage

%% Table des matieres.

\tableofcontents

%% Corps du document.

%%%%%%%%%%%%%%%%%%%%%%%%%%%%%%%%%%%%
%                                  %
% Titre  : s_i.tex                 %
% Sujet  : Manuel de l'utilisateur %
%          du projet 'Scotch'      %
%          Introductions           %
% Auteur : Francois Pellegrini     %
%                                  %
%%%%%%%%%%%%%%%%%%%%%%%%%%%%%%%%%%%%

\section{Introduction}

\subsection{Static mapping}

The efficient execution of a parallel program on a parallel machine
requires that the communicating processes of the program be assigned
to the processors of the machine so as to minimize its overall running
time.
When processes have a limited duration and their logical dependencies
are accounted for, this optimization problem is referred to as
\emph{scheduling}.
When processes are assumed to coexist simultaneously for the entire
duration of the program, it is referred to as \emph{mapping}. It
amounts to balancing the computational weight of the processes among the
processors of the machine, while reducing the cost of communication by
keeping intensively inter-communicating processes on nearby
processors.
In most cases, the underlying computational structure of the parallel
programs to map can be conveniently modeled as a graph in which
vertices correspond to processes that handle distributed pieces of
data, and edges reflect data dependencies. The mapping problem can
then be addressed by assigning processor labels to the vertices of the
graph, so that all processes assigned to some processor are loaded and
run on it.
In a SPMD context, this is equivalent to the \emph{distribution\/}
across processors of the data structures of parallel programs; in this
case, all pieces of data assigned to some processor are handled by a
single process located on this processor.

A mapping is called \emph{static\/} if it is computed prior to the
execution of the program. Static mapping is NP-complete in the general
case~\cite{gajo79}. Therefore, many studies have been carried out in
order to find sub-optimal solutions in reasonable time, including
the development of specific algorithms for common topologies such
as the hypercube~\cite{errasa90,hamm92}.
When the target machine is assumed to have a communication network in
the shape of a complete graph, the static mapping problem turns into
the \emph{partitioning\/} problem, which has also been intensely
studied~\cite{basi94,hele93a,kaku95a,kaku95c,posili90}.
However, when mapping onto parallel machines the communication network
of which is not a bus, not accounting for the topology of the target
machine usually leads to worse running times, because simple cut
minimization can induce more expensive long-distance
communication~\cite{hamm92,wacrevjo95}.

\subsection{Sparse matrix ordering}

Many scientific and engineering problems can be modeled by sparse linear
systems, which are solved either by iterative or direct methods.
To achieve efficiency with direct methods, one must minimize the
fill-in induced by factorization. This fill-in is a direct consequence of
the order in which the unknowns of the linear system are numbered,
and its effects are critical both in terms of memory and computation costs.
\\

An efficient way to compute fill reducing orderings of symmetric
sparse matrices is to use recursive nested dissection~\cite{geli81}.
It amounts to computing a vertex set~$S$ that separates the graph into
two parts~$A$ and~$B$, ordering $S$ with the highest indices that are
still available, and proceeding recursively on parts~$A$ and~$B$ until
their sizes become smaller than some threshold value. This ordering
guarantees that, at each step, no non-zero term can appear in the
factorization process between unknowns of~$A$ and unknowns of~$B$.

The main issue of the nested dissection ordering algorithm is thus to
find small vertex separators that balance the remaining subgraphs as
evenly as possible, in order to minimize fill-in and to increase
concurrency in the factorization process.

\subsection{Contents of this document}

This document describes the capabilities and operations of \scotch,
a software package devoted to static mapping, graph and mesh
partitioning, and sparse matrix block ordering.
\scotch\ allows the user to map efficiently any kind of weighted process
graph onto any kind of weighted architecture graph, and provides high-quality
block orderings of sparse matrices.
The rest of this manual is organized as follows.
Section~\ref{sec-project} presents the goals of the \scotch\ project.
Sections~\ref{sec-algo-map} and~\ref{sec-algo-order} outline the most
important aspects of the mapping and ordering algorithms that it
implements, respectively.
Section~\ref{sec-changes} summarizes the most important changes between
version~\textsc{5.0} and previous versions.
Section~\ref{sec-file} defines the formats of the files used in \scotch,
section~\ref{sec-prog} describes the programs of the
\scotch\ distribution, and section~\ref{sec-lib} defines the interface
and operations of the \libscotch\ library.
Section~\ref{sec-install} explains how to obtain and install the
\scotch\ distribution.
Finally, some practical examples are given in
section~\ref{sec-examples}, and instructions on how to implement new
methods in the \libscotch\ library are provided in
section~\ref{sec-coding}.

\section{The \scotch\ project}
\label{sec-project}

\subsection{Description}

\scotch\ is a project carried out at the \textit{Laboratoire Bordelais de
Recherche en Informatique\/} (LaBRI) of the Universit\'e de Bordeaux and
within the Tadaam team-project of INRIA Bordeaux Sud-Ouest. Its goal
is to study the application of graph theory to scientific computing.

It focused first on static mapping, and has resulted in the
development of the Dual Recursive Bipartitioning (or DRB) mapping
algorithm and in the study of several graph bipartitioning
heuristics~\cite{pell94a}, all of which have been implemented in the
\scotch\ software package~\cite{pero96a}. Then, it focused on the
computation of high-quality vertex separators for the ordering of
sparse matrices by nested dissection, by extending the work that has
been done on graph partitioning in the context of static
mapping~\cite{pero97a,peroam99}. The ordering
capabilities of \scotch\ have then been extended to native mesh structures,
thanks to hypergraph partitioning algorithms. Diffusion-based graph
partitioning methods have also been added~\cite{chpe06a,pell07b}.

Version \textsc{5.0} of \scotch\ was the first one to comprise parallel
graph ordering routines. The parallel features of \scotch\ are referred
to as \ptscotch\ (``\emph{Parallel Threaded \scotch}''). While both
packages share a significant amount of code, because
\ptscotch\ transfers control to the sequential routines of the
\libscotch\ library when the subgraphs on which it operates are
located on a single processor, the two sets of routines have a
distinct user's manual. Readers interested in the parallel features
of \scotch\ should refer to the \emph{\ptscotch\ \textsc{\scotchver}
User's Guide}~\scotchcitepuser.

Version \textsc{6.0} of \scotch\ is oriented towards the development of
new features, namely graph repartitioning and
remapping~\cite{fope11a}. A whole set of direct $k$-way graph
partitioning and mapping algorithms has also been implemented.
Also, new target architectures have been created, to allow \scotch\ to
map efficiently onto parts of regular target
architectures~\cite{pellegrini:hal-01671156}, as it is the case when
considering a potentially non-connected partition of a big machine, as
provided by a batch scheduler.

\subsection{Availability}

Starting from version \textsc{4.0}, which has been developed at INRIA
within the ScAlApplix project, \scotch\ is available under a dual
licensing basis. On the one hand, it is downloadable from the
\scotch\ web page as free/libre software, to all interested parties
willing to use it as a library or to contribute to it as a testbed for
new partitioning and ordering methods. On the other hand, it can also
be distributed, under other types of licenses and conditions, to
parties willing to embed it tightly into closed, proprietary software.
\\

The free/libre software license under which \scotch\ \textsc{\scotchver}
is distributed is the CeCILL-C license~\cite{cecill}, which has
basically the same features as the GNU LGPL (``\textit{Lesser General
Public License}''): ability to link the code as a library to any
free/libre or even proprietary software, ability to modify the code
and to redistribute these modifications. Version \textsc{4.0} of
\scotch\ was distributed under the LGPL itself.
\\

Please refer to section~\ref{sec-install} to see how to obtain the
free/libre distribution of \scotch.

\section{Static mapping algorithms}
\label{sec-algo-map}

The parallel program to be mapped onto the target architecture is modeled
by a valuated unoriented graph $S$ called \emph{source graph\/} or
\emph{process graph}, the vertices of which represent the processes of the
parallel program, and the edges of which the communication channels between
communicating processes.
Vertex- and edge- valuations associate with every vertex $v_S$ and every
edge $e_S$ of $S$ integer numbers $w_S(v_S)$ and $w_S(e_S)$ which
estimate the computation weight of the corresponding process
and the amount of communication to be transmitted on the channel,
respectively.

The target machine onto which is mapped the parallel program is also
modeled by a valuated unoriented graph $T$ called \emph{target graph\/}
or \emph{architecture graph}.
Vertices $v_T$ and edges $e_T$ of $T$ are assigned integer weights
$w_T(v_T)$ and $w_T(e_T)$, which estimate the computational power of the
corresponding processor and the cost of traversal of the inter-processor
link, respectively.

A \emph{mapping} from $S$ to $T$ consists of two applications
$\too{S,T} : V(S) \rghta V(T)$ and
$\roo{S,T} : E(S) \rghta {\cal P}(E(T))$,
where ${\cal P}(E(T))$ denotes the set of all simple loopless paths which
can be built from $E(T)$.
$\too{S,T}(v_S) = v_T$ if process $v_S$ of $S$ is mapped onto processor
$v_T$ of $T$, and $\roo{S,T}(e_S) = \{ e^1_T, e^2_T, \ldots, e^n_T \}$ if
communication channel $e_S$ of $S$ is routed through communication links
$e^1_T$, $e^2_T$, \ldots, $e^n_T$ of $T$.
$|\roo{S,T}(e_S)|$ denotes the dilation of edge $e_S$, that is, the number of
edges of $E(T)$ used to route $e_S$.

\subsection{Cost function and performance criteria}

The computation of efficient static mappings requires an \emph{a priori\/}
knowledge of the dynamic behavior of the target machine with respect to
the programs which are run on it.
This knowledge is synthesized in a \emph{cost function}, the nature of which
determines the characteristics of the desired optimal mappings.
The goal of our mapping algorithm is to minimize some communication cost
function, while keeping the load balance within a specified tolerance.
The communication cost function $f_C$ that we have chosen is the sum,
for all edges, of their dilation multiplied by their weight:
\bn
f_C(\too{S,T},\roo{S,T})
\eqdef \hspace*{-0.25cm}\sum\limits_{e_S\in E(S)}\hspace*{-0.25cm}
w_S(e_S)\,|\roo{S,T}(e_S)|\enspace.
\en
This function, which has already been considered by several authors for
hypercube target topologies~\cite{errasa90,hamm92,hele94b}, has several
interesting properties:
it is easy to compute, allows incremental updates performed by
iterative algorithms, and
its minimization favors the mapping of intensively intercommunicating
processes onto nearby processors;
regardless of the type of routing implemented on the target machine
(store-and-forward or cut-through), it models the traffic on the
interconnection network and thus the risk of congestion.

The strong positive correlation between values of this function and
effective execution times has been experimentally verified by
Hammond~\cite{hamm92} on the CM-2, and by Hendrickson and
Leland~\cite{hele94a} on the nCUBE~2.
\hfill~\\

The quality of mappings is evaluated with respect to the criteria for
quality that we have chosen: the balance of the computation load across
processors, and the minimization of the inter-processor communication cost
modeled by function~$f_C$. These criteria lead to the definition of
several parameters, which are described below.

For load balance, one can define $\mmap$, the average load per
computational power unit (which does not depend on the mapping), and
$\dmap$, the load imbalance ratio, as\\[-0.5em]
\bn
\mmap \eqdef
{\sum\limits_{v_S \in V(S)} w_S(v_S) \over
 \sum\limits_{v_T \in V(T)} w_T(v_T)}
\hspace*{2.5em}\mbox{~and~}
\en
\bn
\dmap \eqdef
{\sum\limits_{v_T \in V(T)}
   \left|\left(\!\!{1 \over w_T(v_T)}\hspace*{-0.3em}
         \sum\limits_{\scriptsize
                      \shortstack{$v_S \in V(S)$\\[-0.2em]
                                  $\too{S,T}(v_S) = v_T$}}
         \hspace*{-0.2em} w_S(v_S)\!\!\right)\:-\:\mmap\right| \over
\sum\limits_{v_S \in V(S)} w_S(v_S)}\enspace.
\en
However, since the maximum load imbalance ratio is provided by the user in
input of the mapping, the information given by these parameters is of little
interest, since what matters is the minimization of the communication cost
function under this load balance constraint.

For communication, the straightforward parameter to consider is $f_C$.
It can be normalized as $\mexp$, the average edge expansion, which can
be compared to $\mdil$, the average edge dilation; these are defined
as\\[-1.3em]
\bn
\mexp \eqdef {f_C \over \sum\limits_{e_S \in E(S)} w_S(e_S)}
\hspace*{2.5em}\mbox{~and~}\hspace*{2.5em}
\mdil \eqdef {\sum\limits_{e_S \in E(S)}|\roo{S,T}(e_S)| \over |E(S)|}
\enspace.
\en
$\dexp \eqdef {\mexp \over \mdil}$ is smaller than $1$ when the mapper
succeeds in putting heavily intercommunicating processes closer to each other
than it does for lightly communicating processes; they are equal if all edges
have same weight.

\subsection{The Dual Recursive Bipartitioning algorithm}
\label{sec-algo-drb}

This mapping algorithm, which is the primary way to compute initial
static mappings, uses a \emph{divide and conquer\/} approach to
recursively allocate subsets of processes to subsets of
processors~\cite{pell94a,pero96b}. It starts by considering a set of
processors, also called \emph{domain}, containing all the processors of
the target machine, and with which is associated the set of all the
processes to map.  At each step, the algorithm bipartitions a yet
unprocessed domain into two disjoint subdomains, and calls a
\emph{graph bipartitioning algorithm\/} to split the subset of
processes associated with the domain across the two subdomains, as
sketched in the following.

\noi
{\renewcommand{\baselinestretch}{0.95}\footnotesize\tt {%
\begin{verbatim}
mapping (D, P)
Set_Of_Processors  D;
Set_Of_Processes   P;
{
  Set_Of_Processors  D0, D1;
  Set_Of_Processes   P0, P1;

  if (|P| == 0) return;  /* If nothing to do.     */
  if (|D| == 1) {        /* If one processor in D */
    result (D, P);       /* P is mapped onto it.  */
    return;
  }

  (D0, D1) = processor_bipartition (D);
  (P0, P1) = process_bipartition   (P, D0, D1);
  mapping (D0, P0);      /* Perform recursion. */
  mapping (D1, P1);
}
\end{verbatim}}}

\noi
The association of a subdomain with every process defines a \emph{partial
mapping\/} of the process graph. As bipartitionings are performed,
the subdomain sizes decrease, up to give a complete mapping when all
subdomains are of size~one.
\\

The above algorithm lies on the ability to define five main objects:
\begin{itemize}
\item
a \emph{domain structure}, which represents a set of processors in the target
architecture;
\item
a \emph{domain bipartitioning function}, which, given a domain, bipartitions
it in two disjoint subdomains;
\item
a \emph{domain distance function}, which gives, in the target graph, a measure
of the distance between two disjoint domains. Since domains may not be convex
nor connected, this distance may be estimated.
However, it must respect certain homogeneity properties, such as
giving more accurate results as domain sizes
decrease~\cite{pero96b,pellegrini:hal-01671156}.
The domain distance function is used by the graph bipartitioning algorithms
to compute the communication function to minimize, since it allows the mapper
to estimate the dilation of the edges that link vertices which belong to
different domains.
Using such a distance function amounts to considering that all routings
will use shortest paths on the target architecture, which is how most
parallel machines actually do.
We have thus chosen that our program would not provide routings for the
communication channels, leaving their handling to the communication system of
the target machine;
\item
a \emph{process subgraph structure}, which represents the subgraph induced by a
subset of the vertex set of the original source graph;
\item
a \emph{process subgraph bipartitioning function}, which bipartitions subgraphs
in two disjoint pieces to be mapped onto the two subdomains computed by
the domain bipartitioning function.
\end{itemize}
All these routines are seen as black boxes by the mapping program, which can
thus accept any kind of target architecture and process bipartitioning
functions.

\subsubsection{Partial cost function}

The production of efficient complete mappings requires that all graph
bipartitionings favor the criteria that we have chosen.
Therefore, the bipartitioning of a subgraph~$S'$ of $S$ should maintain
load balance within the user-specified tolerance, and minimize the
\emph{partial\/} communication cost function $f'_C$, defined as
\bn
f'_C(\too{S,T},\roo{S,T}) \eqdef
\hspace*{-0.45cm}\sum\limits_{\mbox{\scriptsize
             \shortstack{$v\in V(S')$\\
                         $\{v,v'\}\in E(S)$}}}\hspace*{-0.45cm}
w_S(\{v,v'\})\,|\roo{S,T}(\{v,v'\})|\enspace,
\en
which accounts for the dilation of edges internal to subgraph~$S'$ as well as
for the one of edges which belong to the cocycle of $S'$, as shown in
Figure~\ref{fig-bipcost}.
Taking into account the partial mapping results issued by previous
bipartitionings makes it possible to avoid local choices that
might prove globally bad, as explained below.
This amounts to incorporating additional constraints to the standard graph
bipartitioning problem, turning it into a more general optimization problem
termed \emph{skewed graph partitioning\/} by some authors~\cite{heledr97}.

\begin{figure}[hbt]
\hfill
\parbox[b]{4.9cm}{
\hfill
\includegraphics[scale=0.40]{s_f_rua.eps}
\hfill\\
\textbf{a.} Initial position.
}\ \hfill\
\parbox[b]{4.9cm}{
\hfill
\includegraphics[scale=0.40]{s_f_rub.eps}
\hfill\\
\textbf{b.} After one vertex is moved.
}\hfill\
\caption%
{Edges accounted for in the partial communication cost function when
 bipartitioning the subgraph associated with domain~$D$ between
 the two subdomains $D_0$ and $D_1$ of~$D$.
 Dotted edges are of dilation zero, their two ends being mapped onto the
 same subdomain. Thin edges are cocycle edges.}
\label{fig-bipcost}
\end{figure}

\subsubsection{Execution scheme}

From an algorithmic point of view, our mapper behaves as a greedy algorithm,
since the mapping of a process to a subdomain is never reconsidered, and
at each step of which iterative algorithms can be applied.
The double recursive call performed at each step induces a recursion scheme
in the shape of a binary tree, each vertex of which corresponds to a
bipartitioning job, that is, the bipartitioning of both a domain and
its associated subgraph.

In the case of depth-first sequencing, as written in the above sketch,
bipartitioning jobs run in the left branches of the tree have no information
on the distance between the vertices they handle and neighbor vertices to be
processed in the right branches.
On the contrary, sequencing the jobs according to a by-level (breadth-first)
travel of the tree allows any bipartitioning job of a given level to
have information on the subdomains to which all the processes have been
assigned at the previous level.
Thus, when deciding in which subdomain to put a given process, a
bipartitioning job can account for the communication costs induced by
its neighbor processes, whether they are handled by the job itself or not,
since it can estimate in $f'_C$ the dilation of the corresponding edges.
This results in an interesting feedback effect: once an edge has been kept
in a cut between two subdomains, the distance between its end vertices will
be accounted for in the partial communication cost function to be minimized,
and following jobs will be more likely to keep these vertices close to
each other, as illustrated in Figure~\ref{fig-biprub}.
\begin{figure}[hbt]
\hfill
\parbox[b]{5.2cm}{
\hfill
\includegraphics[scale=0.40]{s_f_run.eps}
\hfill\\
\textbf{a.} Depth-first sequencing.
}\ \hfill\
\parbox[b]{5.2cm}{
\hfill
\includegraphics[scale=0.40]{s_f_ruy.eps}
\hfill\\
\textbf{b.} Breadth-first sequencing.
}\hfill\ %
\caption%
{Influence of depth-first and breadth-first sequencings on the
 bipartitioning of a domain~$D$ belonging to the leftmost branch of
 the bipartitioning tree.
 With breadth-first sequencing, the partial mapping data regarding vertices
 belonging to the right branches of the bipartitioning tree are more
 accurate (C.L. stands for ``Cut Level'').}
\label{fig-biprub}
\end{figure}
The relative efficiency of depth-first and breadth-first sequencing schemes
with respect to the structure of the source and target graphs is discussed
in~\cite{pero96b}.

\subsubsection{Clustering by mapping onto variable-sized architectures}
\label{sec-algo-variable}
\index{Architecture|Variable-sized!see Clustering}
\index{Clustering}

Several constrained graph partitioning problems can be modeled as
mapping the problem graph onto a target architecture, the number of
vertices and topology of which depend dynamically on the structure of
the subgraphs to bipartition at each step.

Variable-sized architectures are supported by the DRB algorithm in the
following way: at the end of each bipartitioning step, if any of the
variable subdomains is empty (that is, all vertices of the subgraph
are mapped only to one of the subdomains), then the DRB process stops
for both subdomains, and all of the vertices are assigned to their
parent subdomain; else, if a variable subdomain has only one vertex
mapped onto it, the DRB process stops for this subdomain, and the
vertex is assigned to it.

The moment when to stop the DRB process for a specific subgraph can be
controlled by defining a bipartitioning strategy that checks the
validity of a criterion at each bipartitioning step (see for instance
Section~\ref{sec-lib-func-stratgraphclusterbuild}), and maps all of
the subgraph vertices to one of the subdomains when it becomes false.

\subsection{Static mapping methods}
\label{sec-algo-map-methods}

The core of our static mapping software uses graph mapping methods as
black boxes. It maintains an internal image of the current mapping,
which records the target vertex index onto which each of the source
graph vertices is mapped. It is therefore possible to apply several
mapping methods in sequence, such that the first method computes an
initial mapping to be further refined by the following methods, thus
enabling us to define \emph{static mapping strategies}. The currently
implemented static mapping methods are listed below.

\begin{itemize}
\iteme[\textbf{Multilevel}]\label{sec-algo-mle}
This framework, which has been studied by several
authors~\cite{basi94,hele93b,kaku95a} and should be considered as a strategy
rather than as a method since it uses other methods as parameters, repeatedly
reduces the size of the graph to map by finding matchings that
collapse vertices and edges, computes a mapping of the coarsest
graph obtained, and prolongs the result back to the original graph,
as shown in Figure~\ref{fig-multiproc}.
\begin{figure}[hbt]
~\hfill\includegraphics[scale=0.50]{s_f_mult.eps}\hfill\ ~
\caption%
{The multilevel partitioning process. In the uncoarsening phase, the light
 and bold lines represent for each level the prolonged partition obtained
 from the coarser graph, and the partition obtained after refinement,
 respectively.}
\label{fig-multiproc}
\end{figure}
The multilevel method, when used in conjunction with some local
optimization methods to refine the projected partitions at every
level, usually leads to a significant improvement in quality with
respect to methods operating only on the finest graph. By coarsening
the graphs, the multilevel algorithm broadens the scope of local
optimization algorithms: it makes possible for them to account for
topological structures of the original graph that would else be of a
too high level for them to be encompassed in their local optimization
process.
\iteme[\textbf{Band}]\label{sec-algo-band}
Like the multilevel method above, the band method is a framework, in
the sense that it does not itself compute partitions, but rather helps
other partitioning algorithms perform better. It is a refinement
algorithm which, from a given initial partition, extracts a band graph
of given width (which only contains graph vertices that are at most at
this distance from the frontiers of the parts), calls a partitioning
strategy on this band graph, and projects back the refined partition
on the original graph. This method was designed to be able to use
expensive partitioning heuristics, such as genetic algorithms, on
large graphs, as it dramatically reduces the problem space by several
orders of magnitude. However, it was found that, in a multilevel
context, it also improves partition quality, by coercing partitions in
a problem space that derives from the one which was globally defined
at the coarsest level, thus preventing local optimization refinement
algorithms to be trapped in local optima of the finer
graphs~\cite{chpe06a}.
\iteme[\textbf{Fiduccia-Mattheyses}]
This is a direct $k$-way version of the traditional Fiduccia-Mattheyses
heuristics used for computing bipartitions, that will be presented in
the next section. By default, boundary vertices can only be moved to
parts to which at least one of their neighbors belong.
\iteme[\textbf{Diffusion}]
This is also a $k$-way version of an algorithm that has been first used
in the context of bipartitioning, and which will be presented in the
next section. The $k$-way version differs from the latter as it diffuses
$k$ sorts of liquids rather than just two as in the bipartitioning case.
\iteme[\textbf{Exactifier}]\label{sec-algo-map-exact}
This greedy algorithm refines its input mapping so as to reduce load
imbalance as much as possible. Since this method does not consider
load balance minimization, its use should be restricted to cases where 
achieving load balance is critical and where recursive
bipartitioning may fail to achieve it. It is especially the case when
vertex loads are very irregular: some subdomains may receive only a
few heavy vertices, yielding load balance artifacts when no light
vertices are locally available to compensate.

Graph vertices are sorted by decreasing weights, and considered in
turn. If the current vertex can fit in its initial part without
causing imbalance by excess, it is added to it, and the algorithm goes
on. Else, a candidate part is found by exploring other subdomains in
an order based on an implicit recursive bipartitioning of the
architecture graph. Consequently, such vertices will be placed in
subdomains that tend to be as close as possible to the original
location of the vertex. This method is most likely to result in
disconnected parts.
\iteme[\textbf{Dual recursive bipartitioning}]
This algorithm implements the dual recursive bipartitioning algorithm
that has been presented in Section~\ref{sec-algo-drb}. The DRB
algorithms can be used either directly on the original graph to
partition, or on the coarsest graph yielded by the direct $k$-way
multilevel framework. It uses graph bipartitioning methods, described
below, to compute its bipartitions.
\end{itemize}

\subsection{Graph bipartitioning methods}
\label{sec-algo-bipart}

The core of our dual recursive bipartitioning mapping algorithm uses
process graph bipartitioning methods as black boxes. It allows the
mapper to run any type of graph bipartitioning method compatible with
our criteria for quality.  Bipartitioning jobs maintain an internal
image of the current bipartition, indicating for every vertex of the
job whether it is currently assigned to the first or to the second
subdomain.
It is therefore possible to apply several different methods in sequence,
each one starting from the result of the previous one,
and to select the methods with respect to the job characteristics, thus
enabling us to define \emph{graph bipartitioning strategies}.
The currently implemented graph bipartitioning methods are listed below.
\begin{itemize}
\iteme[\textbf{Diffusion}]
This global optimization method, presented in~\cite{pell07b}, flows two
kinds of antagonistic liquids, scotch and anti-scotch, from two source
vertices, and sets the new frontier as the limit between vertices
which contain scotch and the ones which contain anti-scotch. In order
to add load-balancing constraints to the algorithm, a constant amount
of liquid disappears from every vertex per unit of time, so that no
domain can spread across more than half of the vertices. Because
selecting the source vertices is essential to the obtainment of useful
results, this method has been hard-coded so that the two source
vertices are the two vertices of highest indices, since in the band
method these are the anchor vertices which represent all of the removed
vertices of each part. Therefore, this method must be used on band
graphs only, or on specifically crafted graphs.
\iteme[\textbf{Exactifier}]
This greedy algorithm refines the current partition so as to reduce load
imbalance as much as possible, while keeping the value of the communication
cost function as small as possible.
The vertex set is scanned in order of decreasing vertex weights, and vertices
are moved from one subdomain to the other if doing so reduces load imbalance.
When several vertices have same weight, the vertex whose swap
decreases most the communication cost function is selected first.
This method is used in post-processing of other methods when load balance is
mandatory. For weighted graphs, the strict enforcement of load balance may
cause the swapping of isolated vertices of small weight, thus greatly
increasing the cut. Therefore, great care should be taken when using this
method if connectivity or cut minimization are mandatory.
\iteme[\textbf{Fiduccia-Mattheyses}]\label{sec-algo-fme}
The Fiduccia-Mattheyses heuristics~\cite{fima82} is an almost-linear
improvement of the famous Kernighan-Lin algorithm~\cite{keli70}.
It tries to improve the bipartition that is input to it
by incrementally moving vertices between the subsets of the partition,
as long as it can find sequences of moves that lower its communication cost.
By considering sequences of moves instead of single swaps,
the algorithm allows hill-climbing from local minima of the cost function.
As an extension to the original Fiduccia-Mattheyses algorithm,
we have developed new data structures, based on logarithmic indexings of
arrays, that allow us to handle weighted graphs while preserving the
almost-linearity in time of the algorithm~\cite{pero96b}.

As several authors quoted before~\cite{hele93c,kaku95b},
the Fiduccia-Mattheyses algorithm gives better results when trying to optimize
a good starting partition. Therefore, it should not be used on its own, but
rather after greedy starting methods such as the Gibbs-Poole-Stockmeyer or
the greedy graph growing methods.
\iteme[\textbf{Gibbs-Poole-Stockmeyer}]
This greedy bipartitioning method derives from an algorithm proposed by
Gibbs, Poole, and Stockmeyer to minimize the dilation of graph orderings,
that is, the maximum absolute value of the difference between the numbers of
neighbor vertices~\cite{gipost76}.
The graph is sliced by using a breadth-first spanning tree rooted
at a randomly chosen vertex, and this process is iterated by selecting a new
root vertex within the last layer as long as the number of layers increases.
Then, starting from the current root vertex, vertices are assigned layer after
layer to the first subdomain, until half of the total weight has been
processed. Remaining vertices are then allocated to the second subdomain.

As for the original Gibbs, Poole, and Stockmeyer algorithm, it is assumed that
the maximization of the number of layers results in the minimization of the
sizes --and therefore of the cocycles-- of the layers.
This property has already been used by George and Liu to reorder sparse
linear systems using the nested dissection method~\cite{geli81}, and by
Simon in~\cite{simo91}.
\iteme[\textbf{Greedy graph growing}]\label{sec-algo-ggge}
This greedy algorithm, which has been proposed by Karypis and
Kumar~\cite{kaku95a}, belongs to the GRASP (``\textit{Greedy
Randomized Adaptive Search Procedure\/}'') class~\cite{lafeel94}.
It consists in selecting an initial vertex at random, and repeatedly adding
vertices to this growing subset, such that each added vertex results in the
smallest increase in the communication cost function.
This process, which stops when load balance is achieved, is repeated
several times in order to explore (mostly in a gradient-like fashion)
different areas of the solution space, and the best partition found is kept.
\iteme[\textbf{Multilevel}]
This is a graph bipartition-oriented version of the static mapping
multilevel method described in the previous section,
page~\pageref{sec-algo-mle}.
\end{itemize}

\section{Sparse matrix ordering algorithms}
\label{sec-algo-order}

When solving large sparse linear systems of the form $Ax=b$, it is
common to precede the numerical factorization by a symmetric
reordering. This reordering is chosen in such a way that pivoting down
the diagonal in order on the resulting permuted matrix $PAP^T$
produces much less fill-in and work than computing the factors of $A$
by pivoting down the diagonal in the original order (the fill-in is
the set of zero entries in $A$ that become non-zero in the factored
matrix).

\subsection{Performance criteria}
\label{sec-order-perf}

The quality of orderings is evaluated with respect to several
criteria. The first one, \NNZ, is the number of non-zero terms in the
factored reordered matrix. The second one, \OPC, is the operation
count, that is the number of arithmetic operations required to factor
the matrix. The operation count that we have considered takes into
consideration all operations (additions, subtractions,
multiplications, divisions) required by Cholesky factorization, except
square roots; it is equal to $\sum_c n_c^2$, where $n_c$ is the number
of non-zeros of column $c$ of the factored matrix, diagonal included.
A third criterion for quality is the shape of the elimination tree;
concurrency in parallel solving is all the higher as the elimination tree is
broad and short. To measure its quality, several parameters can be defined:
\hmin, \hmax, and \havg\ denote the minimum, maximum, and average heights
of the tree\footnote%
{We do not consider as leaves the disconnected vertices that are present in
some meshes, since they do not participate in the solving process.},
respectively, and \hdlt\ is the variance, expressed as a percentage of \havg.
Since small separators result in small chains in the elimination tree,
\havg\ should also indirectly reflect the quality of separators.

\subsection{Minimum Degree}

The minimum degree algorithm~\cite{tiwa67} is
a local heuristic that performs its pivot selection by iteratively
selecting from the graph a node of minimum degree.

The minimum degree algorithm is known to be a very fast and general
purpose algorithm, and has received much attention over the last three
decades (see for example~\cite{amdadu96,geli89,liu-85}). However, the
algorithm is intrinsically sequential, and very little can be
theoretically proved about its efficiency.

\subsection{Nested dissection}
\label{sec-algo-nested}

The nested dissection algorithm~\cite{geli81} is a global, heuristic,
recursive algorithm which computes a vertex set~$S$ that separates the
graph into two parts~$A$ and~$B$, ordering $S$ with the highest
remaining indices. It then proceeds recursively on parts~$A$ and~$B$
until their sizes become smaller than some threshold value. This
ordering guarantees that, at each step, no non zero term can appear
in the factorization process between unknowns of~$A$ and unknowns
of~$B$.

Many theoretical results have been carried out on nested dissection
ordering~\cite{chro89,lirota79}, and its divide and conquer nature
makes it easily parallelizable. The main issue of the nested
dissection ordering algorithm is thus to find small vertex separators
that balance the remaining subgraphs as evenly as possible. Most
often, vertex separators are computed by using direct
heuristics~\cite{hero98,lele87}, or from edge separators~\cite[and
included references]{pofa90} by minimum cover
techniques~\cite{duff81,hoka73}, but other techniques such as spectral
vertex partitioning have also been used~\cite{posili90}.

Provided that good vertex separators are found, the nested dissection
algorithm produces orderings which, both in terms of fill-in and
operation count, compare favorably~\cite{gukaku96,kaku95a,pero97a} to
the ones obtained with the minimum degree algorithm~\cite{liu-85}.
Moreover, the elimination trees induced by nested dissection are
broader, shorter, and better balanced, and therefore
exhibit much more concurrency in the context of parallel Cholesky
factorization~\cite[and included
references]{aseilish91,geng89,geheling88,gukaku96,pero97a,shre92}.

\subsection{Hybridization}
\label{sec-algo-nested-hybrid}

Due to their complementary nature, several schemes have been proposed
to hybridize the two methods~\cite{hero98,kaku98a,pero97a}. However,
to our knowledge, only loose couplings have been achieved: incomplete
nested dissection is performed on the graph to order, and the
resulting subgraphs are passed to some minimum degree algorithm. This
results in the fact that the minimum degree algorithm does not have
exact degree values for all of the boundary vertices of the subgraphs,
leading to a misbehavior of the vertex selection process.
\\

Our ordering program implements a tight coupling of the nested dissection
and minimum degree algorithms, that allows each of them to take
advantage of the information computed by the other.
First, the nested dissection algorithm provides exact degree values
for the boundary vertices of the subgraphs passed to the minimum
degree algorithm (called \emph{halo\/} minimum degree since it has a
partial visibility of the neighborhood of the subgraph).
Second, the minimum degree algorithm returns the assembly tree that it
computes for each subgraph, thus allowing for supervariable amalgamation,
in order to obtain column-blocks of a size suitable for BLAS3 block
computations.

As for our mapping program, it is possible to combine ordering methods
into ordering strategies, which allow the user to select the proper
methods with respect to the characteristics of the subgraphs.
\\

The ordering program is completely parametrized by its ordering strategy.
The nested dissection method allows the user to take advantage of all of
the graph partitioning routines that have been developed in the earlier
stages of the \scotch\ project.
Internal ordering strategies for the separators are relevant in the case of
sequential or parallel~\cite{gukaku97,roth94,rogu93,rosc94} block solving,
to select ordering algorithms that minimize the number of extra-diagonal
blocks~\cite{chro89}, thus allowing for efficient use of BLAS3 primitives,
and to reduce inter-processor communication.

\subsection{Ordering methods}

The core of our ordering algorithm uses graph ordering methods as
black boxes, which allows the orderer to run any type of ordering
method. In addition to yielding orderings of the subgraphs that are
passed to them, these methods may compute column block partitions of
the subgraphs, that are recorded in a separate tree structure.
The currently implemented graph ordering methods are listed below.
\begin{itemize}
\iteme[\textbf{Halo approximate minimum degree}]
The halo approximate minimum degree method~\cite{peroam99} is an
improvement of the approximate minimum degree~\cite{amdadu96}
algorithm, suited for use on subgraphs produced by nested dissection
methods. Its interest compared to classical minimum degree algorithms
is that boundary vertices are processed using their real degree in the
global graph rather than their (much smaller) degree in the subgraph,
resulting in smaller fill-in and operation count. This method also
implements amalgamation techniques that result in efficient block
computations in the factoring and the solving processes.
\iteme[\textbf{Halo approximate minimum fill}]
The halo approximate minimum fill method is a variant of the
halo approximate minimum degree algorithm, where the criterion to
select the next vertex to permute is not based on its current
estimated degree but on the minimization of the induced fill.
\iteme[\textbf{Graph compression}]
The graph compression method~\cite{ashc95} merges cliques of vertices
into single nodes, so as to speed-up the ordering of the compressed
graph. It also results in some improvement of the quality of
separators, especially for stiffness matrices.
\iteme[\textbf{Gibbs-Poole-Stockmeyer}]
This method is mainly used on separators to reduce the number and
extent of extra-diagonal blocks.
\iteme[\textbf{Simple method}]
Vertices are ordered consecutively, in the same order as they are
stored in the graph. This is the fastest method to use on separators
when the shape of extra-diagonal structures is not a concern.
\iteme[\textbf{Nested dissection}]
Incomplete nested dissection method. Separators are computed
recursively on subgraphs, and specific ordering methods are applied to
the separators and to the resulting subgraphs (see
sections~\ref{sec-algo-nested} and ~\ref{sec-algo-nested-hybrid}).
\iteme[\textbf{Disconnected subgraph detection}]
This method may be used as a pre-processing step so as to apply the same
ordering strategy on each of the disconnected components of a
graph. While finding the connected components of a graph is expensive,
it may bring an improvement on graph ordering quality in some cases.
\end{itemize}

\subsection{Graph separation methods}

The core of our incomplete nested dissection algorithm uses graph separation
methods as black boxes. It allows the orderer to run any type of graph
separation method compatible with our criteria for quality, that is,
reducing the size of the vertex separator while maintaining the loads of
the separated parts within some user-specified tolerance.
Separation jobs maintain an internal image of the current vertex separator,
indicating for every vertex of the job whether it is currently assigned to
one of the two parts, or to the separator.
It is therefore possible to apply several different methods in sequence,
each one starting from the result of the previous one,
and to select the methods with respect to the job characteristics, thus
enabling the definition of separation strategies.
\\

The currently implemented graph separation methods are listed below.
\begin{itemize}
\iteme[\textbf{Fiduccia-Mattheyses}]
This is a vertex-oriented version of the original, edge-oriented,
Fiduccia-Mattheyses heuristics described in page~\pageref{sec-algo-fme}.
\iteme[\textbf{Greedy graph growing}]
This is a vertex-oriented version of the edge-oriented
greedy graph growing algorithm described in page~\pageref{sec-algo-ggge}.
\iteme[\textbf{Multilevel}]
This is a vertex-oriented version of the edge-oriented
multilevel algorithm described in page~\pageref{sec-algo-mle}.
\iteme[\textbf{Thinner}]
This greedy algorithm refines the current separator by removing all of
the exceeding vertices, that is, vertices that do not have neighbors
in both parts. It is provided as a simple gradient refinement
algorithm for the multilevel method, and is clearly outperformed by
the Fiduccia-Mattheyses algorithm.
\iteme[\textbf{Vertex cover}]
This algorithm computes a vertex separator by first computing an edge
separator, that is, a bipartition of the graph, and then turning it
into a vertex separator by using the method proposed by Pothen and
Fang~\cite{pofa90}. This method requires the computation of maximal
matchings in the bipartite graphs associated with the edge cuts, which
are built using Duff's variant~\cite{duff81} of the Hopcroft and Karp
algorithm~\cite{hoka73}.
Edge separators are computed by using a bipartitioning strategy,
which can use any of the graph bipartitioning methods described in
section~\ref{sec-algo-bipart}, page~\pageref{sec-algo-bipart}.
\end{itemize}
                                  % Introduction
%%%%%%%%%%%%%%%%%%%%%%%%%%%%%%%%%%%%
%                                  %
% Titre  : s_c.tex                 %
% Sujet  : Manuel de l'utilisateur %
%          du projet 'Scotch'      %
%          Changes                 %
% Auteur : Francois Pellegrini     %
%                                  %
%%%%%%%%%%%%%%%%%%%%%%%%%%%%%%%%%%%%

\section{Updates}
\label{sec-changes}

\subsection{Changes in version 6.0 from version 5.1}

The new \texttt{sub} abstract target architecture allows one to map a
graph onto a subset of any given target architecture (including
another \texttt{sub} architecture). This feature is meant to perform
mappings onto potentially disconnected subsets of a parallel machine,
e.g. the set of nodes assigned by a batch scheduler; see
Section~\ref{sec-lib-arch-sub}, page~\pageref{sec-lib-arch-sub} for
further information.
Also, in order to allow decomposition-defined architectures to
scale-up to the sizes of modern machines, a new version of the
\texttt{deco} architecture, called \texttt{deco~2}, has been
designed. This target architecture can be created using the 
\texttt{SCOTCH\_\lbt arch\lbt Build2} routine; see
Section~\ref{sec-lib-arch-build-two},
page~\pageref{sec-lib-arch-build-two} for further information.
For further information on the rationale and implementation of these
two features, please refer to~\cite{pellegrini:hal-01671156}.

Direct k-way graph partitioning and static mapping methods are now
available. They are less expensive than the classical dual recursive
bipartitioning scheme, and improve quality on average for numbers of
parts above a few hundreds. Another new method aims at reducing load
imbalance in the case of source graphs with highly irregular vertex
weights; see Section~\ref{sec-algo-map-methods},
page~\pageref{sec-algo-map-methods}. Users willing to keep using the
old recursive bipartitioning strategies of the \textsc{5.x} branch can
create default strategies with the \texttt{SCOTCH\_\lbt STRATRECURSIVE}
flag set, in addition to other flags; see
Section~\ref{sec-lib-format-strat-default},
page~\pageref{sec-lib-format-strat-default} for further information.

Graph repartitioning and static re-mapping features are now available;
see Sections~\ref{sec-lib-func-graphmapfixed}
to~\ref{sec-lib-func-graphremapfixed}, starting from
page~\pageref{sec-lib-func-graphmapfixed}.

The clustering capabilities of \scotch\ can be used more easily from
the command line and library calls~; see Section~\ref{sec-prog-gmap}
and Section~\ref{sec-lib-func-stratgraphclusterbuild}.

A new set of routines has been created in order to compute
vertex-separated, k-way partitions, that balance the loads of the
parts and of the separator vertices that surround them; see
Sections~\ref{sec-lib-format-part-ovl}
and~\ref{sec-lib-func-graphpartovl}.

A new labeled tree-leaf architecture has been created, for nodes that
label cores in non increasing order. See
Section~\ref{sec-file-target-algo},
page~\pageref{sec-file-target-algo} for the description of the
\texttt{ltleaf} target architecture.

Memory footprint measurement routines are now available to users;
see Section~\ref{sec-lib-misc}, page~\pageref{sec-lib-misc}.

Key algorithms are now multi-threaded. See the installation file
\texttt{INSTALL.txt} in the main directory for instructions on how to
compile \scotch\ with thread support enabled.

A method for computing independently orderings on connected components
of a graph is now available; see Section~\ref{sec-lib-format-ord},
page~\pageref{sec-lib-format-ord}, and the \texttt{SCOTCH\_\lbt
STRAT\lbt DISCON\lbt NECTED} flag in
Section~\ref{sec-lib-format-strat-default},
page~\pageref{sec-lib-format-strat-default}.

\subsection{Changes in version 5.1 from version 5.0}

A new integer index type has been created in the Fortran interface, to
address array indices larger than the maximum value which can be
stored in a regular integer. Please refer to
Section~\ref{sec-install-inttypesize} for more information.

A new set of routines has been designed, to ease the use of the
\libscotch\ as a dynamic library. The \texttt{SCOTCH\_\lbt version}
routine returns the version, release and patch level numbers of the
library being used. The \texttt{SCOTCH\_\lbt *Alloc} routines,
which are only available in the C interface at the time being,
dynamically allocate storage space for the opaque API
\scotch\ structures, which frees application programs from the need
to be systematically recompiled because of possible changes of
\scotch\ structure sizes.
                                  % Changes since previous versions
%%%%%%%%%%%%%%%%%%%%%%%%%%%%%%%%%%%%
%                                  %
% Titre  : s_f.tex                 %
% Sujet  : Manuel de l'utilisateur %
%          du projet 'Scotch'      %
%          Formats de fichiers 6.0 %
% Auteur : Francois Pellegrini     %
%                                  %
%%%%%%%%%%%%%%%%%%%%%%%%%%%%%%%%%%%%

\section{Files and data structures}
\label{sec-file}

For the sake of portability, readability, and reduction of storage space,
all the data files shared by the different programs of the
\scotch\ project are coded in plain ASCII text exclusively.
Although we may speak of ``lines'' when describing file formats,
text-formatting characters such as newlines or tabulations are not
mandatory, and are not taken into account when files are read.
They are only used to provide better readability and understanding.
Whenever numbers are used to label objects, and unless explicitely
stated, \textbf{numberings always start from zero}, not one.

\subsection{Graph files}
\label{sec-file-sgraph}

Graph files, which usually end in ``\texttt{\@.grf}'' or
``\texttt{\@.src}'', describe valuated graphs, which can be valuated
process graphs to be mapped onto target architectures, or graphs
representing the adjacency structures of matrices to order.

Graphs are represented by means of adjacency lists: the definition
of each vertex is accompanied by the list of all of its neighbors, i.e.
all of its adjacent arcs. Therefore, the overall number of edge data is
twice the number of edges.
\\

Since version \textsc{3.3} has been introduced a new file format, referred
to as the ``new-style'' file format, which replaces the previous,
``old-style'', file format. The two advantages of the new-style
format over its predecessor are its greater compacity, which results
in shorter I/O times, and its ability to handle easily graphs output
by C or by Fortran programs.

Starting from version \textsc{4.0}, only the new format is supported. To
convert remaining old-style graph files into new-style graph files,
one should get version \textsc{3.4} of the \scotch\ distribution, which
comprises the \texttt{scv} file converter, and use it to produce
new-style \scotch\ graph files from the old-style \scotch\ graph files
which it is able to read. See section~\ref{sec-prog-gcv} for a
description of \texttt{gcv}, formerly called \texttt{scv}.
\\

The first line of a graph file holds the graph file version number,
which is currently \texttt{0}. The second line holds the number of
vertices of the graph (referred to as \texttt{vertnbr} in \libscotch; see
for instance Figure~\ref{fig-lib-graf-one},
page~\pageref{fig-lib-graf-one}, for a detailed example), followed by
its number of arcs (unappropriately called \texttt{edgenbr}, as it is in
fact equal to twice the actual number of edges). The third line holds
two figures: the graph base index value (\texttt{baseval}), and a numeric
flag.

The graph base index value records the value of the starting index
used to describe the graph; it is usually $0$ when the graph has been
output by C programs, and $1$ for Fortran programs. Its purpose is to
ease the manipulation of graphs within each of these two environments,
while providing compatibility between them.

The numeric flag, similar to the one used by the \chaco\ graph
format~\cite{hele93c}, is made of three decimal digits.
A non-zero value in the units indicates that vertex weights are provided.
A non-zero value in the tenths indicates that edge weights are provided.
A non-zero value in the hundredths indicates that vertex labels are provided;
if it is the case, vertices can be stored in any order in the file; else,
natural order is assumed, starting from the graph base index.

This header data is then followed by as many lines as there are
vertices in the graph, that is, \texttt{vertnbr} lines. Each of these
lines begins with the vertex label, if necessary, the vertex load, if
necessary, and the vertex degree, followed by the description of the
arcs. An arc is defined by the load of the edge, if necessary, and by
the label of its other end vertex.
The arcs of a given vertex can be provided in any order in its
neighbor list. If vertex labels are provided, vertices can also be
stored in any order in the file.

Figure~\ref{fig-file-sgraph} shows the contents of a graph file
modeling a cube with unity vertex and edge weights and base $0$.

\begin{figure}[hbt]
\begin{center}
\begin{minipage}{7.3cm}
{\renewcommand{\baselinestretch}{1.05}
 \footnotesize \tt
\begin{verbatim}
0
8       24
0       000
3       4       2       1
3       5       3       0
3       6       0       3
3       7       1       2
3       0       6       5
3       1       7       4
3       2       4       7
3       3       5       6
\end{verbatim}}
\end{minipage}
\end{center}
\caption{Graph file representing a cube.}
\label{fig-file-sgraph}
\end{figure}

\subsection{Mesh files}
\label{sec-file-smesh}

Mesh files, which usually end in ``\texttt{\@.msh}'', describe valuated
meshes, made of elements and nodes, the elements of which can be
mapped onto target architectures, and the nodes of which can be
reordered.

Meshes are bipartite graphs, in the sense that every element is
connected to the nodes that it comprises, and every node is connected
to the elements to which it belongs. No edge connects any two element
vertices, nor any two node vertices.  One can also think of meshes as
hypergraphs, such that nodes are the vertices of the hypergraph and
elements are hyper-edges which connect multiple nodes, or reciprocally
such that elements are the vertices of the hypergraph and nodes are
hyper-edges which connect multiple elements.

Since meshes are graphs, the structure of mesh files resembles very
much the one of graph files described above in
section~\ref{sec-file-sgraph}, and differs only by its header, which
indicates which of the vertices are node vertices and element
vertices.
\\

The first line of a mesh file holds the mesh file version number,
which is currently \texttt{1}. Graph and mesh version numbers will always
differ, which enables application programs to accept both file formats
and adapt their behavior according to the type of input data.  The
second line holds the number of elements of the mesh (\texttt{velmnbr}),
followed by its number of nodes (\texttt{vnodnbr}), and by its overall
number of arcs (\texttt{edgenbr}, that is, twice the number of edges
which connect elements to nodes and vice-versa).

The third line holds three figures: the base index of the first
element vertex in memory (\texttt{velmbas}), the base index of the first
node vertex in memory (\texttt{vnodbas}), and a numeric flag.

The \scotch\ mesh file format requires that all nodes and all elements
be assigned to contiguous ranges of indices. Therefore, either all
element vertices are defined before all node vertices, or all node
vertices are defined before all element vertices. The node and element
base indices indicate at the same time whether elements or nodes are
put in the first place, as well as the value of the starting index
used to describe the graph. Indeed, if
$\mbox{\texttt{velm}}\-\mbox{\texttt{bas}} <
\mbox{\texttt{vnod}}\-\mbox{\texttt{bas}}$, then elements have the 
smallest indices, \texttt{velmbas} is the base value of the underlying
graph (that is, \texttt{baseval} = \texttt{velmbas}), and
$\mbox{\texttt{velmbas}} + \mbox{\texttt{velmnbr}} =
\mbox{\texttt{vnodbas}}$
holds. Conversely, if $\mbox{\texttt{velm}}\-\mbox{\texttt{bas}}
> \mbox{\texttt{vnod}}\-\mbox{\texttt{bas}}$, then nodes have the
smallest indices, \texttt{vnodbas} is the base value of the underlying
graph, (that is, \texttt{baseval} = \texttt{vnodbas}), and
$\mbox{\texttt{vnodbas}} + \mbox{\texttt{vnodnbr}} =
\mbox{\texttt{velmbas}}$ holds.

The numeric flag, similar to the one used by the \chaco\ graph
format~\cite{hele93c}, is made of three decimal digits.  A non-zero
value in the units indicates that vertex weights are provided.  A
non-zero value in the tenths indicates that edge weights are provided.
A non-zero value in the hundredths indicates that vertex labels are
provided; if it is the case, and if
$\mbox{\texttt{velm}}\-\mbox{\texttt{bas}} <
\mbox{\texttt{vnod}}\-\mbox{\texttt{bas}}$
(resp\@. $\mbox{\texttt{velm}}\-\mbox{\texttt{bas}} >
\mbox{\texttt{vnod}}\-\mbox{\texttt{bas}}$), the \texttt{velmnbr}
(resp\@. \texttt{vnodnbr}) first vertex lines are assumed to be element
(resp\@. node) vertices, irrespective of their vertex labels, and the
\texttt{vnodnbr} (resp\@. \texttt{velmnbr}) remaining vertex lines are
assumed to be node (resp\@. element) vertices; else, natural order is
assumed, starting at the underlying graph base index (\texttt{baseval}).

This header data is then followed by as many lines as there are
node and element vertices in the graph. These lines are similar
to the ones of the graph format, except that, in order to save
disk space, the numberings of nodes and elements all start from
the same base value, that is,
$\min(\mbox{\texttt{velm}}\-\mbox{\texttt{bas}},
\mbox{\texttt{vnod}}\-\mbox{\texttt{bas}})$ (also called
\texttt{baseval}, like for regular graphs).

For example, Figure~\ref{fig-file-smesh} shows the contents of the
mesh file modeling three square elements, with unity vertex and edge
weights, elements defined before nodes, and numbering of the
underlying graph starting from $1$. In memory, the three elements are
labeled from $1$ to $3$, and the eight nodes are labeled from $4$ to
$11$. In the file, the three elements are still labeled from $1$ to $3$,
while the eight nodes are labeled from $1$ to $8$.

When labels are used, elements and nodes may have similar labels,
but not two elements, nor two nodes, should have the same labels.

\begin{figure}[hbt]
\begin{center}
\includegraphics[scale=0.65]{s_f_msf.eps}
\hfil ~\hfil
\begin{minipage}[b]{7cm}
\verb+1+

\noi
\verb+3       8       24+

\noi
\verb+1       4       000+

\verb+4       2 +\makebox[0em][l]{\tiny (= 5)}\verb+      8 +\makebox[0em][l]{\tiny (= 11)}\verb+      4 +\makebox[0em][l]{\tiny (= 7)}\verb+      3 +\makebox[0em][l]{\tiny (= 6)}

\noi
\verb+4       7 +\makebox[0em][l]{\tiny (= 10)}\verb+      2 +\makebox[0em][l]{\tiny (= 5)}\verb+      8 +\makebox[0em][l]{\tiny (= 11)}\verb+      1 +\makebox[0em][l]{\tiny (= 4)}

\noi
\verb+4       5 +\makebox[0em][l]{\tiny (= 8)}\verb+      6 +\makebox[0em][l]{\tiny (= 9)}\verb+      3 +\makebox[0em][l]{\tiny (= 6)}\verb+      4 +\makebox[0em][l]{\tiny (= 7)}

\noi
\verb+1       2+

\noi
\verb+2       2       1+

\noi
\verb+2       1       3+

\noi
\verb+2       1       3+

\noi
\verb+1       3+

\noi
\verb+1       3+

\noi
\verb+1       2+

\noi
\verb+2       2       1+
%\begin{verbatim}
%1
%3       8       24
%1       4       000
%4       5       11      7       6
%4       10      5       11      4
%4       8       9       6       7
%1       2
%2       2       1
%2       1       3
%2       1       3
%1       3
%1       3
%1       2
%2       2       1
%\end{verbatim}
\end{minipage}
\end{center}
\caption{Mesh file representing three square elements, with unity
vertex and edge weights. Elements are defined before nodes, and
numbering of the underlying graph starts from $1$. The left part of
the figure shows the mesh representation in memory, with consecutive
element and node indices. The right part of the figure shows the
contents of the file, with both element and node numberings starting
from $1$, the minimum of the element and node base values.
Corresponding node indices in memory are shown in parentheses for the
sake of comprehension.}
\label{fig-file-smesh}
\end{figure}

\subsection{Geometry files}
\label{sec-file-geom}

Geometry files, which usually end in ``\texttt{\@.xyz}'', hold the coordinates
of the vertices of their associated graph or mesh.
These files are not used in the mapping process itself, since only
topological properties are taken into account then (mappings are
computed regardless of graph geometry).
They are used by visualization programs to compute
graphical representations of mapping results.

The first string to appear in a geometry file codes for its type, or
dimensionality. It is ``\texttt{1}'' if the file contains unidimensional
coordinates, ``\texttt{2}'' for bidimensional coordinates, and ``\texttt{3}'' for
tridimensional coordinates.
It is followed by the number of coordinate data stored in the file, which
should be at least equal to the number of vertices of the associated graph
or mesh, and by that many coordinate lines.
Each coordinate line holds the label of the vertex, plus one, two or three
real numbers which are the (X), (X,Y), or (X,Y,Z), coordinates of the graph
vertices, according to the graph dimensionality.
\\
Vertices can be stored in any order in the file. Moreover, a geometry
file can have more coordinate data than there are vertices in the
associated graph or mesh file; only coordinates the labels of which
match labels of graph or mesh vertices will be taken into account.
This feature allows all subgraphs of a given graph or mesh to share the
same geometry file, provided that graph vertex labels remain unchanged.
For example, Figure~\ref{fig-file-geom} shows the contents of the 3D~geometry
file associated with the graph of Figure~\ref{fig-file-sgraph}.
\begin{figure}[hbt]
\begin{center}
\begin{minipage}{4.6cm}
{\renewcommand{\baselinestretch}{1.05}
 \footnotesize \tt
\begin{verbatim}
3
8
0       0.0     0.0     0.0
1       0.0     0.0     1.0
2       0.0     1.0     0.0
3       0.0     1.0     1.0
4       1.0     0.0     0.0
5       1.0     0.0     1.0
6       1.0     1.0     0.0
7       1.0     1.0     1.0
\end{verbatim}
}\end{minipage}
\end{center}
\caption{Geometry file associated with the graph file of
         Figure~\protect\ref{fig-file-sgraph}.}
\label{fig-file-geom}
\end{figure}

\subsection{Target files}
\label{sec-file-target}

Target files describe the architectures onto which source graphs are mapped.
Instead of containing the structure of the target graph itself, as source
graph files do, target files define how target graphs are bipartitioned and
give the distances between all pairs of vertices (that is, processors).
Keeping the bipartitioning information within target files avoids
recomputing it every time a target architecture is used.
We are allowed to do so because, in our approach, the recursive
bipartitioning of the target graph is fully independent with respect to that
of the source graph (however, the opposite is false).

For space and time saving issues, some classical homogeneous architectures
(2D and 3D meshes and tori, hypercubes, complete graphs, etc\@.) have been
algorithmically coded within the mapper itself by the means of built-in
functions.
Instead of containing the whole graph decomposition data, their target
files hold only a few values, used as parameters by the built-in functions.

\subsubsection{Decomposition-defined architecture files}
\label{sec-file-target-deco}

Decomposition-defined architecture files are the way to describe
irregular target architectures that cannot be represented as
algorithmically-coded architectures.

Two main file formats coexist~: the ``\texttt{deco 0}'' and
``\texttt{deco 2}'' formats. ``\texttt{deco}'' stands for
``decomposition-defined architecture'', followed by the format
number. The ``\texttt{deco 1}'' format is a compiled form of the
``\texttt{deco 0}'' format, which we will not describe here as it is
not meant to be handled by users.

The ``\texttt{deco 0}'' header is followed by two integer numbers,
which are the number of processors and the largest terminal number used
in the decomposition, respectively. Two arrays follow.
The first array has as many lines as there are processors. Each of
these lines holds three numbers: the processor label, the processor
weight (that is an estimation of its computational power), and its terminal
number.
The terminal number associated with every processor is obtained by giving the
initial domain holding all the processors number $1$, and by numbering the
two subdomains of a given domain of number $i$ with numbers $2i$ and $2i+1$.
The second array is a lower triangular diagonal-less matrix that gives the
distance between all pairs of processors. This distance matrix, combined with
the decomposition tree coded by terminal numbers, allows the evaluation
by averaging of the distance between all pairs of domains.
In order for the mapper to behave properly, distances between processors must
be strictly positive numbers. Therefore, null distances are not accepted.
For instance, Figure~\ref{fig-file-targetdeco} shows the contents of the
architecture decomposition file for $\UB(2,3)$, the binary de~Bruijn graph of
dimension~$3$, as computed by the \texttt{amk\_grf} program.
\begin{figure}[hbt]
\begin{tabular}{p{0.69\linewidth}@{}p{0.29\linewidth}}
\begin{center}
\parbox[t]{0.9\linewidth}{\vspace{0pt}\includegraphics[width=0.7\linewidth]{s_f_d.ps}}
\end{center}
&
\begin{center}
{\renewcommand{\baselinestretch}{1.05}
\footnotesize\tt
\begin{verbatim}
deco 0
8	15
0	1	15
1	1	14
2	1	13
3	1	11
4	1	12
5	1	9
6	1	8
7	1	10
1
2 1
2 1 2
1 1 1 2
3 2 1 1 2
2 2 2 1 1 1
3 2 3 1 2 2 1
\end{verbatim}
}
\end{center}
\end{tabular}
\caption{Target decomposition file for $\UB(2,3)$.
         The terminal numbers associated with every processor define a unique
         recursive bipartitioning of the target graph.}
\label{fig-file-targetdeco}
\end{figure}

The ``\texttt{deco 2}'' format was created so as to represent bigger
target architectures. Indeed, the distance matrix of the
``\texttt{deco 0}'' format is quadratic in the number of target
vertices, which is not scalable and prevents users from representing
target architectures bigger than a few thousand vertices. In the
``\texttt{deco 2}'' architecture, distances are computed using in a
multilevel representation of the target graph, in the form of a family
of coarser graphs. Hence, the more distant the vertices are, the
coarsest is the graph to be used to estimate this
distance~\cite{pellegrini:hal-01671156}. The vertices and edges of
these graphs encode their respective cost of traversal, which becomes
less accurate as coarser graphs are used.

\subsubsection{Algorithmically-coded architecture files}
\label{sec-file-target-algo}

Almost all algorithmically-coded architectures are defined with unity
edge and vertex weights. They start with an abbreviation name of the
architecture, followed by parameters specific to the architecture. The
available built-in architecture definitions are listed below.
\begin{itemize}
\iteme[{\texttt{cmplt} {\it size}}]
Defines a complete graph with $\mathit{size}$ vertices.
Its vertex labels are numbers between $0$ and $\mathit{size} - 1$.
%%
\iteme[{\texttt{cmpltw} {\it size} {\it load$_0$} {\it load$_1$}
\ldots\ {\it load$_{\mathit{size} - 1}$}}]
Defines a weighted complete graph with {\it size\/} vertices.
Its vertex labels are numbers between $0$ and $\mathit{size} - 1$,
and vertices are assigned integer weights in the order in which
these are provided.
%%
\iteme[{\texttt{hcub} $\mathit{dim}$}]
Defines a binary hypercube of dimension $\mathit{dim}$.
Graph vertices are numbered according to the value of the binary
representation of their coordinates in the hypercube.
%%
\iteme[{\texttt{ltleaf}
\parbox[t]{11cm}{$\mathit{levlnbr}$ $\mathit{sizeval}_0$ $\mathit{linkval}_0$
\ldots\ $\mathit{sizeval}_{\mathit{levlnbr}-1}$
$\mathit{linkval}_{\mathit{levlnbr}-1}$
\\
$\mathit{permnbr}$ $\mathit{permval}_0$
\ldots\ $\mathit{permval}_{\mathit{permnbr}-1}$}}]
\label{sec-file-target-ltleaf}
The \texttt{ltleaf} (for ``\textit{labeled tree-leaf}'') architecture is
an extended tree-leaf architecture (\texttt{tleaf}, see below) which
models target topologies where cores are not labeled in increasing
order.
\\
The tree structure of the architecture is described just like for a
regular \texttt{tleaf} architecture. $\mathit{permnbr}$ is the length
of the permutation that is used to label cores, followed by this
number of permutation indices, ranging between $0$ and
$(\mathit{permnbr}-1)$. Figure~\ref{fig-file-targetltleaf} presents an
example of such an architecture.
\\
The permutation array must be of a size that matches level
boundaries. Alternatively, a permutation of size $1$, with only index
$0$ given, represents the identity permutation. In this case, the
regular \texttt{tleaf} architecture can be used.
\begin{figure}[hbt]
\begin{center}
\begin{minipage}[b]{6cm}
{\renewcommand{\baselinestretch}{1.05}
\footnotesize\tt
\begin{verbatim}
ltleaf
3 32 10 2 5 4 1
8 0 2 4 6 1 3 5 7
\end{verbatim}
}\end{minipage}
\end{center}
\caption{Labeled tree-leaf architecture with $3$ levels, representing
a system with $32$ nodes of $2$ quad-core processors. Inter-node
communication costs $10$, inter-processor communication within the
same node costs $5$ and inter-core communication within the same
processor costs $1$. Within a $8$-core node, cores are labeled such
that cores $0$, $2$, $4$ and $6$ are located on the first processor,
while cores $1$, $3$, $5$ and $7$ are located on the second processor.}
\label{fig-file-targetltleaf}
\end{figure}
%%
\iteme[{\texttt{mesh2D} {\it dim$_X$} {\it dim$_Y$}}]
Defines a bidimensional array of {\it dim$_X$} columns by {\it dim$_Y$}
rows. The vertex with coordinates $(\mathit{pos_X},\mathit{pos_Y})$
has label $\mathit{pos_X} + \mathit{pos_Y} \times \mathit{dim_X}$.
%%
\iteme[{\texttt{mesh3D} {\it dim$_X$} {\it dim$_Y$} {\it dim$_Z$}}]
Defines a tridimensional array of {\it dim$_X$} columns by {\it dim$_Y$}
rows by {\it dim$_Z$} levels. The vertex with coordinates
($\mathit{pos_X},\mathit{pos_Y},\mathit{pos_Z}$) has label
$\mathit{pos_X} + \mathit{pos_Y} \mathit{dim_X} + \mathit{pos_Z} \mathit{dim_X} \mathit{dim_Y}$.
%%
\iteme[{\texttt{meshXD} {\it ndims} {\it dim$_0$} {\it dim$_1$} \ldots
{\it dim$_{(ndims - 1)}$}}]
Generalization of the \texttt{mesh2D} and \texttt{mesh3D}
architectures. Defines a \textit{ndims}-dimensional array of
dimensions \textit{dim$_0$}, \textit{dim$_1$} \ldots
\textit{dim$_{ndims - 1}$}. The vertex with coordinates 
($\mathit{pos_0},\mathit{pos_1},\ldots,\mathit{pos_{ndims - 1}}$)
has label $\mathit{pos_0} + \sum_{d=1}^{ndims - 1}\left(\mathit{pos_d} \prod_{d'=0}^{d-1}\mathit{dim_{d'}}\right)$.
%%
\iteme[{\texttt{sub} $\mathit{termnbr}$ $\mathit{termnum}_0$
$\mathit{termnum}_1$ \ldots\ $\mathit{termnum}_{\mathit{termnbr}-1}$
$\mathit{architecture}$}]
Defines a sub-architecture of another \textit{architecture}. The
sub-architecture contains $\mathit{termnbr}$ vertices, which have
ranks $\mathit{termnum}_0$, $\mathit{termnum}_1$,
\ldots\ $\mathit{termnum}_{\mathit{termnbr}-1}$ in the prescribed,
original $\mathit{architecture}$. The original architecture must
comprise at least $\mathit{termnbr}$ vertices, and thus cannot be a
variable-sized architecture. The order in which vertex numbers are
provided defines the part indices that will be used as output
mapping data. For instance, in the example shown in
Figure~\ref{fig-file-targetsub}, source vertices that are assigned to
vertex $3$ of the sub-architecture are in fact assigned to vertex $5$
of the original, 2D mesh architecture, according to its canonical
numbering.
\begin{figure}[hbt]
\begin{tabular}{p{0.69\linewidth}@{}p{0.29\linewidth}}
\begin{center}
\parbox[t]{0.9\linewidth}{\vspace{0pt}\includegraphics[width=0.7\linewidth]{m42_a1}}
\end{center}
&
\begin{center}
{\renewcommand{\baselinestretch}{1.05}
\footnotesize\tt
\begin{verbatim}


sub
5 0 4 1 5 7
mesh2D 4 2
\end{verbatim}
}
\end{center}
\end{tabular}
\caption{Sub-architecture of a 4x2 \texttt{mesh2D} 2D grid
  architecture. The sub-architecture comprises $5$ vertices, numbered
  from $0$ to $4$, which correspond to vertices $0$, $4$, $1$, $5$ and
  $7$ of the original architecture, respectively.}
\label{fig-file-targetsub}
\end{figure}
%%
\iteme[{\texttt{tleaf} $\mathit{levlnbr}$ $\mathit{sizeval}_0$
$\mathit{linkval}_0$ \ldots\ $\mathit{sizeval}_{\mathit{levlnbr}-1}$
$\mathit{linkval}_{\mathit{levlnbr}-1}$}]
Defines a hierarchical, tree-shaped, architecture with $\mathit{levlnbr}$
levels and $\sum_{i=0}^{\mathit{levlnbr}-1}\mathit{sizeval}_i$
leaf vertices. This topology is used to model hierarchical NUMA or NUIOA
machines. The mapping is only computed with respect to the leaf
vertices, which represent processing elements, while the upper levels of
the tree model interconnection networks (intra-chip buses, inter-chip
interconnection networks, network routers, etc.), as exemplified in
Figure~\ref{fig-graf-treeleaf}. The communication cost between two
nodes is the cost of the highest common ancestor level.
\begin{figure}[hbt]
\begin{tabular}{p{0.69\linewidth}@{}p{0.29\linewidth}}
\begin{center}
\parbox[t]{0.9\linewidth}{\vspace{0pt}\includegraphics[width=0.95\linewidth]{s_f_lea.eps}}
\end{center}
&
\begin{center}
{\renewcommand{\baselinestretch}{1.05}
\footnotesize\tt
\begin{verbatim}


tleaf
3 3 20 2 7 2 2
\end{verbatim}
}
\end{center}
\end{tabular}
\caption{A ``tree-leaf'' graph with three levels. Processors are drawn
in black and routers in grey. It has $3$ levels, the first level has
$3$ sons and a traversal cost of $20$, the second level has $2$ sons
and a traversal cost of $7$, and the third level has also $2$ sons and
a traversal cost of $2$.}
\label{fig-graf-treeleaf}
\end{figure}
%%
\iteme[{\texttt{torus2D} {\it dim$_X$} {\it dim$_Y$}}]
Defines a bidimensional array of {\it dim$_X$} columns by {\it dim$_Y$}
rows, with wraparound edges.
The vertex with coordinates $(\mathit{pos_X},\mathit{pos_Y})$ has label
$\mathit{pos_X} + \mathit{pos_Y} \times \mathit{dim_X}$.
%%
\iteme[{\texttt{torus3D} {\it dim$_X$} {\it dim$_Y$} {\it dim$_Z$}}]
Defines a tridimensional array of {\it dim$_X$} columns by {\it dim$_Y$}
rows by {\it dim$_Z$} levels, with wraparound edges. The vertex with
coordinates $(\mathit{pos_X},\mathit{pos_Y},\mathit{pos_Z})$ has
label
$\mathit{pos_X} + \mathit{pos_Y} \mathit{dim_X} + \mathit{pos_Z} \mathit{dim_X} \mathit{dim_Y}$.
%%
\iteme[{\texttt{torusXD} {\it ndims} {\it dim$_0$} {\it dim$_1$} \ldots
{\it dim$_{ndims - 1}$}}]
Generalization of the \texttt{torus2D} and \texttt{torus3D}
architectures. Defines a \textit{ndims}-dimensional torus of
dimensions \textit{dim$_0$}, \textit{dim$_1$} \ldots
\textit{dim$_{ndims - 1}$}. The vertex with coordinates 
($\mathit{pos_0},\mathit{pos_1},\ldots,\mathit{pos_{(ndims - 1)}}$)
has label $\mathit{pos_0} + \sum_{d=1}^{ndims - 1}\left(\mathit{pos_d} \prod_{d'=0}^{d-1}\mathit{dim_{d'}}\right)$.
\end{itemize}

\subsubsection{Variable-sized architecture files}
\label{sec-file-target-variable}
\index{Clustering}

Variable-sized architectures are a class of algorithmically-coded
architectures the size of which is not defined {\it a priori}. Domains
of these target architectures can always be bipartitioned, again and
again (until integer overflow occurs in domain indices). These
architectures are used to perform graph clustering (see
Sections~\ref{sec-prog-gmap} and~\ref{sec-lib-func-graphmap}),
using a specifically tailored graph mapping strategy (see for instance
Section~\ref{sec-lib-func-stratgraphclusterbuild}).

As for fixed-size algorithmically-coded architectures, they start with
an abbreviation name of the architecture, followed by parameters
specific to the architecture. The available built-in variable-sized
architecture definitions are listed below.
\begin{itemize}
\iteme[{\texttt{varcmplt}}]
Defines a variable-sized complete graph. Domains are labeled such
that the first domain is labeled $1$, and the two subdomains of
any domain $i$ are labeled $2i$ and $2i + 1$. The distance between
any two subdomains $i$ and $j$ is $0$ if $i=j$ and $1$ else.
\iteme[{\texttt{varhcub}}]
Defines a variable-sized hypercube. Domains are labeled such that
the first domain is labeled $1$, and the two subdomains of any domain
$i$ are labeled $2i$ and $2i + 1$. The distance between any two
domains is the Hamming distance between the common bits of the two
domains, plus half of the absolute difference between the levels of
the two domains, this latter term modeling the average distance on
unknown bits.
For instance, the distance between subdomain $9=1001_B$, of level $3$
(since its leftmost $1$ has been shifted left thrice), and subdomain
$53=110101_B$, of level $5$ (since its leftmost $1$ has been shifted
left five times), is equal to $2$: it is $1$, which is the number of
bits which differ between $1101_B$ (that is, $53=110101_B$ shifted
rightwards twice) and $1001_B$, plus $1$, which is half of the
absolute difference between $5$ and $3$.
\end{itemize}

\subsection{Mapping files}
\label{sec-file-map}

Mapping files, which usually end in ``\texttt{\@.map}'', contain the
result of the mapping of source graphs onto target architectures. They
associate a vertex of the target graph with every vertex of the source
graph.

Mapping files begin with the number of mapping lines which they contain,
followed by that many mapping lines.
Each mapping line holds a mapping pair, made of two integer numbers
which are the label of a source graph vertex and the label
of the target graph vertex onto which it is mapped.
Mapping pairs can be stored in any order in the file; however, labels of
source graph vertices must be all different.
For example, Figure~\ref{fig-file-mapping} shows the result obtained when
mapping the source graph of Figure~\ref{fig-file-sgraph} onto the target
architecture of Figure~\ref{fig-file-targetdeco}.
This one-to-one embedding of $\HY(3)$ into $\UB(2,3)$ has dilation~$1$,
except for one hypercube edge which has dilation~$3$.
\begin{figure}[hbt]
\begin{center}
\begin{minipage}{3cm}
{\renewcommand{\baselinestretch}{1.05}
\footnotesize\tt
\begin{verbatim}
8
0       1
1       3
2       2
3       5
4       0
5       7
6       4
7       6
\end{verbatim}
}\end{minipage}
\end{center}
\caption{Mapping file obtained when mapping the hypercube source graph of
         Figure~\protect\ref{fig-file-sgraph} onto the binary de~Bruijn
         architecture of Figure~\protect\ref{fig-file-targetdeco}.}
\label{fig-file-mapping}
\end{figure}

Mapping files are also used on output of the block orderer to
represent the allocation of the vertices of the original graph to the
column blocks associated with the ordering. In this case, column blocks
are labeled in ascending order, such that the number of a block is
always greater than the ones of its predecessors in the elimination
process, that is, its leaves in the elimination tree.

\subsection{Ordering files}
\label{sec-file-ord}

Ordering files, which usually end in ``\texttt{\@.ord}'', contain the
result of the ordering of source graphs or meshes that represent
sparse matrices. They associate a number with every vertex of the
source graph or mesh.

The structure of ordering files is analogous to the one of mapping
files; they differ only by the meaning of their data.

Ordering files begin with the number of ordering lines which they
contain, that is the number of vertices in the source graph or the
number of nodes in the source mesh, followed by that many ordering
lines. Each ordering line holds an ordering pair, made of two integer
numbers which are the label of a source graph or mesh vertex and its
rank in the ordering. Ranks range from the base value of the graph or
mesh (\texttt{baseval}) to the base value plus the number of vertices
(resp\@. nodes), minus one ($\mbox{\texttt{baseval}} +
\mbox{\texttt{vertnbr}} - 1$ for graphs, and $\mbox{\texttt{baseval}}
+ \mbox{\texttt{vnodnbr}} - 1$ for meshes).  Ordering pairs can be
stored in any order in the file; however, indices of source vertices
must be all different.

For example, Figure~\ref{fig-file-ordering} shows the result obtained
when reordering the source graph of Figure~\ref{fig-file-sgraph}.
\begin{figure}[hbt]
\begin{center}
\begin{minipage}{3cm}
{\renewcommand{\baselinestretch}{1.05}
\footnotesize\tt
\begin{verbatim}
8
0       6
1       3
2       2
3       7
4       1
5       5
6       4
7       0
\end{verbatim}
}\end{minipage}
\end{center}
\caption{Ordering file obtained when reordering the hypercube graph of
Figure~\protect\ref{fig-file-sgraph}.}
\label{fig-file-ordering}
\end{figure}

The advantage of having both graph and mesh orderings start from
\texttt{baseval} (and not \texttt{vnodbas} in the case of meshes) is
that an ordering computed on the nodal graph of some mesh has the same
structure as an ordering computed from the native mesh structure,
allowing for greater modularity. However, in memory, permutation
indices for meshes are numbered from \texttt{vnodbas} to
$\mbox{\texttt{vnodbas}} + \mbox{\texttt{vnodnbr}} - 1$.

\subsection{Vertex list files}

Vertex lists are used by programs that select vertices from graphs.

Vertex lists are coded as lists of integer numbers.
The first integer is the number of vertices in the list and
the other integers are the labels of the selected vertices, given in
any order.
For example, Figure~\ref{fig-file-vertex} shows the list made from
three vertices of labels $2$, $45$, and $7$.
\begin{figure}[hbt]
\begin{center}
\begin{minipage}{3cm}
{\renewcommand{\baselinestretch}{1.05}
 \footnotesize \tt \begin{verbatim}
3   2   45   7
\end{verbatim}}
\end{minipage}
\end{center}
\caption{Example of vertex list with three vertices of labels~$2$, $45$, and~$7$.}
\label{fig-file-vertex}
\end{figure}
                                  % Formats de fichiers
%%%%%%%%%%%%%%%%%%%%%%%%%%%%%%%%%%%%
%                                  %
% Titre  : s_p.tex                 %
% Sujet  : Manuel de l'utilisateur %
%          du projet 'Scotch'      %
%          Programmes              %
% Auteur : Francois Pellegrini     %
%                                  %
%%%%%%%%%%%%%%%%%%%%%%%%%%%%%%%%%%%%

\section{Programs}
\label{sec-prog}

The programs of the \scotch\ project belong to five distinct classes.
\begin{itemize}
\item
Graph handling programs, the names of which begin in ``\texttt{g}'',
that serve to build and test source graphs.
\item
Mesh handling programs, the names of which begin in ``\texttt{m}'',
that serve to build and test source meshes.
\item
Target architecture handling programs, the names of which begin
in ``\texttt{a}'', that allow the user to build and test decomposition-defined
target files, and especially to turn a source graph file into a target file.
\item
The mapping and ordering programs themselves.
\item
Output handling programs, which are
the mapping performance analyzer, the graph factorization program,
and the graph, matrix, and mapping visualization program.
\end{itemize}
The general architecture of the \scotch\ project is displayed in
Figure~\ref{fig-synp}.

\begin{figure}[p]
\centering{\hspace*{-5em}\includegraphics[scale=0.56]{s_f_synp.eps}}
\caption{General architecture of the \scotch\ project. All of the
features offered by the stand-alone programs are also available in
the \libscotch\ library.}
\label{fig-synp}
\end{figure}

\subsection{Invocation}

The programs comprising the \scotch\ project have been designed to run in
command-line mode without any interactive prompting,
so that they can be called easily from other programs by means of
``\mbox{\texttt{system$\,$()}}'' or ``\mbox{\texttt{ popen$\,$()}}''
system calls, or be piped together on a single shell command line. In
order to facilitate this, whenever a stream name is asked for (either
on input or output), the user may put a single ``\texttt{-}'' to
indicate standard input or output. Moreover, programs read their input
in the same order as stream names are given in the command line. It
allows them to read all their data from a single stream (usually the
standard input), provided that these data are ordered properly.

A brief on-line help is provided with all the programs. To get this help,
use the ``\texttt{-h}'' option after the program name.
The case of option letters is not significant, except when
both the lower and upper cases of a letter have different meanings.
When passing parameters to the programs, only the order of file names is
significant; options can be put anywhere in the command line, in any order.
Examples of use of the different programs of the \scotch\ project are provided
in section~\ref{sec-examples}.

Error messages are standardized, but may not be fully explanatory.
However, most of the errors you may run into should be related to file
formats, and located in ``\mbox{\texttt{ \ldots Load}}'' routines.
In this case, compare your data formats with the definitions
given in section~\ref{sec-file}, and use the \texttt{gtst} and \texttt{mtst}
programs to check the consistency of source graphs and meshes.

\subsection{Using compressed files}
\label{sec-prog-compressed}

Starting from version {\sc 5.0.6}, \scotch\ allows users to provide
and retrieve data in compressed form. Since this feature requires that
the compression and decompression tasks run in the same time as data
is read or written, it can only be done on systems which support
multi-threading (Posix threads) or multi-processing (by means of
\texttt{fork} system calls).

To determine if a stream has to be handled in compressed form,
\scotch\ checks its extension. If it is ``\texttt{.gz}'' (\texttt{gzip}
format), ``\texttt{.bz2}'' (\texttt{bzip2} format) or ``\texttt{.lzma}''
(\texttt{lzma} format), the stream is assumed to be compressed according
to the corresponding format. A filter task will then be used to process
it accordingly if the format is implemented in \scotch\ and enabled on
your system.

To date, data can be read and written in \texttt{bzip2} and \texttt{gzip}
formats, and can also be read in the \texttt{lzma} format. Since the
compression ratio of \texttt{lzma} on \scotch\ graphs is $30\%$ better
than the one of \texttt{gzip} and \texttt{bzip2} (which are almost
equivalent in this case), the \texttt{lzma} format is a very good choice
for handling very large graphs. To see how to enable compressed data
handling in \scotch, please refer to Section~\ref{sec-install}.
\\

When the compressed format allows it, several files can be provided on
the same stream, and be uncompressed on the fly. For instance, the
command ``\texttt{cat brol.grf.gz brol.xyz.gz | gout -.gz -.gz -Mn -
brol.iv}'' concatenates the topology and geometry data of some graph
\texttt{brol} and feed them as a single compressed stream to the standard
input of program \texttt{gout}, hence the ''\texttt{-.gz}'' to indicate a
compressed standard stream.

\subsection{Description}

\subsubsection{\texttt{acpl}}

\begin{itemize}
\progsyn
\texttt{acpl} [{\it input\_target\_file} [{\it output\_target\_file}]] {\it options}

\progdes
The program \texttt{acpl} is the decomposition-defined architecture file compiler.
It processes architecture files of type ``\texttt{deco~0}'' built by hand or by
the \texttt{amk\_}* programs, to create a ``\texttt{deco~1}'' compiled
architecture file of about four times the size of the original one;
see section~\ref{sec-file-target-deco}, page~\pageref{sec-file-target-deco},
for a detailed description of decomposition-defined target
architecture file formats.
\\
The mapper can read both original and compiled architecture file formats.
However, compiled architecture files are read much more efficiently, as
they are directly loaded into memory without further processing.
Since the compilation time of a target architecture graph evolves
as the square of its number of vertices, precompiling with \texttt{acpl} can
save some time when many mappings are to be performed onto the same large
target architecture.

\progopt
\begin{itemize}
\iteme[\texttt{-h}]
Display the program synopsis.
\iteme[\texttt{-V}]
Print the program version and copyright.
\end{itemize}
\end{itemize}

\subsubsection{\texttt{amk\_}*}
\relax
\begin{itemize}
\progsyn
\texttt{amk\_ccc} {\it dim} [{\it output\_target\_file}] {\it options}\\
~\\
\texttt{amk\_fft2} {\it dim} [{\it output\_target\_file}] {\it options}\\
~\\
\texttt{amk\_hy} {\it dim} [{\it output\_target\_file}] {\it options}\\
~\\
\texttt{amk\_m2} {\it dimX} [{\it dimY} [{\it output\_target\_file}]] {\it options}\\
~\\
\texttt{amk\_p2} {\it weight0} [{\it weight1} [{\it output\_target\_file}]] {\it options}\\

\progdes
The \texttt{amk\_}* programs make target graphs.
Each of them is devoted to a specific topology, for which it builds target
graphs of any dimension.
\\
These programs are an alternate way between algorithmically-coded
built-in target architectures and decompositions computed by mapping
with \texttt{amk\_grf}.
Like built-in target architectures, their decompositions are
algorithmically computed, and like \texttt{amk\_grf}, their output is
a decomposition-defined target architecture file.
These programs allow the definition and testing of new algorithmically-coded
target architectures without coding them in the core of the mapper.
\\

\noi
Program \texttt{amk\_ccc} outputs the target architecture file of a
Cube-Connected-Cycles graph of dimension {\it dim}.
Vertex $(l,m)$ of $\CCC(dim)$, with $0 \leq l < dim$ and $0 \leq m < 2^{dim}$,
is linked to vertices $((l-1) \bmod dim, m)$, $((l+1) \bmod dim,m)$, and
$(l, m \oplus 2^l)$, and is labeled $l\times 2^{dim} + m$. $\oplus$ denotes
the bitwise exclusive-or binary operator, and $a \bmod b$ the integer
remainder of the euclidian division of $a$ by $b$.
\\

\noi
Program \texttt{amk\_fft2} outputs the target architecture file of a
binary Fast-Fourier-Transform graph of dimension {\it dim}.
Vertex $(l,m)$ of $\FFT(dim)$, with $0 \leq l \leq dim$ and
$0 \leq m < 2^{dim}$, is linked to vertices
$(l-1, m)$, $(l-1, m \bmod 2^{l-1})$, $(l+1, m)$, and $(l+1, m \oplus 2^l)$,
if they exist, and is labeled $l\times 2^{dim} + m$.
\\

\noi
Program \texttt{amk\_hy} outputs the target architecture file of a hypercube graph
of dimension {\it dim}. Vertices are labeled according to the decimal value
of their binary representation.
The decomposition-defined target architectures computed by \texttt{amk\_hy}
do not exactly give the same results as the built-in hypercube targets
because distances are not computed in the same manner, although
the two recursive bipartitionings are identical.
To achieve best performance and save space, use the built-in architecture.
\\

\noi
Program \texttt{amk\_p2} outputs the target architecture file of a weighted path
graph with two vertices, the weights of which are given as parameters.
\\
This simple target topology is used to bipartition a source graph into
two weighted parts with as few cut edges as possible.
In particular, it is used to compute independent partitions of the
processors of a multi-user parallel machine. As a matter of fact,
if the yet unallocated part of the machine is represented by a source
graph with $n$ vertices, and $n'$ processors are requested by a user
in order to run a job (with $n' \leq n$),
mapping the source graph onto the weighted path graph with two
vertices of weights $n'$ and $n-n'$ leads to a partition of the machine in
which the allocated $n'$ processors should be as densely connected as
possible (see Figure~\ref{fig-biparch}).
\begin{figure}[hbt]
\hfill
\parbox[t]{5.8cm}{
\hfill
\includegraphics[scale=0.25]{s_f_do1.ps}
\hfill\ \\
{\bf a.} Construction of a partition with $13$ vertices (in black)
on a $8\times 8$ bidimensional mesh architecture.
}\ \hfill\
\parbox[t]{5.8cm}{
\hfill
\includegraphics[scale=0.25]{s_f_do2.ps}
\hfill\ \\
{\bf b.} Construction of a partition with $17$ vertices (in black)
on the remaining architecture.
}\hfill\
\caption%
{Construction of partitions on a bidimensional $8\times 8$ mesh architecture
 by weighted bipartitioning.}
\label{fig-biparch}
\end{figure}

\progopt
\begin{itemize}
\iteme[\texttt{-h}]
Display the program synopsis.
\iteme[\texttt{-m}{\it method}]
Select the bipartitioning method (for \texttt{amk\_m2} only).
\begin{itemize}
\iteme[\texttt{n}]
Nested dissection.
\iteme[\texttt{o}]
Dimension-per-dimension one-way dissection. This is less efficient
than nested dissection, and this feature exists only for benchmarking
purposes.
\iteme[\texttt{-V}]
Print the program version and copyright.
\end{itemize}
\end{itemize}
\end{itemize}

\subsubsection{\texttt{amk\_grf}}
\label{sec-prog-amkgrf}

\begin{itemize}
\progsyn
\texttt{amk\_grf} [{\it input\_graph\_file} [{\it output\_target\_file}]] {\it options}

\progdes
The program \texttt{amk\_grf} turns a source graph file into a
decomposition-defined target architecture file.

The \texttt{-2} option creates a ``\texttt{deco~2}'' decomposition
rather than a ``\texttt{deco~0}'' one. See
Section~\ref{sec-file-target-deco},
page~\pageref{sec-file-target-deco} for more information on the
different types of decomposition-defined target architectures.

The \texttt{-l} option restricts the target architecture to the vertices indicated
in the given vertex list file. It is therefore possible
to build a target architecture made of several disconnected parts of a bigger
architecture.
Note that this is not equivalent to turning a disconnected source graph into
a target architecture, since doing so would lead to an architecture made of
several independent pieces at infinite distance one from another.
Considering the selected vertices within their original architecture makes
it possible to compute the distance between vertices belonging to distinct
connected components, and therefore to evaluate the cost of the mapping of
two neighbor processes onto disjoint areas of the architecture.
\\
The restriction feature is very useful in the context of multi-user parallel
machines. On these machines, when users request processors in order to run
their jobs, the partitions allocated by the operating system may not be
regular nor connected, because of existing partitions already attributed
to other people.
By feeding \texttt{amk\_grf} with the source graph representing the
whole parallel machine, and the vertex list containing the labels of the
processors allocated by the operating system, it is possible to build a
target architecture corresponding to this partition, and therefore to map
processes on it, automatically, regardless of the partition shape.

The \texttt{-b} option selects the recursive bipartitioning strategy used
to build the ``\texttt{deco~0}'' decomposition of the source
graph. For regular, unweighted, topologies, the \texttt{'-b(g|h)fx'}
recursive bipartitioning strategy should work best. For irregular or
weighted graphs, use the default strategy, which is more flexible. See
also the manual page of function \texttt{SCOTCH\_\lbt arch\lbo Build0},
page~\pageref{sec-lib-arch-build}, for further information.

\progopt
\begin{itemize}
\iteme[\texttt{-b}{\it strategy}]
Use recursive bipartitioning strategy {\it strategy\/} to build the
decomposition of the architecture graph. The format of bipartitioning
strategies is defined within section~\ref{sec-lib-format-map},
at page~\pageref{sec-lib-format-bipart}.
\iteme[\texttt{-h}]
Display the program synopsis.
\iteme[\texttt{-l}{\it input\_vertex\_file}]
Load vertex list from {\it input\_vertex\_file}. As for all other file names,
``\texttt{-}'' may be used to indicate standard input.
\iteme[\texttt{-V}]
Print the program version and copyright.
\end{itemize}
\end{itemize}

\subsubsection{\texttt{atst}}

\begin{itemize}
\progsyn
\texttt{atst} [{\it input\_target\_file} [{\it output\_data\_file}]] {\it options}

\progdes
The program \texttt{atst} is the architecture tester. It gives some statistics on
decomposition-defined target architectures, and in particular the
minimum, maximum, and average communication costs (that is, weighted distance)
between all pairs of processors.

\progopt
\begin{itemize}
\iteme[\texttt{-h}]
Display the program synopsis.
\iteme[\texttt{-V}]
Print the program version and copyright.
\end{itemize}
\end{itemize}

\subsubsection{\texttt{gcv}}
\label{sec-prog-gcv}

\begin{itemize}
\progsyn
\texttt{gcv} [{\it input\_graph\_file} [{\it output\_graph\_file} [{\it output\_geometry\_file}]]] {\it options}

\progdes
The program \texttt{gcv} is the source graph converter. It takes on input a graph
file of the format specified with the \texttt{-i} option, and outputs its
equivalent in the format specified with the \texttt{-o} option, along with its
associated geometry file whenever geometry data is available.
At the time being, it accepts four input formats: the Matrix Market
format~\cite{bopore96}, the Harwell-Boeing collection
format~\cite{dugrle92}, the {\sc Chaco}/\metis\ graph
format~\cite{hele93c}, and the \scotch\ format.  Three output format
are available: the Matrix Market format, the {\sc Chaco}/\metis\ graph
format and the \scotch\ source graph and geometry data format.
\progopt
\begin{itemize}
\iteme[\texttt{-h}]
Display the program synopsis.
\iteme[\texttt{-i}{\it format}]
Specify the type of input graph.
The available input formats are listed below.
\begin{itemize}
\iteme[{\texttt{b}[{\it number}]}]
Harwell-Boeing graph collection format. Only symmetric assembled matrices
are currently supported.
Since files in this format can contain several graphs one after another,
the optional integer {\it number}, starting
from $0$, indicates which graph of the file is considered for conversion.
\iteme[\texttt{c}]
{\sc Chaco v1.0}/\metis\ format.
\iteme[\texttt{m}]
The Matrix Market format.
\iteme[\texttt{s}]
\scotch\ source graph format.
%% \iteme[\texttt{u}]
%% Universal Data Set 2412 format. On output, this node/element structure is
%% turned into a communication graph such that vertices represent elements and
%% there exists an edge between two vertices if the two end elements share a node
%% in the original UDS graph.
%% Since UDS format files are tag files, they do not have a well defined end.
%% Therefore, one cannot append data of different nature to the input stream
%% used to read this graph, since it will make the graph loading routine fail.
\end{itemize}
\iteme[\texttt{-o}{\it format}]
Specify the output graph format. The available output formats are listed below.
\begin{itemize}
\iteme[\texttt{c}]
{\sc Chaco v1.0}/\metis\ format.
\iteme[\texttt{m}]
The Matrix Market format.
\iteme[\texttt{s}]
\scotch\ source graph format.
\end{itemize}
\iteme[\texttt{-V}]
Print the program version and copyright.
\end{itemize}

Default option set is ``\texttt{-Ib0 -Os}''.
\end{itemize}

\subsubsection{\texttt{gmap} / \texttt{gpart}}
\label{sec-prog-gmap}

\begin{itemize}
\progsyn
\texttt{gmap} [{\it input\_graph\_file} [{\it input\_target\_file} [{\it output\_mapping\_file} [{\it output\_log\_file}]]]] {\it options}\\
~\\
\texttt{gpart} {\it number\_\lbt of\_\lbt parts} [{\it input\_graph\_file} [{\it output\_mapping\_file} [{\it output\_log\_file}]]] {\it options}

\progdes
The program \texttt{gmap} is the graph mapper. It uses a partitioning
strategy to map a source graph onto a target graph, so that the weight of
source graph vertices allocated to target vertices is balanced, and the
communication cost function $f_C$ is minimized.

The program \texttt{gpart} is the graph partitioner. It uses a
partitioning strategy to split a source graph into the prescribed
number of parts, using vertex or edge separators, depending whether
the \texttt{-o} option is set or not.

The implemented mapping methods mainly derive from graph theory.  In
particular, graph geometry is never used, even if it is available;
only topological properties are taken into account. Mapping methods
are used to define mapping strategies by means of selection,
combination, grouping, and condition operators.
\\

Mapping methods implemented in version~{\sc 6.0} comprise direct k-way
methods, including a k-way multilevel framework and k-way local
refinement methods, as well as the Dual
Recursive Bipartitioning algorithm, which uses graph bipartitioning
methods. Available bipartitioning methods include a multilevel
framework that uses other bipartitioning methods to compute the
initial and refined bipartitions: an improved implementation of the
Fiduccia--Mattheyses heuristic designed to handle weighted graphs,
a diffusion-based algorithm, a greedy method derived from the Gibbs,
Poole, and Stockmeyer algorithm, a greedy graph growing heuristic, a
greedy ``exactifying'' refinement algorithm designed to balance vertex
loads as much as possible, etc.

\texttt{gpart} is a simplified interface to \texttt{gmap}, which performs
graph partitioning instead of static mapping. Consequently, the
desired number of parts has to be provided, in lieu of the target
architecture.

The \texttt{-b} and \texttt{-c} options allow the user to set preferences on
the behavior of the mapping strategy which is used by default. The
\texttt{-m} option allows the user to define a custom mapping strategy.

Both programs can be used to perform clustering, by means of the
\texttt{-q} option. \texttt{gpart} will perform topology-independent
clustering, while \texttt{gmap} may compute locality-preserving clusters
when mapping onto variable-sized, non-complete, architectures (see
Section~\ref{sec-file-target-variable}).

If mapping statistics are wanted rather than the mapping output itself,
mapping output can be set to \texttt{/dev/null}, with option \texttt{-vmt}
to get mapping statistics and timings.

\progopt\\*
Since the program is devoted to experimental studies, it has many
optional parameters, used to test various execution modes. Values
set by default will give best results in most cases.
\begin{itemize}
\iteme[\texttt{-b}{\it rat}]
Set the maximum load imbalance ratio to \textit{rat}, which should
be a value comprised between $0$ and $1$. This option can be used in
conjunction with option \texttt{-c}, but is incompatible with option
\texttt{-m}.
\iteme[\texttt{-c}{\it flags}]
Tune the default mapping strategy according to the given preference
flags. Some of these flags are antagonistic, while others can be
combined. See Section~\ref{sec-lib-format-strat-default} for more
information. The currently available flags are the following.
\begin{itemize}
\iteme[\texttt{b}]
Enforce load balance as much as possible.
\iteme[\texttt{q}]
Privilege quality over speed.
\iteme[\texttt{r}]
Only use recursive bipartitioning methods.
\iteme[\texttt{s}]
Privilege speed over quality.
\iteme[\texttt{t}]
Use only safe methods in the strategy.
\end{itemize}
This option can be used in conjunction with option \texttt{-b}, but is
incompatible with option \texttt{-m}.
The resulting strategy string can be displayed by means
of the \texttt{-vs} option.
\iteme[\texttt{-h}]
Display the program synopsis.
\iteme[\texttt{-m}{\it strat\/}]
Apply mapping strategy {\it strat}. In the case of static mapping or
of edge-based graph partitioning, the format of mapping strategies
should comply with the format defined in
Section~\ref{sec-lib-format-map}.  If the \texttt{-o} option is used (see
below), strategies must be vertex partitioning strategies, which are
described in Section~\ref{sec-lib-format-part-ovl}.
This option is incompatible with options \texttt{-b} and
\texttt{-c}.
\iteme[\texttt{-o}]
Compute vertex-based partitions rather than static mappings or
edge-based partitions. This option is only valid for \texttt{gpart}, or
when \texttt{gmap} is called with a target architecture which is an
unweighted complete graph.
\iteme[\texttt{-q}] (for \texttt{gpart})
\iteme[\texttt{-q}{\it pwght}] (for \texttt{gmap})
Perform clustering instead of partitioning or mapping. Clustering is
achieved by means of a specific strategy string that performs
recursive bipartitioning until the size of the parts is smaller than
some threshold value. For \texttt{gpart}, this value replaces the desired
number of parts as the first argument passed to the program. For
\texttt{gmap}, the threshold must be given just after the \texttt{-q} option.
\iteme[\texttt{-s}{\it obj}]
Mask source edge and vertex weights. This option allows the user to
``unweight'' weighted source graphs by removing weights from edges and
vertices at loading time. {\it obj\/} may contain several of the following
switches.
\begin{itemize}
\iteme[\texttt{e}]
Remove edge weights, if any.
\iteme[\texttt{v}]
Remove vertex weights, if any.
\end{itemize}
\iteme[\texttt{-V}]
Print the program version and copyright.
\iteme[\texttt{-v}{\it verb}]
Set verbose mode to {\it verb}, which may contain several of the following
switches. For a detailed description of the data displayed, please
refer to the manual page of \texttt{gmtst} below.
\begin{itemize}
\iteme[\texttt{m}]
Mapping or partitioning information, depending whether the \texttt{-o}
option has been set or not.
\iteme[\texttt{s}]
Strategy information. This parameter displays the mapping
strategy which will be used by \texttt{gmap} or \texttt{gpart}.
\iteme[\texttt{t}]
Timing information.
\end{itemize}
\iteme[\texttt{-V}]
Print the program version and copyright.
\end{itemize}
\end{itemize}

\subsubsection{\texttt{gmk\_}*}

\begin{itemize}
\progsyn
\texttt{gmk\_hy} {\it dim} [{\it output\_graph\_file}] {\it options}\\
~\\
\texttt{gmk\_m2} {\it dimX} [{\it dimY} [{\it output\_graph\_file}]] {\it options}\\
~\\
\texttt{gmk\_m3} {\it dimX} [{\it dimY} [{\it dimZ} [{\it output\_graph\_file}]]] {\it options}\\
~\\
\texttt{gmk\_ub2} {\it dim} [{\it output\_graph\_file}] {\it options}

\progdes
The \texttt{gmk\_}* programs make source graphs.
Each of them is devoted to a specific topology, for which it builds target
graphs of any dimension.
\\
The \texttt{gmk\_}* programs are mainly used in conjunction with \texttt{amk\_grf}.
Most \texttt{gmk\_}* programs build source graphs
describing parallel machines, which are used by \texttt{amk\_grf} to generate
corresponding target sub-architectures, by means of its \texttt{-l}
option.
Such a procedure is shown in section~\ref{sec-examples}, which builds a target
architecture from five vertices of a binary de~Bruijn graph of dimension~$3$.
\\

\noi
Program \texttt{gmk\_hy} outputs the source file of a hypercube graph of
dimension {\it dim}. Vertices are labeled according to the decimal value
of their binary representation.
\\

\noi
Program \texttt{gmk\_m2} outputs the source file of a bidimensional mesh
with {\it dimX\/} columns and {\it dimY\/} rows. If the \texttt{-t}
option is set, tori are built instead of meshes. The vertex of
coordinates $(\mbox{\it posX},\mbox{\it posY\/})$ is labeled
$\mbox{\it posY} \times \mbox{\it dimX} + \mbox{\it posX}$.
\\

\noi
Program \texttt{gmk\_m3} outputs the source file of a tridimensional mesh
with {\it dimZ} layers of {\it dimY\/} rows by {\it dimX\/}
columns. If the \texttt{-t} option is set, tori are built instead of
meshes. The vertex of coordinates $(\mbox{\it posX},\mbox{\it
posY\/})$ is labeled $(\mbox{\it posZ} \times \mbox{\it dimY} +
\mbox{\it posY}) \times \mbox{\it dimX} + \mbox{\it posX}$.
\\

\noi
Program \texttt{gmk\_ub2} outputs the source file of a binary unoriented
de~Bruijn graph of dimension {\it dim}. Vertices are labeled according to
the decimal value of their binary representation.

\progopt
\begin{itemize}
\iteme[\texttt{-g}{\it output\_geometry\_file}]
Output graph geometry to file {\it output\_geometry\_file}
(for \texttt{gmk\_m2} only).
As for all other file names, ``\texttt{-}''
may be used to indicate standard output.
\iteme[\texttt{-h}]
Display the program synopsis.
\iteme[\texttt{-t}]
Build a torus rather than a mesh (for \texttt{gmk\_m2} only).
\iteme[\texttt{-V}]
Print the program version and copyright.
\end{itemize}
\end{itemize}

\subsubsection{\texttt{gmk\_msh}}
\label{sec-prog-gmkmsh}

\begin{itemize}
\progsyn
\texttt{gmk\_msh} [{\it input\_mesh\_file} [{\it output\_graph\_file}]] {\it options}\\

\progdes
The \texttt{gmk\_msh} program builds a graph file from a mesh file. All
of the nodes of the mesh are turned into graph vertices, and edges
are created between all pairs of vertices that share an element (that
is, elements are turned into cliques).

\progopt
\begin{itemize}
\iteme[\texttt{-h}]
Display the program synopsis.
\iteme[\texttt{-V}]
Print the program version and copyright.
\end{itemize}
\end{itemize}

\subsubsection{\texttt{gmtst}}

\begin{itemize}
\progsyn
\texttt{gmtst} [{\it input\_graph\_file} [{\it input\_\lbt target\_\lbt file} [{\it input\_\lbt mapping\_\lbt file} [{\it output\_\lbt data\_\lbt file}]]]] {\it options}

\progdes
The program \texttt{gmtst} is the graph mapping tester. It outputs some
statistics on the given mapping, regarding load balance and inter-processor
communication.
\\
The two first statistics lines deal with process mapping statistics, while
the following ones deal with communication statistics.
The first mapping line gives the number of processors used by
the mapping, followed by the number of processors available in the
architecture, and the ratio of these two numbers, written between parentheses.
The second mapping line gives the minimum, maximum, and average loads of the
processors, followed by the variance of the load distribution, and an imbalance
ratio equal to the maximum load over the average load.
The first communication line gives the minimum and maximum number of neighbors
over all blocks of the mapping, followed by the sum of the number of neighbors
over all blocks of the mapping, that is the total number of messages
that have to be sent to exchange data between all neighboring blocks.
The second communication line gives the average dilation of the edges,
followed by the sum of all edge dilations.
The third communication line gives the average expansion of the edges,
followed by the value of function $f_C$.
The fourth communication line gives the average cut of the edges,
followed by the number of cut edges.
The fifth communication line shows the ratio of the average expansion over
the average dilation; it is smaller than $1$ when the mapper succeeds in
putting heavily intercommunicating processes closer to each other than it
does for lightly communicating processes; it is equal to $1$ if all edges
have the same weight.
The remaining lines form a distance histogram, which shows the amount of
communication load that involves processors located at increasing distances.

\texttt{gmtst} allows the testing of cross-architecture mappings. By inputing it
a target architecture different from the one that has been used to compute the
mapping, but with compatible vertex labels, one can see what the mapping
would yield on this new target architecture.

\progopt
\begin{itemize}
\iteme[\texttt{-h}]
Display the program synopsis.
\iteme[\texttt{-V}]
Print the program version and copyright.
\end{itemize}
\end{itemize}

\subsubsection{\texttt{gord}}

\begin{itemize}
\progsyn
\texttt{gord} [{\it input\_graph\_file} [{\it output\_ordering\_file} [{\it output\_log\_file}]]] {\it options}

\progdes
The \texttt{gord} program is the block sparse matrix graph orderer. It uses an
ordering strategy to compute block orderings of sparse matrices
represented as source graphs, whose vertex weights indicate the number
of DOFs per node (if this number is non homogeneous) and whose edges
are unweighted, in order to minimize fill-in and operation count.

Since its main purpose is to provide orderings that exhibit high
concurrency for parallel block factorization, it comprises a nested
dissection method~\cite{geli81}, but classical~\cite{liu-85} and
state-of-the-art~\cite{amdadu96,peroam99} minimum degree algorithms
are implemented as well.
Ordering methods are used to define ordering strategies by means of
selection, grouping, and condition operators.

For the nested dissection method, vertex separation methods comprise
algorithms that directly compute vertex separators, as well as methods
that build vertex separators from edge separators, \ie graph
bipartitions (all of the graph bipartitioning methods available in the
static mapper \texttt{gmap} can be used in this latter case).

The \texttt{-o} option allows the user to define the ordering
strategy. The \texttt{-c} option allows the user to set preferences on
the behavior of the ordering strategy which is used by default.
\\

When the graphs to order are very large, the same results can
be obtained by using the \texttt{dgord} parallel program of the
\ptscotch\ distribution, which can read centralized graph
files too.

\progopt\\*
Since the program is devoted to experimental studies, it has many
optional parameters, used to test various execution modes. Values
set by default will give best results in most cases.
\begin{itemize}
\iteme[\texttt{-c}{\it flags}]
Tune the default ordering strategy according to the given preference
flags. Some of these flags are antagonistic, while others can be
combined. See Section~\ref{sec-lib-format-strat-default} for more
information. The resulting strategy string can be displayed by means
of the \texttt{-vs} option.
\begin{itemize}
\iteme[\texttt{b}]
Enforce load balance as much as possible.
\iteme[\texttt{q}]
Privilege quality over speed. This is the default behavior.
\iteme[\texttt{s}]
Privilege speed over quality.
\iteme[\texttt{t}]
Use only safe methods in the strategy.
\end{itemize}
\iteme[\texttt{-h}]
Display the program synopsis.
\iteme[\texttt{-m}{\it output\_mapping\_file}]
Write to {\it output\_mapping\_file\/} the mapping of graph vertices to
column blocks. All of the separators and leaves produced by the nested
dissection method are considered as distinct column blocks, which may
be in turn split by the ordering methods that are applied to them.
Distinct integer numbers are associated with each of the column blocks,
such that the number of a block is always greater than the ones of its
predecessors in the elimination process, that is, its descendants in the
elimination tree.
The structure of mapping files is given in section~\ref{sec-file-map}.

When the geometry of the graph is available, this mapping file may be
processed by program \texttt{gout} to display the vertex separators and
supervariable amalgamations that have been computed.
\iteme[{\texttt{-o}{\it strat}}]
Apply ordering strategy {\it strat}. The format of ordering
strategies is defined in section~\ref{sec-lib-format-ord}.
\iteme[\texttt{-t}{\it output\_tree\_file}]
Write to {\it output\_tree\_file\/} the structure of the separator
tree. The data that is written resembles much the one of a mapping
file: after a first line that contains the number of lines to follow,
there are that many lines of mapping pairs, which associate an integer
number with every graph vertex index. This integer number is the
number of the column block which is the parent of the column block to
which the vertex belongs, or $-1$ if the column block to which the
vertex belongs is a root of the separator tree (there can be several
roots, if the graph is disconnected).

Combined to the column block mapping data produced by option \texttt{-m},
the tree structure allows one to rebuild the separator tree.
\iteme[\texttt{-V}]
Print the program version and copyright.
\iteme[\texttt{-v}{\it verb}]
Set verbose mode to {\it verb}, which may contain several of the following
switches.
%For a detailed description of the data displayed, please
%refer to the manual page of \texttt{gotst}.
\begin{itemize}
\iteme[\texttt{s}]
Strategy information. This parameter displays the ordering
strategy which will be used by \texttt{gord}.
\iteme[\texttt{t}]
Timing information.
\end{itemize}
\end{itemize}
\end{itemize}

\subsubsection{\texttt{gotst}}
\label{sec-prog-gotst}

\begin{itemize}
\progsyn
\texttt{gotst} [{\it input\_graph\_file} [{\it input\_ordering\_file} [{\it output\_data\_file}]]] {\it options}

\progdes
The program \texttt{gotst} is the ordering tester. It gives some
statistics on orderings, including the number of non-zeros and
the operation count of the factored matrix, as well as statistics
regarding the elimination tree. Since it performs the factorization
of the reordered matrix, it can take a very long time and consume
a large amount of memory when applied to large graphs.
\\
The first two statistics lines deal with the elimination tree. The
first one displays the number of leaves, while the second shows
the minimum height of the tree (that is, the
length of the shortest path from any leaf to the --or a-- root node),
its maximum height, its average height, and the variance of the
heights with respect to the average.
The third line displays the number of non-zero terms in the factored
matrix, the amount of index data that is necessary to maintain the
block structure of the factored matrix, and the number of operations
required to factor the matrix by means of Cholesky factorization.

\progopt
\begin{itemize}
\iteme[\texttt{-h}]
Display the program synopsis.
\iteme[\texttt{-V}]
Print the program version and copyright.
\end{itemize}
\end{itemize}

\subsubsection{\texttt{gout}}
\label{sec-prog-gout}

\begin{itemize}
\progsyn
\texttt{gout} [{\it input\_graph\_file} [{\it input\_geometry\_file} [{\it input\_\lbt mapping\_\lbt file} [{\it output\_\lbt visualization\_\lbt file}]]]] {\it options}

\progdes
The \texttt{gout} program is the graph, matrix, and mapping viewer program. It
takes on input a source graph, its geometry file, and optionally a mapping
result file, and produces a file suitable for display.
At the time being, \texttt{gout} can generate plain and encapsulated PostScript
files for the display of adjacency matrix patterns and the
display of planar graphs (although tridimensional objects
can be displayed by means of isometric projection, the display of
tridimensional mappings is not efficient), and {\sc Open Inventor}
files~\cite{oinv} for the interactive visualization of
tridimensional graphs.
\\
In the case of mapping display,
the number of mapping pairs contained in the input mapping file may
differ from the number of vertices of the input source graph;
only mapping pairs the source labels of which match labels of source graph
vertices will be taken into account for display.
This feature allows the user to show the result of the mapping of a subgraph
drawn on the whole graph, or else to outline the most important aspects
of a mapping by restricting the display to a limited portion of the graph.
For example, Figure~\ref{fig-out-ps}\@.b shows how the result of the mapping of
a subgraph of the bidimensional mesh $\MD(4,4)$ onto the complete graph
$\KP(2)$ can be displayed on the whole $\MD(4,4)$ graph,
and Figure~\ref{fig-out-ps}\@.c shows how the display of the same mapping
can be restricted to a subgraph of the original graph.
% gmk_m2 4 4 s_f_out1.src -gs_f_out.xyz
% map s_f_out2.src ../tgt/k2.tgt s_f_out2.map
% out s_f_out2.src s_f_out.xyz -m - s_f_out1.ps '-Op{e,g,l}'
% out s_f_out1.src s_f_out.xyz s_f_out2.map s_f_out2.ps '-Op{e,g,l}'
% out s_f_out3.src s_f_out.xyz s_f_out2.map s_f_out3.ps '-Op{e,g,l}'
\begin{figure}[hbt]
\hfill
\parbox[t]{4.5cm}{
\hfill
\includegraphics[scale=0.25]{s_f_out1.ps}
\hfill\ \\
{\bf a.} A subgraph of $\MD(4,4)$ to be mapped onto $\KP(2)$.
}\ \hfill\
\parbox[t]{4.5cm}{
\hfill
\includegraphics[scale=0.25]{s_f_out2.ps}
\hfill\ \\
{\bf b.} Mapping result displayed on the full $\MD(4,4)$ graph.
}\ \hfill\
\parbox[t]{4.5cm}{
\hfill
\includegraphics[scale=0.25]{s_f_out3.ps}
\hfill\ \\
{\bf c.} Mapping result displayed on another subgraph of $\MD(4,4)$.
}\hfill\
\caption{PostScript diplay of a single mapping file with different
         subgraphs of the same source graph. Vertices covered with disks
         of the same color are mapped onto the same processor.}
\label{fig-out-ps}
\end{figure}

\progopt
\begin{itemize}
\iteme[\texttt{-g}{\it parameters}]
Geometry parameters.
\begin{itemize}
\iteme[\texttt{n}]
Do not read geometry data. This option can be used in conjunction with
option \texttt{-om} to avoid reading the geometry file when displaying
the pattern of the adjacency matrix associated with the source graph,
since geometry data are not needed in this case.
If this option is set, the geometry file is not read. However, if an
{\it output\_\lbt visualization\_\lbt file} name is given in the
command line, dummy {\it input\_\lbt geometry\_\lbt file\/} and {\it
input\_\lbt mapping\_\lbt file\/} names must be specified so that the
file argument count is correct.  In this case, use the ``\texttt{-}''
parameter to take standard input as a dummy geometry input stream.  In
practice, the \texttt{-om} and \texttt{-gn} options always imply the
\texttt{-mn} option.
\iteme[\texttt{r}]
For bidimensional geometry only, rotate geometry data by $90$ degrees,
counter-clockwise.
\end{itemize}
\iteme[\texttt{-h}]
Display the program synopsis.
\iteme[\texttt{-mn}]
Do not read mapping data, and display the graph without any mapping
information. If this option is set, the mapping file is not
read. However, if an {\it output\_\lbt visualization\_\lbt file\/}
name is given in the command line, a dummy {\it input\_\lbt
mapping\_\lbt file\/} name must be specified so that the file argument
count is correct. In this case, use the ``\texttt{-}'' parameter to take
standard input as a dummy mapping input stream.
\iteme[{\texttt{-o}{\it format}[\texttt{\{}{\it parameters}\texttt{\}}]}]
Specify the type of output, with optional parameters within curly braces
and separated by commas. The output formats are listed below.
\begin{itemize}
\iteme[\texttt{i}]
Output the graph in SGI's {\sc Open Inventor} format, in ASCII mode,
suitable for display by the \texttt{ivview} program~\cite{oinv}. The
optional parameters are given below.
\begin{itemize}
\iteme[\texttt{c}]
Color output, using $16$ different colors. Opposite of \texttt{g}.
\iteme[\texttt{g}]
Grey-level output, using $8$ different levels. Opposite of \texttt{c}.
\iteme[\texttt{r}]
Remove cut edges. Edges the ends of which are mapped onto different
processors are not displayed. Opposite of \texttt{v}.
\iteme[\texttt{v}]
View cut edges. All graph edges are displayed.
Opposite of \texttt{r}.
\end{itemize}
\iteme[\texttt{m}]
Output the pattern of the adjacency matrix associated with the source graph,
in Adobe's PostScript format. The optional parameters are given below.
\begin{itemize}
\iteme[\texttt{e}]
Encapsulated PostScript output, suitable for \LaTeX\ use with \texttt{epsf}.
Opposite of \texttt{f}.
\iteme[\texttt{f}]
Full-page PostScript output, suitable for direct printing.
Opposite of \texttt{e}.
\end{itemize}
\iteme[\texttt{p}]
Output the graph in Adobe's PostScript format.
The optional parameters are given below.
\begin{itemize}
\iteme[\texttt{a}]
Avoid displaying the mapping disks. Opposite of \texttt{d}.
\iteme[\texttt{c}]
Color PostScript output, using $16$ different colors. Opposite of \texttt{g}.
\iteme[\texttt{d}]
Display the mapping disks. Opposite of \texttt{a}.
\iteme[\texttt{e}]
Encapsulated PostScript output, suitable for \LaTeX\ use with \texttt{epsf}.
Opposite of \texttt{f}.
\iteme[\texttt{f}]
Full-page PostScript output, suitable for direct printing.
Opposite of \texttt{e}.
\iteme[\texttt{g}]
Grey-level PostScript output. Opposite of \texttt{c}.
\iteme[\texttt{l}]
Large clipping. Mapping disks are included in the clipping area computation.
Opposite of \texttt{s}.
\iteme[\texttt{r}]
Remove cut edges. Edges the ends of which are mapped onto different
processors are not displayed.
Opposite of \texttt{v}.
\iteme[\texttt{s}]
Small clipping. Mapping disks are excluded from the clipping area computation.
Opposite of \texttt{l}.
\iteme[\texttt{v}]
View cut edges. All graph edges are displayed.
Opposite of \texttt{r}.
\iteme[\texttt{x=}{\it val}]
Minimum X relative clipping position (in [0.0;1.0]).
\iteme[\texttt{X=}{\it val}]
Maximum X relative clipping position (in [0.0;1.0]).
\iteme[\texttt{y=}{\it val}]
Minimum Y relative clipping position (in [0.0;1.0]).
\iteme[\texttt{Y=}{\it val}]
Maximum Y relative clipping position (in [0.0;1.0]).
\end{itemize}
\iteme[\texttt{-V}]
Print the program version and copyright.
\end{itemize}
\end{itemize}

Default option set is ``\texttt{-Oi\{v\}}''.
\end{itemize}

\subsubsection{\texttt{gtst}}

\begin{itemize}
\progsyn
\texttt{gtst} [{\it input\_graph\_file} [{\it output\_data\_file}]] {\it options}

\progdes
The program \texttt{gtst} is the source graph tester. It checks the
consistency of the input source graph structure (matching of arcs,
number of vertices and edges, etc\@.), and gives some statistics
regarding edge weights, vertex weights, and vertex degrees.
\\

When the graphs to test are very large, the same results can
be obtained by using the \texttt{dgtst} parallel program of the
\ptscotch\ distribution, which can read centralized graph
files too.

\progopt
\begin{itemize}
\iteme[\texttt{-h}]
Display the program synopsis.
\iteme[\texttt{-V}]
Print the program version and copyright.
\end{itemize}
\end{itemize}

\subsubsection{\texttt{mcv}}
\label{sec-prog-mcv}

\begin{itemize}
\progsyn
\texttt{mcv} [{\it input\_mesh\_file} [{\it output\_mesh\_file} [{\it output\_geometry\_file}]]] {\it options}

\progdes
The program \texttt{mcv} is the source mesh converter. It takes on input a mesh
file of the format specified with the \texttt{-i} option, and outputs its
equivalent in the format specified with the \texttt{-o} option, along with its
associated geometry file whenever geometrical data is available.
At the time being, it only accepts one external input format: the
Harwell-Boeing format~\cite{dugrle92}, for square elemental matrices only.
The only output format to date is the \scotch\ source mesh and
geometry data format.

\progopt
\begin{itemize}
\iteme[\texttt{-h}]
Display the program synopsis.
\iteme[\texttt{-i}{\it format}]
Specify the type of input mesh.
The available input formats are listed below.
\begin{itemize}
\iteme[{\texttt{b}[{\it number}]}]
Harwell-Boeing mesh collection format. Only symmetric elemental matrices
are currently supported.
Since files in this format can contain several meshes one after another,
the optional integer {\it number}, starting
from $0$, indicates which mesh of the file is considered for conversion.
\iteme[\texttt{s}]
\scotch\ source mesh format.
\end{itemize}
\iteme[\texttt{-o}{\it format}]
Specify the output graph format. The available output formats are listed below.
\begin{itemize}
\iteme[\texttt{s}]
\scotch\ source graph format.
\end{itemize}
\iteme[\texttt{-V}]
Print the program version and copyright.
\end{itemize}

Default option set is ``\texttt{-Ib0 -Os}''.
\end{itemize}

\subsubsection{\texttt{mmk\_}*}

\begin{itemize}
\progsyn
\texttt{mmk\_m2} {\it dimX} [{\it dimY} [{\it output\_mesh\_file}]] {\it options}\\
~\\
\texttt{mmk\_m3} {\it dimX} [{\it dimY} [{\it dimZ} [{\it output\_mesh\_file}]]] {\it options}\\

\progdes
The \texttt{mmk\_}* programs make source meshes.
\\

\noi
Program \texttt{mmk\_m2} outputs the source file of a bidimensional mesh
with $\mbox{{\it dimX\/}} \times \mbox{{\it dimY\/}}$ elements and
$(\mbox{{\it dimX\/}}+1) \times (\mbox{{\it dimY\/}}+1)$ nodes.
The element of coordinates $(\mbox{\it posX},\mbox{\it posY\/})$ is
labeled $\mbox{\it posY} \times \mbox{\it dimX} + \mbox{\it posX}$.
\\

\noi
Program \texttt{mmk\_m3} outputs the source file of a tridimensional mesh
with $\mbox{{\it dimX\/}} \times \mbox{{\it dimY\/}} \times
\mbox{{\it dimZ\/}}$ elements and $(\mbox{{\it dimX\/}}+1) \times
(\mbox{{\it dimY\/}}+1) \times (\mbox{{\it dimZ\/}}+1)$ nodes.
\\

\progopt
\begin{itemize}
\iteme[\texttt{-g}{\it output\_geometry\_file}]
Output mesh geometry to file {\it output\_geometry\_file}
(for \texttt{mmk\_m2} only).
As for all other file names, ``\texttt{-}''
may be used to indicate standard output.
\iteme[\texttt{-h}]
Display the program synopsis.
\iteme[\texttt{-V}]
Print the program version and copyright.
\end{itemize}
\end{itemize}

\subsubsection{\texttt{mord}}

\begin{itemize}
\progsyn
\texttt{mord} [{\it input\_mesh\_file} [{\it output\_ordering\_file} [{\it output\_log\_file}]]] {\it options}

\progdes
The \texttt{mord} program is the block sparse matrix mesh orderer. It
uses an ordering strategy to compute block orderings of sparse matrices
represented as source meshes, whose node vertex weights indicate the
number of DOFs per node (if this number is non homogeneous), in order
to minimize fill-in and operation count.

Since its main purpose is to provide orderings that exhibit high
concurrency for parallel block factorization, it comprises a nested
dissection method~\cite{geli81}, but classical~\cite{liu-85} and
state-of-the-art~\cite{amdadu96,peroam99} minimum degree algorithms
are implemented as well.
Ordering methods are used to define ordering strategies by means of
selection, grouping, and condition operators.

The \texttt{-o} option allows the user to define the ordering
strategy. The \texttt{-c} option allows the user to set preferences on
the behavior of the ordering strategy which is used by default.

\progopt\\*
Since the program is devoted to experimental studies, it has many
optional parameters, used to test various execution modes. Values
set by default will give best results in most cases.
\begin{itemize}
\iteme[\texttt{-c}{\it flags}]
Tune the default ordering strategy according to the given preference
flags. Some of these flags are antagonistic, while others can be
combined. See Section~\ref{sec-lib-format-strat-default} for more
information. The resulting strategy string can be displayed by means
of the \texttt{-vs} option.
\begin{itemize}
\iteme[\texttt{b}]
Enforce load balance as much as possible.
\iteme[\texttt{q}]
Privilege quality over speed. This is the default behavior.
\iteme[\texttt{s}]
Privilege speed over quality.
\iteme[\texttt{t}]
Use only safe methods in the strategy.
\end{itemize}
\iteme[\texttt{-h}]
Display the program synopsis.
\iteme[\texttt{-m}{\it output\_mapping\_file}]
Write to {\it output\_mapping\_file\/} the mapping of mesh node
vertices to column blocks. All of the separators and leaves produced
by the nested dissection method are considered as distinct column
blocks, which may be in turn split by the ordering methods that are
applied to them. Distinct integer numbers are associated with each
of the column blocks, such that the number of a block is always
greater than the ones of its predecessors in the elimination process,
that is, its leaves in the elimination tree.
The structure of mapping files is given in section~\ref{sec-file-map}.

When the coordinates of the node vertices are available, the mapping
file may be processed by program \texttt{gout}, along with the graph
structure that can be created from the source mesh file by means of
the \texttt{gmk\_\lbt msh} program, to display the node vertex separators
and supervariable amalgamations that have been computed.
\iteme[{\texttt{-o}{\it strat}}]
Apply ordering strategy {\it strat}. The format of ordering
strategies is defined in section~\ref{sec-lib-format-ord}.
\iteme[\texttt{-t}{\it output\_tree\_file}]
Write to {\it output\_tree\_file\/} the structure of the separator
tree. The data that is written resembles much the one of a mapping
file: after a first line that contains the number of lines to follow,
there are that many lines of mapping pairs, which associate an integer
number with every node vertex index. This integer number is the
number of the column block which is the parent of the column block to
which the node vertex belongs, or $-1$ if the column block to which the
node vertex belongs is a root of the separator tree (there can be
several roots, if the mesh is disconnected).

Combined to the column block mapping data produced by option \texttt{-m},
the tree structure allows one to rebuild the separator tree.
\iteme[\texttt{-V}]
Print the program version and copyright.
\iteme[\texttt{-v}{\it verb}]
Set verbose mode to {\it verb}, which may contain several of the following
switches.
%For a detailed description of the data displayed, please
%refer to the manual page of \texttt{gotst}.
\begin{itemize}
\iteme[\texttt{s}]
Strategy information. This parameter displays the default ordering
strategy used by \texttt{mord}.
\iteme[\texttt{t}]
Timing information.
\end{itemize}
\end{itemize}
\end{itemize}

\subsubsection{\texttt{mtst}}

\begin{itemize}
\progsyn
\texttt{mtst} [{\it input\_mesh\_file} [{\it output\_data\_file}]] {\it options}

\progdes
The program \texttt{mtst} is the source mesh tester. It checks the
consistency of the input source mesh structure (matching of arcs
that link elements to nodes and nodes to elements, number of elements,
nodes, and edges, etc\@.), and gives some statistics regarding element
and node weights, edge weights, and element and node degrees.

\progopt
\begin{itemize}
\iteme[\texttt{-h}]
Display the program synopsis.
\iteme[\texttt{-V}]
Print the program version and copyright.
\end{itemize}
\end{itemize}
                                  % Programmes
%%%%%%%%%%%%%%%%%%%%%%%%%%%%%%%%%%%%
%                                  %
% Titre  : s_l.tex                 %
% Sujet  : Manuel de l'utilisateur %
%          du projet 'Scotch'      %
%          Bibliotheque            %
% Auteur : Francois Pellegrini     %
%                                  %
%%%%%%%%%%%%%%%%%%%%%%%%%%%%%%%%%%%%

\section{Library}
\label{sec-lib}

All of the features provided by the programs of the
\scotch\ distribution may be directly accessed by calling
the appropriate functions of the \libscotch\ library, archived
in files {\tt libscotch.a} and {\tt libscotcherr.a}.
These routines belong to six distinct classes:
\begin{itemize}
\item
source graph and source mesh handling routines, which serve to declare, build,
load, save, and check the consistency of source graphs and meshes, along with
their geometry data;
\item
target architecture handling routines, which allow the user to
declare, build, load, and save target architectures;
\item
strategy handling routines, which allow the user to declare and build
mapping and ordering strategies;
\item
mapping routines, which serve to declare, compute, and save
mappings of source graphs to target architectures by means of
mapping strategies;
\item
a partitioning-with-overlap routine, which computes a vertex separator
that splits a graph into a prescribed number of parts, such that the
vertex load of each part and of its neighboring separator vertices are
balanced;
\item
ordering routines, which allow the user to declare, compute, and save
orderings of source graphs and meshes;
\item
error handling routines, which allow the user either to provide his own
error servicing routines, or to use the default routines provided
in the \libscotch\ distribution.
\end{itemize}

A \metis\ compatibility library, called {\tt lib\lbo scotch\lbo metis.a},
is also available. It allows users who were previously using
\metis\ in their software to take advantage of the efficieny of
\scotch\ without having to modify their code. The
services provided by this library are described in
Section~\ref{sec-lib-metis}.

\subsection{Calling the routines of {\sc libScotch}}

\subsubsection{Calling from C}

All of the C routines of the \libscotch\ library are prefixed with
``{\tt SCOTCH\_}''. The remainder of the function names is made of the
name of the type of object to which the functions apply (e\@.g\@.
``{\tt graph}'', ``{\tt mesh}'', ``{\tt arch}'', ``{\tt map}'', etc.),
followed by the type of action performed on this object: ``{\tt
Init}'' for the initialization of the object, ``{\tt Exit}'' for the
freeing of its internal structures, ``{\tt Load}'' for loading the
object from a stream, and so on.

Typically, functions that return an error code return zero if the
function succeeds, and a non-zero value in case of error.

For instance, the {\tt SCOTCH\_\lbt graph\lbt Init} and
{\tt SCOTCH\_\lbt graph\lbt Load} routines, described in
sections~\ref{sec-lib-func-graphinit}
and~\ref{sec-lib-func-graphload}, respectively, can
be called from C by using the following code.
{\tt
\begin{verbatim}
#include <stdio.h>
#include "scotch.h"
  ...
  SCOTCH_Graph      grafdat;
  FILE *            fileptr;

  if (SCOTCH_graphInit (&grafdat) != 0) {
    ... /* Error handling */
  }
  if ((fileptr = fopen ("brol.grf", "r")) == NULL) {
    ... /* Error handling */
  }
  if (SCOTCH_graphLoad (&grafdat, fileptr, -1, 0) != 0) {
    ... /* Error handling */
  }
  ...
\end{verbatim}
}

Since ``{\tt scotch.h}'' uses several system objects which are
declared in ``{\tt stdio.h}'', this latter file must be
included beforehand in your application code.

Although the ``{\tt scotch.h}'' and ``{\tt ptscotch.h}'' files may
look very similar on your system, never mistake them, and always use
the ``{\tt scotch.h}'' file as the include file for compiling a
program which uses only the sequential routines of the \libscotch\
library.

\subsubsection{Calling from Fortran}

The routines of the \libscotch\ library can also be called from
Fortran. For any C function named {\tt SCOTCH\_\lbt {\it type\lbt
Action}()} which is documented in this manual, there exists a {\tt
SCOTCHF\lbt {\it TYPE\lbt ACTION\/}()} Fortran counterpart, in which
the separating underscore character is replaced by an ``{\tt
F}''. In most cases, the Fortran routines have exactly the same
parameters as the C functions, save for an added trailing {\tt
INTEGER} argument to store the return value yielded by the function
when the return type of the C function is not {\tt void}.
\\

Since all the data structures used in \libscotch\ are
opaque, equivalent declarations for these structures must
be provided in Fortran. These structures must therefore
be defined as arrays of {\tt DOUBLEPRECISION}s, of sizes
given in file {\tt scotchf.h}, which must be included whenever
necessary.

For routines which read or write data using a {\tt FILE~*} stream
in C, the Fortran counterpart uses an {\tt INTEGER} parameter which
is the numer of the Unix file descriptor corresponding to the logical
unit from which to read or write. In most Unix implementations of
Fortran, standard descriptors 0 for standard input (logical unit 5),
1 for standard output (logical unit 6) and 2 for standard error are
opened by default. However, for files which are opened using
{\tt OPEN} statements, an additional function must be used to obtain
the number of the Unix file descriptor from the number of the logical
unit. This function is called \texttt{PXFFILENO} in the normalized
POSIX Fortran API, and files which use it should include the
\texttt{USE IFPOSIX} directive whenever necessary. An alternate, non
normalized, function also exists in most Unix implementations of
Fortran, and is called {\tt FNUM}.

For instance, the {\tt SCOTCH\_\lbt graph\lbt Init} and
{\tt SCOTCH\_\lbt graph\lbt Load} routines, described in
sections~\ref{sec-lib-func-graphinit}
and~\ref{sec-lib-func-graphload}, respectively, can
be called from Fortran by using the following code.
{\tt
\begin{verbatim}
        INCLUDE "scotchf.h"
        DOUBLEPRECISION GRAFDAT(SCOTCH_GRAPHDIM)
        INTEGER RETVAL
        ...
        CALL SCOTCHFGRAPHINIT (GRAFDAT (1), RETVAL)
        IF (RETVAL .NE. 0) THEN
        ...
        OPEN (10, FILE='brol.grf')
        CALL SCOTCHFGRAPHLOAD (GRAFDAT (1), FNUM (10), 1, 0, RETVAL)
        CLOSE (10)
        IF (RETVAL .NE. 0) THEN
        ...
\end{verbatim}
}

Although the ``{\tt scotchf.h}'' and ``{\tt ptscotchf.h}'' files may
look very similar on your system, never mistake them, and always use
the ``{\tt scotchf.h}'' file as the include file for compiling a
program which uses only the sequential routines of the \libscotch\
library.

\subsubsection{Compiling and linking}

The compilation of C or Fortran routines which use routines of
the \libscotch\ library requires that either ``{\tt scotch.h}'' or
``{\tt scotchf.h}'' be included, respectively.

The routines of the \libscotch\ library are grouped in a library
file called {\tt libscotch.a}. Default error routines that print
an error message and exit are provided in library file
{\tt libscotcherr.a}.

Therefore, the linking of applications that make use of the
\libscotch\ library with standard error handling is carried out by
using the following options: ``{\tt -lscotch -lscotcherr -lm}''.
If you want to handle errors by yourself, you should not link with
library file {\tt libscotcherr.a}, but rather provide a
{\tt SCOTCH\_\lbt error\lbt Print()} routine.
Please refer to section~\ref{sec-lib-error} for more information.

Programs that call both sequential and parallel routines of
\scotch\ should use only the parallel versions of the include file and
of the library. Please refer to the equivalent section of the
\ptscotch\ user's manual for more information.

\subsubsection{Dynamic library issues}
\label{sec-lib-dynalloc}

The advantage of dynamic libraries is that application code may not
need to be recompiled when the library is updated. Whether this is
true or not depends on the extent of the changes. One of the cases
when recompilation is mandatory is when API data structures change:
code that statically reserves space for them may be subject to
boundary overflow errors when the size of library data structures
increase, so that library routines operate on more space than what was
statically allocated by the compiler based on the header files of the
old version of the library.

In order to alleviate this problem, the \libscotch\ proposes a set of
routines to dynamically allocate storage space for the opaque API
\scotch\ structures. Because these routines return pointers, these
{\tt SCOTCH\_\lbt *Alloc} routines, as well as the {\tt SCOTCH\_\lbt
free} routine, are only available in the C interface.

\subsubsection{Machine word size issues}
\label{sec-lib-inttypesize}

Graph indices are represented in \scotch\ as integer values of type
{\tt SCOTCH\_\lbt Num}. By default, this type equates to the {\tt int}
C type, that is, an integer type of size equal to the one of 
the machine word. However, it can represent any other integer
type. Indeed, the size of the {\tt SCOTCH\_\lbt Num} integer type can
be coerced to 32 or 64 bits by using the ``{\tt -DINTSIZE32}'' or
``{\tt -DINTSIZE64}'' compilation flags, respectively, or else by
using the ``{\tt -DINT=}'' definition (see
Section~\ref{sec-install-inttypesize} for more information on the
setting of these compilation flags).

Consequently, the C interface of \scotch\ uses two types of integers.
Graph-related quantities are passed as {\tt SCOTCH\_\lbt Num}s,
while system-related values such as file handles, as well as
return values of \libscotch\ routines, are always passed as
{\tt int}s.

Because of the variability of library integer type sizes, one must be
careful when using the Fortran interface of \scotch, as it does not
provide any prototyping information, and consequently cannot produce
any warning at link time. In the manual pages of the
\libscotch\ routines, Fortran prototypes are written using three types
of {\tt INTEGER}s. As for the C interface, the regular {\tt INTEGER}
type is used for system-based values, such as file handles and MPI
communicators, as well as for return values of the
\libscotch\ routines, while the {\tt INTEGER*}{\it num} type
should be used for all graph-related values, in accordance to the size
of the {\tt SCOTCH\_\lbt Num} type, as set by the
``{\tt -DINTSIZE}{\it x\/}'' compilation flags. Also, the
{\tt INTEGER*}{\it idx} type represents an integer type of a size
equivalent to the one of a {\tt SCOTCH\_\lbt Idx}, as set by the
``{\tt -DIDXSIZE}{\it x\/}'' compilation flags. Values of this type
are used in the Fortran interface to represent arbitrary array indices
which can span across the whole address space, and consequently
deserve special treatment.

In practice, when \scotch\ is compiled on a 32-bit architecture so as
to use 64-bit {\tt SCOTCH\_\lbt Num}s, graph indices should be
declared as {\tt INTEGER*8}, while error return values
should still be declared as plain {\tt INTEGER} (that is,
{\tt INTEGER*4}) values. On a 32\_64-bit architecture, irrespective of
whether {\tt SCOTCH\_\lbt Num}s are defined as {\tt INTEGER*4}
or {\tt INTEGER*8} quantities, the {\tt SCOTCH\_\lbt Idx} type
should always be defined as a 64-bit quantity, that is, an
{\tt INTEGER*8}, because it stores differences between memory
addresses, which are represented by 64-bit values.
The above is no longer a problem if \scotch\ is compiled such that
{\tt int}s equate 64-bit integers. In this case, there is no need to
use any type coercing definition.
\\

The \metis\ v3 compatibility library provided by \scotch\ can also
run on a 64-bit architecture. Yet, if you are willing to use it this
way, you will have to replace all {\tt int}'s that are passed to the
\metis\ routines by 64-bit integer {\tt SCOTCH\_\lbt Num} values (even
the option configuration values). However, in this case, you will no
longer be able to link against the service routines of the genuine
\metis\ v3 library, as they are only available as a 32-bit
implementation.

\subsection{Data formats}

All of the data used in the \libscotch\ interface are of integer type
{\tt SCOTCH\_Num}. To hide the internals of \scotch\ to callers, all
of the data structures are opaque, that is, declared within ``{\tt
scotch.h}'' as dummy arrays of double precision values, for the sake of
data alignment. Accessor routines, the names of which end in ``{\tt
Size}'' and ``{\tt Data}'', allow callers to retrieve information from
opaque structures.
\\

In all of the following, whenever arrays are defined, passed, and
accessed, it is assumed that the first element of these arrays is
always labeled as {\tt baseval}, whether {\tt baseval} is set to $0$
(for C-style arrays) or $1$ (for Fortran-style arrays). \scotch\
internally manages with base values and array pointers so as to
process these arrays accordingly.

\subsubsection{Architecture format}

Target architecture structures are completely opaque. The only way
to describe an architecture is by means of a graph passed to the
{\tt SCOTCH\_\lbt arch\lbt Build} or {\tt SCOTCH\_\lbt arch\lbt
Build2} routines.

\subsubsection{Graph format}
\label{sec-lib-format-graph}

Source graphs are described by means of adjacency lists. The
description of a graph requires several {\tt SCOTCH\_Num} scalars and
arrays, as shown in Figures~\ref{fig-lib-graf-one}
and~\ref{fig-lib-graf-two}. They have the following meaning:
\begin{itemize}
\iteme[{\tt baseval}]
Base value for all array indexings.
\iteme[{\tt vertnbr}]
Number of vertices in graph.
\iteme[{\tt edgenbr}]
Number of arcs in graph. Since edges are represented by both of their
ends, the number of edge data in the graph is twice the number of
graph edges.
\iteme[{\tt verttab}]
Array of start indices in $\mathtt{edgetab}$ of vertex adjacency
sub-arrays.
\iteme[{\tt vendtab}]
Array of after-last indices in {\tt edgetab} of vertex adjacency
sub-arrays.
For any vertex $i$, with $\mathtt{baseval} \leq i < (\mathtt{baseval}
+ \mathtt{vertnbr})$, $\mathtt{vendtab}\mbox{\tt [}\mathit{i}\mbox{\tt
]} - \mathtt{verttab}\mbox{\tt [}\mathit{i}\mbox{\tt ]}$ is the degree
of vertex $i$, and the indices of the neighbors of $i$ are stored in
$\mathtt{edgetab}$ from $\mathtt{edgetab}\lbt \mbox{\tt
[}\mathtt{verttab}\mbox{\tt [}i\mbox{\tt ]]}$ to
$\mathtt{edgetab}\lbt \mbox{\tt [}\mathtt{vendtab}\mbox{\tt
[}i\mbox{\tt ]} - 1\mbox{\tt ]}$, inclusive.

When all vertex adjacency lists are stored in order in {\tt edgetab},
it is possible to save memory by not allocating the physical memory
for {\tt vendtab}. In this case, illustrated in
Figure~\ref{fig-lib-graf-one}, {\tt verttab} is of size
$\mathtt{vertnbr} + 1$ and $\mathtt{vendtab}$ points to
$\mathtt{verttab} + 1$. This case is referred to as the ``compact edge
array'' case, such that $\mathtt{verttab}$ is sorted in ascending
order, $\mathtt{verttab}\mbox{\tt [}\lbt\mathtt{baseval}\mbox{\tt ]} =
\mathtt{baseval}$ and $\mathtt{verttab}\lbt\mbox{\tt
[}\mathtt{baseval} + \mathtt{vertnbr}\mbox{\tt ]} =
(\mathtt{baseval} + \mathtt{edgenbr})$. 
\iteme[{\tt velotab}]
Optional array, of size {\tt vertnbr}, holding the integer load
associated with every vertex.
\iteme[{\tt edgetab}]
Array, of a size equal at least to
$\left(\max_{i}(\mathtt{vendtab}\mbox{\tt [}i\mbox{\tt ]}) -
\mathtt{baseval}\right)$, holding the adjacency array of every
vertex.
\iteme[{\tt edlotab}]
Optional array, of a size equal at least to $\left(\max_{i}(\mbox{\tt
vendtab[} i \mathtt{]}) - \mathtt{baseval}\right)$, holding the
integer load associated with every arc. Matching arcs should always
have identical loads.
\end{itemize}

\begin{figure}
\centering\includegraphics[scale=0.47]{s_f_gr1.eps}
\caption{Sample graph and its description by \libscotch\ arrays using
a compact edge array. Numbers within vertices are vertex indices, bold
numbers close to vertices are vertex loads, and numbers close to edges
are edge loads. Since the edge array is compact, {\tt verttab} is of
size $\mathtt{vertnbr} + 1$ and {\tt vendtab} points to
$\mathtt{verttab} + 1$.}
\label{fig-lib-graf-one}
\end{figure}

\begin{figure}
\centering\includegraphics[scale=0.47]{s_f_gr2.eps}
\caption{Adjacency structure of the sample graph of
Figure~\protect\ref{fig-lib-graf-one} with disjoint edge and
edge load arrays. Both {\tt verttab} and {\tt vendtab} are of
size {\tt vertnbr}. This allows for the handling of dynamic
graphs, the structure of which can evolve with time.}
\label{fig-lib-graf-two}
\end{figure}

Dynamic graphs can be handled elegantly by using the {\tt vendtab}
array. In order to dynamically manage graphs, one just has to allocate
{\tt verttab}, {\tt vendtab} and {\tt edgetab} arrays that are large
enough to contain all of the expected new vertex and edge
data. Original vertices are labeled starting from {\tt baseval},
leaving free space at the end of the arrays. To remove some vertex
$i$, one just has to replace ${\tt verttab}\mathtt{[}i\mbox{\tt
]}$ and ${\tt vendtab}\mathtt{[}i\mathtt{]}$ with the values of
{\tt verttab\lbt [vertnbr\lbt -1]} and {\tt vendtab\lbt [vertnbr\lbt
-1]}, respectively, and browse the adjacencies of all neighbors of
former vertex {\tt vertnbr-1} such that all {\tt (vertnbr-1)} indices
are turned into $i$s. Then, {\tt vertnbr} must be decremented, and
{\tt SCOTCH\_\lbt graphBuild()} must be called to account for the
change of topology. If a graph building routine such as {\tt
SCOTCH\_\lbt graph\lbt Load()} or {\tt SCOTCH\_\lbt graph\lbt Build()}
had already been called on the {\tt SCOTCH\_\lbt Graph} structure,
{\tt SCOTCH\_\lbt graph\lbt Free()} has to be called first in order to
free the internal structures associated with the older version of the
graph, else these data would be lost, which would result in memory
leakage.

To add a new vertex, one has to fill {\tt verttab\lbt [vertnbr\lbt
-1]} and {\tt vendtab\lbt [vertnbr\lbt -1]} with the starting and end
indices of the adjacency sub-array of the new vertex. Then, the
adjacencies of its neighbor vertices must also be updated to account
for it. If free space had been reserved at the end of each of the
neighbors, one just has to increment the ${\tt vendtab}\mbox{\tt
[}i\mathtt{]}$ values of every neighbor $i$, and add the index of
the new vertex at the end of the adjacency sub-array. If the sub-array
cannot be extended, then it has to be copied elsewhere in the edge
array, and both ${\tt verttab}\mathtt{[}i\mathtt{]}$ and ${\tt
vendtab}\mathtt{[}i\mathtt{]}$ must be updated accordingly. With
simple housekeeping of free areas of the edge array, dynamic arrays
can be updated with as little data movement as possible.

\subsubsection{Mesh format}
\label{sec-lib-format-mesh}

Since meshes are basically bipartite graphs, source meshes are also
described by means of adjacency lists. The description of a mesh
requires several {\tt SCOTCH\_Num} scalars and arrays, as shown in
Figure~\ref{fig-lib-mesh-one}. They have the following meaning:
\begin{itemize}
\iteme[{\tt velmbas}]
Base value for element indexings.
\iteme[{\tt vnodbas}]
Base value for node indexings. The base value of the underlying graph,
{\tt baseval}, is set as $\min(\mathtt{velmbas}, \mathtt{vnodbas})$.
\iteme[{\tt velmnbr}]
Number of element vertices in mesh.
\iteme[{\tt vnodnbr}]
Number of node vertices in mesh.
The overall number of vertices in the underlying graph, {\tt vertnbr},
is set as $\mathtt{velmnbr} +\mathtt{vnodnbr}$.
\iteme[{\tt edgenbr}]
Number of arcs in mesh. Since edges are represented by both of their
ends, the number of edge data in the mesh is twice the number of
edges.
\iteme[{\tt verttab}]
Array of start indices in {\tt edgetab} of vertex (that is, both
elements and nodes) adjacency sub-arrays.
\iteme[{\tt vendtab}]
Array of after-last indices in {\tt edgetab} of vertex adjacency
sub-arrays. For any element or node vertex $i$, with
$\mathtt{baseval} \leq i <
(\mathtt{baseval} + \mathtt{vertnbr})$, $\mbox{\tt
vendtab[}i\mathtt{]} - \mathtt{verttab[}i\mathtt{]}$ is the
degree of vertex $i$, and the indices of the neighbors of $i$ are
stored in {\tt edgetab} from {\tt edgetab\lbt [verttab[$i$]]} to {\tt
edgetab\lbt [vendtab[$i$]\lbt $- 1$]}, inclusive.

When all vertex adjacency lists are stored in order in {\tt edgetab},
it is possible to save memory by not allocating the physical memory
for {\tt vendtab}. In this case, illustrated in
Figure~\ref{fig-lib-mesh-one}, {\tt verttab} is of size $\mbox{\tt
vertnbr} + 1$ and {\tt vendtab} points to $\mathtt{verttab} + 1$.
This case is referred to as the ``compact edge array'' case, such that
{\tt verttab} is sorted in ascending order,
$\mathtt{verttab[}\lbt\mathtt{baseval]} = \mathtt{baseval}$
and $\mathtt{verttab[}\lbt\mathtt{baseval} + \mbox{\tt
vertnbr]} = (\mathtt{baseval} + \mathtt{edgenbr})$.
\iteme[{\tt velotab}]
Array, of size {\tt vertnbr}, holding the integer load associated with
each vertex.
\end{itemize}

\begin{figure}
\centering\includegraphics[scale=0.47]{s_f_me1.eps}
\caption{Sample mesh and its description by \libscotch\ arrays using a
compact edge array. Numbers within vertices are vertex indices. Since the
edge array is compact, {\tt verttab} is of size $\mathtt{vertnbr} +
1$ and {\tt vendtab} points to $\mathtt{verttab} + 1$.}
\label{fig-lib-mesh-one}
\end{figure}

As for graphs, it is possible to handle elegantly dynamic meshes by
means of the {\tt verttab} and {\tt vendtab} arrays. There is,
however, an additional constraint, which is that mesh nodes and
elements must be ordered consecutively. The solution to fulfill this
constraint in the context of mesh ordering is to keep a set of empty
elements (that is, elements which have no node adjacency attached to
them) between the element and node arrays. For instance,
Figure~\ref{fig-lib-mesh-two} represents a $4$-element mesh with $6$
nodes, and such that $4$ element vertex slots have been reserved for
new elements and nodes. These slots are empty elements for which
{\tt verttab[i]} equals {\tt vendtab[i]}, irrespective of these
values, since they will not lead to any memory access in {\tt edgetab}.

\begin{figure}
\centering\includegraphics[scale=0.47]{s_f_me2.eps}
\caption{Sample mesh and its description by \libscotch\ arrays, with
nodes numbered first and elements numbered last. In order to allow for
dynamic re-meshing, empty elements (in grey) have been inserted between
existing node and element vertices.}
\label{fig-lib-mesh-two}
\end{figure}

Using this layout of vertices, new nodes and elements can be created
by growing the element and node sub-arrays into the empty element
sub-array, by both of its sides, without having to re-write the whole
mesh structure, as illustrated in
Figure~\ref{fig-lib-mesh-three}. Empty elements are transparent to the
mesh ordering routines, which base their work on node vertices only.
Users who want to update the arrays of a mesh that has already
been declared using the {\tt SCOTCH\_\lbt mesh\lbo Build} routine must
call {\tt SCOTCH\_\lbt mesh\lbo Exit} prior to updating the mesh arrays,
and then call {\tt SCOTCH\_\lbt mesh\lbo Build} again after the arrays
have been updated, so that the {\tt SCOTCH\_\lbt Mesh} structure remains
consistent with the new mesh data.

\begin{figure}
\centering\includegraphics[scale=0.47]{s_f_me3.eps}
\caption{Re-meshing of the mesh of Figure~\protect\ref{fig-lib-mesh-two}.
New node vertices have been added at the end of the vertex sub-array,
new elements have been added at the beginning of the element sub-array,
and vertex base values have been updated accordingly. Node adjacency
lists that could not fit in place have been added at the end of the edge
array, and some of the freed space has been re-used for new adjacency
lists. Element adjacency lists do not require moving in this case, as
all of the elements have the name number of nodes.}
\label{fig-lib-mesh-three}
\end{figure}

\subsubsection{Geometry format}
\label{sec-lib-format-geom}

Geometry data is always associated with a graph or a mesh. It is
simply made of a single array of double-precision values which
represent the coordinates of the vertices of a graph, or of
the node vertices of a mesh, in vertex order. The fields of
a geometry structure are the following:
\begin{itemize}
\iteme[{\tt dimnnbr}]
Number of dimensions of the graph or of the mesh, which can be $1$,
$2$, or $3$.
\iteme[{\tt geomtab}]
Array of coordinates. This is an array of double precision values
organized as an array of $(x)$, or $(x,y)$, or $(x,y,z)$ tuples,
according to $\mathtt{dimnnbr}$. Coordinates that are not used
(e.g. the $z$ coordinates for a bidimentional object) are not
allocated. Therefore, the $x$ coordinate of some graph vertex $i$ is
located at $\mathtt{geomtab}\mbox{\tt [}(i - \mathtt{baseval}) *
\mathtt{dimnnbr}  + \mathtt{baseval}\mbox{\tt ]}$, its $y$ coordinate
is located at $\mathtt{geomtab}\mbox{\tt [}(i - \mathtt{baseval}) *
  \mathtt{dimnnbr} + \mathtt{baseval} + 1\mbox{\tt ]}$ if
$\mathtt{dimnnbr} \geq 2$, and its $z$ coordinate is located at
$\mathtt{geomtab}\mbox{\tt [}(i - \mathtt{baseval}) * \mathtt{dimnnbr}
+ \mathtt{baseval} + 2\mbox{\tt ]}$ if $\mathtt{dimnnbr} =
3$. Whenever the geometry is associated with a mesh, 
only node vertices are considered, so the $x$ coordinate of some mesh
node vertex $i$, with $\mathtt{vnodbas} \leq i$, is located
at $\mathtt{geomtab}\mbox{\tt [}(i - \mathtt{vnodbas}) *
\mathtt{dimnnbr} + \mathtt{baseval}\mbox{\tt ]}$, its $y$ coordinate
is located at $\mathtt{geomtab}\mbox{\tt [}(i - \mathtt{vnodbas}) *
\mathtt{dimnnbr} + \mathtt{baseval} + 1\mbox{\tt ]}$ if
$\mathtt{dimnnbr} \geq 2$, and its $z$ coordinate is located at
$\mathtt{geomtab}\mbox{\tt [}(i - \mathtt{vnodbas}) * \mathtt{dimnnbr}
+ \mathtt{baseval} + 2\mbox{\tt ]}$ if $\mbox{\tt dimnnbr} = 3$.
\end{itemize}

\subsubsection{Block ordering format}
\label{sec-lib-format-order}

Block orderings associated with graphs and meshes are described by
means of block and permutation arrays, made of {\tt SCOTCH\_Num}s, as
shown in Figure~\ref{fig-lib-ord-block}. In order for all orderings to
have the same structure, irrespective of whether they are created from
graphs or meshes, all ordering data indices start from {\tt baseval},
even when they refer to a mesh the node vertices of which are labeled
from a $\mathtt{vnodbas}$ index such that $\mathtt{vnodbas} >
\mathtt{baseval}$. Consequently, row indices are related to vertex
indices in memory in the following way: row $i$ is associated with
vertex $i$ of the {\tt SCOTCH\_\lbt Graph} structure if the ordering
was computed from a graph, and with node vertex $i + (\mbox{\tt
vnodbas} - \mathtt{baseval})$ of the {\tt SCOTCH\_\lbt Mesh}
structure if the ordering was computed from a mesh.
Block orderings are made of the following data:
\begin{itemize}
\iteme[{\tt permtab}]
Array holding the permutation of the reordered matrix. Thus, if $k =
\mathtt{permtab}\mbox{\tt [}\mathit{i}\mbox{\tt ]}$, then row $i$ of
the original matrix is now row $k$ of the reordered matrix, that is,
row $i$ is the $k^{\mathrm{th}}$ pivot.
\iteme[{\tt peritab}]
Inverse permutation of the reordered matrix. Thus, if
$i = \mathtt{peritab}\mbox{\tt [}\mathit{k}\mbox{\tt ]}$, then row $k$
of the reordered matrix was row $i$ of the original matrix.
\iteme[{\tt cblknbr}]
Number of column blocks (that is, supervariables) in the block ordering.
\iteme[{\tt rangtab}]
Array of ranges for the column blocks. Column block $c$, with
$\mathtt{baseval} \leq c < (\mathtt{cblknbr} + \mathtt{baseval})$,
contains columns with indices ranging from
$\mathtt{rangtab}\mbox{\tt [}\mathit{i}\mbox{\tt ]}$ to
$\mathtt{rangtab}\mbox{\tt [}\mathit{i} + 1\mbox{\tt ]}$, exclusive,
in the reordered matrix. Indices in $\mathtt{rangtab}$ are
based. Therefore,
$\mathtt{rangtab}\mbox{\tt [}\mathtt{baseval}\mbox{\tt ]}$ is always
equal to $\mathtt{baseval}$, and
$\mathtt{rangtab}\mbox{\tt [}\mathtt{cblknbr} +
\mathtt{baseval}\mbox{\tt ]}$ is always equal to
$\mathtt{vertnbr} + \mathtt{baseval}$ for graphs and
to $\mathtt{vnodnbr} + \mathtt{baseval}$ for meshes.
In order to avoid memory errors when column blocks are all single
columns, the size of {\tt rangtab} must always be one more than the
number of columns, that is, $\mathtt{vertnbr} + 1$ for graphs and
$\mathtt{vnodnbr} + 1$ for meshes.
\iteme[{\tt treetab}]
Array of ascendants of permuted column blocks in the separators tree.
{\tt treetab[i]} is the index of the father of column block $i$ in the
separators tree, or $-1$ if column block $i$ is the root of the
separators tree. Whenever separators or leaves of the separators tree
are split into subblocks, as the block splitting, minimum fill or minimum
degree methods do, all subblocks of the same level are linked to the
column block of higher index belonging to the closest separator
ancestor. Indices in {\tt treetab} are based, in the same way as for
the other blocking structures. See Figure~\ref{fig-lib-ord-block} for
a complete example.
\end{itemize}

\begin{figure}
\centering\includegraphics[scale=0.47]{s_f_orb.eps}
\caption{Arrays resulting from the ordering by complete nested
dissection of a 4 by 3 grid based from $1$. Leftmost grid is the
original grid, and righmost grid is the reordered grid, with
separators shown and column block indices written in bold.}
% using strategy  n{sep=hf{bal=0},ole=s,ose=s}
\label{fig-lib-ord-block}
\end{figure}

\subsection{Strategy strings}

The behavior of the mapping and block ordering routines of the
\libscotch\ library is parametrized by means of strategy strings,
which describe how and when given partitioning or ordering methods
should be applied to graphs and subgraphs, or to meshes and submeshes.

\subsubsection{Using default strategy strings}
\label{sec-lib-format-strat-default}

While strategy strings can be built by hand, according to the syntax
given in the next sections, users who do not have specific needs can
take advantage of default strategies already implemented in the
\libscotch, which will yield very good results in most cases. By
doing so, they will spare themselves the hassle of updating their
strategies to comply to subsequent syntactic changes, and they will
benefit from the availability of new partitioning or ordering methods
as soon as they are released.

The simplest way to use default strategy strings is to avoid
specifying any. By initializing a strategy object, by means of the
{\tt SCOTCH\_\lbt stratInit} routine, and by using the initialized
strategy object as is, without further parametrization, this object
will be filled with a default strategy when passing it as a parameter
to the next partitioning or ordering routine to be called. On return,
the strategy object will contain a fully specified strategy, tailored
for the type of operation which has been requested. Consequently, a
fresh strategy object that was used to partition a graph cannot be
used afterward as a default strategy when calling an ordering routine,
for instance, as partitioning and ordering strategies are incompatible.

The \libscotch\ also provides helper routines which allow users to
express their preferences on the kind of strategy that they
need. These helper routines, which are of the form
{\tt SCOTCH\_\lbt strat\lbt *\lbt Build} (see
Section~\ref{sec-lib-func-stratgraphclusterbuild} and after), tune
default strategy strings according to parameters provided by the user,
such as the requested number of parts (used as a hint to select the
most efficient partitioning routines), the desired maximum load
imbalance ratio, and a set of preference flags. While some of these
flags are antagonistic, most of them can be combined, by means of
addition or ``binary or'' operators. These flags are the following.
They are grouped by application class.

\paragraph{Global flags}

\begin{itemize}
\iteme[{\tt SCOTCH\_STRATDEFAULT}]
Default behavior. No flags are set.
\iteme[{\tt SCOTCH\_STRATBALANCE}]
Enforce load balance as much as possible.
\iteme[{\tt SCOTCH\_STRATQUALITY}]
Privilege quality over speed.
\iteme[{\tt SCOTCH\_STRATSAFETY}]
Do not use methods that can lead to the occurrence of problematic
events, such as floating point exceptions, which could not be properly
handled by the calling software.
\iteme[{\tt SCOTCH\_STRATSPEED}]
Privilege speed over quality.
\end{itemize}

\paragraph{Mapping and partitioning flags}

\begin{itemize}
\iteme[{\tt SCOTCH\_STRATRECURSIVE}]
Use only recursive bipartitioning methods, and not direct k-way
methods. When this flag is not set, any combination of methods can be
used, so as to achieve the best result according to other user
preferences.
\iteme[{\tt SCOTCH\_STRATREMAP}]
Use the strategy for remapping an existing partition.
\end{itemize}

\paragraph{Ordering flags}

\begin{itemize}
\iteme[{\tt SCOTCH\_STRATDISCONNECTED}]
Find and handle independently disconnected components.
\iteme[{\tt SCOTCH\_STRATLEVELMAX}]
Create at most the prescribed levels of nested dissection separators.
\iteme[{\tt SCOTCH\_STRATLEVELMIN}]
Create at least the prescribed levels of nested dissection separators.
When used in conjunction with {\tt SCOTCH\_\lbt STRAT\lbt LEVEL\lbt
MAX}, the exact number of nested dissection levels will be performed,
unless the graph to order is too small.
\iteme[{\tt SCOTCH\_STRATLEAFSIMPLE}]
Order nested dissection leaves as cheaply as possible.
\iteme[{\tt SCOTCH\_STRATSEPASIMPLE}]
Order nested dissection separators as cheaply as possible.
\end{itemize}

\subsubsection{Mapping strategy strings}
\label{sec-lib-format-map}

At the time being, mapping methods only apply to graphs, as there is
not yet a mesh mapping tool in the \scotch\ package.
Mapping strategies are made of methods, with optional parameters enclosed
between curly braces, and separated by commas, in the form of
{\it method\/}{[{\tt \{}{\it parameters\/}{\tt \}}]}\enspace.
The currently available mapping methods are the following.

\begin{itemize}
\iteme[{\tt b}]
Band method. This method builds a band graph of given width around the
current frontier of the k-way partition to which it is applied, and
calls a graph mapping strategy to refine the equivalent k-way
partition of the band graph. Then, the refined frontier of the band
graph is projected back to the current graph. This method was
initially presented in~\cite{chpe06a} in the case of bipartitioning.
The parameters of the band bipartitioning method are listed below.
\begin{itemize}
\iteme[{\tt bnd=}{\it strat}]
Set the graph mapping strategy to be used on the band graph.
\iteme[{\tt org=}{\it strat}]
Set the fallback graph mapping strategy to be used on the
original graph if the band graph strategy could not be used. The three
cases which require the use of this fallback strategy are the
following. First, if the separator of the original graph is empty,
which makes it impossible to compute a band graph. Second, if any part
of the band graph to be built is of the same size as the one of the
original graph. Third, if the application of the {\tt bnd}
bipartitioning method to the band graph leads to a situation where any
two anchor vertices are placed in the same part.
\iteme[{\tt width=}{\it val}]
Set the width of the band graph. All graph vertices that are at a
distance less than or equal to {\it val} from any frontier vertex
are kept in the band graph.
\end{itemize}
\iteme[{\tt d}]
Diffusion method. This method, presented in~\cite{pell07b} in the case
of bipartitioning, flows $k$ kinds of antagonistic liquids from $k$
source vertices, and sets the new frontier as the limit between vertices
which contain different kinds of liquids. Because
selecting the source vertices is essential to the obtainment of useful
results, this method has been hard-coded so that the $k$ source
vertices are the $k$ vertices of highest indices, since in the band
method these are the anchor vertices which represent all of the removed
vertices of each part. Therefore, this method must be used on band
graphs only, or on specifically crafted graphs. Applying it to any
other graphs is very likely to lead to extremely poor results.
The physical analogy of this method loses weight when it is applied to
target architectures that are not complete graphs.
The parameters of the diffusion mapping method are listed below.
\begin{itemize}
\iteme[{\tt dif=}{\it rat}]
Fraction of liquid which is diffused to neighbor vertices at each
pass. To achieve convergence, the sum of the {\tt dif} and {\tt rem}
parameters must be equal to $1$, but in order to speed-up the diffusion
process, other combinations of higher sum can be tried. In this case,
the number of passes must be kept low, to avoid numerical overflows
which would make the results useless.
\iteme[{\tt pass=}{\it nbr}]
Set the number of diffusion sweeps performed by the algorithm. This
number depends on the width of the band graph to which the diffusion
method is applied. Useful values range from $30$ to $500$ according
to chosen {\tt dif} and {\tt rem} coefficients.
\iteme[{\tt rem=}{\it rat}]
Fraction of liquid which remains on vertices at each pass. See above.
\end{itemize}
\iteme[{\tt f}]
$k$-way Fiduccia-Mattheyses method. The parameters of the
Fiduccia-Mattheyses method are listed below.
\begin{itemize}
\iteme[{\tt bal=}{\it rat}]
Set the maximum weight imbalance ratio to the given fraction of
the subgraph vertex weight. Common values are around $0.01$, that
is, one percent.
\iteme[{\tt move=}{\it nbr}]
Maximum number of hill-climbing moves that can be performed before a
pass ends. During each of its passes, the Fiduccia-Mattheyses
algorithm repeatedly swaps vertices between parts so as to
minimize the cost function. A pass completes either when all of the
vertices have been moved once, or if too many swaps that do not
decrease the value of the cost function have been performed. Setting
this value to zero turns the Fiduccia-Mattheyses algorithm into a
gradient-like method, which may be used to quickly refine partitions
during the uncoarsening phase of the multilevel method.
\iteme[{\tt pass=}{\it nbr}]
Set the maximum number of optimization passes performed by the
algorithm. The Fiduccia-Mattheyses algorithm stops as soon as a pass
has not yielded any improvement of the cost function, or when the
maximum number of passes has been reached. Value $-1$ stands for an
infinite number of passes, that is, as many as needed by the algorithm
to converge.
\end{itemize}
\iteme[{\tt m}]
Multilevel method. The parameters of the multilevel method are listed below.
\begin{itemize}
\iteme[{\tt asc=}{\it strat}]
Set the strategy that is used to refine the mappings obtained
at ascending levels of the uncoarsening phase by projection of the
mappings computed for coarser graphs.
This strategy is not applied to the coarsest graph, for which only the
{\tt low} strategy is used.
\iteme[{\tt low=}{\it strat}]
Set the strategy that is used to compute the mapping of the
coarsest graph, at the lowest level of the coarsening process.
\iteme[{\tt rat=}{\it rat}]
Set the threshold maximum coarsening ratio over which graphs are no longer
coarsened. The ratio of any given coarsening cannot be less that $0.5$
(case of a perfect matching), and cannot be greater than $1.0$.
Coarsening stops when either the coarsening ratio is above the maximum
coarsening ratio, or the graph has fewer vertices than the minimum number of
vertices allowed.
\iteme[{\tt vert=}{\it nbr}]
Set the threshold under which graphs are no longer
coarsened. Coarsening stops when either the coarsening ratio is above
the maximum coarsening ratio, or the graph would have fewer vertices
than the minimum number of vertices allowed. When the target
architecture is a variable-sized architecture, coarsening stops when
the coarsened graph would have less than \mbox{\it nbr} vertices. When
the target architecture is a regular, fixed-size, architecture,
coarsening stops when each subdomain would have less than \mbox{\it
  nbr} vertices, that is, when the size of the coarsened graph would
have less than $\mbox{\it nbr}\times\mathtt{domnnbr}$ vertices,
where $\mathtt{domnnbr}$ is the number of vertices in the target
architecture.
\end{itemize}
\iteme[{\tt r}]
Dual Recursive Bipartitioning mapping algorithm, as defined in
section~\ref{sec-algo-drb}. The parameters of the DRB mapping method are
listed below.
\begin{itemize}
\iteme[{\tt job=}{\it tie}]
The {\it tie\/} flag defines how new jobs are stored in job pools.
\begin{itemize}
\iteme[{\tt t}]
Tie job pools together. Subjobs are stored in same pool as their parent job.
This is the default behavior, as it proves the most efficient in practice.
\iteme[{\tt u}]
Untie job pools. Subjobs are stored in the next job pool to be processed.
\end{itemize}
\iteme[{\tt map=}{\it tie}]
The {\it tie\/} flag defines how results of bipartitioning jobs are
propagated to jobs still in pools.
\begin{itemize}
\iteme[{\tt t}]
Tie both mapping tables together. Results are immediately available to jobs
in the same job pool. This is the default behavior.
\iteme[{\tt u}]
Untie mapping tables. Results are only available to jobs of next pool to be
processed.
\end{itemize}
\iteme[{\tt poli=}{\it policy}]
Select jobs according to policy {\it policy}. Job selection policies
define how bipartitioning jobs are ordered within the currently active job
pool. Valid policy flags are
\begin{itemize}
\iteme[{\tt L}]
Most neighbors of higher level.
\iteme[{\tt l}]
Highest level.
\iteme[{\tt r}]
Random.
\iteme[{\tt S}]
Most neighbors of smaller size. This is the default behavior.
\iteme[{\tt s}]
Biggest size.
\end{itemize}
\iteme[{\tt sep=}{\it strat}]
Apply bipartitioning strategy {\it strat\/} to each bipartitioning job.
A bipartitioning strategy is made of one or several bipartitioning methods,
which can be combined by means of strategy operators. Graph
bipartitioning strategies are described below.
\end{itemize}
\iteme[{\tt x}]
Exactifier method, as defined in Section~\ref{sec-algo-map-exact}.
This greedy algorithm refines the current mapping so as to reduce load
imbalance as much as possible. Since this method does not consider
communication minimization, its use should be restricted to cases
where achieving load balance is critical and where recursive
bipartitioning may fail to achieve it, because of very irregular
vertex loads.
\end{itemize}

\subsubsection{Graph bipartitioning strategy strings}
\label{sec-lib-format-bipart}

A graph bipartitioning strategy is made of one or several graph
bipartitioning methods, which can be combined by means of strategy
operators. Strategy operators are listed below, by increasing
precedence.

\begin{itemize}
\iteme[{\it strat1\/}{\tt |}{\it strat2}]
Selection operator. The result of the selection is the best bipartition of
the two that are obtained by the separate application of {\it strat1\/} and
{\it strat2\/} to the current bipartition.
\iteme[{\it strat1$\:$}{\it strat2}]
Combination operator. Strategy {\it strat2\/} is applied to the bipartition
resulting from the application of strategy {\it strat1\/} to the current
bipartition. Typically, the first method used should compute an initial
bipartition from scratch, and every following method should use the
result of the previous one at its starting point.
\iteme[{\tt (}{\it strat\/}{\tt )}]
Grouping operator.
The strategy enclosed within the parentheses is treated as a single
bipartitioning method.
\iteme[{\tt /}{\it cond\/}{\tt ?}{\it strat1\/}{[{\tt :}{\it strat2}]{\tt ;}}]
Condition operator. According to the result of the evaluation of condition
{\it cond}, either {\it strat1\/} or {\it strat2\/} (if it is present) is
applied. The condition applies to the characteristics of the current active
graph, and can be built from logical and relational operators. Conditional
operators are listed below, by increasing precedence.
\begin{itemize}
\iteme[{\it cond1\/}{\tt |}{\it cond2}]
Logical or operator. The result of the condition is true if {\it cond1\/}
or {\it cond2\/} are true, or both.
\iteme[{\it cond1\/}{\tt \&}{\it cond2}]
Logical and operator. The result of the condition is true only if both
{\it cond1\/} and {\it cond2\/} are true.
\iteme[{\tt !}{\it cond}]
Logical not operator. The result of the condition is true only if
{\it cond\/} is false.
\iteme[{\it var} {\it relop} {\it val}]
Relational operator, where {\it var\/} is a graph variable, {\it val\/} is
either a graph variable or a constant of the type of variable {\it var\/}, and
{\it relop\/} is one of '{\tt\verb+<+}', '{\tt\verb+=+}', and '{\tt\verb+>+}'.
The graph variables are listed below, along with their types.
\begin{itemize}
\iteme[{\tt deg}]
The average degree of the current graph.
Float.
\iteme[{\tt edge}]
The number of arcs (which is twice the number of edges) of the current graph.
Integer.
\iteme[{\tt load}]
The overall vertex load (weight) of the current graph.
Integer.
\iteme[{\tt load0}]
The vertex load of the first subset of the current bipartition of the current
graph.
Integer.
\iteme[{\tt vert}]
The number of vertices of the current graph.
Integer.
\end{itemize}
\end{itemize}
\iteme[{\it method\/}{[{\tt \{}{\it parameters\/}{\tt \}}]}]
Bipartitioning method. For bipartitioning methods that can be
parametrized, parameter settings may be provided after the method
name. Parameters must be separated by commas, and the whole list
be enclosed between curly braces.
\end{itemize}
The currently available graph bipartitioning methods are the following.
\begin{itemize}
\iteme[{\tt b}]
Band method. This method builds a band graph of given width around the
current frontier of the graph to which it is applied, and
calls a graph bipartitioning strategy to refine the equivalent
bipartition of the band graph. Then, the refined frontier of the band
graph is projected back to the current graph. This method, presented
in~\cite{chpe06a}, was created to reduce the cost of vertex separator
refinement algorithms in a multilevel context, but it improves
partition quality too. The same behavior is observed for graph
bipartitioning. The parameters of the band bipartitioning method are
listed below.
\begin{itemize}
\iteme[{\tt bnd=}{\it strat}]
Set the graph bipartitioning strategy to be used on the band graph.
\iteme[{\tt org=}{\it strat}]
Set the fallback graph bipartitioning strategy to be used on the
original graph if the band graph strategy could not be used. The three
cases which require the use of this fallback strategy are the
following. First, if the separator of the original graph is empty,
which makes it impossible to compute a band graph. Second, if any part
of the band graph to be built is of the same size as the one of the
original graph. Third, if the application of the {\tt bnd}
bipartitioning method to the band graph leads to a situation where both
anchor vertices are placed in the same part.
\iteme[{\tt width=}{\it val}]
Set the width of the band graph. All graph vertices that are at a
distance less than or equal to {\it val} from any frontier vertex
are kept in the band graph.
\end{itemize}
\iteme[{\tt d}]
Diffusion method. This method, presented in~\cite{pell07b}, flows two
kinds of antagonistic liquids, scotch and anti-scotch, from two source
vertices, and sets the new frontier as the limit between vertices
which contain scotch and the ones which contain anti-scotch. Because
selecting the source vertices is essential to the obtainment of useful
results, this method has been hard-coded so that the two source
vertices are the two vertices of highest indices, since in the band
method these are the anchor vertices which represent all of the removed
vertices of each part. Therefore, this method must be used on band
graphs only, or on specifically crafted graphs. Applying it to any
other graphs is very likely to lead to extremely poor results.
The parameters of the diffusion bipartitioning method are
listed below.
\begin{itemize}
\iteme[{\tt dif=}{\it rat}]
Fraction of liquid which is diffused to neighbor vertices at each
pass. To achieve convergence, the sum of the {\tt dif} and {\tt rem}
parameters must be equal to $1$, but in order to speed-up the diffusion
process, other combinations of higher sum can be tried. In this case,
the number of passes must be kept low, to avoid numerical overflows
which would make the results useless.
\iteme[{\tt pass=}{\it nbr}]
Set the number of diffusion sweeps performed by the algorithm. This
number depends on the width of the band graph to which the diffusion
method is applied. Useful values range from $30$ to $500$ according
to chosen {\tt dif} and {\tt rem} coefficients.
\iteme[{\tt rem=}{\it rat}]
Fraction of liquid which remains on vertices at each pass. See above.
\end{itemize}
\iteme[{\tt f}]
Fiduccia-Mattheyses method. The parameters of the Fiduccia-Mattheyses method
are listed below.
\begin{itemize}
\iteme[{\tt bal=}{\it rat}]
Set the maximum weight imbalance ratio to the given fraction of
the subgraph vertex weight. Common values are around $0.01$, that
is, one percent.
\iteme[{\tt move=}{\it nbr}]
Maximum number of hill-climbing moves that can be performed before a
pass ends. During each of its passes, the Fiduccia-Mattheyses
algorithm repeatedly swaps vertices between the two parts so as to
minimize the cost function. A pass completes either when all of the
vertices have been moved once, or if too many swaps that do not
decrease the value of the cost function have been performed. Setting
this value to zero turns the Fiduccia-Mattheyses algorithm into a
gradient-like method, which may be used to quickly refine partitions
during the uncoarsening phase of the multilevel method.
\iteme[{\tt pass=}{\it nbr}]
Set the maximum number of optimization passes performed by the
algorithm. The Fiduccia-Mattheyses algorithm stops as soon as a pass
has not yielded any improvement of the cost function, or when the
maximum number of passes has been reached. Value $-1$ stands for an
infinite number of passes, that is, as many as needed by the algorithm
to converge.
\end{itemize}
\iteme[{\tt g}]
Gibbs-Poole-Stockmeyer method. This method has only one parameter.
\begin{itemize}
\iteme[{\tt pass=}{\it nbr}]
Set the number of sweeps performed by the algorithm.
\end{itemize}
\iteme[{\tt h}]
Greedy-graph-growing method. This method has only one parameter.
\begin{itemize}
\iteme[{\tt pass=}{\it nbr}]
Set the number of runs performed by the algorithm.
\end{itemize}
\iteme[{\tt m}]
Multilevel method. The parameters of the multilevel method are listed below.
\begin{itemize}
\iteme[{\tt asc=}{\it strat}]
Set the strategy that is used to refine the partitions obtained
at ascending levels of the uncoarsening phase by projection of the
bipartitions computed for coarser graphs.
This strategy is not applied to the coarsest graph, for which only the
{\tt low} strategy is used.
\iteme[{\tt low=}{\it strat}]
Set the strategy that is used to compute the partition of the
coarsest graph, at the lowest level of the coarsening process.
\iteme[{\tt rat=}{\it rat}]
Set the threshold maximum coarsening ratio over which graphs are no longer
coarsened. The ratio of any given coarsening cannot be less that $0.5$
(case of a perfect matching), and cannot be greater than $1.0$.
Coarsening stops when either the coarsening ratio is above the maximum
coarsening ratio, or the graph has fewer vertices than the minimum number of
vertices allowed.
\iteme[{\tt vert=}{\it nbr}]
Set the threshold minimum graph size under which graphs are no longer
coarsened. Coarsening stops when either the coarsening ratio is above the
maximum coarsening ratio, or the coarsened graph would have fewer
vertices than the minimum number of vertices allowed.
\end{itemize}
\iteme[{\tt x}]
Exactifying method.
\iteme[{\tt z}]
Zero method. This method moves all of the vertices to the first
part. Its main use is to stop the bipartitioning process, if some
condition is true, when mapping onto variable-sized architectures (see
section~\ref{sec-algo-variable}).
\end{itemize}

\subsubsection{Vertex partitioning strategy strings}
\label{sec-lib-format-part-ovl}

Vertex partitioning is a special form of graph partitioning, in which
graphs are partitioned into a prescribed number of parts by means of
vertex separators rather than of edge separators like in
Section~\ref{sec-lib-format-map}. The load balance criterion also
differs from common practice: the load to be balanced across all parts
comprises not only the load of the vertices which belong to the part,
but also the load of all the separator vertices which are their
immediate neighbors. Consequently, the load of every separator vertex
is accounted for several times, in each of the parts it separates.

Vertex partitioning strategies are made of methods, with optional
parameters enclosed between curly braces, and separated by commas, in
the form of {\it method\/}{[{\tt \{}{\it parameters\/}{\tt \}}]}\enspace.
The currently available mapping methods are the following.

\begin{itemize}
\iteme[{\tt f}]
Fiduccia-Mattheyses method. The parameters of the Fiduccia-Mattheyses method
are listed below.
\begin{itemize}
\iteme[{\tt bal=}{\it rat}]
Set the maximum weight imbalance ratio to the given fraction of
the subgraph vertex weight. Common values are around $0.01$, that
is, one percent.
\iteme[{\tt move=}{\it nbr}]
Maximum number of hill-climbing moves that can be performed before a
pass ends. During each of its passes, the Fiduccia-Mattheyses
algorithm repeatedly moves vertices between parts so as to
minimize the cost function. A pass completes either when all of the
vertices have been moved once, or if too many swaps that do not
decrease the value of the cost function have been performed. Setting
this value to zero turns the Fiduccia-Mattheyses algorithm into a
gradient-like method, which may be used to quickly refine partitions
during the uncoarsening phase of the multilevel method.
\iteme[{\tt pass=}{\it nbr}]
Set the maximum number of optimization passes performed by the
algorithm. The Fiduccia-Mattheyses algorithm stops as soon as a pass
has not yielded any improvement of the cost function, or when the
maximum number of passes has been reached. Value $-1$ stands for an
infinite number of passes, that is, as many as needed by the algorithm
to converge.
\end{itemize}
\iteme[{\tt g}]
Gibbs-Poole-Stockmeyer method. This is a k-way version of the
original algorithm, in which parts are grown one after the
other. Consequently, depending on graph topology, this method is
likely to yield disconnected parts, with higher probability as the
number of part increases. This method has only one parameter.
\begin{itemize}
\iteme[{\tt pass=}{\it nbr}]
Set the number of sweeps performed by the algorithm.
\end{itemize}
\iteme[{\tt h}]
Greedy-graph-growing method. This is a $k$-way version of the
original algorithm, in which parts are grown one after the
other. Consequently, depending on graph topology, this method is
likely to yield disconnected parts, with higher probability as the
number of part increases. This method has only one parameter.
\begin{itemize}
\iteme[{\tt pass=}{\it nbr}]
Set the number of runs performed by the algorithm.
\end{itemize}
\iteme[{\tt m}]
Multilevel method. The parameters of the multilevel method are listed below.
\begin{itemize}
\iteme[{\tt asc=}{\it strat}]
Set the strategy that is used to refine the partitions obtained
at ascending levels of the uncoarsening phase by projection of the
bipartitions computed for coarser graphs.
This strategy is not applied to the coarsest graph, for which only the
{\tt low} strategy is used.
\iteme[{\tt low=}{\it strat}]
Set the strategy that is used to compute the partition of the
coarsest graph, at the lowest level of the coarsening process.
\iteme[{\tt rat=}{\it rat}]
Set the threshold maximum coarsening ratio over which graphs are no longer
coarsened. The ratio of any given coarsening cannot be less that $0.5$
(case of a perfect matching), and cannot be greater than $1.0$.
Coarsening stops when either the coarsening ratio is above the maximum
coarsening ratio, or the graph has fewer vertices than the minimum number of
vertices allowed.
\iteme[{\tt vert=}{\it nbr}]
Set the threshold minimum number of vertices per part under which
graphs are no longer coarsened. Coarsening stops when either the
coarsening ratio is above the maximum coarsening ratio, or the graph
has fewer vertices than the minimum number of vertices allowed.
\end{itemize}
\iteme[{\tt r}]
Recursive bipartitioning algorithm. The parameters of the recursive
bipartitioning method are listed below.
\begin{itemize}
\iteme[{\tt sep=}{\it strat}]
Apply vertex (node) separation strategy {\it strat\/} to each
bipartitioning job. A node separation strategy is made of one or
several node separation methods, which can be combined by means of
strategy operators. Node separation strategies are described in
Section~\ref{sec-lib-format-nsep}.
\end{itemize}
\end{itemize}

\subsubsection{Ordering strategy strings}
\label{sec-lib-format-ord}

Ordering strategies are available both for graphs and for meshes.
An ordering strategy is made of one or several ordering methods, which
can be combined by means of strategy operators. The strategy
operators that can be used in ordering strategies are listed below, by
increasing precedence.

\begin{itemize}
\iteme[{\tt (}{\it strat\/}{\tt )}]
Grouping operator.
The strategy enclosed within the parentheses is treated as a single
ordering method.
\iteme[{\tt /}{\it cond\/}{\tt ?}{\it strat1\/}{[{\tt :}{\it strat2}]{\tt ;}}]
Condition operator. According to the result of the evaluation of
condition {\it cond}, either {\it strat1\/} or {\it strat2\/} (if it
is present) is applied. The condition applies to the characteristics
of the current node of the separators tree, and can be built from
logical and relational operators. Conditional operators are listed
below, by increasing precedence.
\begin{itemize}
\iteme[{\it cond1\/}{\tt |}{\it cond2}]
Logical or operator. The result of the condition is true if {\it cond1\/}
or {\it cond2\/} are true, or both.
\iteme[{\it cond1\/}{\tt \&}{\it cond2}]
Logical and operator. The result of the condition is true only if both
{\it cond1\/} and {\it cond2\/} are true.
\iteme[{\tt !}{\it cond}]
Logical not operator. The result of the condition is true only if
{\it cond\/} is false.
\iteme[{\it var} {\it relop} {\it val}]
Relational operator, where {\it var\/} is a node variable, {\it val\/} is
either a node variable or a constant of the type of variable {\it var}, and
{\it relop\/} is one of '{\tt\verb+<+}', '{\tt\verb+=+}', and '{\tt\verb+>+}'.
The node variables are listed below, along with their types.
\begin{itemize}
\iteme[{\tt edge}]
The number of vertices of the current subgraph.
Integer.
\iteme[{\tt levl}]
The level of the subgraph in the separators tree, starting from zero
for the initial graph at the root of the tree.
Integer.
\iteme[{\tt load}]
The overall vertex load (weight) of the current subgraph.
Integer.
\iteme[{\tt mdeg}]
The maximum degree of the current subgraph.
Integer.
\iteme[{\tt vert}]
The number of vertices of the current subgraph.
Integer.
\end{itemize}
\end{itemize}
\iteme[{\it method\/}{[{\tt \{}{\it parameters\/}{\tt \}}]}]
Graph or mesh ordering method. Available ordering methods are listed
below.
\end{itemize}
The currently available ordering methods are the following.
\begin{itemize}
\iteme[{\tt b}]
Blocking method. This method does not perform ordering by itself, but
is used as post-processing to cut into blocks of smaller sizes the
separators or large blocks produced by other ordering methods. This is
not useful in the context of direct solving methods, because the
off-diagonal blocks created by the splitting of large diagonal blocks
are likely to be filled at factoring time. However, in the context of
incomplete solving methods such as ILU(k)~\cite{heperaro04a}, it can
lead to a significant reduction of the required memory space and time,
because it helps carving large triangular blocks.
The parameters of the blocking method are described below.
\begin{itemize}
\iteme[{\tt cmin=}{\it size}]
Set the minimum size of the resulting subblocks, in number of columns.
Blocks larger than twice this minimum size are cut into sub-blocks of
equal sizes (within one), having a number of columns comprised between
{\it size} and $2${\it size}.
\\
The definition of {\it size} depends on the size of the graph
to order. Large graphs cannot afford very small values, because the
number of blocks becomes much too large and limits the acceleration
of BLAS~3 routines, while large values do not help reducing enough
the complexity of ILU(k) solving.
\iteme[{\tt strat=}{\it strat}]
Ordering strategy to be performed. After the ordering strategy is
applied, the resulting separators tree is traversed and all of the
column blocks that are larger than $2${\it size} are split into
smaller column blocks, without changing the ordering that has been
computed.
\end{itemize}
\iteme[{\tt c}]
Compression method~\cite{ashc95}.
The parameters of the compression method are listed below.
\begin{itemize}
\iteme[{\tt rat=}{\it rat}]
Set the compression ratio over which graphs and meshes will not be
compressed. Useful values range between $0.7$ and $0.8$.
\iteme[{\tt cpr=}{\it strat}]
Ordering strategy to use on the compressed graph or mesh if its size
is below the compression ratio times the size of the original graph
or mesh.
\iteme[{\tt unc=}{\it strat}]
Ordering strategy to use on the original graph or mesh if the size of
the compressed graph or mesh were above the compression ratio times
the size of the original graph or mesh.
\end{itemize}
\iteme[{\tt d}]
Block Halo Approximate Minimum Degree method~\cite{peroam99}.
The parameters of the Halo Approximate Minimum Degree method are
listed below. The Block Halo Approximate Minimum Fill method,
described below, is more efficient and should be preferred.
\begin{itemize}
\iteme[{\tt cmin=}{\it size}]
Minimum number of columns per column block. All column blocks
of width smaller than {\it size\/} are amalgamated to their
parent column block in the elimination tree, provided that it
does not violate the {\tt cmax} constraint.
\iteme[{\tt cmax=}{\it size}]
Maximum number of column blocks over which some column block
will not amalgamate one of its descendents in the elimination
tree. This parameter is mainly designed to provide an upper
bound for block size in the context of BLAS3 computations~;
else, a huge value should be provided.
\iteme[{\tt frat=}{\it rat}]
Fill-in ratio over which some column block will not amalgamate
one of its descendents in the elimination tree. Typical values
range from $0.05$ to $0.10$.
\end{itemize}
\iteme[{\tt f}]
Block Halo Approximate Minimum Fill method.
The parameters of the Halo Approximate Minimum Fill method are
listed below.
\begin{itemize}
\iteme[{\tt cmin=}{\it size}]
Minimum number of columns per column block. All column blocks
of width smaller than {\it size\/} are amalgamated to their
parent column block in the elimination tree, provided that it
does not violate the {\tt cmax} constraint.
\iteme[{\tt cmax=}{\it size}]
Maximum number of column blocks over which some column block
will not amalgamate one of its descendents in the elimination
tree. This parameter is mainly designed to provide an upper
bound for block size in the context of BLAS3 computations~;
else, a huge value should be provided.
\iteme[{\tt frat=}{\it rat}]
Fill-in ratio over which some column block will not amalgamate
one of its descendents in the elimination tree. Typical values
range from $0.05$ to $0.10$.
\end{itemize}
\iteme[{\tt g}]
Gibbs-Poole-Stockmeyer method. This method is used on separators
to reduce the number and extent of extra-diagonal blocks. If
the number of extra-diagonal blocks is not relevant, the {\tt s}
method should be preferred. This method has only one parameter.
\begin{itemize}
\iteme[{\tt pass=}{\it nbr}]
Set the number of sweeps performed by the algorithm.
\end{itemize}
\iteme[{\tt n}]
Nested dissection method. The parameters of the nested dissection method are
given below.
\begin{itemize}
\iteme[{\tt ole=}{\it strat}]
Set the ordering strategy that is used on every leaf of the separators tree
if the node separation strategy {\tt sep} has failed to separate it further.
\iteme[{\tt ose=}{\it strat}]
Set the ordering strategy that is used on every separator of the separators
tree.
\iteme[{\tt sep=}{\it strat}]
Set the node separation strategy that is used on every leaf of the
separators tree to make it grow. Node separation strategies are
described below, in section~\ref{sec-lib-format-nsep}.
\end{itemize}
\iteme[{\tt o}]
Disconnected subgraph detection method. This method is used at the
global level to search for connected components, and run independently
the provided graph ordering strategy on each of them.
\begin{itemize}
\iteme[{\tt strat=}{\it strat}]
Ordering strategy to apply to each of the connected components.
\end{itemize}
\iteme[{\tt s}]
Simple method. Vertices are ordered in their natural order. This
method is fast, and should be used to order separators if the number
of extra-diagonal blocks is not relevant~; else, the {\tt g} method
should be preferred.
\iteme[{\tt v}]
Mesh-to-graph method. Available only for mesh ordering strategies.
From the mesh to which this method applies is derived a graph,
such that a graph vertex is associated with every node of the
mesh, and a clique is created between all vertices which represent
nodes that belong to the same element. A graph ordering strategy is
then applied to the derived graph, and this ordering is projected
back to the nodes of the mesh. This method is here for evaluation
purposes only, as mesh ordering methods are generally more
efficient than their graph ordering counterpart.
\begin{itemize}
\iteme[{\tt strat=}{\it strat}]
Graph ordering strategy to apply to the associated graph.
\end{itemize}
\end{itemize}

\subsubsection{Node separation strategy strings}
\label{sec-lib-format-nsep}

A node separation strategy is made of one or several node separation
methods, which can be combined by means of strategy
operators. Strategy operators are listed below, by increasing
precedence.

\begin{itemize}
\iteme[{\it strat1\/}{\tt |}{\it strat2}]
Selection operator. The result of the selection is the best vertex separator of
the two that are obtained by the distinct application of {\it strat1\/} and
{\it strat2\/} to the current separator.
\iteme[{\it strat1$\:$}{\it strat2}]
Combination operator. Strategy {\it strat2\/} is applied to the vertex
separator resulting from the application of strategy {\it strat1\/} to the
current separator. Typically, the first method used should compute an initial
separation from scratch, and every following method should use the
result of the previous one as a starting point.
\iteme[{\tt (}{\it strat\/}{\tt )}]
Grouping operator.
The strategy enclosed within the parentheses is treated as a single
separation method.
\iteme[{\tt /}{\it cond\/}{\tt ?}{\it strat1\/}{[{\tt :}{\it strat2}]{\tt ;}}]
Condition operator. According to the result of the evaluation of condition
{\it cond}, either {\it strat1\/} or {\it strat2\/} (if it is present) is
applied. The condition applies to the characteristics of the current subgraph,
and can be built from logical and relational operators. Conditional
operators are listed below, by increasing precedence.
\begin{itemize}
\iteme[{\it cond1\/}{\tt |}{\it cond2}]
Logical or operator. The result of the condition is true if {\it cond1\/}
or {\it cond2\/} are true, or both.
\iteme[{\it cond1\/}{\tt \&}{\it cond2}]
Logical and operator. The result of the condition is true only if both
{\it cond1\/} and {\it cond2\/} are true.
\iteme[{\tt !}{\it cond}]
Logical not operator. The result of the condition is true only if
{\it cond\/} is false.
\iteme[{\it var} {\it relop} {\it val}]
Relational operator, where {\it var\/} is a graph or node variable,
{\it val\/} is either a graph or node variable or a constant of the type of
variable {\it var\/}, and {\it relop\/} is one of
'{\tt\verb+<+}', '{\tt\verb+=+}', and '{\tt\verb+>+}'.
The graph and node variables are listed below, along with their types.
\begin{itemize}
\iteme[{\tt levl}]
The level of the subgraph in the separators tree, starting from zero at the root
of the tree.
Integer.
\iteme[{\tt proc}]
The number of processors on which the current subgraph is distributed
at this level of the separators tree. This variable is available only
when calling from routines of the \ptscotch\ parallel library.
Integer.
\iteme[{\tt rank}]
The rank of the current processor among the group of processors on
which the current subgraph is distributed at this level of the
separators tree. This variable is available only
when calling from routines of the \ptscotch\ parallel library, for
instance to decide which node separation strategy should be used on
which processor in a multi-sequential approach.
Integer.
\iteme[{\tt vert}]
The number of vertices of the current subgraph.
Integer.
\end{itemize}
\end{itemize}
\end{itemize}
The currently available vertex separation methods are the following.
\begin{itemize}
\iteme[{\tt b}]
Band method. Available only for graph separation strategies. This
method builds a band graph of given width around the current separator
of the graph to which it is applied, and calls a graph
separation strategy to refine the equivalent separator of the band
graph. Then, the refined separator of the band graph is projected back
to the current graph. This method, presented in~\cite{chpe06a}, was
created to reduce the cost of separator refinement algorithms in a
multilevel context, but it improves partition quality too.
The parameters of the band separation method are listed below.
\begin{itemize}
\iteme[{\tt bnd=}{\it strat}]
Set the vertex separation strategy to be used on the band graph.
\iteme[{\tt org=}{\it strat}]
Set the fallback vertex separation strategy to be used on the
original graph if the band graph strategy could not be used. The three
cases which require the use of this fallback strategy are the
following. First, if the separator of the original graph is empty,
which makes it impossible to compute a band graph. Second, if any part
of the band graph to be built is of the same size as the one of the
original graph. Third, if the application of the {\tt bnd}
vertex separation method to the band graph leads to a situation where both
anchor vertices are placed in the same part.
\iteme[{\tt width=}{\it val}]
Set the width of the band graph. All graph vertices that are at a
distance less than or equal to {\it val} from any separator vertex
are kept in the band graph.
\end{itemize}
\iteme[{\tt e}]
Edge-separation method. Available only for graph separation strategies.
This method builds vertex separators from edge separators, by the
method proposed by Pothen and Fang~\cite{pofa90}, which uses a variant
of the Hopcroft and Karp algorithm due to Duff~\cite{duff81}. This
method is expensive and most often yields poorer results than direct
vertex-oriented methods such as the vertex vertex Greedy-graph-growing
and the vertex Fiduccia-Mattheyses algorithms. The parameters of the
edge-separation method are listed below.
\begin{itemize}
\iteme[{\tt bal=}{\it val}]
Set the load imbalance tolerance to {\it val}, which is a floating-point
ratio expressed with respect to the ideal load of the partitions.
\iteme[{\tt strat=}{\it strat}]
Set the graph bipartitioning strategy that is used to compute the edge
bipartition. The syntax of bipartitioning
strategy strings is defined within section~\ref{sec-lib-format-bipart},
at page~\pageref{sec-lib-format-bipart}.
\iteme[{\tt width=}{\it type}]
Select the width of the vertex separators built from edge separators.
When {\it type\/} is set to {\tt f}, fat vertex separators are built,
that hold all of the ends of the edges of the edge cut. When it is set
to {\tt t}, a thin vertex separator is built by removing as many
vertices as possible from the fat separator.
\end{itemize}
\iteme[{\tt f}]
Vertex Fiduccia-Mattheyses method. The parameters of the vertex
Fiduccia-Mattheyses method are listed below.
\begin{itemize}
\iteme[{\tt bal=}{\it rat}]
Set the maximum weight imbalance ratio to the given fraction of
the weight of all node vertices. Common values are around $0.01$,
that is, one percent.
\iteme[{\tt move=}{\it nbr}]
Maximum number of hill-climbing moves that can be performed before a
pass ends. During each of its passes, the vertex Fiduccia-Mattheyses
algorithm repeatedly moves vertices from the separator to any of the
two parts, so as to minimize the size of the separator. A pass
completes either when all of the vertices have been moved once, or if
too many swaps that do not decrease the size of the separator have
been performed.
\iteme[{\tt pass=}{\it nbr}]
Set the maximum number of optimization passes performed by the
algorithm. The vertex Fiduccia-Mattheyses algorithm stops as soon as a
pass has not yielded any reduction of the size of the separator, or
when the maximum number of passes has been reached. Value -1 stands
for an infinite number of passes, that is, as many as needed by the
algorithm to converge.
\end{itemize}
\iteme[{\tt g}]
Gibbs-Poole-Stockmeyer method. Available only for graph separation strategies.
This method has only one parameter.
\begin{itemize}
\iteme[{\tt pass=}{\it nbr}]
Set the number of sweeps performed by the algorithm.
\end{itemize}
\iteme[{\tt h}]
Vertex greedy-graph-growing method. This method has only one parameter.
\begin{itemize}
\iteme[{\tt pass=}{\it nbr}]
Set the number of runs performed by the algorithm.
\end{itemize}
\iteme[{\tt m}]
Vertex multilevel method. The parameters of the vertex
multilevel method are listed below.
\begin{itemize}
\iteme[{\tt asc=}{\it strat}]
Set the strategy that is used to refine the vertex separators obtained
at ascending levels of the uncoarsening phase by projection of the
separators computed for coarser graphs or meshes.  This strategy is
not applied to the coarsest graph or mesh, for which only the
{\tt low} strategy is used.
\iteme[{\tt low=}{\it strat}]
Set the strategy that is used to compute the vertex separator of the
coarsest graph or mesh, at the lowest level of the coarsening process.
\iteme[{\tt rat=}{\it rat}]
Set the threshold maximum coarsening ratio over which graphs or meshes
are no longer coarsened. The ratio of any given coarsening cannot be
less that $0.5$ (case of a perfect matching), and cannot be greater
than $1.0$. Coarsening stops when either the coarsening ratio is
above the maximum coarsening ratio, or the graph or mesh has fewer
node vertices than the minimum number of vertices allowed.
\iteme[{\tt vert=}{\it nbr}]
Set the threshold minimum size under which graphs or meshes are no longer
coarsened. Coarsening stops when either the coarsening ratio is above the
maximum coarsening ratio, or the graph or mesh has fewer node vertices
than the minimum number of vertices allowed.
\end{itemize}
\iteme[{\tt t}]
Thinner method. Available only for graph separation strategies.
This method quickly eliminates all useless vertices of
the current separator. It searches the separator for vertices that
have no neighbors in one of the two parts, and moves these vertices to
the part they are connected to. This method may be used to refine
separators during the uncoarsening phase of the multilevel method,
and is faster than a vertex Fiduccia-Mattheyses algorithm with
{\tt \{move\lbt =\lbt 0\}}.
\iteme[{\tt v}]
Mesh-to-graph method. Available only for mesh separation strategies.
From the mesh to which this method applies is derived a graph,
such that a graph vertex is associated with every node of the
mesh, and a clique is created between all vertices which represent
nodes that belong to the same element. A graph separation strategy is
then applied to the derived graph, and the separator is projected
back to the nodes of the mesh. This method is here for evaluation
purposes only, as mesh separation methods are generally more
efficient than their graph separation counterpart.
\begin{itemize}
\iteme[{\tt strat=}{\it strat}]
Graph separation strategy to apply to the associated graph.
\end{itemize}
\iteme[{\tt w}]
Graph separator viewer. Available only for graph separation strategies.
Every call to this method results in the creation, in the current
subdirectory, of partial mapping files called ``{\tt vgraph\lbt
separate\lbt vw\_\lbt output\_\lbt {\it nnnnnnnn}.map}'', where
``{\it nnnnnnnn}'' are increasing decimal numbers, which contain the
current state of the two parts and the separator. These mapping files
can be used as input by the {\tt gout} program to produce displays of
the evolving shape of the current separator and parts. This is mostly
a debugging feature, but it can also have an illustrative interest.
While it is only available for graph separation strategies, mesh
separation strategies can indirectly use it through the mesh-to-graph
separation method.
\iteme[{\tt z}]
Zero method. This method moves all of the node vertices to the first
part, resulting in an empty separator. Its main use is to stop the
separation process whenever some condition is true.
\end{itemize}

\subsection{Target architecture handling routines}
\label{sec-lib-arch-handling}

\subsubsection{{\tt SCOTCH\_archExit}}

\begin{itemize}
\progsyn

{\tt\begin{tabular}{l@{}ll}
void SCOTCH\_archExit ( & SCOTCH\_Arch * & archptr)
\end{tabular}}

{\tt\begin{tabular}{l@{}ll}
scotchfarchexit ( & doubleprecision (*) & archdat)
\end{tabular}}

\progdes

The {\tt SCOTCH\_archExit} function frees the contents of a
{\tt SCOTCH\_\lbt Arch} structure previously initialized by
{\tt SCOTCH\_\lbt archInit}. All subsequent calls to
{\tt SCOTCH\_\lbt arch} routines other than {\tt SCOTCH\_\lbt
archInit}, using this structure as parameter, may yield
unpredictable results.
\end{itemize}

\subsubsection{{\tt SCOTCH\_archInit}}

\begin{itemize}
\progsyn

{\tt\begin{tabular}{l@{}ll}
int SCOTCH\_archInit ( & SCOTCH\_Arch * & archptr)
\end{tabular}}

{\tt\begin{tabular}{l@{}ll}
scotchfarchinit ( & doubleprecision (*) & archdat, \\
                  & integer             & ierr)
\end{tabular}}

\progdes

The {\tt SCOTCH\_archInit} function initializes a {\tt SCOTCH\_\lbt
Arch} structure so as to make it suitable for future operations. It
should be the first function to be called upon a {\tt SCOTCH\_\lbt
Arch} structure.  When the target architecture data is no longer of
use, call function {\tt SCOTCH\_\lbt archExit} to free its internal
structures.

\progret

{\tt SCOTCH\_archInit} returns $0$ if the graph structure has been
successfully initialized, and $1$ else.
\end{itemize}

\subsubsection{{\tt SCOTCH\_archLoad}}

\begin{itemize}
\progsyn

{\tt\begin{tabular}{l@{}ll}
int SCOTCH\_archLoad ( & SCOTCH\_Arch * & archptr, \\
                       & FILE *         & stream)
\end{tabular}}

{\tt\begin{tabular}{l@{}ll}
scotchfarchload ( & doubleprecision (*) & archdat, \\
                  & integer             & fildes, \\
                  & integer             & ierr)
\end{tabular}}

\progdes

The {\tt SCOTCH\_archLoad} routine fills the {\tt
SCOTCH\_\lbt Arch} structure pointed to by {\tt archptr} with the
source graph description available from stream {\tt stream} in the
\scotch\ target architecture format (see
Section~\ref{sec-file-target}).

Fortran users must use the {\tt PXFFILENO} or {\tt FNUM} functions to
obtain the number of the Unix file descriptor {\tt fildes} associated
with the logical unit of the architecture file.

\progret

{\tt SCOTCH\_archLoad} returns $0$ if the target architecture structure
has been successfully allocated and filled with the data read, and $1$
else.
\end{itemize}

\subsubsection{{\tt SCOTCH\_archName}}

\begin{itemize}
\progsyn

{\tt\begin{tabular}{l@{}ll}
const char * SCOTCH\_archName ( & const SCOTCH\_Arch * & archptr)
\end{tabular}}

{\tt\begin{tabular}{l@{}ll}
scotchfarchname ( & doubleprecision (*) & archdat, \\
                  & character (*)       & chartab, \\
                  & integer             & charnbr)
\end{tabular}}

\progdes

The {\tt SCOTCH\_archName} function returns a string containing the
name of the architecture pointed to by {\tt archptr}. Since Fortran
routines cannot return string pointers, the {\tt scotchf\lbt arch\lbt
name} routine takes as second and third parameters a {\tt character()}
array to be filled with the name of the architecture, and the {\tt
integer} size of the array, respectively. If the array is of
sufficient size, a trailing nul character is appended to the string to
materialize the end of the string (this is the C style of handling
character strings).

\progret

{\tt SCOTCH\_archName} returns a non-null character pointer that points
to a null-terminated string describing the type of the architecture.
\end{itemize}

\subsubsection{{\tt SCOTCH\_archSave}}

\begin{itemize}
\progsyn

{\tt\begin{tabular}{l@{}ll}
int SCOTCH\_archSave ( & const SCOTCH\_Arch * & archptr, \\
                       & FILE *               & stream)
\end{tabular}}

{\tt\begin{tabular}{l@{}ll}
scotchfarchsave ( & doubleprecision (*) & archdat, \\
                  & integer             & fildes, \\
                  & integer             & ierr)
\end{tabular}}

\progdes

The {\tt SCOTCH\_archSave} routine saves the contents of the {\tt
SCOTCH\_\lbt Arch} structure pointed to by {\tt archptr} to stream
{\tt stream}, in the \scotch\ target architecture format (see
section~\ref{sec-file-target}).

Fortran users must use the {\tt PXFFILENO} or {\tt FNUM} functions to
obtain the number of the Unix file descriptor {\tt fildes} associated
with the logical unit of the architecture file.

\progret

{\tt SCOTCH\_archSave} returns $0$ if the graph structure has been
successfully written to {\tt stream}, and $1$ else.
\end{itemize}

\subsubsection{{\tt SCOTCH\_archSize}}

\begin{itemize}
\progsyn

{\tt\begin{tabular}{l@{}ll}
SCOTCH\_Num SCOTCH\_archSize ( & const SCOTCH\_Arch * & archptr)
\end{tabular}}

{\tt\begin{tabular}{l@{}ll}
scotchfarchsize ( & doubleprecision (*) & archdat, \\
                  & integer*{\it num}   & archnbr)
\end{tabular}}

\progdes

The {\tt SCOTCH\_archSize} function returns the number of nodes of
the given target architecture. The Fortran routine has a second
parameter, of integer type, which is set on return with the number
of nodes of the target architecture.

\progret

{\tt SCOTCH\_archSize} returns the number of nodes of the target
architecture.
\end{itemize}

\subsection{Target architecture creation routines}
\label{sec-lib-arch-create}

\subsubsection{{\tt SCOTCH\_archBuild0} / {\tt SCOTCH\_archBuild}}
\label{sec-lib-arch-build}

\begin{itemize}
\progsyn

{\tt\begin{tabular}{l@{}ll}
int SCOTCH\_archBuild0 ( & SCOTCH\_Arch *        & archptr, \\
                         & const SCOTCH\_Graph * & grafptr, \\
                         & const SCOTCH\_Num     & listnbr, \\
                         & const SCOTCH\_Num *   & listtab, \\
                         & const SCOTCH\_Strat * & straptr) \\
~\\
int SCOTCH\_archBuild ( & SCOTCH\_Arch *        & archptr, \\
                        & const SCOTCH\_Graph * & grafptr, \\
                        & const SCOTCH\_Num     & listnbr, \\
                        & const SCOTCH\_Num *   & listtab, \\
                        & const SCOTCH\_Strat * & straptr)
\end{tabular}}

{\tt\begin{tabular}{l@{}ll}
scotchfarchbuild0 ( & doubleprecision (*)   & archdat, \\
                    & doubleprecision (*)   & grafdat, \\
                    & integer*{\it num}     & listnbr, \\
                    & integer*{\it num} (*) & listtab, \\
                    & doubleprecision (*)   & stradat, \\
                    & integer               & ierr) \\
~\\
scotchfarchbuild ( & doubleprecision (*)   & archdat, \\
                   & doubleprecision (*)   & grafdat, \\
                   & integer*{\it num}     & listnbr, \\
                   & integer*{\it num} (*) & listtab, \\
                   & doubleprecision (*)   & stradat, \\
                   & integer               & ierr)
\end{tabular}}

\progdes

The {\tt SCOTCH\_archBuild0} routine fills the architecture structure
pointed to by {\tt archptr} with the ``\texttt{deco 1}'' (that is, a
compiled form of a ``\texttt{deco 0}'') decomposition-defined target
architecture computed by applying the graph bipartitioning strategy
pointed to by {\tt straptr} to the architecture graph pointed to by
{\tt grafptr}.

When {\tt listptr} is not {\tt NULL} and {\tt listnbr} is greater than
zero, the decomposition-defined architecture is restricted to the
{\tt listnbr} vertices whose indices are given in the array pointed to
by {\tt listtab}, from {\tt listtab\lbt [0]} to {\tt listtab\lbt
[listnbr - 1]}. These indices should have the same base value as the
one of the graph pointed to by {\tt grafptr}, that is, be in the
range from $0$ to $\mathtt{vertnbr} - 1$ if the graph base is
$0$, and from $1$ to $\mathtt{vertnbr}$ if the graph base is $1$.

Graph bipartitioning strategies are declared by means of the
{\tt SCOTCH\_\lbt strat\lbt Graph\lbt Bipart} function, described in
page~\pageref{sec-lib-strat-graph-bipart}. The syntax of bipartitioning
strategy strings is defined in section~\ref{sec-lib-format-map},
page~\pageref{sec-lib-format-bipart}.
Additional information may be obtained from the manual page of
{\tt amk\_\lbt grf}, the stand-alone executable that builds
decomposition-defined target architecture files from source graph
files, available at page~\pageref{sec-prog-amkgrf}.

At the time being, {\tt SCOTCH\_arch\lbt Build} is equivalent to {\tt
SCOTCH\_\lbt arch\lbt Build0}. In future releases, it is planned that
{\tt SCOTCH\_\lbt arch\lbt Build} will either behave as
{\tt SCOTCH\_\lbt arch\lbt Build0} or {\tt SCOTCH\_\lbt arch\lbt
Build2}, depending on target graph size. For target graphs of small
sizes, users are invited to use explicitly the {\tt SCOTCH\_\lbt
arch\lbt Build0} routine.

\progret

{\tt SCOTCH\_archBuild0} returns $0$ if the decomposition-defined
architecture has been successfully computed, and $1$ else.
\end{itemize}

\subsubsection{{\tt SCOTCH\_archBuild2}}
\label{sec-lib-arch-build-two}

\begin{itemize}
\progsyn

{\tt\begin{tabular}{l@{}ll}
int SCOTCH\_archBuild2 ( & SCOTCH\_Arch *        & archptr, \\
                         & const SCOTCH\_Graph * & grafptr, \\
                         & const SCOTCH\_Num     & listnbr, \\
                         & const SCOTCH\_Num *   & listtab)
\end{tabular}}

{\tt\begin{tabular}{l@{}ll}
scotchfarchbuild2 ( & doubleprecision (*)   & archdat, \\
                    & doubleprecision (*)   & grafdat, \\
                    & integer*{\it num}     & listnbr, \\
                    & integer*{\it num} (*) & listtab, \\
                    & integer               & ierr)
\end{tabular}}

\progdes

The {\tt SCOTCH\_archBuild2} routine fills the architecture structure
pointed to by {\tt archptr} with the ``\texttt{deco 2}''
decomposition-defined target architecture corresponding to the graph
pointed to by {\tt grafptr}. Since the computation of the
decomposition is performed by means of graph coarsening,
unlike {\tt SCOTCH\_\lbt arch\lbt Build}, no bipartitioning strategy
has to be provided.

When {\tt listptr} is not {\tt NULL} and {\tt listnbr} is greater than
zero, the decomposition-defined architecture is restricted to the
{\tt listnbr} vertices whose indices are given in the array pointed to
by {\tt listtab}, from {\tt listtab\lbt [0]} to {\tt listtab\lbt
[listnbr - 1]}. These indices should have the same base value as that
of the graph pointed to by {\tt grafptr}, that is, be in the
range from $0$ to $\mathtt{vertnbr} - 1$ if the graph base is
$0$, and from $1$ to $\mathtt{vertnbr}$ if the graph base is $1$.

Additional information may be obtained from the manual page of
{\tt amk\_\lbt grf}, the stand-alone executable that builds
decomposition-defined target architecture files from source graph
files, available at page~\pageref{sec-prog-amkgrf}.

\progret

{\tt SCOTCH\_archBuild} returns $0$ if the decomposition-defined
architecture has been successfully computed, and $1$ else.
\end{itemize}

\subsubsection{{\tt SCOTCH\_archCmplt}}

\begin{itemize}
\progsyn

{\tt\begin{tabular}{l@{}ll}
int SCOTCH\_archCmplt ( & SCOTCH\_Arch *    & archptr, \\
                        & const SCOTCH\_Num & vertnbr)
\end{tabular}}

{\tt\begin{tabular}{l@{}ll}
scotchfarchcmplt ( & doubleprecision (*) & archdat, \\
                   & integer*{\it num}   & vertnbr, \\
                   & integer             & ierr)
\end{tabular}}

\progdes

The {\tt SCOTCH\_archCmplt} routine fills the {\tt SCOTCH\_\lbt Arch}
structure pointed to by {\tt archptr} with the description of a complete
graph architecture with {\tt vertnbr} processors, which can be used as
input to {\tt SCOTCH\_\lbt graph\lbt Map} to perform graph partitioning.
A shortcut to this is to use the {\tt SCOTCH\_\lbt graph\lbt Part}
routine.

\progret

{\tt SCOTCH\_archCmplt} returns $0$ if the complete graph target
architecture has been successfully built, and $1$ else.
\end{itemize}

\subsubsection{{\tt SCOTCH\_archCmpltw}}
\label{sec-lib-arch-cmpltw}

\begin{itemize}
\progsyn

{\tt\begin{tabular}{l@{}ll}
int SCOTCH\_archCmpltw ( & SCOTCH\_Arch *            & archptr, \\
                         & const SCOTCH\_Num         & vertnbr, \\
                         & const SCOTCH\_Num * const & velotab)
\end{tabular}}

{\tt\begin{tabular}{l@{}ll}
scotchfarchcmplt ( & doubleprecision (*)   & archdat, \\
                   & integer*{\it num}     & vertnbr, \\
                   & integer*{\it num} (*) & velotab, \\
                   & integer               & ierr)
\end{tabular}}

\progdes

The {\tt SCOTCH\_archCmpltw} routine fills the {\tt SCOTCH\_\lbt Arch}
structure pointed to by {\tt archptr} with the description of a
weighted complete graph architecture with {\tt vertnbr} processors.
The relative weights of the processors are given in the {\tt velotab}
array. Once the target architecture has been created, it can be
used as input to {\tt SCOTCH\_\lbt graph\lbt Map} to perform weighted
graph partitioning.

\progret

{\tt SCOTCH\_archCmpltw} returns $0$ if the weighted complete graph target
architecture has been successfully built, and $1$ else.
\end{itemize}

\subsubsection{{\tt SCOTCH\_archHcub}}

\begin{itemize}
\progsyn

{\tt\begin{tabular}{l@{}ll}
int SCOTCH\_archHcub ( & SCOTCH\_Arch *    & archptr, \\
                       & const SCOTCH\_Num & hdimval)
\end{tabular}}

{\tt\begin{tabular}{l@{}ll}
scotchfarchhcub ( & doubleprecision (*) & archdat, \\
                  & integer*{\it num}   & hdimval, \\
                  & integer             & ierr)
\end{tabular}}

\progdes

The {\tt SCOTCH\_archHcub} routine fills the {\tt SCOTCH\_\lbt Arch}
structure pointed to by {\tt archptr} with the description of a
hypercube graph of dimension {\tt hdimval}.

\progret

{\tt SCOTCH\_archHcub} returns $0$ if the hypercube target
architecture has been successfully built, and $1$ else.
\end{itemize}

\subsubsection{{\tt SCOTCH\_archLtleaf}}

\begin{itemize}
\progsyn

{\tt\begin{tabular}{l@{}ll}
int SCOTCH\_archLtleaf ( & SCOTCH\_Arch *      & archptr, \\
                         & const SCOTCH\_Num   & levlnbr, \\
                         & const SCOTCH\_Num * & sizetab, \\
                         & const SCOTCH\_Num * & linktab, \\
                         & const SCOTCH\_Num   & permnbr, \\
                         & const SCOTCH\_Num * & permtab)
\end{tabular}}

{\tt\begin{tabular}{l@{}ll}
scotchfarchltleaf ( & doubleprecision (*)    & archdat, \\
                    & integer*{\it num}      & levlnbr, \\
                    & integer*{\it num} (*)  & sizetab, \\
                    & integer*{\it num} (*)  & linktab, \\
                    & integer*{\it num}      & permnbr, \\
                    & integer*{\it num} (*)  & permtab, \\
                    & integer                & ierr)
\end{tabular}}

\progdes

The {\tt SCOTCH\_archLtleaf} routine fills the {\tt SCOTCH\_\lbt Arch}
structure pointed to by {\tt archptr} with the description of a
labeled, tree-shaped, hierarchical graph architecture with
$\sum_{i=0}^{\mathtt{levlnbr}-1}\mathtt{sizetab}\mbox{\tt [}i\mbox{\tt ]}$
processors. Level $0$ is the root of the tree. For each level $i$,
with $0 \leq i < \mathtt{levlnbr}$, $\mathtt{sizetab}\mbox{\tt [}i\mbox{\tt
]}$ is the number of childs at level $(i+1)$ of each node at level $i$,
and {\tt linktab[}$i${\tt ]} is the cost of communication between
processors the first common ancestor of which belongs to this
level. See Section~\ref{sec-file-target-algo},
page~\pageref{sec-file-target-ltleaf}, for an example of this
architecture.

\progret

{\tt SCOTCH\_archLtleaf} returns $0$ if the labeled tree-leaf target
architecture has been successfully built, and $1$ else.
\end{itemize}

\subsubsection{{\tt SCOTCH\_archMesh2}}

\begin{itemize}
\progsyn

{\tt\begin{tabular}{l@{}ll}
int SCOTCH\_archMesh2 ( & SCOTCH\_Arch *    & archptr, \\
                        & const SCOTCH\_Num & xdimval, \\
                        & const SCOTCH\_Num & ydimval) \\
\end{tabular}}

{\tt\begin{tabular}{l@{}ll}
scotchfarchmesh2 ( & doubleprecision (*) & archdat, \\
                   & integer*{\it num}   & xdimval, \\
                   & integer*{\it num}   & ydimval, \\
                   & integer             & ierr)
\end{tabular}}

\progdes

The {\tt SCOTCH\_archMesh2} routine fills the {\tt SCOTCH\_\lbt Arch}
structure pointed to by {\tt archptr} with the description of a
2D mesh architecture with $\mathtt{xdimval} \times \mathtt{ydimval}$
processors.

\progret

{\tt SCOTCH\_archMesh2} returns $0$ if the 2D mesh target
architecture has been successfully built, and $1$ else.
\end{itemize}

\subsubsection{{\tt SCOTCH\_archMesh3}}

\begin{itemize}
\progsyn

{\tt\begin{tabular}{l@{}ll}
int SCOTCH\_archMesh3 ( & SCOTCH\_Arch *    & archptr, \\
                        & const SCOTCH\_Num & xdimval, \\
                        & const SCOTCH\_Num & ydimval, \\
                        & const SCOTCH\_Num & zdimval) \\
\end{tabular}}

{\tt\begin{tabular}{l@{}ll}
scotchfarchmesh3 ( & doubleprecision (*) & archdat, \\
                   & integer*{\it num}   & xdimval, \\
                   & integer*{\it num}   & ydimval, \\
                   & integer*{\it num}   & zdimval, \\
                   & integer             & ierr)
\end{tabular}}

\progdes

The {\tt SCOTCH\_archMesh3} routine fills the {\tt SCOTCH\_\lbt Arch}
structure pointed to by {\tt archptr} with the description of a
3D mesh architecture with $\mathtt{xdimval} \times \mathtt{ydimval}
\times \mathtt{zdimval}$ processors.

\progret

{\tt SCOTCH\_archMesh3} returns $0$ if the 3D mesh target
architecture has been successfully built, and $1$ else.
\end{itemize}

\subsubsection{{\tt SCOTCH\_archMeshX}}

\begin{itemize}
\progsyn

\texttt{\begin{tabular}{l@{}ll}
int SCOTCH\_archMeshX ( & SCOTCH\_Arch *      & archptr, \\
                        & const SCOTCH\_Num   & dimnnbr, \\
                        & const SCOTCH\_Num * & dimntab) \\
\end{tabular}}

\texttt{\begin{tabular}{l@{}ll}
scotchfarchmeshx ( & doubleprecision (*) & archdat, \\
                   & integer*{\it num}   & dimnnbr, \\
                   & integer*{\it num}   & dimntab, \\
                   & integer             & ierr)
\end{tabular}}

\progdes

The \texttt{SCOTCH\_archMeshX} routine fills the
\texttt{SCOTCH\_\lbt Arch} structure pointed to by \texttt{archptr}
with the description of a \texttt{dimnnbr}-dimension mesh architecture
with $\prod_d\mathtt{dimntab[}d\mathtt{]}$ processors. The maximum
number of dimensions is defined at compile-time.

\progret

\texttt{SCOTCH\_archMeshX} returns $0$ if the
\texttt{dimnnbr}-dimension mesh target architecture has been
successfully built, and $1$ else.
\end{itemize}

\subsubsection{{\tt SCOTCH\_archSub}}
\label{sec-lib-arch-sub}

\begin{itemize}
\progsyn

{\tt\begin{tabular}{l@{}ll}
int SCOTCH\_archSub ( & SCOTCH\_Arch *      & subarchptr, \\
                      & SCOTCH\_Arch *      & orgarchptr, \\
                      & const SCOTCH\_Num   & vnumnbr,    \\
                      & const SCOTCH\_Num * & vnumtab)    \\
\end{tabular}}

{\tt\begin{tabular}{l@{}ll}
scotchfarchsub ( & doubleprecision (*) & subarchdat, \\
                 & doubleprecision (*) & orgarchdat, \\
                 & integer*{\it num}   & vnumnbr,    \\
                 & integer*{\it num}   & vnumtab,    \\
                 & integer             & ierr)
\end{tabular}}

\progdes

The \texttt{SCOTCH\_archSub} routine fills the
\texttt{SCOTCH\_\lbt Arch} structure pointed to by \texttt{subarchptr}
with the description of a subset of the \texttt{orgarchptr}
architecture, restricted to \texttt{vertnbr} processors which are
listed in the \texttt{vnumtab} array. The order in which these
processor indices in the original architecture are stored in the
\texttt{vnumtab} array defines the rank of these processors in the
sub-architecture.

\progret

{\tt SCOTCH\_archSub} returns $0$ if the target sub-architecture
has been successfully built, and $1$ else.
\end{itemize}

\subsubsection{{\tt SCOTCH\_archTleaf}}

\begin{itemize}
\progsyn

{\tt\begin{tabular}{l@{}ll}
int SCOTCH\_archTleaf ( & SCOTCH\_Arch *      & archptr, \\
                        & const SCOTCH\_Num   & levlnbr, \\
                        & const SCOTCH\_Num * & sizetab, \\
                        & const SCOTCH\_Num * & linktab)
\end{tabular}}

{\tt\begin{tabular}{l@{}ll}
scotchfarchtleaf ( & doubleprecision (*)   & archdat, \\
                   & integer*{\it num}     & levlnbr, \\
                   & integer*{\it num} (*) & sizetab, \\
                   & integer*{\it num} (*) & linktab, \\
                   & integer               & ierr)
\end{tabular}}

\progdes

The {\tt SCOTCH\_archTleaf} routine fills the {\tt SCOTCH\_\lbt Arch}
structure pointed to by {\tt archptr} with the description of a
tree-shaped, hierarchical graph architecture with
$\sum_{i=0}^{\mathtt{levlnbr}-1}\mathtt{sizetab}{\tt [}i{\tt ]}$
processors. Level $0$ is the root of the tree. For each level $i$,
with $0 \leq i < \mathtt{levlnbr}$, {\tt sizetab[}$i${\tt
]} is the number of childs at level $(i+1)$ of each node at level $i$,
and {\tt linktab[}$i${\tt ]} is the cost of communication between
processors the first common ancestor of which belongs to this
level. See Section~\ref{sec-file-target-algo},
page~\pageref{sec-file-target-algo}, for an example of this
architecture.

\progret

{\tt SCOTCH\_archTleaf} returns $0$ if the tree-leaf target
architecture has been successfully built, and $1$ else.
\end{itemize}

\subsubsection{{\tt SCOTCH\_archTorus2}}

\begin{itemize}
\progsyn

{\tt\begin{tabular}{l@{}ll}
int SCOTCH\_archTorus2 ( & SCOTCH\_Arch *    & archptr, \\
                         & const SCOTCH\_Num & xdimval, \\
                         & const SCOTCH\_Num & ydimval) \\
\end{tabular}}

{\tt\begin{tabular}{l@{}ll}
scotchfarchtorus2 ( & doubleprecision (*) & archdat, \\
                    & integer*{\it num}   & xdimval, \\
                    & integer*{\it num}   & ydimval, \\
                    & integer             & ierr)
\end{tabular}}

\progdes

The {\tt SCOTCH\_archTorus2} routine fills the {\tt SCOTCH\_\lbt Arch}
structure pointed to by {\tt archptr} with the description of a
2D torus architecture with $\mathtt{xdimval} \times \mathtt{ydimval}$
processors.

\progret

{\tt SCOTCH\_archTorus2} returns $0$ if the 2D torus target
architecture has been successfully built, and $1$ else.
\end{itemize}

\subsubsection{{\tt SCOTCH\_archTorus3}}

\begin{itemize}
\progsyn

{\tt\begin{tabular}{l@{}ll}
int SCOTCH\_archTorus3 ( & SCOTCH\_Arch *    & archptr, \\
                         & const SCOTCH\_Num & xdimval, \\
                         & const SCOTCH\_Num & ydimval, \\
                         & const SCOTCH\_Num & zdimval) \\
\end{tabular}}

{\tt\begin{tabular}{l@{}ll}
scotchfarchtorus3 ( & doubleprecision (*) & archdat, \\
                    & integer*{\it num}   & xdimval, \\
                    & integer*{\it num}   & ydimval, \\
                    & integer*{\it num}   & zdimval, \\
                    & integer             & ierr)
\end{tabular}}

\progdes

The {\tt SCOTCH\_archTorus3} routine fills the {\tt SCOTCH\_\lbt Arch}
structure pointed to by {\tt archptr} with the description of a
3D torus architecture with $\mathtt{xdimval} \times \mbox{\tt
ydimval} \times \mathtt{zdimval}$ processors.

\progret

{\tt SCOTCH\_archTorus3} returns $0$ if the 3D torus target
architecture has been successfully built, and $1$ else.
\end{itemize}

\subsubsection{{\tt SCOTCH\_archTorusX}}

\begin{itemize}
\progsyn

\texttt{\begin{tabular}{l@{}ll}
int SCOTCH\_archTorusX ( & SCOTCH\_Arch *      & archptr, \\
                         & const SCOTCH\_Num   & dimnnbr, \\
                         & const SCOTCH\_Num * & dimntab) \\
\end{tabular}}

\texttt{\begin{tabular}{l@{}ll}
scotchfarchtorusx ( & doubleprecision (*) & archdat, \\
                    & integer*{\it num}   & dimnnbr, \\
                    & integer*{\it num}   & dimntab, \\
                    & integer             & ierr)
\end{tabular}}

\progdes

The \texttt{SCOTCH\_archTorusX} routine fills the
\texttt{SCOTCH\_\lbt Arch}
structure pointed to by \texttt{archptr} with the description of a
\texttt{dimnnbr}-dimension torus architecture with
$\prod_d\mathtt{dimntab[}d\mathtt{]}$ processors. The maximum number
of dimensions is defined at compile-time.

\progret

\texttt{SCOTCH\_archTorusX} returns $0$ if the
\texttt{dimnnbr}-dimension mesh target architecture has been
successfully built, and $1$ else.
\end{itemize}

\subsection{Graph handling routines}
\label{sec-lib-graph}

\subsubsection{{\tt SCOTCH\_graphAlloc}}

\begin{itemize}
\progsyn

{\tt\begin{tabular}{l@{}l}
SCOTCH\_Graph * SCOTCH\_graphAlloc ( & void)
\end{tabular}}

\progdes

The {\tt SCOTCH\_graphAlloc} function allocates a memory area of a
size sufficient to store a {\tt SCOTCH\_\lbt Graph} structure. It is
the user's responsibility to free this memory when it is no longer
needed, using the {\tt SCOTCH\_\lbt mem\lbt Free} routine. The
allocated space must be initialized before use, by means of the
{\tt SCOTCH\_\lbt graph\lbt Init} routine.

\progret

{\tt SCOTCH\_graphAlloc} returns the pointer to the memory area if it
has been successfully allocated, and {\tt NULL} else.
\end{itemize}

\subsubsection{{\tt SCOTCH\_graphBase}}

\begin{itemize}
\progsyn

{\tt\begin{tabular}{l@{}ll}
int SCOTCH\_graphBase ( & SCOTCH\_Graph * & grafptr, \\
                        & SCOTCH\_Num     & baseval)
\end{tabular}}

{\tt\begin{tabular}{l@{}ll}
scotchfgraphbase ( & doubleprecision (*) & grafdat, \\
                   & integer*{\it num}   & baseval, \\
                   & integer*{\it num}   & oldbaseval)
\end{tabular}}

\progdes

The {\tt SCOTCH\_graphBase} routine sets the base of all graph indices
according to the given base value, and returns the old base value.
This routine is a helper for applications that do not handle base
values properly.

In Fortan, the old base value is returned in the third parameter of
the function call.

\progret

{\tt SCOTCH\_graphBase} returns the old base value.

\end{itemize}

\subsubsection{{\tt SCOTCH\_graphBuild}}

\begin{itemize}
\progsyn

{\tt\begin{tabular}{l@{}ll}
int SCOTCH\_graphBuild ( & SCOTCH\_Graph *     & grafptr, \\
                         & const SCOTCH\_Num   & baseval, \\
                         & const SCOTCH\_Num   & vertnbr, \\
                         & const SCOTCH\_Num * & verttab, \\
                         & const SCOTCH\_Num * & vendtab, \\
                         & const SCOTCH\_Num * & velotab, \\
                         & const SCOTCH\_Num * & vlbltab, \\
                         & const SCOTCH\_Num   & edgenbr, \\
                         & const SCOTCH\_Num * & edgetab, \\
                         & const SCOTCH\_Num * & edlotab)
\end{tabular}}

{\tt\begin{tabular}{l@{}ll}
scotchfgraphbuild ( & doubleprecision (*)   & grafdat, \\
                    & integer*{\it num}     & baseval, \\
                    & integer*{\it num}     & vertnbr, \\
                    & integer*{\it num} (*) & verttab, \\
                    & integer*{\it num} (*) & vendtab, \\
                    & integer*{\it num} (*) & velotab, \\
                    & integer*{\it num} (*) & vlbltab, \\
                    & integer*{\it num}     & edgenbr, \\
                    & integer*{\it num} (*) & edgetab, \\
                    & integer*{\it num} (*) & edlotab, \\
                    & integer               & ierr)
\end{tabular}}

\progdes

The {\tt SCOTCH\_graphBuild} routine fills the source graph structure
pointed to by {\tt grafptr} with all of the data that are passed to it.

{\tt baseval} is the graph base value for index arrays (typically $0$ for
structures built from C and $1$ for structures built from Fortran).
{\tt vertnbr} is the number of vertices.
{\tt verttab} is the adjacency index array, of size $({\tt vertnbr} +
1)$ if the edge array is compact (that is, if {\tt vendtab} equals
$\mathtt{verttab}+1$ or {\tt NULL}), or of size {\tt vertnbr} else.
{\tt vendtab} is the adjacency end index array, of size {\tt vertnbr} if
it is disjoint from {\tt verttab}.
{\tt velotab} is the vertex load array, of size {\tt vertnbr} if it exists.
{\tt vlbltab} is the vertex label array, of size {\tt vertnbr} if it exists.
{\tt edgenbr} is the number of arcs (that is, twice the number of edges).
{\tt edgetab} is the adjacency array, of size at least {\tt edgenbr}
(it can be more if the edge array is not compact). {\tt edlotab} is
the arc load array, of size {\tt edgenbr} if it exists.

The {\tt vendtab}, {\tt velotab}, {\tt vlbltab} and {\tt edlotab}
arrays are optional, and a {\tt NULL} pointer can be passed as
argument whenever they are not defined.
Since, in Fortran, there is no null reference, passing the
{\tt scotchf\lbt graph\lbt build} routine a reference equal to
{\tt verttab} in the {\tt velotab} or {\tt vlbltab} fields makes them
be considered as missing arrays. The same holds for {\tt edlotab}
when it is passed a reference equal to {\tt edgetab}. Setting
{\tt vendtab} to refer to one cell after {\tt verttab} yields the
same result, as it is the exact semantics of a compact vertex array.

To limit memory consumption, {\tt SCOTCH\_\lbt graph\lbo Build} does
not copy array data, but instead references them in the {\tt
SCOTCH\_\lbt Graph} structure. Therefore, great care should be taken
not to modify the contents of the arrays passed to {\tt SCOTCH\_\lbt
graph\lbo Build} as long as the graph structure is in use. Every
update of the arrays should be preceded by a call to {\tt SCOTCH\_\lbt
graph\lbo Free}, to free internal graph structures, and eventually
followed by a new call to {\tt SCOTCH\_\lbt graph\lbo Build} to
re-build these internal structures so as to be able to use the new
graph.

To ensure that inconsistencies in user data do not result in an
erroneous behavior of the \libscotch\ routines, it is recommended, at
least in the development stage, to call the {\tt SCOTCH\_\lbt
graph\lbt Check} routine on the newly created {\tt SCOTCH\_\lbt Graph}
structure before calling any other \libscotch\ routine.

\progret

{\tt SCOTCH\_graphBuild} returns $0$ if the graph structure has been
successfully set with all of the input data, and $1$ else.
\end{itemize}

\subsubsection{{\tt SCOTCH\_graphCheck}}

\begin{itemize}
\progsyn

{\tt\begin{tabular}{l@{}ll}
int SCOTCH\_graphCheck ( & const SCOTCH\_Graph * & grafptr)
\end{tabular}}

{\tt\begin{tabular}{l@{}ll}
scotchfgraphcheck ( & doubleprecision (*) & grafdat, \\
                    & integer             & ierr)
\end{tabular}}

\progdes

The {\tt SCOTCH\_graphCheck} routine checks the consistency of the
given {\tt SCOTCH\_\lbt Graph} structure. It can be used in client
applications to determine if a graph that has been created from
used-generated data by means of the {\tt SCOTCH\_\lbt graph\lbt Build}
routine is consistent, prior to calling any other routines of the
\libscotch\ library.

\progret

{\tt SCOTCH\_graphCheck} returns $0$ if graph data are consistent, and
$1$ else.

\end{itemize}

\subsubsection{{\tt SCOTCH\_graphCoarsen}}

\begin{itemize}
\progsyn

{\tt\begin{tabular}{l@{}ll}
int SCOTCH\_graphCoarsen ( & SCOTCH\_Graph * const & finegrafptr, \\
                           & const SCOTCH\_Num     & coarvertnbr, \\
                           & const double          & coarrat,     \\
                           & const SCOTCH\_Num     & flagval,     \\
                           & SCOTCH\_Graph * const & coargrafptr, \\
                           & SCOTCH\_Num * const   & coarmulttab) \\
\end{tabular}}

{\tt\begin{tabular}{l@{}ll}
scotchfgraphcoarsen ( & doubleprecision (*)   & finegrafdat, \\
                      & integer*{\it num}     & coarvertnbr, \\
                      & doubleprecision       & coarrat,     \\
                      & integer*{\it num}     & flagval,     \\
                      & doubleprecision (*)   & coargrafdat, \\
                      & integer*{\it num} (*) & coarmulttab, \\
                      & integer               & ierr)
\end{tabular}}

\progdes

The {\tt SCOTCH\_graphCoarsen} routine creates, in the
{\tt SCOTCH\_\lbt Graph} structure {\tt coar\lbt graf\lbt dat} pointed
to by {\tt coar\lbt graf\lbt ptr}, a graph coarsened from the
{\tt SCOTCH\_\lbt Graph} structure {\tt fine\lbt graf\lbt dat} pointed
to by {\tt fine\lbt graf\lbt ptr}. The coarsened graph is created only
if it comprises more than {\tt coar\lbt vert\lbt nbr} vertices, or if
the coarsening ratio is lower than {\tt coarrat}. Valid coarsening
ratio values range from $0.5$ (in the case of a perfect matching) to
$1.0$ (if no vertex could be coarsened). Classical threshold values
range from $0.7$ to $0.8$.

The {\tt flagval} flag specifies the type of coarsening.
% Several groups of flags can be combined, by means of addition or
% ``binary or'' operators.
When {\tt SCOTCH\_\lbt COARSEN\lbt NO\lbt MERGE} is set, isolated
vertices are never merged with other vertices. This preserves the
topology of the graph, at the expense of a higher coarsening ratio.

The {\tt multloctab} array should be of a size big enough to store
multinode data for the resulting coarsened graph. Hence, the size of
the array must be at least twice the maximum expected number of local
coarse vertices, according to the prescribed coarsening ratio
{\tt coarrat}. Upon successful completion, this array will contain
pairs of consecutive {\tt SCOTCH\_\lbt Num} values, representing the
indices of the two fine vertices that have been coarsened into each of
the coarse vertices. When a vertex has been coarsened with itself, its
two multinode values are identical.

{\tt coargrafdat} must have been initialized with the
{\tt SCOTCH\_\lbt graph\lbt Init} routine before
{\tt SCOTCH\_graph\lbt Coarsen} is called.

\progret

{\tt SCOTCH\_graphCoarsen} returns $0$ if the coarse graph
structure has been successfully created, $1$ if the coarse graph was
not created because it did not enforce the threshold parameters, and
$2$ on error.
\end{itemize}

\subsubsection{{\tt SCOTCH\_graphCoarsenBuild}}

\begin{itemize}
\progsyn

{\tt\begin{tabular}{l@{}ll}
int SCOTCH\_graphCoarsenBuild ( & SCOTCH\_Graph * const & finegrafptr, \\
                                & const SCOTCH\_Num     & coarvertnbr, \\
                                & SCOTCH\_Num * const   & finematetab, \\
                                & SCOTCH\_Graph * const & coargrafptr, \\
                                & SCOTCH\_Num * const   & coarmulttab) \\
\end{tabular}}

{\tt\begin{tabular}{l@{}ll}
scotchfgraphcoarsenbuild ( & doubleprecision (*)   & finegrafdat, \\
                           & integer*{\it num}     & coarvertnbr, \\
                           & integer*{\it num} (*) & finematetab, \\
                           & doubleprecision (*)   & coargrafdat, \\
                           & integer*{\it num} (*) & coarmulttab, \\
                           & integer               & ierr)
\end{tabular}}

\progdes

The {\tt SCOTCH\_graphCoarsenBuild} routine creates, in the
{\tt SCOTCH\_\lbt Graph} structure {\tt coar\lbt graf\lbt dat} pointed
to by {\tt coar\lbt graf\lbt ptr}, a graph with {\tt coar\lbt vert\lbt nbr}
vertices, coarsened from the {\tt SCOTCH\_\lbt Graph} structure
{\tt fine\lbt graf\lbt dat} pointed to by {\tt fine\lbt graf\lbt ptr},
using the matching provided by {\tt fine\lbt mate\lbt tab}.

On input, the {\tt fine\lbt mate\lbt tab} mating array should contain
the indices of the mates chosen for each vertex of the fine
graph. When some vertex is mated to itself, its array cell value is
equal to its own index. Upon successful completion, this array is
updated so as to contain fine-to-coarse indices: each array cell
contains the index of the coarse vertex created from the given fine
vertex.

The {\tt fine\lbt mate\lbt tab} mating array and its associated number
of coarse vertices {\tt coar\lbt vert\lbt nbr} may have been computed
using the {\tt SCOTCH\_\lbt graph\lbt Coarsen\lbt Match}
routine. Indeed, calling the
{\tt SCOTCH\_\lbt graph\lbt Coarsen\lbt Match} and
{\tt SCOTCH\_\lbt graph\lbt Coarsen\lbt Build} routines in sequence
amounts to calling the {\tt SCOTCH\_\lbt graph\lbt Coarsen} routine,
yet additionally publicizing the {\tt fine\lbt mate\lbt tab} array.

The {\tt multloctab} array should be of a size big enough to store
multinode data for the resulting coarsened graph, that is, twice the
value of {\tt coar\lbt vert\lbt nbr}. Upon successful completion,
this array will contain pairs of consecutive {\tt SCOTCH\_\lbt Num}
values, representing the indices of the two fine vertices that have
been coarsened into each of the coarse vertices. When a vertex has
been coarsened with itself, the two multinode values are identical.

{\tt coargrafdat} must have been initialized with the
{\tt SCOTCH\_\lbt graph\lbt Init} routine before
{\tt SCOTCH\_graph\lbt Coarsen\lbt Build} is called.

\progret

{\tt SCOTCH\_graphCoarsenBuild} returns $0$ if the coarse graph
structure has been successfully created, and $1$ on error.
\end{itemize}

\subsubsection{{\tt SCOTCH\_graphCoarsenMatch}}

\begin{itemize}
\progsyn

{\tt\begin{tabular}{l@{}ll}
int SCOTCH\_graphCoarsenMatch ( & SCOTCH\_Graph * const & finegrafptr, \\
                                & SCOTCH\_Num * const   & coarvertptr, \\
                                & const double          & coarrat,     \\
                                & const SCOTCH\_Num     & flagval,     \\
                                & SCOTCH\_Num * const   & finematetab) \\
\end{tabular}}

{\tt\begin{tabular}{l@{}ll}
scotchfgraphcoarsenmatch ( & doubleprecision (*)   & finegrafdat, \\
                           & integer*{\it num}     & coarvertnbr, \\
                           & doubleprecision       & coarrat,     \\
                           & integer*{\it num}     & flagval,     \\
                           & integer*{\it num} (*) & finematetab, \\
                           & integer               & ierr)
\end{tabular}}

\progdes

The {\tt SCOTCH\_graphCoarsenMatch} routine fills the
{\tt fine\lbt mate\lbt tab} array with a matching of the vertices of
the {\tt SCOTCH\_\lbt Graph} structure {\tt fine\lbt graf\lbt dat}
pointed to by {\tt fine\lbt graf\lbt ptr}. The matching is
computed only if it amounts to the creation of more than
{\tt coar\lbt vert\lbt nbr} (that is, the value pointed to by
{\tt coar\lbt vert\lbt ptr} in the C interface) coarse vertices, or if
the coarsening ratio is lower than {\tt coarrat}. Valid coarsening
ratio values range from $0.5$ (in the case of a perfect matching) to
$1.0$ (if no vertex could be coarsened). Classical threshold values
range from $0.7$ to $0.8$.

The {\tt flagval} flag specifies the type of matching.
% Several groups of flags can be combined, by means of addition or
% ``binary or'' operators.
When {\tt SCOTCH\_\lbt COARSEN\lbt NO\lbt MERGE} is set, isolated
vertices are never matched with other vertices. This preserves the
topology of the graph, at the expense of a higher coarsening ratio.

The {\tt finematetab} array must be of a size sufficient to hold as
many {\tt SCOTCH\_\lbt Num} values as the number of vertices in the
{\tt fine\lbt graf\lbt dat} graph. Upon successful completion, this
array will contain the indices of the mates chosen for each vertex of
the provided graph. When some vertex is mated to itself, its array
cell value is equal to its own index. Additionally, {\tt coarvertnbr}
will be set to the number of coarse vertices associated with the
matching. This number is equal to the number of vertices in the
provided graph, minus the number of matched pairs of vertices,
since in a subsequent coarsening process, each pair should see its two
matched vertices collapsed into a single coarse vertex.

The mating array and its associated number of coarse vertices can be
used by the {\tt SCOTCH\_\lbt graph\lbt Coarsen\lbt Build} routine.
Indeed, calling the {\tt SCOTCH\_\lbt graph\lbt Coarsen\lbt Match} and
{\tt SCOTCH\_\lbt graph\lbt Coarsen\lbt Build} routines in sequence
amounts to calling the {\tt SCOTCH\_\lbt graph\lbt Coarsen} routine,
yet additionally publicizing the {\tt fine\lbt mate\lbt tab} array.

\progret

{\tt SCOTCH\_graphCoarsenMatch} returns $0$ if a matching has been
successfully computed, $1$ if the matching was not computed because it
did not enforce the threshold parameters, and $2$ on error.
\end{itemize}

\subsubsection{{\tt SCOTCH\_graphColor}}

\begin{itemize}
\progsyn

{\tt\begin{tabular}{l@{}ll}
int SCOTCH\_graphColor ( & const SCOTCH\_Graph * & grafptr, \\
                         & SCOTCH\_Num *         & colotab, \\
                         & SCOTCH\_Num *         & coloptr, \\
                         & SCOTCH\_Num           & flagval)
\end{tabular}}

{\tt\begin{tabular}{l@{}ll}
scotchfgraphcolor ( & doubleprecision (*)   & grafdat, \\
                    & integer*{\it num} (*) & colotab, \\
                    & integer{\it num}      & colonbr, \\
                    & integer{\it num}      & flagval, \\
                    & integer               & ierr)

\end{tabular}}

\progdes

The {\tt SCOTCH\_graphColor} routine computes a coloring of the graph
vertices. The {\tt colotab} array is filled with color values, and the
number of colors found is placed into the integer variable
{\tt colonbr}, pointed to by {\tt coloptr}.

The computed coloring is not guaranteed to be maximal. Indeed,
the only algorithm currently implemented is a variant of Luby's
algorithm. Due to the operations of this algorithm, the first colors
are likely to have many more representatives than the last colors.

Like for partition arrays, color values are \textit{not} based: color
values range from $0$ to $(\mathtt{colonbr} - 1)$.

The flag value {\tt flagval} is currently not used. It may be used in
the future to select a coloring method. At the time being, a value of
$0$ should be provided.

\progret

{\tt SCOTCH\_graphColor} returns $0$ if the graph coloring has been
successfully computed, and $1$ else.
\end{itemize}

\subsubsection{{\tt SCOTCH\_graphData}}
\label{sec-lib-func-graphdata}

\begin{itemize}
\progsyn

{\tt\begin{tabular}{l@{}ll}
void SCOTCH\_graphData ( & const SCOTCH\_Graph * & grafptr, \\
                         & SCOTCH\_Num *         & baseptr, \\
                         & SCOTCH\_Num *         & vertptr, \\
                         & SCOTCH\_Num **        & verttab, \\
                         & SCOTCH\_Num **        & vendtab, \\
                         & SCOTCH\_Num **        & velotab, \\
                         & SCOTCH\_Num **        & vlbltab, \\
                         & SCOTCH\_Num *         & edgeptr, \\
                         & SCOTCH\_Num **        & edgetab, \\
                         & SCOTCH\_Num **        & edlotab)
\end{tabular}}

{\tt\begin{tabular}{l@{}ll}
scotchfgraphdata ( & doubleprecision (*)   & grafdat, \\
                   & integer*{\it num} (*) & indxtab, \\
                   & integer*{\it num}     & baseval, \\
                   & integer*{\it num}     & vertnbr, \\
                   & integer*{\it idx}     & vertidx, \\
                   & integer*{\it idx}     & vendidx, \\
                   & integer*{\it idx}     & veloidx, \\
                   & integer*{\it idx}     & vlblidx, \\
                   & integer*{\it num}     & edgenbr, \\
                   & integer*{\it idx}     & edgeidx, \\
                   & integer*{\it num}     & edloidx)
\end{tabular}}

\progdes

The {\tt SCOTCH\_graphData} routine is the dual of the
{\tt SCOTCH\_\lbt graph\lbo Build} routine. It is a multiple
accessor that returns scalar values and array references.

{\tt baseptr} is the pointer to a location that will hold the graph base
value for index arrays (typically $0$ for
structures built from C and $1$ for structures built from Fortran).
{\tt vertptr} is the pointer to a location that will hold the number of
vertices.
{\tt verttab} is the pointer to a location that will hold the reference to
the adjacency index array, of size $\mathtt{*vertptr} + 1$ if the
adjacency array is compact, or of size {\tt *vertptr} else.
{\tt vendtab} is the pointer to a location that will hold the reference to
the adjacency end index array, and is equal to $\mathtt{verttab} +
1$ if the adjacency array is compact.
{\tt velotab} is the pointer to a location that will hold the reference to
the vertex load array, of size {\tt *vertptr}.
{\tt vlbltab} is the pointer to a location that will hold the reference to
the vertex label array, of size {\tt vertnbr}.
{\tt edgeptr} is the pointer to a location that will hold the number of arcs
(that is, twice the number of edges).
{\tt edgetab} is the pointer to a location that will hold the reference to
the adjacency array, of size at least {\tt *edgeptr}.
{\tt edlotab} is the pointer to a location that will hold the reference to
the arc load array, of size {\tt *edgeptr}.

Any of these pointers can be set to {\tt NULL} on input if the
corresponding information is not needed. Else, the reference to a
dummy area can be provided, where all unwanted data will be written.

Since there are no pointers in Fortran, a specific mechanism is used
to allow users to access graph arrays. The {\tt scotchf\lbt graph\lbt
data} routine is passed an integer array, the first element of which is used
as a base address from which all other array indices are
computed. Therefore, instead of returning references, the routine
returns integers, which represent the starting index of each of the
relevant arrays with respect to the base input array, or {\tt
vertidx}, the index of {\tt verttab}, if they do not exist. For
instance, if some base array {\tt myarray\lbt (1)} is passed as
parameter {\tt indxtab}, then the first cell of array {\tt verttab}
will be accessible as {\tt myarray\lbt (vertidx)}.
In order for this feature to behave properly, the {\tt indxtab}
array must be word-aligned with the graph arrays. This is
automatically enforced on most systems, but some care should be
taken on systems that allow one to access data that is not
word-aligned. On such systems, declaring the array after a
dummy {\tt double\lbt precision} array can coerce the compiler
into enforcing the proper alignment. Also, on 32\_64 architectures,
such indices can be larger than the size of a regular
{\tt INTEGER}. This is why the indices to be returned are defined by
means of a specific integer type. See
Section~\ref{sec-lib-inttypesize} for more information on this
issue.
\end{itemize}




\subsubsection{{\tt SCOTCH\_graphDiamPV}}

\begin{itemize}
\progsyn

{\tt\begin{tabular}{l@{}ll}
SCOTCH\_Num SCOTCH\_graphDiamPV ( & const SCOTCH\_Graph * & grafptr)
\end{tabular}}

{\tt\begin{tabular}{l@{}ll}
scotchfgraphdiampv ( & doubleprecision (*) & grafdat, \\
                     & integer{\it num}    & diamval)

\end{tabular}}

\progdes

The {\tt SCOTCH\_graphDiamPV} routine computes the edge-weighted
(pseudo-)diameter value of the given graph.

To do so, it selects a random vertex, computes the set of vertices at
maximum distance from this vertex by means of Dijkstra's algorithm,
selects a vertex from this set, and repeats the process as long as
this maximum distance value increases. If the graph is not
edge-weighted, neighboring vertices are assumed to be at distance $1$
from each other; else, edge weights represent distances between
vertices.

\progret

{\tt SCOTCH\_graphDiamPV} returns a positive value if the graph
diameter has been successfully computed, the \texttt{SCOTCH\_\lbt
NUMMAX} maximum positive value if the graph is disconnected, and $-1$
on error.
\end{itemize}

\subsubsection{{\tt SCOTCH\_graphExit}}

\begin{itemize}
\progsyn

{\tt\begin{tabular}{l@{}ll}
void SCOTCH\_graphExit ( & SCOTCH\_Graph * & grafptr)
\end{tabular}}

{\tt\begin{tabular}{l@{}ll}
scotchfgraphexit ( & doubleprecision (*) & grafdat)
\end{tabular}}

\progdes

The {\tt SCOTCH\_graphExit} function frees the contents of a
{\tt SCOTCH\_\lbt Graph} structure previously initialized by
{\tt SCOTCH\_\lbt graphInit}. All subsequent calls to
{\tt SCOTCH\_\lbt graph} routines other than {\tt SCOTCH\_\lbt
graphInit}, using this structure as parameter, may yield
unpredictable results.
\end{itemize}

\subsubsection{{\tt SCOTCH\_graphFree}}

\begin{itemize}
\progsyn

{\tt\begin{tabular}{l@{}ll}
void SCOTCH\_graphFree ( & SCOTCH\_Graph * & grafptr)
\end{tabular}}

{\tt\begin{tabular}{l@{}ll}
scotchfgraphfree ( & doubleprecision (*) & grafdat)
\end{tabular}}

\progdes

The {\tt SCOTCH\_graphFree} function frees the graph data of a {\tt
SCOTCH\_\lbt Graph} structure previously initialized by {\tt
SCOTCH\_\lbt graph\lbt Init}, but preserves its internal data
structures. This call is equivalent to a call to {\tt SCOTCH\_\lbt
graph\lbt Exit} immediately followed by a call to {\tt SCOTCH\_\lbt
graph\lbt Init}. Consequently, the given {\tt SCOTCH\_\lbt Graph}
structure remains ready for subsequent calls to any routine of the
\libscotch\ library.

\end{itemize}

\subsubsection{{\tt SCOTCH\_graphInduceList}}

\begin{itemize}
\progsyn

{\tt\begin{tabular}{l@{}ll}
int SCOTCH\_graphInduceList ( & const SCOTCH\_Graph * & orggrafptr, \\
                              & SCOTCH\_Num           & vnumnbr, \\
                              & SCOTCH\_Num *         & vnumtab, \\
                              & SCOTCH\_Graph *       & indgrafptr)
\end{tabular}}

{\tt\begin{tabular}{l@{}ll}
scotchfgraphinducelist ( & doubleprecision (*)   & orggrafdat, \\
                         & integer*{\it num}     & vnumnbr, \\
                         & integer{\it num} (*)  & vnumtab, \\
                         & doubleprecision (*)   & indgrafdat, \\
                         & integer               & ierr)

\end{tabular}}

\progdes

The {\tt SCOTCH\_graphInduceList} routine computes an induced graph 
\texttt{indgrafdat} from the original graph \texttt{orggrafdat}. The
vertices that are kept in the induced graph are the \texttt{vnumnbr}
vertices whose based indices in the original graph are provided in the
\texttt{vnumtab} array, in its first \texttt{vnumnbr} cells.

\progret

{\tt SCOTCH\_graphInduceList} returns $0$ if the induced graph has
been successfully computed, and $1$ else.
\end{itemize}

\subsubsection{{\tt SCOTCH\_graphInducePart}}

\begin{itemize}
\progsyn

{\tt\begin{tabular}{l@{}ll}
int SCOTCH\_graphInducePart ( & const SCOTCH\_Graph * & orggrafptr, \\
                              & SCOTCH\_Num           & vnumnbr, \\
                              & SCOTCH\_GraphPart2 *  & parttab, \\
                              & SCOTCH\_GraphPart2    & partval, \\
                              & SCOTCH\_Graph *       & indgrafptr)
\end{tabular}}

{\tt\begin{tabular}{l@{}ll}
scotchfgraphinducepart ( & doubleprecision (*)    & orggrafdat, \\
                         & integer*{\it num}      & vnumnbr, \\
                         & character{\it num} (*) & parttab, \\
                         & character{\it num}     & partval, \\
                         & doubleprecision (*)    & indgrafdat, \\
                         & integer                & ierr)

\end{tabular}}

\progdes

The {\tt SCOTCH\_graphInducePart} routine computes an induced graph 
\texttt{indgrafdat} from the original graph \texttt{orggrafdat}. The
vertices that are kept in the induced graph are the \texttt{vnumnbr}
vertices whose part number in the \texttt{parttab} array are equal to
\texttt{partval}. The \texttt{SCOTCH\_\lbt Graph\lbt Part2} type,
being a very small integer (most likely, an \texttt{unsigned char}),
is assumed to hold only small values, e.g. \texttt{0} or \texttt{1}.

\progret

{\tt SCOTCH\_graphInducePart} returns $0$ if the induced graph has
been successfully computed, and $1$ else.
\end{itemize}

\subsubsection{{\tt SCOTCH\_graphInit}}
\label{sec-lib-func-graphinit}

\begin{itemize}
\progsyn

{\tt\begin{tabular}{l@{}ll}
int SCOTCH\_graphInit ( & SCOTCH\_Graph * & grafptr)
\end{tabular}}

{\tt\begin{tabular}{l@{}ll}
scotchfgraphinit ( & doubleprecision (*) & grafdat, \\
                   & integer             & ierr)
\end{tabular}}

\progdes

The {\tt SCOTCH\_graphInit} function initializes a {\tt SCOTCH\_\lbt
Graph} structure so as to make it suitable for future operations. It
should be the first function to be called upon a {\tt SCOTCH\_\lbt
Graph} structure. When the graph data is no longer of use, call
function {\tt SCOTCH\_\lbt graph\lbt Exit} to free its internal
structures.

\progret

{\tt SCOTCH\_graphInit} returns $0$ if the graph structure has been
successfully initialized, and $1$ else.
\end{itemize}

\subsubsection{{\tt SCOTCH\_graphLoad}}
\label{sec-lib-func-graphload}

\begin{itemize}
\progsyn

{\tt\begin{tabular}{l@{}ll}
int SCOTCH\_graphLoad ( & SCOTCH\_Graph * & grafptr, \\
                        & FILE *          & stream,  \\
                        & SCOTCH\_Num     & baseval, \\
                        & SCOTCH\_Num     & flagval)
\end{tabular}}

{\tt\begin{tabular}{l@{}ll}
scotchfgraphload ( & doubleprecision (*) & grafdat, \\
                   & integer             & fildes,  \\
                   & integer*{\it num}   & baseval, \\
                   & integer*{\it num}   & flagval, \\
                   & integer             & ierr)
\end{tabular}}

\progdes

The {\tt SCOTCH\_graphLoad} routine fills the {\tt
SCOTCH\_\lbt Graph} structure pointed to by {\tt grafptr} with the
source graph description available from stream {\tt stream} in the
\scotch\ graph format (see section~\ref{sec-file-sgraph}).

To ease the handling of source graph files by programs written in C as
well as in Fortran, the base value of the graph to read can be set
to {\tt 0} or {\tt 1}, by setting the {\tt baseval} parameter to the
proper value. A value of {\tt -1} indicates that the graph base should
be the same as the one provided in the graph description that is read
from {\tt stream}.

The {\tt flagval} value is a combination of the following integer values,
that may be added or bitwise-ored:
\begin{itemize}
\iteme[{\tt 0}]
Keep vertex and edge weights if they are present in the {\tt stream} data.
\iteme[{\tt 1}]
Remove vertex weights. The graph read will have all of its vertex weights
set to one, regardless of what is specified in the {\tt stream} data.
\iteme[{\tt 2}]
Remove edge weights. The graph read will have all of its edge weights
set to one, regardless of what is specified in the {\tt stream} data.
\end{itemize}

Fortran users must use the {\tt PXFFILENO} or {\tt FNUM} functions to
obtain the number of the Unix file descriptor {\tt fildes} associated
with the logical unit of the graph file.

\progret

{\tt SCOTCH\_graphLoad} returns $0$ if the graph structure has been
successfully allocated and filled with the data read, and $1$ else.
\end{itemize}

\subsubsection{{\tt SCOTCH\_graphSave}}

\begin{itemize}
\progsyn

{\tt\begin{tabular}{l@{}ll}
int SCOTCH\_graphSave ( & const SCOTCH\_Graph * & grafptr, \\
                        & FILE *                & stream)
\end{tabular}}

{\tt\begin{tabular}{l@{}ll}
scotchfgraphsave ( & doubleprecision (*) & grafdat, \\
                   & integer             & fildes,  \\
                   & integer             & ierr)
\end{tabular}}

\progdes

The {\tt SCOTCH\_graphSave} routine saves the contents of the {\tt
SCOTCH\_\lbt Graph} structure pointed to by {\tt grafptr} to stream
{\tt stream}, in the \scotch\ graph format (see
section~\ref{sec-file-sgraph}).

Fortran users must use the {\tt PXFFILENO} or {\tt FNUM} functions to
obtain the number of the Unix file descriptor {\tt fildes} associated
with the logical unit of the graph file.

\progret

{\tt SCOTCH\_graphSave} returns $0$ if the graph structure has been
successfully written to {\tt stream}, and $1$ else.
\end{itemize}

\subsubsection{{\tt SCOTCH\_graphSize}}

\begin{itemize}
\progsyn

{\tt\begin{tabular}{l@{}ll}
void SCOTCH\_graphSize ( & const SCOTCH\_Graph * & grafptr, \\
                         & SCOTCH\_Num *         & vertptr, \\
                         & SCOTCH\_Num *         & edgeptr)
\end{tabular}}

{\tt\begin{tabular}{l@{}ll}
scotchfgraphsize ( & doubleprecision (*) & grafdat, \\
                   & integer*{\it num}   & vertnbr, \\
                   & integer*{\it num}   & edgenbr)
\end{tabular}}

\progdes

The {\tt SCOTCH\_graphSize} routine fills the two areas of type
{\tt SCOTCH\_\lbt Num} pointed to by {\tt vertptr} and {\tt edgeptr}
with the number of vertices and arcs (that is, twice the number
of edges) of the given graph pointed to by {\tt grafptr},
respectively.

Any of these pointers can be set to {\tt NULL} on input if the
corresponding information is not needed. Else, the reference to a
dummy area can be provided, where all unwanted data will be written.

This routine is useful to get the size of a graph read by means
of the {\tt SCOTCH\_\lbt graph\lbo Load} routine, in order to allocate
auxiliary arrays of proper sizes. If the whole structure of the
graph is wanted, function {\tt SCOTCH\_graph\lbo Data} should be
preferred.
\end{itemize}

\subsubsection{{\tt SCOTCH\_graphStat}}

\begin{itemize}
\progsyn

{\tt\begin{tabular}{l@{}ll}
void SCOTCH\_graphStat ( & const SCOTCH\_Graph * & grafptr, \\
                         & SCOTCH\_Num *         & velominptr, \\
                         & SCOTCH\_Num *         & velomaxptr, \\
                         & SCOTCH\_Num *         & velosumptr, \\
                         & double *              & veloavgptr, \\
                         & double *              & velodltptr, \\
                         & SCOTCH\_Num *         & degrminptr, \\
                         & SCOTCH\_Num *         & degrmaxptr, \\
                         & double *              & degravgptr, \\
                         & double *              & degrdltptr, \\
                         & SCOTCH\_Num *         & edlominptr, \\
                         & SCOTCH\_Num *         & edlomaxptr, \\
                         & SCOTCH\_Num *         & edlosumptr, \\
                         & double *              & edloavgptr, \\
                         & double *              & edlodltptr)
\end{tabular}}

{\tt\begin{tabular}{l@{}ll}
scotchfgraphstat ( & doubleprecision (*) & grafdat, \\
                   & integer*{\it num}   & velomin, \\
                   & integer*{\it num}   & velomax, \\
                   & integer*{\it num}   & velosum, \\
                   & doubleprecision     & veloavg, \\
                   & doubleprecision     & velodlt, \\
                   & integer*{\it num}   & degrmin, \\
                   & integer*{\it num}   & degrmax, \\
                   & doubleprecision     & degravg, \\
                   & doubleprecision     & degrdlt, \\
                   & integer*{\it num}   & edlomin, \\
                   & integer*{\it num}   & edlomax, \\
                   & integer*{\it num}   & edlosum, \\
                   & doubleprecision     & edloavg, \\
                   & doubleprecision     & edlodlt)
\end{tabular}}

\progdes

The {\tt SCOTCH\_graphStat} routine produces some statistics regarding
the graph structure pointed to by {\tt grafptr}.
{\tt velomin}, {\tt velomax}, {\tt velosum}, {\tt veloavg} and
{\tt velodlt} are the minimum vertex load, the maximum vertex
load, the sum of all vertex loads, the average vertex load,
and the variance of the vertex loads, respectively.
{\tt degrmin}, {\tt degrmax}, {\tt degravg} and
{\tt degrdlt} are the minimum vertex degree, the maximum vertex
degree, the average vertex degree, and the variance of the vertex
degrees, respectively.
{\tt edlomin}, {\tt edlomax}, {\tt edlosum}, {\tt edloavg} and
{\tt edlodlt} are the minimum edge load, the maximum edge
load, the sum of all edge loads, the average edge load,
and the variance of the edge loads, respectively.
\end{itemize}

\subsection{High-level graph partitioning, mapping and clustering routines}
\label{sec-lib-func-part-map}
\index{Clustering}

The routines presented in this section provide high-level
functionalities and free the user from the burden of calling in
sequence several of the low-level routines described in the next
section.

\subsubsection{{\tt SCOTCH\_graphMap}}
\label{sec-lib-func-graphmap}

\begin{itemize}
\progsyn

{\tt\begin{tabular}{l@{}ll}
int SCOTCH\_graphMap ( & const SCOTCH\_Graph * & grafptr, \\
                       & const SCOTCH\_Arch *  & archptr, \\
                       & const SCOTCH\_Strat * & straptr, \\
                       & SCOTCH\_Num *         & parttab)
\end{tabular}}

{\tt\begin{tabular}{l@{}ll}
scotchfgraphmap ( & doubleprecision (*)   & grafdat, \\
                  & doubleprecision (*)   & archdat, \\
                  & doubleprecision (*)   & stradat, \\
                  & integer*{\it num} (*) & parttab, \\
                  & integer               & ierr)
\end{tabular}}

\progdes

The {\tt SCOTCH\_graphMap} routine computes a mapping of the source
graph structure pointed to by {\tt grafptr} onto the target
architecture pointed to by {\tt archptr}, using the mapping strategy
pointed to by {\tt straptr} (as defined in
Section~\ref{sec-lib-format-map}), and returns the mapping data in the
array pointed to by {\tt parttab}.

The {\tt parttab} array should have been previously allocated, of a
size sufficient to hold as many {\tt SCOTCH\_\lbt Num} integers as
there are vertices in the source graph.

On return, every cell of the mapping array holds the number of the
target vertex to which the corresponding source vertex is mapped.
The numbering of target values is {\em not\/} based: target vertices
are numbered from $0$ to the number of target vertices minus $1$.
This semantics aims at complying with standards such as MPI, in
which process ranks start from $0$.

When a variable-sized architecture is used (see
Section~\ref{sec-file-target-variable}) and a proper strategy is
provided (see Section~\ref{sec-lib-func-stratgraphclusterbuild}),
the {\tt SCOTCH\_graph\lbt Map} routine can cluster\index{Clustering}
the given graph by means of recursive bipartitioning.
In this case, clusters are labeled according to a binary scheme: the
part equal to the whole graph is numbered $1$, its two bipartitioned
descendants are labeled $2$ and $3$, the two descendants of part $2$
are labeled $4$ and $5$, and so on. More generally, clusters are
labeled such that the two descendants of any cluster $i$ that has been
split are labeled $2i$ and $2i + 1$.

Classical clustering strategies perform recursive bipartitioning of
process graphs until some criterion is met: either parts become
smaller than some size threshold, or edge density becomes higher than
some ratio, etc. If graph mapping is performed using a variable-sized
architecture and a classical mapping strategy, recursive
bipartitioning will halt only when the load imbalance criterion allows
for one of the bipartitioned parts to be empty (that is, most often,
parts contains a single vertex).

\progret

{\tt SCOTCH\_graphMap} returns $0$ if the mapping of the graph has
been successfully computed, and $1$ else. In this last case, the
{\tt parttab} array may however have been partially or completely
filled, but its contents are not significant.
\end{itemize}

\subsubsection{{\tt SCOTCH\_graphMapFixed}}
\label{sec-lib-func-graphmapfixed}

\begin{itemize}
\progsyn

{\tt\begin{tabular}{l@{}ll}
int SCOTCH\_graphMapFixed ( & const SCOTCH\_Graph * & grafptr, \\
                            & const SCOTCH\_Arch *  & archptr, \\
                            & const SCOTCH\_Strat * & straptr, \\
                            & SCOTCH\_Num *         & parttab)
\end{tabular}}

{\tt\begin{tabular}{l@{}ll}
scotchfgraphmapfixed ( & doubleprecision (*)   & grafdat, \\
                       & doubleprecision (*)   & archdat, \\
                       & doubleprecision (*)   & stradat, \\
                       & integer*{\it num} (*) & parttab, \\
                       & integer               & ierr)
\end{tabular}}

\progdes

The {\tt SCOTCH\_graphMapFixed} routine computes a mapping of the
source graph structure pointed to by {\tt grafptr} onto the target
architecture pointed to by {\tt archptr}, using the mapping strategy
pointed to by {\tt straptr} (as defined in
Section~\ref{sec-lib-format-map}), and fills the array pointed to by
{\tt parttab} with the mapping data regarding vertices which have not
been pre-assigned by the user.

The {\tt parttab} array should have been previously allocated, of a
size sufficient to hold as many {\tt SCOTCH\_\lbt Num} integers as
there are vertices in the source graph. It must also have been filled
in advance by the user, with data indicating whether vertices have
been already pre-assigned to a fixed position or are to be processed
by the routine. In each cell of the {\tt parttab} array, a value of
$-1$ indicates that the vertex is movable, while a value between $0$
and the number of target vertices minus $1$ indicates that the vertex
has been pre-assigned to the given part.

On return, every cell of the mapping array that contained a $-1$ will
hold the number of the target vertex to which the corresponding source
vertex is mapped. The numbering of target values is {\em not\/}
based: target vertices are numbered from $0$ to the number of target
vertices minus $1$. This semantics aims at complying with standards
such as MPI, in which process ranks start from $0$.

\progret

{\tt SCOTCH\_graphMapFixed} returns $0$ if the mapping of the graph
has been successfully computed, and $1$ else. In this last case, the
{\tt parttab} array may however have been partially or completely
filled, but its contents are not significant.
\end{itemize}

\subsubsection{{\tt SCOTCH\_graphPart}}

\begin{itemize}
\progsyn

{\tt\begin{tabular}{l@{}ll}
int SCOTCH\_graphPart ( & const SCOTCH\_Graph * & grafptr, \\
                        & const SCOTCH\_Num     & partnbr, \\
                        & const SCOTCH\_Strat * & straptr, \\
                        & SCOTCH\_Num *         & parttab)
\end{tabular}}

{\tt\begin{tabular}{l@{}ll}
scotchfgraphpart ( & doubleprecision (*)   & grafdat, \\
                   & integer*{\it num}     & partnbr, \\
                   & doubleprecision (*)   & stradat, \\
                   & integer*{\it num} (*) & parttab, \\
                   & integer               & ierr)
\end{tabular}}

\progdes

The {\tt SCOTCH\_graphPart} routine computes an edge-separated
partition, into {\tt partnbr} parts, of the source graph structure
pointed to by {\tt grafptr}, using the graph edge partitioning
strategy pointed to by {\tt stratptr} (as defined in
Section~\ref{sec-lib-format-map}), and returns the partition data in
the array pointed to by {\tt parttab}.

The {\tt parttab} array should have been previously allocated, of a
size sufficient to hold as many {\tt SCOTCH\_\lbt Num} integers as
there are vertices in the source graph.

On return, every cell of the mapping array holds the number of the
target vertex to which the corresponding source vertex is mapped. The
numbering of target values is {\em not\/} based: target vertices are
numbered from $0$ to $\mathtt{partnbr} - 1$. This semantics aims at
complying with standards such as MPI, in which process ranks start
from $0$.

\progret

{\tt SCOTCH\_graphPart} returns $0$ if the graph partition has
been successfully computed, and $1$ else. In the latter case, the
{\tt parttab} array may however have been partially or completely
filled, but its contents are not significant.
\end{itemize}

\subsubsection{{\tt SCOTCH\_graphPartFixed}}
\label{sec-lib-func-graphpartfixed}

\begin{itemize}
\progsyn

{\tt\begin{tabular}{l@{}ll}
int SCOTCH\_graphPartFixed ( & const SCOTCH\_Graph * & grafptr, \\
                             & const SCOTCH\_Num     & partnbr, \\
                             & const SCOTCH\_Strat * & straptr, \\
                             & SCOTCH\_Num *         & parttab)
\end{tabular}}

{\tt\begin{tabular}{l@{}ll}
scotchfgraphpartfixed ( & doubleprecision (*)   & grafdat, \\
                        & integer*{\it num}     & partnbr, \\
                        & doubleprecision (*)   & stradat, \\
                        & integer*{\it num} (*) & parttab, \\
                        & integer               & ierr)
\end{tabular}}

\progdes

The {\tt SCOTCH\_graphPartFixed} routine computes an edge-separated
partition, into {\tt partnbr} parts, of the source graph structure
pointed to by {\tt grafptr}, using the graph edge partitioning
strategy pointed to by {\tt stratptr} (as defined in
Section~\ref{sec-lib-format-map}), and fills the array pointed to by
{\tt parttab} with the partitioning data regarding vertices which have
not been pre-assigned by the user.

The {\tt parttab} array should have been previously allocated, of a
size sufficient to hold as many {\tt SCOTCH\_\lbt Num} integers as
there are vertices in the source graph. It must also have been filled
in advance by the user, with data indicating whether vertices have
been already pre-assigned to a fixed position or are to be processed
by the routine. In each cell of the {\tt parttab} array, a value of
$-1$ indicates that the vertex is movable, while a value between $0$
and the number of target vertices minus $1$ indicates that the vertex
has been pre-assigned to the given part.

On return, every cell of the mapping array that contained a $-1$ will
hold the number of the target vertex to which the corresponding source
vertex is assigned. The numbering of target values is {\em not\/}
based: target vertices are numbered from $0$ to the number of target
vertices minus $1$. This semantics aims at complying with standards
such as MPI, in which process ranks start from $0$.

\progret

{\tt SCOTCH\_graphPartFixed} returns $0$ if the graph partition
has been successfully computed, and $1$ else. In the latter case, the
{\tt parttab} array may however have been partially or completely
filled, but its contents are not significant.
\end{itemize}

\subsubsection{{\tt SCOTCH\_graphPartOvl}}
\label{sec-lib-func-graphpartovl}

\begin{itemize}
\progsyn

{\tt\begin{tabular}{l@{}ll}
int SCOTCH\_graphPartOvl ( & const SCOTCH\_Graph * & grafptr, \\
                           & const SCOTCH\_Num     & partnbr, \\
                           & const SCOTCH\_Strat * & straptr, \\
                           & SCOTCH\_Num *         & parttab)
\end{tabular}}

{\tt\begin{tabular}{l@{}ll}
scotchfgraphpartovl ( & doubleprecision (*)   & grafdat, \\
                      & integer*{\it num}     & partnbr, \\
                      & doubleprecision (*)   & stradat, \\
                      & integer*{\it num} (*) & parttab, \\
                      & integer               & ierr)
\end{tabular}}

\progdes

The {\tt SCOTCH\_graphPartOvl} routine computes an overlapped
vertex-separated partition, into {\tt partnbr} parts, of the source
graph structure pointed to by {\tt grafptr}, using the graph vertex
partitioning with overlap strategy pointed to by {\tt stratptr} (as
defined in Section~\ref{sec-lib-format-part-ovl}), and returns the
partition data in the array pointed to by {\tt parttab}.

The {\tt parttab} array should have been previously allocated, of a
size sufficient to hold as many {\tt SCOTCH\_\lbt Num} integers as
there are vertices in the source graph.

On return, every array cell holds the number of the part to which the
corresponding vertex is mapped. Regular parts are numbered from $0$ to
$\mathtt{partnbr} - 1$, and separator vertices are labeled with part
number {\tt -1}.

While {\tt SCOTCH\_graphMap} and {\tt SCOTCH\_\lbt graph\lbt Part} are
based on edge partitioning methods,
{\tt SCOTCH\_\lbt graph\lbt Part\lbt Ovl} relies on a completely
distinct set of routines to compute vertex separators. This is why
{\tt SCOTCH\_\lbt graph\lbt Part\lbt Ovl} requires strategy strings of
a different kind, created by the
{\tt SCOTCH\_\lbt strat\lbt Graph\lbt   Part\lbt Ovl*} routines only
(see Sections~\ref{sec-lib-func-stratgraphpartovl}
and~\ref{sec-lib-func-stratgraphpartovlbuild}).

\progret

{\tt SCOTCH\_graphPartOvl} returns $0$ if the partition of the graph
has been successfully computed, and $1$ else. In the latter case, the
{\tt parttab} array may however have been partially or completely
filled, but its contents are not significant.
\end{itemize}

\subsubsection{{\tt SCOTCH\_graphRemap}}

\begin{itemize}
\progsyn

{\tt\begin{tabular}{l@{}ll}
int SCOTCH\_graphRemap ( & const SCOTCH\_Graph * & grafptr, \\
                         & const SCOTCH\_Arch *  & archptr, \\
                         & const SCOTCH\_Num *   & parotab, \\
                         & const double          & emraval, \\
                         & const SCOTCH\_Num *   & vmlotab, \\
                         & const SCOTCH\_Strat * & straptr, \\
                         & SCOTCH\_Num *         & parttab)
\end{tabular}}

{\tt\begin{tabular}{l@{}ll}
scotchfgraphremap ( & doubleprecision (*)   & grafdat, \\
                    & doubleprecision (*)   & archdat, \\
                    & integer*{\it num} (*) & parotab, \\
                    & doubleprecision       & emraval, \\
                    & integer*{\it num} (*) & vmlotab, \\
                    & doubleprecision (*)   & stradat, \\
                    & integer*{\it num} (*) & parttab, \\
                    & integer               & ierr)
\end{tabular}}

\progdes

The {\tt SCOTCH\_graphRemap} routine computes a remapping of the
source graph structure pointed to by {\tt grafptr} onto the target
architecture pointed to by {\tt archptr}, based on the old partition
array pointed to by {\tt parotab}, using the mapping strategy pointed
to by {\tt straptr} (as defined in Section~\ref{sec-lib-format-map}),
and returns the mapping data in the array pointed to by {\tt parttab}.

The {\tt parotab} array stores the old partition that is used to
compute migration costs. Every cell contains values from $0$ to the
number of target vertices minus $1$, or $-1$ for vertices that did not
belong to the old partition (e.g., vertices newly created by graph
adaptation, which can be placed at no cost before their associated
data is interpolated).

With every source graph vertex is associated an individual integer
migration cost, stored in the {\tt vmlotab} array. These costs are
accounted for in the communication cost function to minimize as
multiples of the individual migration cost {\tt emraval}. Since this
value is provided as a floating point number, migration costs can be
set as fractions or as non-integer multiples of the cut metric
communication costs stored as integer edge loads.

The {\tt parttab} array should have been previously allocated, of a
size sufficient to hold as many {\tt SCOTCH\_\lbt Num} integers as
there are vertices in the source graph.

On return, every cell of the mapping array holds the number of the
target vertex to which the corresponding source vertex is mapped.
The numbering of target values is {\em not\/} based: target vertices
are numbered from $0$ to the number of target vertices minus $1$.
This semantics aims at complying with standards such as MPI, in
which process ranks start from $0$.

\progret

{\tt SCOTCH\_graphRemap} returns $0$ if the mapping of the graph has
been successfully computed, and $1$ else. In this last case, the
{\tt parttab} array may however have been partially or completely
filled, but its contents are not significant.
\end{itemize}

\subsubsection{{\tt SCOTCH\_graphRemapFixed}}
\label{sec-lib-func-graphremapfixed}

\begin{itemize}
\progsyn

{\tt\begin{tabular}{l@{}ll}
int SCOTCH\_graphRemapFixed ( & const SCOTCH\_Graph * & grafptr, \\
                              & const SCOTCH\_Arch *  & archptr, \\
                              & const SCOTCH\_Num *   & parotab, \\
                              & const double          & emraval, \\
                              & const SCOTCH\_Num *   & vmlotab, \\
                              & const SCOTCH\_Strat * & straptr, \\
                              & SCOTCH\_Num *         & parttab)
\end{tabular}}

{\tt\begin{tabular}{l@{}ll}
scotchfgraphremapfixed ( & doubleprecision (*)   & grafdat, \\
                         & doubleprecision (*)   & archdat, \\
                         & integer*{\it num} (*) & parotab, \\
                         & doubleprecision       & emraval, \\
                         & integer*{\it num} (*) & vmlotab, \\
                         & doubleprecision (*)   & stradat, \\
                         & integer*{\it num} (*) & parttab, \\
                         & integer               & ierr)
\end{tabular}}

\progdes

The {\tt SCOTCH\_graphRemapFixed} routine computes a remapping of the
source graph structure pointed to by {\tt grafptr} onto the target
architecture pointed to by {\tt archptr}, based on the old partition
array pointed to by {\tt parotab}, using the mapping strategy pointed
to by {\tt straptr} (as defined in Section~\ref{sec-lib-format-map}),
and fills the array pointed to by {\tt parttab} with the mapping data
regarding vertices which have not been pre-assigned by the user.

The {\tt parotab} array stores the old partition that is used to
compute migration costs. Every cell contains values from $0$ to the
number of target vertices minus $1$, or $-1$ for vertices that did not
belong to the old partition (e.g., vertices newly created by graph
adaptation, which can be placed at no cost before their associated
data is interpolated).

With every source graph vertex is associated an individual integer
migration cost, stored in the {\tt vmlotab} array. These costs are
accounted for in the communication cost function to minimize as
multiples of the individual migration cost {\tt emraval}. Since this
value is provided as a floating point number, migration costs can be
set as fractions or as non-integer multiples of the cut metric
communication costs stored as integer edge loads.

The {\tt parttab} array should have been previously allocated, of a
size sufficient to hold as many {\tt SCOTCH\_\lbt Num} integers as
there are vertices in the source graph. It must also have been filled
in advance by the user, with data indicating whether vertices have
been already pre-assigned to a fixed position or are to be processed
by the routine. In each cell of the {\tt parttab} array, a value of
$-1$ indicates that the vertex is movable, while a value between $0$
and the number of target vertices minus $1$ indicates that the vertex
has been pre-assigned to the given part.

On return, every cell of the mapping array that contained a $-1$ will
hold the number of the target vertex to which the corresponding source
vertex is mapped. The numbering of target values is {\em not\/}
based: target vertices are numbered from $0$ to the number of target
vertices minus $1$. This semantics aims at complying with standards
such as MPI, in which process ranks start from $0$.

\progret

{\tt SCOTCH\_graphRemapFixed} returns $0$ if the mapping of the
graph has been successfully computed, and $1$ else. In this last case,
the {\tt parttab} array may however have been partially or completely
filled, with some $-1$'s removed, but its contents are not significant.
\end{itemize}

\subsubsection{{\tt SCOTCH\_graphRepart}}

\begin{itemize}
\progsyn

{\tt\begin{tabular}{l@{}ll}
int SCOTCH\_graphRepart ( & const SCOTCH\_Graph * & grafptr, \\
                          & const SCOTCH\_Num     & partnbr, \\
                          & const SCOTCH\_Num *   & parotab, \\
                          & const double          & emraval, \\
                          & const SCOTCH\_Num *   & vmlotab, \\
                          & const SCOTCH\_Strat * & straptr, \\
                          & SCOTCH\_Num *         & parttab)
\end{tabular}}

{\tt\begin{tabular}{l@{}ll}
scotchfgraphrepart ( & doubleprecision (*)   & grafdat, \\
                     & integer*{\it num}     & partnbr, \\
                     & integer*{\it num} (*) & parotab, \\
                     & doubleprecision       & emraval, \\
                     & integer*{\it num} (*) & vmlotab, \\
                     & doubleprecision (*)   & stradat, \\
                     & integer*{\it num} (*) & parttab, \\
                     & integer               & ierr)
\end{tabular}}

\progdes

The {\tt SCOTCH\_graphRepart} routine computes an edge-separated
repartition, into {\tt partnbr} parts, of the source graph structure
pointed to by {\tt grafptr}, based on the old partition array pointed
to by {\tt parotab}, using the partitioning strategy pointed to by
{\tt straptr} (as defined in Section~\ref{sec-lib-format-map}), and
returns the partition data in the array pointed to by {\tt parttab}.

The {\tt parotab} array stores the old partition that is used to
compute migration costs. Every cell contains values from $0$ to the
number of target vertices minus $1$, or $-1$ for vertices that did not
belong to the old partition (e.g., vertices newly created by graph
adaptation, which can be assigned to any part at no cost before
their associated data is interpolated).

With every source graph vertex is associated an individual integer
migration cost, stored in the {\tt vmlotab} array. These costs are
accounted for in the communication cost function to minimize as
multiples of the individual migration cost {\tt emraval}. Since this
value is provided as a floating point number, migration costs can be
set as fractions or as non-integer multiples of the cut metric
communication costs stored as integer edge loads.

The {\tt parttab} array should have been previously allocated, of a
size sufficient to hold as many {\tt SCOTCH\_\lbt Num} integers as
there are vertices in the source graph.

On return, every cell of the mapping array holds the number of the
target vertex to which the corresponding source vertex is mapped.
The numbering of target values is {\em not\/} based: target vertices
are numbered from $0$ to the number of target vertices minus $1$.
This semantics aims at complying with standards such as MPI, in
which process ranks start from $0$.

\progret

{\tt SCOTCH\_graphRepart} returns $0$ if the graph partition has
been successfully computed, and $1$ else. In the latter case, the
{\tt parttab} array may however have been partially or completely
filled, but its contents are not significant.
\end{itemize}

\subsubsection{{\tt SCOTCH\_graphRepartFixed}}
\label{sec-lib-func-graphrepartfixed}

\begin{itemize}
\progsyn

{\tt\begin{tabular}{l@{}ll}
int SCOTCH\_graphRepartFixed ( & const SCOTCH\_Graph * & grafptr, \\
                               & const SCOTCH\_Num     & partnbr, \\
                               & const SCOTCH\_Num *   & parotab, \\
                               & const double          & emraval, \\
                               & const SCOTCH\_Num *   & vmlotab, \\
                               & const SCOTCH\_Strat * & straptr, \\
                               & SCOTCH\_Num *         & parttab)
\end{tabular}}

{\tt\begin{tabular}{l@{}ll}
scotchfgraphrepartfixed ( & doubleprecision (*)   & grafdat, \\
                          & integer*{\it num}     & partnbr, \\
                          & integer*{\it num} (*) & parotab, \\
                          & doubleprecision       & emraval, \\
                          & integer*{\it num} (*) & vmlotab, \\
                          & doubleprecision (*)   & stradat, \\
                          & integer*{\it num} (*) & parttab, \\
                          & integer               & ierr)
\end{tabular}}

\progdes

The {\tt SCOTCH\_graphRepartFixed} routine computes an edge-separated
repartition, into {\tt partnbr} parts, of the source graph structure
pointed to by {\tt grafptr}, based on the old partition array pointed
to by {\tt parotab}, using the partitioning strategy pointed to by
{\tt straptr} (as defined in Section~\ref{sec-lib-format-map}),
and fills the array pointed to by {\tt parttab} with the mapping data
regarding vertices which have not been pre-assigned by the user.

The {\tt parotab} array stores the old partition that is used to
compute migration costs. Every cell contains values from $0$ to the
number of target vertices minus $1$, or $-1$ for vertices that did not
belong to the old partition (e.g., vertices newly created by graph
adaptation, which can be assigned to any part at no cost before
their associated data is interpolated).

With every source graph vertex is associated an individual integer
migration cost, stored in the {\tt vmlotab} array. These costs are
accounted for in the communication cost function to minimize as
multiples of the individual migration cost {\tt emraval}. Since this
value is provided as a floating point number, migration costs can be
set as fractions or as non-integer multiples of the cut metric
communication costs stored as integer edge loads.

The {\tt parttab} array should have been previously allocated, of a
size sufficient to hold as many {\tt SCOTCH\_\lbt Num} integers as
there are vertices in the source graph. It must also have been filled
in advance by the user, with data indicating whether vertices have
been already pre-assigned to a fixed position or are to be processed
by the routine. In each cell of the {\tt parttab} array, a value of
$-1$ indicates that the vertex is movable, while a value between $0$
and the number of target vertices minus $1$ indicates that the vertex
has been pre-assigned to the given part.

On return, every cell of the mapping array that contained a $-1$ will
hold the number of the target vertex to which the corresponding source
vertex is mapped. The numbering of target values is {\em not\/}
based: target vertices are numbered from $0$ to the number of target
vertices minus $1$. This semantics aims at complying with standards
such as MPI, in which process ranks start from $0$.

\progret

{\tt SCOTCH\_graphRepartFixed} returns $0$ if the graph partition has
has been successfully computed, and $1$ else. In this last case, the
{\tt parttab} array may however have been partially or completely
filled, with some $-1$'s removed, but its contents are not
significant.
\end{itemize}

\subsection{Low-level graph partitioning, mapping and clustering routines}
\label{sec-lib-func-part-map-low}

All of the following routines operate on a {\tt SCOTCH\_\lbt Mapping}
structure that contains references to the partition and mapping arrays
to be filled during the mapping or remapping process.

\subsubsection{{\tt SCOTCH\_graphMapCompute}}

\begin{itemize}
\progsyn

{\tt\begin{tabular}{l@{}ll}
int SCOTCH\_graphMapCompute ( & const SCOTCH\_Graph * & grafptr, \\
                              & SCOTCH\_Mapping *     & mappptr, \\
                              & const SCOTCH\_Strat * & straptr)
\end{tabular}}

{\tt\begin{tabular}{l@{}ll}
scotchfgraphmapcompute ( & doubleprecision (*) & grafdat, \\
                         & doubleprecision (*) & mappdat, \\
                         & doubleprecision (*) & stradat, \\
                         & integer             & ierr)
\end{tabular}}

\progdes

The {\tt SCOTCH\_graphMapCompute} routine computes a mapping
on the given {\tt SCOTCH\_\lbt Mapping} structure pointed
to by {\tt mappptr} using the mapping strategy pointed to
by {\tt stratptr}.

On return, every cell of the mapping array defined by
{\tt SCOTCH\_\lbt map\lbt Init} holds the number of the target
vertex to which the corresponding source vertex is mapped. The
numbering of target values is {\em not\/} based: target vertices are
numbered from $0$ to the number of target vertices, minus $1$.

\progret

{\tt SCOTCH\_graphMapCompute} returns $0$ if the mapping has been
successfully computed, and $1$ else. In this latter case, the mapping
array may however have been partially or completely filled, but its
contents are not significant.
\end{itemize}

\subsubsection{{\tt SCOTCH\_graphMapExit}}

\begin{itemize}
\progsyn

{\tt\begin{tabular}{l@{}ll}
void SCOTCH\_graphMapExit ( & const SCOTCH\_Graph * & grafptr, \\
                            & SCOTCH\_Mapping *     & mappptr)
\end{tabular}}

{\tt\begin{tabular}{l@{}ll}
scotchfgraphmapexit ( & doubleprecision (*) & grafdat, \\
                      & doubleprecision (*) & mappdat)
\end{tabular}}

\progdes

The {\tt SCOTCH\_graphMapExit} function frees the contents of a
{\tt SCOTCH\_\lbt Mapping} structure previously initialized by
{\tt SCOTCH\_\lbt graph\lbt Map\lbt Init}. All subsequent calls to
{\tt SCOTCH\_\lbt graph\lbt Map*} routines other than
{\tt SCOTCH\_\lbt graph\lbt Map\lbt Init}, using this structure
as parameter, may yield unpredictable results.
\end{itemize}

\subsubsection{{\tt SCOTCH\_graphMapFixedCompute}}

\begin{itemize}
\progsyn

{\tt\begin{tabular}{l@{}ll}
int SCOTCH\_graphMapFixedCompute ( & const SCOTCH\_Graph * & grafptr, \\
                                   & SCOTCH\_Mapping *     & mappptr, \\
                                   & const SCOTCH\_Strat * & straptr)
\end{tabular}}

{\tt\begin{tabular}{l@{}ll}
scotchfgraphmapfixedcompute ( & doubleprecision (*) & grafdat, \\
                              & doubleprecision (*) & mappdat, \\
                              & doubleprecision (*) & stradat, \\
                              & integer             & ierr)
\end{tabular}}

\progdes

The {\tt SCOTCH\_graphMapFixedCompute} routine computes a mapping on
the given {\tt SCOTCH\_\lbt Mapping} structure pointed to by
{\tt mappptr} using the mapping strategy pointed to by
{\tt stratptr}. The mapping must have been built so that its partition
array has been filled in advance by the user, with data indicating
whether vertices have been already pre-assigned to a fixed position or
are to be processed by the routine. In each cell of the {\tt parttab}
array, a value of $-1$ indicates that the vertex is movable, while a
value between $0$ and the number of target vertices minus $1$
indicates that the vertex has been pre-assigned to the given part.

On return, every cell of the mapping array defined by
{\tt SCOTCH\_\lbt map\lbt Init} that contained a $-1$ will hold
the number of the target vertex to which the corresponding source
vertex is mapped. The numbering of target values is {\em not\/} based:
target vertices are numbered from $0$ to the number of target
vertices, minus $1$.

\progret

{\tt SCOTCH\_graphMapFixedCompute} returns $0$ if the mapping has been
successfully computed, and $1$ else. In this latter case, the mapping
array may however have been partially or completely filled, with some
$-1$'s removed, but its contents are not significant.
\end{itemize}

\subsubsection{{\tt SCOTCH\_graphMapInit}}

\begin{itemize}
\progsyn

{\tt\begin{tabular}{l@{}ll}
int SCOTCH\_graphMapInit ( & const SCOTCH\_Graph * & grafptr, \\
                           & SCOTCH\_Mapping *     & mappptr, \\
                           & const SCOTCH\_Arch *  & archptr, \\
                           & SCOTCH\_Num *         & parttab)
\end{tabular}}

{\tt\begin{tabular}{l@{}ll}
scotchfgraphmapinit ( & doubleprecision (*)   & grafdat, \\
                      & doubleprecision (*)   & mappdat, \\
                      & doubleprecision (*)   & archdat, \\
                      & integer*{\it num} (*) & parttab, \\
                      & integer               & ierr)
\end{tabular}}

\progdes

The {\tt SCOTCH\_graphMapInit} routine fills the mapping structure
pointed to by {\tt mappptr} with all of the data that is passed to it.
Thus, all subsequent calls to ordering routines such as {\tt
SCOTCH\_\lbt graph\lbt Map\lbt Compute}, using this mapping structure
as parameter, will place mapping results in field {\tt parttab}.

{\tt parttab} is the pointer to an array of as many {\tt SCOTCH\_\lbt
Num}s as there are vertices in the graph pointed to by {\tt grafptr},
and which will receive the indices of the vertices of the target
architecture pointed to by {\tt archptr}.

It should be the first function to be called upon a {\tt SCOTCH\_\lbt
Mapping} structure. When the mapping structure is no longer of use,
call function {\tt SCOTCH\_graph\lbt \lbt Map\lbt Exit} to free its
internal structures.

\progret

{\tt SCOTCH\_graphMapInit} returns $0$ if the mapping structure has been
successfully initialized, and $1$ else.
\end{itemize}

\subsubsection{{\tt SCOTCH\_graphMapLoad}}
\label{sec-lib-graph-map-load}

\begin{itemize}
\progsyn

{\tt\begin{tabular}{l@{}ll}
int SCOTCH\_graphMapLoad ( & const SCOTCH\_Graph * & grafptr, \\
                           & SCOTCH\_Mapping *     & mappptr, \\
                           & FILE *                & stream)
\end{tabular}}

{\tt\begin{tabular}{l@{}ll}
scotchfgraphmapload ( & doubleprecision (*) & grafdat, \\
                      & doubleprecision (*) & mappdat, \\
                      & integer             & fildes, \\
                      & integer             & ierr)
\end{tabular}}

\progdes

The {\tt SCOTCH\_graphMapLoad} routine fills the
{\tt SCOTCH\_\lbt Mapping} structure pointed to by
{\tt mappptr} with the mapping data available in
the \scotch\ mapping format (see section~\ref{sec-file-map})
from stream {\tt stream}. If the source graph has vertex labels
attached to its vertices, mapping indices in the input stream are
assumed to be vertex labels as well.

Users willing to have subsequent access to the partition data
rather than to fill an opaque {\tt SCOTCH\_\lbt Mapping} structure
are invited to use the {\tt SCOTCH\_\lbt graph\lbt Tab\lbt Load}
routine instead.

Fortran users must use the {\tt PXFFILENO} or {\tt FNUM} functions to
obtain the number of the Unix file descriptor {\tt fildes} associated
with the logical unit of the mapping file.

\progret

{\tt SCOTCH\_graphMapLoad} returns $0$ if the mapping structure
has been successfully loaded from {\tt stream}, and $1$ else.
\end{itemize}

\subsubsection{{\tt SCOTCH\_graphMapSave}}

\begin{itemize}
\progsyn

{\tt\begin{tabular}{l@{}ll}
int SCOTCH\_graphMapSave ( & const SCOTCH\_Graph *   & grafptr, \\
                           & const SCOTCH\_Mapping * & mappptr, \\
                           & FILE *                  & stream)
\end{tabular}}

{\tt\begin{tabular}{l@{}ll}
scotchfgraphmapsave ( & doubleprecision (*) & grafdat, \\
                      & doubleprecision (*) & mappdat, \\
                      & integer             & fildes,  \\
                      & integer             & ierr)
\end{tabular}}

\progdes

The {\tt SCOTCH\_graphMapSave} routine saves the contents of the {\tt
SCOTCH\_\lbt Mapping} structure pointed to by {\tt mappptr} to stream
{\tt stream}, in the \scotch\ mapping format (see
section~\ref{sec-file-map}).

Fortran users must use the {\tt PXFFILENO} or {\tt FNUM} functions to
obtain the number of the Unix file descriptor {\tt fildes} associated
with the logical unit of the mapping file.

\progret

{\tt SCOTCH\_graphMapSave} returns $0$ if the mapping structure
has been successfully written to {\tt stream}, and $1$ else.
\end{itemize}

\subsubsection{{\tt SCOTCH\_graphMapView}}

\begin{itemize}
\progsyn

{\tt\begin{tabular}{l@{}ll}
int SCOTCH\_graphMapView ( & const SCOTCH\_Graph *   & grafptr, \\
                           & const SCOTCH\_Mapping * & mappptr, \\
                           & FILE *                  & stream)
\end{tabular}}

{\tt\begin{tabular}{l@{}ll}
scotchfgraphmapview ( & doubleprecision (*) & grafdat, \\
                      & doubleprecision (*) & mappdat, \\
                      & integer             & fildes,  \\
                      & integer             & ierr)
\end{tabular}}

\progdes

The {\tt SCOTCH\_mapView} routine summarizes statistical information
on the mapping pointed to by {\tt mappptr} (load of target processors,
number of neighboring domains, average dilation and expansion, edge
cut size, distribution of edge dilations), and prints these results to
stream {\tt stream}.

Fortran users must use the {\tt PXFFILENO} or {\tt FNUM} functions to
obtain the number of the Unix file descriptor {\tt fildes} associated
with the logical unit of the output data file.

\progret

{\tt SCOTCH\_mapView} returns $0$ if the data has been successfully
written to {\tt stream}, and $1$ else.
\end{itemize}

\subsubsection{{\tt SCOTCH\_graphRemapCompute}}

\begin{itemize}
\progsyn

{\tt\begin{tabular}{l@{}ll}
int SCOTCH\_graphRemapCompute ( & const SCOTCH\_Graph * & grafptr, \\
                                & SCOTCH\_Mapping *     & mappptr, \\
                                & SCOTCH\_Mapping *     & mapoptr, \\
                                & const double          & emraval, \\
                                & const SCOTCH\_Num *   & vmlotab, \\
                                & const SCOTCH\_Strat * & straptr)
\end{tabular}}

{\tt\begin{tabular}{l@{}ll}
scotchfgraphremapcompute ( & doubleprecision (*)   & grafdat, \\
                           & doubleprecision (*)   & mappdat, \\
                           & doubleprecision (*)   & mapodat, \\
                           & doubleprecision       & emraval, \\
                           & integer*{\it num} (*) & vmlotab, \\
                           & doubleprecision (*)   & stradat, \\
                           & integer               & ierr)
\end{tabular}}

\progdes

The {\tt SCOTCH\_graphRemapCompute} routine computes a mapping
on the given {\tt SCOTCH\_\lbt Mapping} structure pointed
to by {\tt mappptr}, using the mapping strategy pointed to
by {\tt stratptr}, and accounting for migration costs computed
based on the already computed partition pointed to by
{\tt mapoptr}. This partition should have been created from the same
graph and target architecture as the one pointer to by {\tt mappptr}.

With every source graph vertex is associated an individual integer
migration cost, stored in the {\tt vmlotab} array. These costs are
accounted for in the communication cost function to minimize as
multiples of the individual migration cost {\tt emraval}. Since this
value is provided as a floating point number, migration costs can be
set as fractions or as non-integer multiples of the cut metric
communication costs stored as integer edge loads.

On return, every cell of the new mapping array defined by
{\tt SCOTCH\_\lbt map\lbt Init} holds the number of the target
vertex to which the corresponding source vertex is mapped. The
numbering of target values is {\em not\/} based: target vertices are
numbered from $0$ to the number of target vertices, minus $1$.

\progret

{\tt SCOTCH\_graphRemapCompute} returns $0$ if the remapping has been
successfully computed, and $1$ else. In this latter case, the mapping
array may however have been partially or completely filled, but its
contents are not significant.
\end{itemize}

\subsubsection{{\tt SCOTCH\_graphRemapFixedCompute}}

\begin{itemize}
\progsyn

{\tt\begin{tabular}{l@{}ll}
int SCOTCH\_graphRemapFixedCompute ( & const SCOTCH\_Graph * & grafptr, \\
                                     & SCOTCH\_Mapping *     & mappptr, \\
                                     & SCOTCH\_Mapping *     & mapoptr, \\
                                     & const double          & emraval, \\
                                     & const SCOTCH\_Num *   & vmlotab, \\
                                     & const SCOTCH\_Strat * & straptr)
\end{tabular}}

{\tt\begin{tabular}{l@{}ll}
scotchfgraphremapfixedcompute ( & doubleprecision (*)   & grafdat, \\
                                & doubleprecision (*)   & mappdat, \\
                                & doubleprecision (*)   & mapodat, \\
                                & doubleprecision       & emraval, \\
                                & integer*{\it num} (*) & vmlotab, \\
                                & doubleprecision (*)   & stradat, \\
                                & integer               & ierr)
\end{tabular}}

\progdes

The {\tt SCOTCH\_graphRemapFixedCompute} routine computes a mapping
on the given {\tt SCOTCH\_\lbt Mapping} structure pointed
to by {\tt mappptr}, using the mapping strategy pointed to
by {\tt stratptr}, and accounting for migration costs computed
based on the already computed partition pointed to by
{\tt mapoptr}. This partition should have been created from the same
graph and target architecture as the one pointer to by {\tt mappptr}.

The partition array of the mapping pointed to by {\tt mappptr} must
have been filled in advance by the user, with data indicating whether
vertices have been already pre-assigned to a fixed position or are to
be processed by the routine. A value of $-1$ indicates that the vertex
is movable, while a value between $0$ and the number of target
vertices minus $1$ indicates that the vertex has been pre-assigned to
the given part.

With every source graph vertex is associated an individual integer
migration cost, stored in the {\tt vmlotab} array. These costs are
accounted for in the communication cost function to minimize as
multiples of the individual migration cost {\tt emraval}. Since this
value is provided as a floating point number, migration costs can be
set as fractions or as non-integer multiples of the cut metric
communication costs stored as integer edge loads.

On return, every cell of the new mapping array defined by
{\tt SCOTCH\_\lbt map\lbt Init} that contained a $-1$ holds the number
of the target vertex to which the corresponding source vertex is
mapped. The numbering of target values is {\em not\/} based: target
vertices are numbered from $0$ to the number of target vertices, minus
$1$.

\progret

{\tt SCOTCH\_graphRemapFixedCompute} returns $0$ if the remapping has
been successfully computed, and $1$ else. In this latter case, the
mapping array may however have been partially or completely filled,
with some $-1$'s removed, but its contents are not significant.
\end{itemize}

\subsubsection{{\tt SCOTCH\_graphTabLoad}}
\label{sec-lib-graph-tab-load}

\begin{itemize}
\progsyn

{\tt\begin{tabular}{l@{}ll}
int SCOTCH\_graphTabLoad ( & const SCOTCH\_Graph * & grafptr, \\
                           & SCOTCH\_Num *         & parttab, \\
                           & FILE *                & stream)
\end{tabular}}

{\tt\begin{tabular}{l@{}ll}
scotchfgraphmapload ( & doubleprecision (*)   & grafdat, \\
                      & integer*{\it num} (*) & parttab, \\
                      & integer               & fildes, \\
                      & integer               & ierr)
\end{tabular}}

\progdes

The {\tt SCOTCH\_graphTabLoad} routine fills the
{\tt parttab} part array pointed to by {\tt parttab} with the mapping
data available in the \scotch\ mapping format (see section~\ref{sec-file-map})
from stream {\tt stream}.

This routine allows users to fill plain partition arrays rather than
opaque mapping structures, as routine {\tt SCOTCH\_\lbt graph\lbt
Map\lbt Load} does.

The {\tt parttab} array should have been previously allocated, of a
size sufficient to hold as many {\tt SCOTCH\_\lbt Num} integers as
there are vertices in the source graph. Upon completion, array cells
contain the indices of the parts to which vertices belong according to
the input mapping stream, or {\tt -1} if they were not mentioned in
the stream. If the source graph has vertex labels attached to its
vertices, mapping indices in the input stream are assumed to be vertex
labels as well.

Fortran users must use the {\tt PXFFILENO} or {\tt FNUM} functions to
obtain the number of the Unix file descriptor {\tt fildes} associated
with the logical unit of the mapping file.

\progret

{\tt SCOTCH\_graphMapLoad} returns $0$ if the mapping structure
has been successfully loaded from {\tt stream}, and $1$ else.
\end{itemize}

\subsection{High-level graph ordering routines}

This routine provides high-level functionality and frees the
user from the burden of calling in sequence several of the low-level
routines described in the next section.

\subsubsection{{\tt SCOTCH\_graphOrder}}

\begin{itemize}
\progsyn

{\tt\begin{tabular}{l@{}ll}
int SCOTCH\_graphOrder ( & const SCOTCH\_Graph * & grafptr, \\
                         & const SCOTCH\_Strat * & straptr, \\
                         & SCOTCH\_Num *         & permtab, \\
                         & SCOTCH\_Num *         & peritab, \\
                         & SCOTCH\_Num *         & cblkptr, \\
                         & SCOTCH\_Num *         & rangtab, \\
                         & SCOTCH\_Num *         & treetab)
\end{tabular}}

{\tt\begin{tabular}{l@{}ll}
scotchfgraphorder ( & doubleprecision (*)   & grafdat, \\
                    & doubleprecision (*)   & stradat, \\
                    & integer*{\it num} (*) & permtab, \\
                    & integer*{\it num} (*) & peritab, \\
                    & integer*{\it num}     & cblknbr, \\
                    & integer*{\it num} (*) & rangtab, \\
                    & integer*{\it num} (*) & treetab, \\
                    & integer               & ierr)
\end{tabular}}

\progdes

The {\tt SCOTCH\_graphOrder} routine computes a block ordering of the
unknowns of the symmetric sparse matrix the adjacency structure of
which is represented by the source graph structure pointed to by {\tt
grafptr}, using the ordering strategy pointed to by {\tt stratptr},
and returns ordering data in the scalar pointed to by {\tt cblkptr}
and the four arrays {\tt permtab}, {\tt peritab}, {\tt rangtab} and
{\tt treetab}.

The {\tt permtab}, {\tt peritab}, {\tt rangtab} and {\tt treetab}
arrays should have been previously allocated, of a size sufficient to
hold as many {\tt SCOTCH\_\lbt Num} integers as there are vertices in
the source graph, plus one in the case of {\tt rangtab}. Any of the
five output fields can be set to {\tt NULL} if the corresponding
information is not needed. Since, in Fortran, there is no null
reference, passing a reference to {\tt grafptr} in these fields will
have the same effect.

On return, {\tt permtab} holds the direct permutation of the unknowns,
that is, vertex $i$ of the original graph has index {\tt permtab[$i$]}
in the reordered graph, while {\tt peritab} holds the inverse
permutation, that is, vertex $i$ in the reordered graph had index {\tt
peritab[$i$]} in the original graph. All of these indices are numbered
according to the base value of the source graph: permutation indices
are numbered from {\tt baseval} to
$\mathtt{vertnbr} + \mathtt{baseval} - 1$, that is,
from $0$ to $\mathtt{vertnbr} - 1$ if the graph base is
$0$, and from $1$ to $\mathtt{vertnbr}$ if the graph base is $1$.

The three other result fields, {\tt *cblkptr}, {\tt rangtab} and {\tt
treetab}, contain data related to the block structure. {\tt *cblkptr}
holds the number of column blocks of the produced ordering, and {\tt
rangtab} holds the starting indices of each of the permuted column
blocks, in increasing order, so that column block $i$ starts at index
{\tt rangtab\lbt [$i$]} and ends at index $(\mbox{\tt
rangtab}\lbt\mathtt{[}i + 1\mathtt{]} - 1)$, inclusive, in the new
ordering. {\tt treetab} holds the separators tree structure, that is,
{\tt treetab[$i$]} is the index of the father of column block $i$ in
the separators tree, or $-1$ if column block $i$ is the root of the
separators tree. Please refer to Section~\ref{sec-lib-format-order}
for more information.

\progret

{\tt SCOTCH\_graphOrder} returns $0$ if the ordering of the graph has
been successfully computed, and $1$ else. In this last case, the
{\tt rangtab}, {\tt permtab}, and {\tt peritab} arrays may however have
been partially or completely filled, but their contents are not significant.
\end{itemize}

\subsection{Low-level graph ordering routines}

All of the following routines operate on a {\tt SCOTCH\_\lbt Ordering}
structure that contains references to the permutation arrays to be
filled during the graph ordering process.

\subsubsection{{\tt SCOTCH\_graphOrderCheck}}

\begin{itemize}
\progsyn

{\tt\begin{tabular}{l@{}ll}
int SCOTCH\_graphOrderCheck ( & const SCOTCH\_Graph *    & grafptr, \\
                              & const SCOTCH\_Ordering * & ordeptr)
\end{tabular}}

{\tt\begin{tabular}{l@{}ll}
scotchfgraphordercheck ( & doubleprecision (*) & grafdat, \\
                         & doubleprecision (*) & ordedat, \\
                         & integer             & ierr)
\end{tabular}}

\progdes

The {\tt SCOTCH\_graphOrderCheck} routine checks the consistency of
the given {\tt SCOTCH\_\lbt Ordering} structure pointed to by {\tt ordeptr}.

\progret

{\tt SCOTCH\_graphOrderCheck} returns $0$ if ordering data are consistent, and
$1$ else.

\end{itemize}

\subsubsection{{\tt SCOTCH\_graphOrderCompute}}

\begin{itemize}
\progsyn

{\tt\begin{tabular}{l@{}ll}
int SCOTCH\_graphOrderCompute ( & const SCOTCH\_Graph * & grafptr, \\
                                & SCOTCH\_Ordering *    & ordeptr, \\
                                & const SCOTCH\_Strat * & straptr)
\end{tabular}}

{\tt\begin{tabular}{l@{}ll}
scotchfgraphordercompute ( & doubleprecision (*) & grafdat, \\
                           & doubleprecision (*) & ordedat, \\
                           & doubleprecision (*) & stradat, \\
                           & integer             & ierr)
\end{tabular}}

\progdes

The {\tt SCOTCH\_graphOrderCompute} routine computes a block ordering
of the graph structure pointed to by {\tt grafptr}, using the ordering
strategy pointed to by {\tt stratptr}, and stores its result in the
ordering structure pointed to by {\tt ordeptr}.

On return, the ordering structure holds a block ordering of the
given graph (see section~\ref{sec-lib-graph-order-init} for a
description of the ordering fields).

\progret

{\tt SCOTCH\_graphOrderCompute} returns $0$ if the ordering has been
successfully computed, and $1$ else. In this latter case, the ordering
arrays may however have been partially or completely filled, but their
contents are not significant.
\end{itemize}

\subsubsection{{\tt SCOTCH\_graphOrderComputeList}}
\label{sec-lib-graph-order-compute-list}

\begin{itemize}
\progsyn

{\tt\begin{tabular}{l@{}ll}
int SCOTCH\_graphOrderComputeList ( & const SCOTCH\_Graph * & grafptr, \\
                                    & SCOTCH\_Ordering *    & ordeptr, \\
                                    & SCOTCH\_Num           & listnbr, \\
                                    & SCOTCH\_Num *         & listtab, \\
                                    & const SCOTCH\_Strat * & straptr)
\end{tabular}}

{\tt\begin{tabular}{l@{}ll}
scotchfgraphordercompute ( & doubleprecision (*)   & grafdat, \\
                           & doubleprecision (*)   & ordedat, \\
                           & integer*{\it num}     & listnbr, \\
                           & integer*{\it num} (*) & listtab, \\
                           & doubleprecision (*)   & stradat, \\
                           & integer               & ierr)
\end{tabular}}

\progdes

The {\tt SCOTCH\_graphOrderComputeList} routine computes a block
ordering of a subgraph of the graph structure pointed to by {\tt
grafptr}, using the ordering strategy pointed to by {\tt stratptr},
and stores its result in the ordering structure pointed to by {\tt
ordeptr}. The induced subgraph is described by means of a vertex
list: {\tt listnbr} holds the number of vertices to keep in the
induced subgraph, the indices of which are given, in any order,
in the {\tt listtab} array.

On return, the ordering structure holds a block ordering of the
induced subgraph (see section~\ref{sec-lib-format-order} for a
description of the ordering fields). To compute this ordering, graph
ordering methods such as the minimum degree and minimum fill methods
will base on the original degree of the induced graph vertices, their
non-induced neighbors being considered as halo vertices (see
Section~\ref{sec-algo-nested-hybrid} for more information on halo
vertices).

Because an ordering always refers to the full graph, the ordering
computed by {\tt SCOTCH\_\lbt graph\lbt Order\lbt Compute\lbt List} is
divided into two distinct parts: the induced graph vertices are ordered
by applying to the induced graph the strategy provided by the {\tt
stratptr} parameter, while non-induced vertex are ordered
consecutively with the highest available indices. Consequently, the
permuted indices of induced vertices range from {\tt baseval} to
$(\mathtt{listnbr} + \mathtt{baseval} - 1)$, while the permuted
indices of the remaining vertices range from $(\mathtt{listnbr} +
\mathtt{baseval})$ to $(\mathtt{vertnbr} + \mathtt{baseval} -
1)$, inclusive. The separation tree yielded by {\tt SCOTCH\_\lbt
graph\lbt Order\lbt Compute\lbt List} reflects this property: it
is made of two branches, the first one corresponding to the induced
subgraph, and the second one to the remaining vertices. Since these
two subgraphs are not considered to be connected, both will have their
own root, represented by a $-1$ value in the {\tt treetab} array of the
ordering.

\progret

{\tt SCOTCH\_graphOrderComputeList} returns $0$ if the ordering has been
successfully computed, and $1$ else. In this latter case, the ordering
arrays may however have been partially or completely filled, but their
contents are not significant.
\end{itemize}

\subsubsection{{\tt SCOTCH\_graphOrderExit}}

\begin{itemize}
\progsyn

{\tt\begin{tabular}{l@{}ll}
void SCOTCH\_graphOrderExit ( & const SCOTCH\_Graph * & grafptr, \\
                              & SCOTCH\_Ordering *    & ordeptr)
\end{tabular}}

{\tt\begin{tabular}{l@{}ll}
scotchfgraphorderexit ( & doubleprecision (*) & grafdat, \\
                        & doubleprecision (*) & ordedat)
\end{tabular}}

\progdes

The {\tt SCOTCH\_graphOrderExit} function frees the contents of a
{\tt SCOTCH\_\lbt Ordering} structure previously initialized by
{\tt SCOTCH\_\lbt graph\lbt Order\lbt Init}. All subsequent calls to
{\tt SCOTCH\_\lbt graph\lbt Order*} routines other than
{\tt SCOTCH\_\lbt graph\lbt Order\lbt Init}, using this structure
as parameter, may yield unpredictable results.
\end{itemize}

\subsubsection{{\tt SCOTCH\_graphOrderInit}}
\label{sec-lib-graph-order-init}

\begin{itemize}
\progsyn

{\tt\begin{tabular}{l@{}ll}
int SCOTCH\_graphOrderInit ( & const SCOTCH\_Graph * & grafptr, \\
                             & SCOTCH\_Ordering *    & ordeptr, \\
                             & SCOTCH\_Num *         & permtab, \\
                             & SCOTCH\_Num *         & peritab, \\
                             & SCOTCH\_Num *         & cblkptr, \\
                             & SCOTCH\_Num *         & rangtab, \\
                             & SCOTCH\_Num *         & treetab)
\end{tabular}}

{\tt\begin{tabular}{l@{}ll}
scotchfgraphorderinit ( & doubleprecision (*)   & grafdat, \\
                        & doubleprecision (*)   & ordedat, \\
                        & integer*{\it num} (*) & permtab, \\
                        & integer*{\it num} (*) & peritab, \\
                        & integer*{\it num}     & cblknbr, \\
                        & integer*{\it num} (*) & rangtab, \\
                        & integer*{\it num} (*) & treetab, \\
                        & integer               & ierr)
\end{tabular}}

\progdes

The {\tt SCOTCH\_graph\lbt Order\lbt Init} routine fills the ordering
structure pointed to by {\tt ordeptr} with all of the data that are
passed to it. Thus, all subsequent calls to ordering routines such as
{\tt SCOTCH\_\lbt graph\lbt Order\lbt Compute}, using this ordering
structure as parameter, will place ordering results in fields {\tt
permtab}, {\tt peritab}, {\tt *cblkptr}, {\tt rangtab} or {\tt
treetab}, if they are not set to {\tt NULL}.

{\tt permtab} is the ordering permutation array, of size ${\tt
vertnbr}$, {\tt peritab} is the inverse ordering permutation array,
of size ${\tt vertnbr}$, {\tt cblkptr} is the pointer to a
{\tt SCOTCH\_\lbt Num} that will receive the number of produced
column blocks, {\tt rangtab} is the array that holds the column block
span information, of size $\mathtt{vertnbr} + 1$, and {\tt treetab}
is the array holding the structure of the separators tree, of size
${\tt vertnbr}$. See the above manual page of
{\tt SCOTCH\_graph\lbt Order}, as well as
section~\ref{sec-lib-format-order}, for an explanation of the
semantics of all of these fields.

The {\tt SCOTCH\_\lbt graph\lbt Order\lbt Init} routine should be the
first function to be called upon a {\tt SCOTCH\_\lbt Ordering}
structure for ordering graphs. When the ordering structure is no
longer of use, the {\tt SCOTCH\_\lbt graph\lbt Order\lbt Exit}
function must be called, in order to to free its internal structures.

\progret

{\tt SCOTCH\_graphOrderInit} returns $0$ if the ordering structure has
been successfully initialized, and $1$ else.
\end{itemize}

\subsubsection{{\tt SCOTCH\_graphOrderLoad}}

\begin{itemize}
\progsyn

{\tt\begin{tabular}{l@{}ll}
int SCOTCH\_graphOrderLoad ( & const SCOTCH\_Graph * & grafptr, \\
                             & SCOTCH\_Ordering *    & ordeptr, \\
                             & FILE *                & stream)
\end{tabular}}

{\tt\begin{tabular}{l@{}ll}
scotchfgraphorderload ( & doubleprecision (*) & grafdat, \\
                        & doubleprecision (*) & ordedat, \\
                        & integer             & fildes,  \\
                        & integer             & ierr)
\end{tabular}}

\progdes

The {\tt SCOTCH\_graphOrderLoad} routine fills the
{\tt SCOTCH\_\lbt Ordering} structure pointed to by
{\tt ordeptr} with the ordering data available in
the \scotch\ ordering format (see section~\ref{sec-file-ord})
from stream {\tt stream}.

Fortran users must use the {\tt PXFFILENO} or {\tt FNUM} functions to
obtain the number of the Unix file descriptor {\tt fildes} associated
with the logical unit of the ordering file.

\progret

{\tt SCOTCH\_graphOrderLoad} returns $0$ if the ordering structure
has been successfully loaded from {\tt stream}, and $1$ else.
\end{itemize}

\subsubsection{{\tt SCOTCH\_graphOrderSave}}

\begin{itemize}
\progsyn

{\tt\begin{tabular}{l@{}ll}
int SCOTCH\_graphOrderSave ( & const SCOTCH\_Graph *    & grafptr, \\
                             & const SCOTCH\_Ordering * & ordeptr, \\
                             & FILE *                   & stream)
\end{tabular}}

{\tt\begin{tabular}{l@{}ll}
scotchfgraphordersave ( & doubleprecision (*) & grafdat, \\
                        & doubleprecision (*) & ordedat, \\
                        & integer             & fildes, \\
                        & integer             & ierr)
\end{tabular}}

\progdes

The {\tt SCOTCH\_graphOrderSave} routine saves the contents of the {\tt
SCOTCH\_\lbt Ordering} structure pointed to by {\tt ordeptr} to stream
{\tt stream}, in the \scotch\ ordering format (see
section~\ref{sec-file-ord}).

Fortran users must use the {\tt PXFFILENO} or {\tt FNUM} functions to
obtain the number of the Unix file descriptor {\tt fildes} associated
with the logical unit of the ordering file.

\progret

{\tt SCOTCH\_graphOrderSave} returns $0$ if the ordering structure
has been successfully written to {\tt stream}, and $1$ else.
\end{itemize}

\subsubsection{{\tt SCOTCH\_graphOrderSaveMap}}

\begin{itemize}
\progsyn

{\tt\begin{tabular}{l@{}ll}
int SCOTCH\_graphOrderSaveMap ( & const SCOTCH\_Graph *    & grafptr, \\
                                & const SCOTCH\_Ordering * & ordeptr, \\
                                & FILE *                   & stream)
\end{tabular}}

{\tt\begin{tabular}{l@{}ll}
scotchfgraphordersavemap ( & doubleprecision (*) & grafdat, \\
                           & doubleprecision (*) & ordedat, \\
                           & integer             & fildes,  \\
                           & integer             & ierr)
\end{tabular}}

\progdes

The {\tt SCOTCH\_graphOrderSaveMap} routine saves the block
partitioning data associated with the {\tt SCOTCH\_\lbt Ordering}
structure pointed to by {\tt ordeptr} to stream {\tt stream},
in the \scotch\ mapping format (see section~\ref{sec-file-map}).
A target domain number is associated with every block, such that
all node vertices belonging to the same block are shown as belonging
to the same target vertex.
The resulting mapping file can be used by the {\tt gout} program
(see Section~\ref{sec-prog-gout}) to produce pictures showing
the different separators and blocks.

Fortran users must use the {\tt PXFFILENO} or {\tt FNUM} functions to
obtain the number of the Unix file descriptor {\tt fildes} associated
with the logical unit of the mapping file.

\progret

{\tt SCOTCH\_graphOrderSaveMap} returns $0$ if the ordering structure
has been successfully written to {\tt stream}, and $1$ else.
\end{itemize}

\subsubsection{{\tt SCOTCH\_graphOrderSaveTree}}

\begin{itemize}
\progsyn

{\tt\begin{tabular}{l@{}ll}
int SCOTCH\_graphOrderSaveTree ( & const SCOTCH\_Graph *    & grafptr, \\
                                 & const SCOTCH\_Ordering * & ordeptr, \\
                                 & FILE *                   & stream)
\end{tabular}}

{\tt\begin{tabular}{l@{}ll}
scotchfgraphordersavetree ( & doubleprecision (*) & grafdat, \\
                            & doubleprecision (*) & ordedat, \\
                            & integer             & fildes,  \\
                            & integer             & ierr)
\end{tabular}}

\progdes

The {\tt SCOTCH\_graphOrderSaveTree} routine saves the tree
hierarchy information associated with the {\tt SCOTCH\_\lbt Ordering}
structure pointed to by {\tt ordeptr} to stream {\tt stream}.

The format of the tree output file resembles the one of a mapping or
ordering file: it is made up of as many lines as there are vertices in
the ordering. Each of these lines holds two integer numbers. The first
one is the index or the label of the vertex, and the second one is the
index of its parent node in the separators tree, or $-1$ if the vertex
belongs to a root node.

Fortran users must use the {\tt PXFFILENO} or {\tt FNUM} functions to
obtain the number of the Unix file descriptor {\tt fildes} associated
with the logical unit of the tree mapping file.

\progret

{\tt SCOTCH\_graphOrderSaveTree} returns $0$ if the separators tree
structure has been successfully written to {\tt stream}, and $1$ else.
\end{itemize}

\subsection{Mesh handling routines}
\label{sec-lib-mesh}

\subsubsection{{\tt SCOTCH\_meshAlloc}}

\begin{itemize}
\progsyn

{\tt\begin{tabular}{l@{}l}
SCOTCH\_Mesh * SCOTCH\_meshAlloc ( & void)
\end{tabular}}

\progdes

The {\tt SCOTCH\_meshAlloc} function allocates a memory area of a
size sufficient to store a {\tt SCOTCH\_\lbt Mesh} structure. It is
the user's responsibility to free this memory when it is no longer
needed, using the {\tt SCOTCH\_\lbt mem\lbt Free} routine. The
allocated space must be initialized before use, by means of the
{\tt SCOTCH\_\lbt mesh\lbt Init} routine.

\progret

{\tt SCOTCH\_meshAlloc} returns the pointer to the memory area if it
has been successfully allocated, and {\tt NULL} else.
\end{itemize}

\subsubsection{{\tt SCOTCH\_meshBuild}}

\begin{itemize}
\progsyn

{\tt\begin{tabular}{l@{}ll}
int SCOTCH\_meshBuild ( & SCOTCH\_Mesh *       & meshptr, \\
                        & const SCOTCH\_Num   & velmbas, \\
                        & const SCOTCH\_Num   & vnodbas, \\
                        & const SCOTCH\_Num   & velmnbr, \\
                        & const SCOTCH\_Num   & vnodnbr, \\
                        & const SCOTCH\_Num * & verttab, \\
                        & const SCOTCH\_Num * & vendtab, \\
                        & const SCOTCH\_Num * & velotab, \\
                        & const SCOTCH\_Num * & vnlotab, \\
                        & const SCOTCH\_Num * & vlbltab, \\
                        & const SCOTCH\_Num   & edgenbr, \\
                        & const SCOTCH\_Num * & edgetab)
\end{tabular}}

{\tt\begin{tabular}{l@{}ll}
scotchfmeshbuild ( & doubleprecision (*)   & meshdat, \\
                   & integer*{\it num}     & velmbas, \\
                   & integer*{\it num}     & vnodbas, \\
                   & integer*{\it num}     & velmnbr, \\
                   & integer*{\it num}     & vnodnbr, \\
                   & integer*{\it num} (*) & verttab, \\
                   & integer*{\it num} (*) & vendtab, \\
                   & integer*{\it num} (*) & velotab, \\
                   & integer*{\it num} (*) & vnlotab, \\
                   & integer*{\it num} (*) & vlbltab, \\
                   & integer*{\it num}     & edgenbr, \\
                   & integer*{\it num} (*) & edgetab, \\
                   & integer*{\it num}     & ierr)
\end{tabular}}

\progdes

The {\tt SCOTCH\_meshBuild} routine fills the source mesh structure
pointed to by {\tt meshptr} with all of the data that is passed to it.

{\tt velmbas} and {\tt vnodbas} are the base values for the element
and node vertices, respectively.
{\tt velmnbr} and {\tt vnodnbr} are the number of element and node
vertices, respectively, such that either $\mathtt{velmbas}
+\mathtt{velmnbr}=\mathtt{vnodnbr}$ or $\mathtt{vnodbas}
+\mathtt{vnodnbr}=\mathtt{velmnbr}$ holds, and typically
$\min(\mathtt{velmbas}, \mathtt{vnodbas})$ is $0$ for
structures built from C and $1$ for structures built from Fortran.
{\tt verttab} is the adjacency index array, of size $({\tt velmnbr} +
{\tt vnodnbr} + 1)$ if the edge array is compact (that is, if
{\tt vendtab} equals $\mathtt{vendtab}+1$ or {\tt NULL}), or of
size $({\tt velmnbr} + {\tt vnodnbr}1)$ else.
{\tt vendtab} is the adjacency end index array, of size $({\tt velmnbr} +
{\tt vnodnbr})$ if it is disjoint from {\tt verttab}.
{\tt velotab} is the element vertex load array, of size {\tt velmnbr}
if it exists.
{\tt vnlotab} is the node vertex load array, of size {\tt vnodnbr} if
it exists.
{\tt vlbltab} is the vertex label array, of size $({\tt velmnbr} +
{\tt vnodnbr})$ if it exists.
{\tt edgenbr} is the number of arcs (that is, twice the number of edges).
{\tt edgetab} is the adjacency array, of size at least {\tt edgenbr}
(it can be more if the edge array is not compact).

The {\tt vendtab}, {\tt velotab}, {\tt vnlotab} and {\tt vlbltab}
arrays are optional, and a {\tt NULL} pointer can be passed as
argument whenever they are not defined.
Since, in Fortran, there is no null reference, passing the
{\tt scotchf\lbt mesh\lbt build} routine a reference equal to
{\tt verttab} in the {\tt velotab}, {\tt vnlotab} or {\tt vlbltab}
fields makes them be considered as missing arrays. Setting
{\tt vendtab} to refer to one cell after {\tt verttab} yields the
same result, as it is the exact semantics of a compact vertex array.

To limit memory consumption, {\tt SCOTCH\_\lbt mesh\lbo Build} does
not copy array data, but instead references them in the {\tt
SCOTCH\_\lbt Mesh} structure. Therefore, great care should be taken
not to modify the contents of the arrays passed to {\tt SCOTCH\_\lbt
mesh\lbo Build} as long as the mesh structure is in use. Every
update of the arrays should be preceded by a call to {\tt SCOTCH\_\lbt
mesh\lbo Exit}, to free internal mesh structures, and eventually
followed by a new call to {\tt SCOTCH\_\lbt mesh\lbo Build} to
re-build these internal structures so as to be able to use the new
mesh.

To ensure that inconsistencies in user data do not result in an
erroneous behavior of the \libscotch\ routines, it is recommended, at
least in the development stage, to call the {\tt SCOTCH\_\lbt mesh\lbt Check}
routine on the newly created {\tt SCOTCH\_\lbt Mesh} structure, prior
to any other calls to \libscotch\ routines.

\progret

{\tt SCOTCH\_meshBuild} returns $0$ if the mesh structure has been
successfully set with all of the input data, and $1$ else.
\end{itemize}

\subsubsection{{\tt SCOTCH\_meshCheck}}

\begin{itemize}
\progsyn

{\tt\begin{tabular}{l@{}ll}
int SCOTCH\_meshCheck ( & const SCOTCH\_Mesh * & meshptr)
\end{tabular}}

{\tt\begin{tabular}{l@{}ll}
scotchfmeshcheck ( & doubleprecision (*) & meshdat, \\
                   & integer             & ierr)
\end{tabular}}

\progdes

The {\tt SCOTCH\_meshCheck} routine checks the consistency of the
given {\tt SCOTCH\_\lbt Mesh} structure. It can be used in client
applications to determine if a mesh that has been created from
used-generated data by means of the {\tt SCOTCH\_\lbt mesh\lbt Build}
routine is consistent, prior to calling any other routines of the
\libscotch\ library.

\progret

{\tt SCOTCH\_meshCheck} returns $0$ if mesh data are consistent, and
$1$ else.

\end{itemize}

\subsubsection{{\tt SCOTCH\_meshData}}
\label{sec-lib-func-meshdata}

\begin{itemize}
\progsyn

{\tt\begin{tabular}{l@{}ll}
void SCOTCH\_meshData ( & const SCOTCH\_Mesh * & meshptr, \\
                        & SCOTCH\_Num *        & vebaptr, \\
                        & SCOTCH\_Num *        & vnbaptr, \\
                        & SCOTCH\_Num *        & velmptr, \\
                        & SCOTCH\_Num *        & vnodptr, \\
                        & SCOTCH\_Num **       & verttab, \\
                        & SCOTCH\_Num **       & vendtab, \\
                        & SCOTCH\_Num **       & velotab, \\
                        & SCOTCH\_Num **       & vnlotab, \\
                        & SCOTCH\_Num **       & vlbltab, \\
                        & SCOTCH\_Num *        & edgeptr, \\
                        & SCOTCH\_Num **       & edgetab, \\
                        & SCOTCH\_Num *        & degrptr)
\end{tabular}}

{\tt\begin{tabular}{l@{}ll}
scotchfmeshdata ( & doubleprecision (*)   & meshdat, \\
                  & integer*{\it num} (*) & indxtab, \\
                  & integer*{\it num}     & velobas, \\
                  & integer*{\it num}     & vnlobas, \\
                  & integer*{\it num}     & velmnbr, \\
                  & integer*{\it num}     & vnodnbr, \\
                  & integer*{\it idx}     & vertidx, \\
                  & integer*{\it idx}     & vendidx, \\
                  & integer*{\it idx}     & veloidx, \\
                  & integer*{\it idx}     & vnloidx, \\
                  & integer*{\it idx}     & vlblidx, \\
                  & integer*{\it num}     & edgenbr, \\
                  & integer*{\it idx}     & edgeidx, \\
                  & integer*{\it num}     & degrmax)
\end{tabular}}

\progdes

The {\tt SCOTCH\_meshData} routine is the dual of the
{\tt SCOTCH\_\lbt mesh\lbo Build} routine. It is a multiple
accessor that returns scalar values and array references.

{\tt vebaptr} and {\tt vnbaptr} are pointers to locations that
will hold the mesh base value for elements and nodes, respectively
(the minimum of these two values is typically $0$ for structures built
from C and $1$ for structures built from Fortran).
{\tt velmptr} and {\tt vnodptr} are pointers to locations that
will hold the number of element and node vertices, respectively.
{\tt verttab} is the pointer to a location that will hold the reference to
the adjacency index array, of size $(\mathtt{*velmptr} +
\mathtt{*vnodptr} + 1)$ if the adjacency array is compact, or
of size $(\mathtt{*velmptr} + \mathtt{*vnodptr})$ else.
{\tt vendtab} is the pointer to a location that will hold the
reference to the adjacency end index array, and is equal to
$\mathtt{verttab} + 1$ if the adjacency array is compact.
{\tt velotab} and {\tt vnlotab} are pointers to locations that will
hold the reference to the element and node vertex load arrays, of
sizes {\tt *velmptr} and {\tt *vnodptr}, respectively.
{\tt vlbltab} is the pointer to a location that will hold the reference to
the vertex label array, of size $(\mathtt{*velmptr} + \mathtt{*vnodptr})$.
{\tt edgeptr} is the pointer to a location that will hold the number of arcs
(that is, twice the number of edges).
{\tt edgetab} is the pointer to a location that will hold the reference to
the adjacency array, of size at least {\tt edgenbr}.
{\tt degrptr} is the pointer to a location that will hold the maximum
vertex degree computed across all element and node vertices.

Any of these pointers can be set to {\tt NULL} on input if the
corresponding information is not needed. Else, the reference to a
dummy area can be provided, where all unwanted data will be written.

Since there are no pointers in Fortran, a specific mechanism is used
to allow users to access mesh arrays. The {\tt scotchf\lbt mesh\lbt
data} routine is passed an integer array, the first element of which is used
as a base address from which all other array indices are
computed. Therefore, instead of returning references, the routine
returns integers, which represent the starting index of each of the
relevant arrays with respect to the base input array, or {\tt
vertidx}, the index of {\tt verttab}, if they do not exist. For
instance, if some base array {\tt myarray\lbt (1)} is passed as
parameter {\tt indxtab}, then the first cell of array {\tt verttab}
will be accessible as {\tt myarray\lbt (vertidx)}.
In order for this feature to behave properly, the {\tt indxtab}
array must be word-aligned with the mesh arrays. This is
automatically enforced on most systems, but some care should be
taken on systems that allow one to access data that is not
word-aligned. On such systems, declaring the array after a
dummy {\tt double\lbt precision} array can coerce the compiler
into enforcing the proper alignment. Also, on 32\_64 architectures,
such indices can be larger than the size of a regular
{\tt INTEGER}. This is why the indices to be returned are defined by
means of a specific integer type. See
Section~\ref{sec-lib-inttypesize} for more information on this
issue.
\end{itemize}

\subsubsection{{\tt SCOTCH\_meshExit}}

\begin{itemize}
\progsyn

{\tt\begin{tabular}{l@{}ll}
void SCOTCH\_meshExit ( & SCOTCH\_Mesh * & meshptr)
\end{tabular}}

{\tt\begin{tabular}{l@{}ll}
scotchfmeshexit ( & doubleprecision (*) & meshdat)
\end{tabular}}

\progdes

The {\tt SCOTCH\_meshExit} function frees the contents of a
{\tt SCOTCH\_\lbt Mesh} structure previously initialized by
{\tt SCOTCH\_\lbt meshInit}. All subsequent calls to
{\tt SCOTCH\_\lbt mesh*} routines other than {\tt SCOTCH\_\lbt
meshInit}, using this structure as parameter, may yield
unpredictable results.
\end{itemize}

\subsubsection{{\tt SCOTCH\_meshGraph}}

\begin{itemize}
\progsyn

{\tt\begin{tabular}{l@{}ll}
int SCOTCH\_meshGraph ( & const SCOTCH\_Mesh * & meshptr, \\
                        & SCOTCH\_Graph *      & grafptr)
\end{tabular}}

{\tt\begin{tabular}{l@{}ll}
scotchfmeshgraph ( & doubleprecision (*) & meshdat, \\
                   & doubleprecision (*) & grafdat, \\
                   & integer             & ierr)
\end{tabular}}

\progdes

The {\tt SCOTCH\_meshGraph} routine builds a graph from a mesh. It
creates in the {\tt SCOTCH\_\lbt Graph} structure pointed to by {\tt
grafptr} a graph having as many vertices as there are nodes in the
{\tt SCOTCH\_\lbt Mesh} structure pointed to by {\tt meshptr}, and
where there is an edge between any two graph vertices if and only if
there exists in the mesh an element containing both of the associated
nodes. Consequently, all of the elements of the mesh are turned into
cliques in the resulting graph.

In order to save memory space as well as computation time, in the
current implementation of {\tt SCOTCH\_meshGraph}, some mesh
arrays are shared with the graph structure. Therefore, one should make
sure that the graph must no longer be used after the mesh structure
is freed. The graph structure can be freed before or after the mesh
structure, but must not be used after the mesh structure is freed.

\progret

{\tt SCOTCH\_meshGraph} returns $0$ if the graph structure has been
successfully allocated and filled, and $1$ else.
\end{itemize}

\subsubsection{{\tt SCOTCH\_meshInit}}
\label{sec-lib-func-meshinit}

\begin{itemize}
\progsyn

{\tt\begin{tabular}{l@{}ll}
int SCOTCH\_meshInit ( & SCOTCH\_Mesh * & meshptr)
\end{tabular}}

{\tt\begin{tabular}{l@{}ll}
scotchfmeshinit ( & doubleprecision (*) & meshdat, \\
                  & integer             & ierr)
\end{tabular}}

\progdes

The {\tt SCOTCH\_meshInit} function initializes a {\tt SCOTCH\_\lbt
Mesh} structure so as to make it suitable for future operations. It
should be the first function to be called upon a {\tt SCOTCH\_\lbt
Mesh} structure. When the mesh data is no longer of use, call
function {\tt SCOTCH\_\lbt mesh\lbt Exit} to free its internal
structures.

\progret

{\tt SCOTCH\_meshInit} returns $0$ if the mesh structure has been
successfully initialized, and $1$ else.
\end{itemize}

\subsubsection{{\tt SCOTCH\_meshLoad}}
\label{sec-lib-func-meshload}

\begin{itemize}
\progsyn

{\tt\begin{tabular}{l@{}ll}
int SCOTCH\_meshLoad ( & SCOTCH\_Mesh * & meshptr, \\
                       & FILE *         & stream, \\
                       & SCOTCH\_Num    & baseval)
\end{tabular}}

{\tt\begin{tabular}{l@{}ll}
scotchfmeshload ( & doubleprecision (*) & meshdat, \\
                  & integer             & fildes,  \\
                  & integer*{\it num}   & baseval, \\
                  & integer             & ierr)
\end{tabular}}

\progdes

The {\tt SCOTCH\_meshLoad} routine fills the {\tt
SCOTCH\_\lbt Mesh} structure pointed to by {\tt meshptr} with the
source mesh description available from stream {\tt stream} in the
\scotch\ mesh format (see section~\ref{sec-file-smesh}).

To ease the handling of source mesh files by programs written in C as
well as in Fortran, The base value of the mesh to read can be set
to {\tt 0} or {\tt 1}, by setting the {\tt baseval} parameter to the
proper value. A value of {\tt -1} indicates that the mesh base should
be the same as the one provided in the mesh description that is read
from {\tt stream}.

Fortran users must use the {\tt PXFFILENO} or {\tt FNUM} functions to
obtain the number of the Unix file descriptor {\tt fildes} associated
with the logical unit of the mesh file.

\progret

{\tt SCOTCH\_meshLoad} returns $0$ if the mesh structure has been
successfully allocated and filled with the data read, and $1$ else.
\end{itemize}

\subsubsection{{\tt SCOTCH\_meshSave}}

\begin{itemize}
\progsyn

{\tt\begin{tabular}{l@{}ll}
int SCOTCH\_meshSave ( & const SCOTCH\_Mesh * & meshptr, \\
                       & FILE *               & stream)
\end{tabular}}

{\tt\begin{tabular}{l@{}ll}
scotchfmeshsave ( & doubleprecision (*) & meshdat, \\
                  & integer             & fildes,  \\
                  & integer             & ierr)
\end{tabular}}

\progdes

The {\tt SCOTCH\_meshSave} routine saves the contents of the {\tt
SCOTCH\_\lbt Mesh} structure pointed to by {\tt meshptr} to stream
{\tt stream}, in the \scotch\ mesh format (see
section~\ref{sec-file-smesh}).

Fortran users must use the {\tt PXFFILENO} or {\tt FNUM} functions to
obtain the number of the Unix file descriptor {\tt fildes} associated
with the logical unit of the mesh file.

\progret

{\tt SCOTCH\_meshSave} returns $0$ if the mesh structure has been
successfully written to {\tt stream}, and $1$ else.
\end{itemize}

\subsubsection{{\tt SCOTCH\_meshSize}}

\begin{itemize}
\progsyn

{\tt\begin{tabular}{l@{}ll}
void SCOTCH\_meshSize ( & const SCOTCH\_Mesh * & meshptr, \\
                        & SCOTCH\_Num *        & velmptr, \\
                        & SCOTCH\_Num *        & vnodptr, \\
                        & SCOTCH\_Num *        & edgeptr)
\end{tabular}}

{\tt\begin{tabular}{l@{}ll}
scotchfmeshsize ( & doubleprecision (*) & meshdat, \\
                  & integer*{\it num}   & velmnbr, \\
                  & integer*{\it num}   & vnodnbr, \\
                  & integer*{\it num}   & edgenbr)
\end{tabular}}

\progdes

The {\tt SCOTCH\_meshSize} routine fills the three areas of type
{\tt SCOTCH\_\lbt Num} pointed to by {\tt velmptr}, {\tt vnodptr}
and {\tt edgeptr} with the number of element vertices, node
vertices and arcs (that is, twice the number of edges) of the
given mesh pointed to by {\tt meshptr}, respectively.

Any of these pointers can be set to {\tt NULL} on input if the
corresponding information is not needed. Else, the reference to a
dummy area can be provided, where all unwanted data will be written.

This routine is useful to get the size of a mesh read by means
of the {\tt SCOTCH\_\lbt mesh\lbo Load} routine, in order to allocate
auxiliary arrays of proper sizes. If the whole structure of the
mesh is wanted, function {\tt SCOTCH\_mesh\lbo Data} should be
preferred.
\end{itemize}

\subsubsection{{\tt SCOTCH\_meshStat}}

\begin{itemize}
\progsyn

{\tt\begin{tabular}{l@{}ll}
void SCOTCH\_meshStat ( & const SCOTCH\_Mesh * & meshptr, \\
                        & SCOTCH\_Num *        & vnlominptr, \\
                        & SCOTCH\_Num *        & vnlomaxptr, \\
                        & SCOTCH\_Num *        & vnlosumptr, \\
                        & double *             & vnloavgptr, \\
                        & double *             & vnlodltptr, \\
                        & SCOTCH\_Num *        & edegminptr, \\
                        & SCOTCH\_Num *        & edegmaxptr, \\
                        & double *             & edegavgptr, \\
                        & double *             & edegdltptr, \\
                        & SCOTCH\_Num *        & ndegminptr, \\
                        & SCOTCH\_Num *        & ndegmaxptr, \\
                        & double *             & ndegavgptr, \\
                        & double *             & ndegdltptr)
\end{tabular}}

{\tt\begin{tabular}{l@{}ll}
scotchfmeshstat ( & doubleprecision (*) & meshdat, \\
                  & integer*{\it num}   & vnlomin, \\
                  & integer*{\it num}   & vnlomax, \\
                  & integer*{\it num}   & vnlosum, \\
                  & doubleprecision     & vnloavg, \\
                  & doubleprecision     & vnlodlt, \\
                  & integer*{\it num}   & edegmin, \\
                  & integer*{\it num}   & edegmax, \\
                  & doubleprecision     & edegavg, \\
                  & doubleprecision     & edegdlt, \\
                  & integer*{\it num}   & ndegmin, \\
                  & integer*{\it num}   & ndegmax, \\
                  & doubleprecision     & ndegavg, \\
                  & doubleprecision     & ndegdlt)
\end{tabular}}

\progdes

The {\tt SCOTCH\_meshStat} routine produces some statistics regarding
the mesh structure pointed to by {\tt meshptr}.
{\tt vnlomin}, {\tt vnlomax}, {\tt vnlosum}, {\tt vnloavg} and
{\tt vnlodlt} are the minimum node vertex load, the maximum node
vertex load, the sum of all node vertex loads, the average node
vertex load, and the variance of the node vertex loads, respectively.
{\tt edegmin}, {\tt edegmax}, {\tt edegavg} and
{\tt edegdlt} are the minimum element vertex degree, the maximum
element vertex degree, the average element vertex degree, and the
variance of the element vertex degrees, respectively.
{\tt ndegmin}, {\tt ndegmax}, {\tt ndegavg} and
{\tt ndegdlt} are the minimum element vertex degree, the maximum
element vertex degree, the average element vertex degree, and the
variance of the element vertex degrees, respectively.
\end{itemize}

\subsection{High-level mesh ordering routines}

This routine provides high-level functionality and frees the
user from the burden of calling in sequence several of the low-level
routines described afterward.

\subsubsection{{\tt SCOTCH\_meshOrder}}

\begin{itemize}
\progsyn

{\tt\begin{tabular}{l@{}ll}
int SCOTCH\_meshOrder ( & const SCOTCH\_Mesh *  & meshptr, \\
                        & const SCOTCH\_Strat * & straptr, \\
                        & SCOTCH\_Num *         & permtab, \\
                        & SCOTCH\_Num *         & peritab, \\
                        & SCOTCH\_Num *         & cblkptr, \\
                        & SCOTCH\_Num *         & rangtab, \\
                        & SCOTCH\_Num *         & treetab)
\end{tabular}}

{\tt\begin{tabular}{l@{}ll}
scotchfmeshorder ( & doubleprecision (*)   & meshdat, \\
                   & doubleprecision (*)   & stradat, \\
                   & integer*{\it num} (*) & permtab, \\
                   & integer*{\it num} (*) & peritab, \\
                   & integer*{\it num}     & cblknbr, \\
                   & integer*{\it num} (*) & rangtab, \\
                   & integer*{\it num} (*) & treetab, \\
                   & integer               & ierr)
\end{tabular}}

\progdes

The {\tt SCOTCH\_meshOrder} routine computes a block ordering of the
unknowns of the symmetric sparse matrix the adjacency structure of
which is represented by the elements that connect the nodes of the
source mesh structure pointed to by {\tt meshptr}, using the ordering
strategy pointed to by {\tt stratptr}, and returns ordering data in
the scalar pointed to by {\tt cblkptr} and the four arrays
{\tt permtab}, {\tt peritab}, {\tt rangtab} and {\tt treetab}.

The {\tt permtab}, {\tt peritab}, {\tt rangtab} and {\tt treetab}
arrays should have been previously allocated, of a size sufficient to
hold as many {\tt SCOTCH\_\lbt Num} integers as there are node
vertices in the source mesh, plus one in the case of {\tt rangtab}.
Any of the five output fields can be set to {\tt NULL} if the
corresponding information is not needed. Since, in Fortran, there is
no null reference, passing a reference to {\tt meshptr} in these
fields will have the same effect.

On return, {\tt permtab} holds the direct permutation of the unknowns,
that is, node vertex $i$ of the original mesh has index {\tt permtab[$i$]}
in the reordered mesh, while {\tt peritab} holds the inverse
permutation, that is, node vertex $i$ in the reordered mesh had index
{\tt peritab[$i$]} in the original mesh. All of these indices are numbered
according to the base value of the source mesh: permutation indices
are numbered from $\min(\mathtt{velmbas},\mathtt{vnodbas})$ to
$\mathtt{vnodnbr} + \min(\mathtt{velmbas},\mathtt{vnodbas}) -
1$, that is, from $0$ to $\mathtt{vnodnbr} - 1$ if the mesh base is
$0$, and from $1$ to $\mathtt{vnodnbr}$ if the mesh base is $1$.
The base value for mesh orderings is taken as
$\min(\mathtt{velmbas},\mathtt{vnodbas})$, and not just as
{\tt vnodbas}, such that orderings that are computed on some mesh
have exactly the same index range as orderings that would be computed
on the graph obtained from the original mesh by means of the
{\tt SCOTCH\_\lbt mesh\lbt Graph} routine.

The three other result fields, {\tt *cblkptr}, {\tt rangtab} and {\tt
treetab}, contain data related to the block structure. {\tt *cblkptr}
holds the number of column blocks of the produced ordering, and {\tt
rangtab} holds the starting indices of each of the permuted column
blocks, in increasing order, so that column block $i$ starts at index
{\tt rangtab\lbt [$i$]} and ends at index $(\mbox{\tt
rangtab}\lbt\mathtt{[}i + 1\mathtt{]} - 1)$, inclusive, in the new
ordering. {\tt treetab} holds the separators tree structure, that is,
{\tt treetab[$i$]} is the index of the father of column block $i$ in
the separators tree, or $-1$ if column block $i$ is the root of the
separators tree. Please refer to Section~\ref{sec-lib-format-order}
for more information.

\progret

{\tt SCOTCH\_meshOrder} returns $0$ if the ordering of the mesh has
been successfully computed, and $1$ else. In this last case, the
{\tt rangtab}, {\tt permtab}, and {\tt peritab} arrays may however have
been partially or completely filled, but their contents are not significant.
\end{itemize}

\subsection{Low-level mesh ordering routines}

All of the following routines operate on a {\tt SCOTCH\_\lbt Ordering}
structure that contains references to the permutation arrays to be
filled during the mesh ordering process.

\subsubsection{{\tt SCOTCH\_meshOrderCheck}}

\begin{itemize}
\progsyn

{\tt\begin{tabular}{l@{}ll}
int SCOTCH\_meshOrderCheck ( & const SCOTCH\_Mesh *     & meshptr, \\
                             & const SCOTCH\_Ordering * & ordeptr)
\end{tabular}}

{\tt\begin{tabular}{l@{}ll}
scotchfmeshordercheck ( & doubleprecision (*) & meshdat, \\
                        & doubleprecision (*) & ordedat, \\
                        & integer             & ierr)
\end{tabular}}

\progdes

The {\tt SCOTCH\_meshOrderCheck} routine checks the consistency of
the given {\tt SCOTCH\_\lbt Ordering} structure pointed to by {\tt ordeptr}.

\progret

{\tt SCOTCH\_meshOrderCheck} returns $0$ if ordering data are consistent, and
$1$ else.

\end{itemize}

\subsubsection{{\tt SCOTCH\_meshOrderCompute}}

\begin{itemize}
\progsyn

{\tt\begin{tabular}{l@{}ll}
int SCOTCH\_meshOrderCompute ( & const SCOTCH\_Mesh *  & meshptr, \\
                               & SCOTCH\_Ordering *    & ordeptr, \\
                               & const SCOTCH\_Strat * & straptr)
\end{tabular}}

{\tt\begin{tabular}{l@{}ll}
scotchfmeshordercompute ( & doubleprecision (*) & meshdat, \\
                          & doubleprecision (*) & ordedat, \\
                          & doubleprecision (*) & stradat, \\
                          & integer             & ierr)
\end{tabular}}

\progdes

The {\tt SCOTCH\_meshOrderCompute} routine computes a block ordering
of the mesh structure pointed to by {\tt grafptr}, using the mapping
strategy pointed to by {\tt stratptr}, and stores its result in the
ordering structure pointed to by {\tt ordeptr}.

On return, the ordering structure holds a block ordering of the
given mesh (see section~\ref{sec-lib-mesh-order-init} for a
description of the ordering fields).

\progret

{\tt SCOTCH\_meshOrderCompute} returns $0$ if the ordering has been
successfully computed, and $1$ else. In this latter case, the ordering
arrays may however have been partially or completely filled, but their
contents are not significant.
\end{itemize}

\subsubsection{{\tt SCOTCH\_meshOrderExit}}

\begin{itemize}
\progsyn

{\tt\begin{tabular}{l@{}ll}
void SCOTCH\_meshOrderExit ( & const SCOTCH\_Mesh * & meshptr, \\
                             & SCOTCH\_Ordering *   & ordeptr)
\end{tabular}}

{\tt\begin{tabular}{l@{}ll}
scotchfmeshorderexit ( & doubleprecision (*) & meshdat, \\
                       & doubleprecision (*) & ordedat)
\end{tabular}}

\progdes

The {\tt SCOTCH\_meshOrderExit} function frees the contents of a
{\tt SCOTCH\_\lbt Ordering} structure previously initialized by
{\tt SCOTCH\_\lbt mesh\lbt Order\lbt Init}. All subsequent calls to
{\tt SCOTCH\_\lbt mesh\lbt Order*} routines other than
{\tt SCOTCH\_\lbt mesh\lbt Order\lbt Init}, using this structure
as parameter, may yield unpredictable results.
\end{itemize}

\subsubsection{{\tt SCOTCH\_meshOrderInit}}
\label{sec-lib-mesh-order-init}

\begin{itemize}
\progsyn

{\tt\begin{tabular}{l@{}ll}
int SCOTCH\_meshOrderInit ( & const SCOTCH\_Mesh * & meshptr, \\
                            & SCOTCH\_Ordering *   & ordeptr, \\
                            & SCOTCH\_Num *        & permtab, \\
                            & SCOTCH\_Num *        & peritab, \\
                            & SCOTCH\_Num *        & cblkptr, \\
                            & SCOTCH\_Num *        & rangtab, \\
                            & SCOTCH\_Num *        & treetab)
\end{tabular}}

{\tt\begin{tabular}{l@{}ll}
scotchfmeshorderinit ( & doubleprecision (*)   & meshdat, \\
                       & doubleprecision (*)   & ordedat, \\
                       & integer*{\it num} (*) & permtab, \\
                       & integer*{\it num} (*) & peritab, \\
                       & integer*{\it num}     & cblknbr, \\
                       & integer*{\it num} (*) & rangtab, \\
                       & integer*{\it num} (*) & treetab, \\
                       & integer               & ierr)
\end{tabular}}

\progdes

The {\tt SCOTCH\_mesh\lbt Order\lbt Init} routine fills the ordering
structure pointed to by {\tt ordeptr} with all of the data that are
passed to it. Thus, all subsequent calls to ordering routines such as
{\tt SCOTCH\_\lbt mesh\lbt Order\lbt Compute}, using this ordering
structure as parameter, will place ordering results in fields {\tt
permtab}, {\tt peritab}, {\tt *cblkptr}, {\tt rangtab} or {\tt
treetab}, if they are not set to {\tt NULL}.

{\tt permtab} is the ordering permutation array, of size ${\tt
vnodnbr}$, {\tt peritab} is the inverse ordering permutation array,
of size ${\tt vnodnbr}$, {\tt cblkptr} is the pointer to a
{\tt SCOTCH\_\lbt Num} that will receive the number of produced
column blocks, {\tt rangtab} is the array that holds the column block
span information, of size $\mathtt{vnodnbr} + 1$, and {\tt treetab}
is the array holding the structure of the separators tree, of size
${\tt vnodnbr}$. See the above manual page of
{\tt SCOTCH\_mesh\lbt Order}, as well as
section~\ref{sec-lib-format-order}, for an explanation of the
semantics of all of these fields.

The {\tt SCOTCH\_\lbt mesh\lbt Order\lbt Init} routine should be the
first function to be called upon a {\tt SCOTCH\_\lbt Ordering}
structure for ordering meshes. When the ordering structure is no
longer of use, the {\tt SCOTCH\_\lbt mesh\lbt Order\lbt Exit}
function must be called, in order to to free its internal structures.

\progret

{\tt SCOTCH\_meshOrderInit} returns $0$ if the ordering structure has
been successfully initialized, and $1$ else.
\end{itemize}

\subsubsection{{\tt SCOTCH\_meshOrderSave}}

\begin{itemize}
\progsyn

{\tt\begin{tabular}{l@{}ll}
int SCOTCH\_meshOrderSave ( & const SCOTCH\_Mesh *     & meshptr, \\
                            & const SCOTCH\_Ordering * & ordeptr, \\
                            & FILE *                   & stream)
\end{tabular}}

{\tt\begin{tabular}{l@{}ll}
scotchfmeshordersave ( & doubleprecision (*) & meshdat, \\
                       & doubleprecision (*) & ordedat, \\
                       & integer             & fildes,  \\
                       & integer             & ierr)
\end{tabular}}

\progdes

The {\tt SCOTCH\_meshOrderSave} routine saves the contents of the {\tt
SCOTCH\_\lbt Ordering} structure pointed to by {\tt ordeptr} to stream
{\tt stream}, in the \scotch\ ordering format (see
section~\ref{sec-file-ord}).

\progret

{\tt SCOTCH\_meshOrderSave} returns $0$ if the ordering structure
has been successfully written to {\tt stream}, and $1$ else.
\end{itemize}

\subsubsection{{\tt SCOTCH\_meshOrderSaveMap}}

\begin{itemize}
\progsyn

{\tt\begin{tabular}{l@{}ll}
int SCOTCH\_meshOrderSaveMap ( & const SCOTCH\_Mesh *     & meshptr, \\
                               & const SCOTCH\_Ordering * & ordeptr, \\
                               & FILE *                   & stream)
\end{tabular}}

{\tt\begin{tabular}{l@{}ll}
scotchfmeshordersavemap ( & doubleprecision (*) & meshdat, \\
                          & doubleprecision (*) & ordedat, \\
                          & integer             & fildes, \\
                          & integer             & ierr)
\end{tabular}}

\progdes

The {\tt SCOTCH\_meshOrderSaveMap} routine saves the block
partitioning data associated with the {\tt SCOTCH\_\lbt Ordering}
structure pointed to by {\tt ordeptr} to stream {\tt stream},
in the \scotch\ mapping format (see section~\ref{sec-file-map}).
A target domain number is associated with every block, such that
all node vertices belonging to the same block are shown as belonging
to the same target vertex.

This mapping file can then be used by the {\tt gout} program
(see section~\ref{sec-prog-gout}) to produce pictures showing
the different separators and blocks. Since {\tt gout} only takes
graphs as input, the mesh has to be converted into a graph by
means of the {\tt gmk\_\lbt msh} program (see
section~\ref{sec-prog-gmkmsh}).

\progret

{\tt SCOTCH\_meshOrderSaveMap} returns $0$ if the ordering structure
has been successfully written to {\tt stream}, and $1$ else.
\end{itemize}

\subsubsection{{\tt SCOTCH\_meshOrderSaveTree}}

\begin{itemize}
\progsyn

{\tt\begin{tabular}{l@{}ll}
int SCOTCH\_meshOrderSaveTree ( & const SCOTCH\_Mesh *     & meshptr, \\
                                & const SCOTCH\_Ordering * & ordeptr, \\
                                & FILE *                   & stream)
\end{tabular}}

{\tt\begin{tabular}{l@{}ll}
scotchfmeshordersavetree ( & doubleprecision (*) & meshdat, \\
                           & doubleprecision (*) & ordedat, \\
                           & integer             & fildes,  \\
                           & integer             & ierr)
\end{tabular}}

\progdes

The {\tt SCOTCH\_meshOrderSaveTree} routine saves the tree
hierarchy information associated with the {\tt SCOTCH\_\lbt Ordering}
structure pointed to by {\tt ordeptr} to stream {\tt stream}.

The format of the tree output file resembles the one of a mapping or
ordering file: it is made up of as many lines as there are node
vertices in the ordering. Each of these lines holds two integer
numbers. The first one is the index or the label of the node vertex,
starting from {\tt baseval}, and the second one is the index of its
parent node in the separators tree, or $-1$ if the vertex belongs to a
root node.

Fortran users must use the {\tt PXFFILENO} or {\tt FNUM} functions to
obtain the number of the Unix file descriptor {\tt fildes} associated
with the logical unit of the tree mapping file.

\progret

{\tt SCOTCH\_meshOrderSaveTree} returns $0$ if the separators tree
structure has been successfully written to {\tt stream}, and $1$ else.
\end{itemize}

\subsection{Strategy handling routines}
\label{sec-lib-strat}

\subsubsection{{\tt SCOTCH\_stratAlloc}}

\begin{itemize}
\progsyn

{\tt\begin{tabular}{l@{}l}
SCOTCH\_Strat * SCOTCH\_stratAlloc ( & void)
\end{tabular}}

\progdes

The {\tt SCOTCH\_stratAlloc} function allocates a memory area of a
size sufficient to store a {\tt SCOTCH\_\lbt Strat} structure. It is
the user's responsibility to free this memory when it is no longer
needed, using the {\tt SCOTCH\_\lbt mem\lbt Free} routine. The
allocated space must be initialized before use, by means of the
{\tt SCOTCH\_\lbt strat\lbt Init} routine.

\progret

{\tt SCOTCH\_stratAlloc} returns the pointer to the memory area if it
has been successfully allocated, and {\tt NULL} else.
\end{itemize}

\subsubsection{{\tt SCOTCH\_stratExit}}

\begin{itemize}
\progsyn

{\tt\begin{tabular}{l@{}ll}
void SCOTCH\_stratExit ( & SCOTCH\_Strat * & archptr)
\end{tabular}}

{\tt\begin{tabular}{l@{}ll}
scotchfstratexit ( & doubleprecision (*) & stradat)
\end{tabular}}

\progdes

The {\tt SCOTCH\_stratExit} function frees the contents of a
{\tt SCOTCH\_\lbt Strat} structure previously initialized by
{\tt SCOTCH\_\lbt strat\lbt Init}. All subsequent calls to
{\tt SCOTCH\_\lbt strat} routines other than {\tt SCOTCH\_\lbt
strat\lbt Init}, using this structure as parameter, may yield
unpredictable results.
\end{itemize}

\subsubsection{{\tt SCOTCH\_stratInit}}

\begin{itemize}
\progsyn

{\tt\begin{tabular}{l@{}ll}
int SCOTCH\_stratInit ( & SCOTCH\_Strat * & straptr)
\end{tabular}}

{\tt\begin{tabular}{l@{}ll}
scotchfstratinit ( & doubleprecision (*) & stradat, \\
                   & integer             & ierr)
\end{tabular}}

\progdes

The {\tt SCOTCH\_stratInit} function initializes a {\tt SCOTCH\_\lbt
Strat} structure so as to make it suitable for future operations. It
should be the first function to be called upon a {\tt SCOTCH\_\lbt
Strat} structure. When the strategy data is no longer of use, call
function {\tt SCOTCH\_\lbt strat\lbt Exit} to free its internal
structures.

\progret

{\tt SCOTCH\_stratInit} returns $0$ if the strategy structure has been
successfully initialized, and $1$ else.
\end{itemize}

\subsubsection{{\tt SCOTCH\_stratSave}}

\begin{itemize}
\progsyn

{\tt\begin{tabular}{l@{}ll}
int SCOTCH\_stratSave ( & const SCOTCH\_Strat * & straptr, \\
                        & FILE *                & stream)
\end{tabular}}

{\tt\begin{tabular}{l@{}ll}
scotchfstratsave ( & doubleprecision (*) & stradat, \\
                   & integer             & fildes, \\
                   & integer             & ierr)
\end{tabular}}

\progdes

The {\tt SCOTCH\_stratSave} routine saves the contents of the {\tt
SCOTCH\_\lbt Strat} structure pointed to by {\tt straptr} to stream
{\tt stream}, in the form of a text string. The methods and
parameters of the strategy string depend on the type of the strategy,
that is, whether it is a bipartitioning, mapping, or ordering
strategy, and to which structure it applies, that is, graphs or
meshes.

Fortran users must use the {\tt PXFFILENO} or {\tt FNUM} functions to
obtain the number of the Unix file descriptor {\tt fildes} associated
with the logical unit of the output file.

\progret

{\tt SCOTCH\_stratSave} returns $0$ if the strategy string has been
successfully written to {\tt stream}, and $1$ else.
\end{itemize}

\subsection{Strategy creation routines}
\label{sec-lib-strat-creation}

Strategy creation routines parse the user-provided strategy string and
populate the given opaque strategy object with a tree-shaped structure
that represents the parsed expression. It is this structure that will
be later traversed by the generic routines for partitioning, mapping or
ordering, so as to determine which specific partitioning, mapping or
ordering method to be called on a subgraph being considered.

Because strategy creation routines call third-party lexical analyzers
that may have been implemented in a non-reentrant way, no guarantee is
given on the reentrance of these routines. Consequently, strategy
creation routines that might be called simultaneously by multiple
threads should be protected by a mutex.

\subsubsection{{\tt SCOTCH\_stratGraphBipart}}
\label{sec-lib-strat-graph-bipart}

\begin{itemize}
\progsyn

{\tt\begin{tabular}{l@{}ll}
int SCOTCH\_stratGraphBipart ( & SCOTCH\_Strat * & straptr, \\
                               & const char *    & string)
\end{tabular}}

{\tt\begin{tabular}{l@{}ll}
scotchfstratgraphbipart ( & doubleprecision (*) & stradat, \\
                          & character (*)       & string,  \\
                          & integer             & ierr)
\end{tabular}}

\progdes

The {\tt SCOTCH\_stratGraphBipart} routine fills the strategy structure
pointed to by {\tt straptr} with the graph bipartitioning strategy
string pointed to by {\tt string}. From this point, the strategy
structure can only be used as a graph bipartitioning strategy, to be
used by function {\tt SCOTCH\_\lbt arch\lbt Build}, for instance.

When using the C interface, the array of characters pointed to by
{\tt string} must be null-terminated.

\progret

{\tt SCOTCH\_stratGraphBipart} returns $0$ if the strategy string has
been successfully set, and $1$ else.
\end{itemize}

\subsubsection{{\tt SCOTCH\_stratGraphClusterBuild}}
\label{sec-lib-func-stratgraphclusterbuild}
\index{Clustering}

\begin{itemize}
\progsyn

{\tt\begin{tabular}{l@{}ll}
int SCOTCH\_stratGraphClusterBuild ( & SCOTCH\_Strat *   & straptr, \\
                                     & const SCOTCH\_Num & flagval, \\
                                     & const SCOTCH\_Num & pwgtmax, \\
                                     & const double      & densmin, \\
                                     & const double      & bbalval)
\end{tabular}}

{\tt\begin{tabular}{l@{}ll}
scotchfstratgraphclusterbuild ( & doubleprecision (*) & stradat, \\
                                & integer*{\it num}   & flagval, \\
                                & integer*{\it num}   & pwgtmax, \\
                                & doubleprecision     & densmin, \\
                                & doubleprecision     & bbalval, \\
                                & integer             & ierr)
\end{tabular}}

\progdes

The {\tt SCOTCH\_stratGraphClusterBuild} routine fills the strategy
structure pointed to by {\tt straptr} with a default clustering
strategy tuned according to the preference flags passed as
{\tt flagval}, the maximum cluster vertex weight {\tt pwgtmax},
the minimum edge density {\tt densmin}, and the bipartition imbalance
ratio {\tt bbalval}. From this point, the strategy structure
can only be used as a mapping strategy, to be used by a mapping
function such as {\tt SCOTCH\_\lbt graph\lbt Map}.

Recursive bipartitioning will be applied to the graph, every
bipartition allowing for an imbalance tolerance of {\tt bbalval}.
Recursion will stop if either cluster size becomes smaller than
{\tt pwgtmax}, or cluster edge density becomes higher than
{\tt densmin}, which represents the fraction of edges internal to the
cluster with respect to a complete graph. See
Section~\ref{sec-lib-format-strat-default} for a description of the
available flags.

\progret

{\tt SCOTCH\_stratGraphClusterBuild} returns $0$ if the strategy string
has been successfully set, and $1$ else.
\end{itemize}

\subsubsection{{\tt SCOTCH\_stratGraphMap}}

\begin{itemize}
\progsyn

{\tt\begin{tabular}{l@{}ll}
int SCOTCH\_stratGraphMap ( & SCOTCH\_Strat * & straptr, \\
                            & const char *    & string)
\end{tabular}}

{\tt\begin{tabular}{l@{}ll}
scotchfstratgraphmap ( & doubleprecision (*) & stradat, \\
                       & character (*)       & string,  \\
                       & integer             & ierr)
\end{tabular}}

\progdes

The {\tt SCOTCH\_stratGraphMap} routine fills the strategy
structure pointed to by {\tt straptr} with the graph mapping
strategy string pointed to by {\tt string}. From this point,
the strategy structure can only be used as a mapping strategy,
to be used by function {\tt SCOTCH\_\lbt graph\lbt Map},
for instance.

When using the C interface, the array of characters pointed to by
{\tt string} must be null-terminated.

\progret

{\tt SCOTCH\_stratGraphMap} returns $0$ if the strategy string
has been successfully set, and $1$ else.
\end{itemize}

\subsubsection{{\tt SCOTCH\_stratGraphMapBuild}}
\label{sec-lib-func-stratgraphmapbuild}

\begin{itemize}
\progsyn

{\tt\begin{tabular}{l@{}ll}
int SCOTCH\_stratGraphMapBuild ( & SCOTCH\_Strat *   & straptr, \\
                                 & const SCOTCH\_Num & flagval, \\
                                 & const SCOTCH\_Num & partnbr, \\
                                 & const double      & balrat)
\end{tabular}}

{\tt\begin{tabular}{l@{}ll}
scotchfstratgraphmapbuild ( & doubleprecision (*) & stradat, \\
                            & integer*{\it num}   & flagval, \\
                            & integer*{\it num}   & partnbr, \\
                            & doubleprecision     & balrat,  \\
                            & integer             & ierr)
\end{tabular}}

\progdes

The {\tt SCOTCH\_stratGraphMapBuild} routine fills the strategy
structure pointed to by {\tt straptr} with a default mapping strategy
tuned according to the preference flags passed as {\tt flagval} and
to the desired number of parts {\tt partnbr} and imbalance ratio
{\tt balrat}. From this point, the strategy structure can only be used
as a mapping strategy, to be used by function
{\tt SCOTCH\_\lbt graph\lbt Map}, for instance. See
Section~\ref{sec-lib-format-strat-default} for a description of the
available flags.

\progret

{\tt SCOTCH\_stratGraphMapBuild} returns $0$ if the strategy string
has been successfully set, and $1$ else.
\end{itemize}

\subsubsection{{\tt SCOTCH\_stratGraphPartOvl}}
\label{sec-lib-func-stratgraphpartovl}

\begin{itemize}
\progsyn

{\tt\begin{tabular}{l@{}ll}
int SCOTCH\_stratGraphPartOvl ( & SCOTCH\_Strat * & straptr, \\
                                & const char *    & string)
\end{tabular}}

{\tt\begin{tabular}{l@{}ll}
scotchfstratgraphpartovl ( & doubleprecision (*) & stradat, \\
                           & character (*)       & string,  \\
                           & integer             & ierr)
\end{tabular}}

\progdes

The {\tt SCOTCH\_stratGraphPartOvl} routine fills the strategy
structure pointed to by {\tt straptr} with the graph partitioning with
overlap strategy string pointed to by {\tt string}. From this point,
the strategy structure can only be used as a partitioning with overlap
strategy, to be used by function
{\tt SCOTCH\_\lbt graph\lbt Part\lbt Ovl} only.

When using the C interface, the array of characters pointed to by
{\tt string} must be null-terminated.

\progret

{\tt SCOTCH\_stratGraphPartOvl} returns $0$ if the strategy string
has been successfully set, and $1$ else.
\end{itemize}

\subsubsection{{\tt SCOTCH\_stratGraphPartOvlBuild}}
\label{sec-lib-func-stratgraphpartovlbuild}

\begin{itemize}
\progsyn

{\tt\begin{tabular}{l@{}ll}
int SCOTCH\_stratGraphPartOvlBuild ( & SCOTCH\_Strat *   & straptr, \\
                                     & const SCOTCH\_Num & flagval, \\
                                     & const SCOTCH\_Num & partnbr, \\
                                     & const double      & balrat)
\end{tabular}}

{\tt\begin{tabular}{l@{}ll}
scotchfstratgraphpartovlbuild ( & doubleprecision (*) & stradat, \\
                                & integer*{\it num}   & flagval, \\
                                & integer*{\it num}   & partnbr, \\
                                & doubleprecision     & balrat,  \\
                                & integer             & ierr)
\end{tabular}}

\progdes

The {\tt SCOTCH\_stratGraphPartOvlBuild} routine fills the strategy
structure pointed to by {\tt straptr} with a default partitioning with
overlap strategy tuned according to the preference flags passed as
{\tt flagval} and to the desired number of parts {\tt partnbr} and
imbalance ratio {\tt balrat}. From this point, the strategy structure
can only be used as a partitioning with overlap strategy, to be used
by function {\tt SCOTCH\_\lbt graph\lbt Part\lbt Ovl} only. See
Section~\ref{sec-lib-format-strat-default} for a description of the
available flags.

\progret

{\tt SCOTCH\_stratGraphPartOvlBuild} returns $0$ if the strategy string
has been successfully set, and $1$ else.
\end{itemize}

\subsubsection{{\tt SCOTCH\_stratGraphOrder}}

\begin{itemize}
\progsyn

{\tt\begin{tabular}{l@{}ll}
int SCOTCH\_stratGraphOrder ( & SCOTCH\_Strat * & straptr, \\
                              & const char *    & string)
\end{tabular}}

{\tt\begin{tabular}{l@{}ll}
scotchfstratgraphorder ( & doubleprecision (*) & stradat, \\
                         & character (*)       & string,  \\
                         & integer             & ierr)
\end{tabular}}

\progdes

The {\tt SCOTCH\_stratGraphOrder} routine fills the strategy
structure pointed to by {\tt straptr} with the graph ordering
strategy string pointed to by {\tt string}. From this point,
the strategy structure can only be used as a graph ordering
strategy, to be used by function
{\tt SCOTCH\_\lbt graph\lbt Order}, for instance.

When using the C interface, the array of characters pointed to by
{\tt string} must be null-terminated.

\progret

{\tt SCOTCH\_stratGraphOrder} returns $0$ if the strategy string
has been successfully set, and $1$ else.
\end{itemize}

\subsubsection{{\tt SCOTCH\_stratGraphOrderBuild}}

\begin{itemize}
\progsyn

{\tt\begin{tabular}{l@{}ll}
int SCOTCH\_stratGraphOrderBuild ( & SCOTCH\_Strat *   & straptr, \\
                                   & const SCOTCH\_Num & flagval, \\
                                   & const SCOTCH\_Num & levlnbr, \\
                                   & const double      & balrat)
\end{tabular}}

{\tt\begin{tabular}{l@{}ll}
scotchfstratgraphorderbuild ( & doubleprecision (*) & stradat, \\
                              & integer*{\it num}   & flagval, \\
                              & integer*{\it num}   & levlnbr, \\
                              & doubleprecision     & balrat,  \\
                              & integer             & ierr)
\end{tabular}}

\progdes

The {\tt SCOTCH\_stratGraphOrderBuild} routine fills the strategy
structure pointed to by {\tt straptr} with a default sequential
ordering strategy tuned according to the preference flags passed as
{\tt flagval} and to the desired nested dissection imbalance ratio
{\tt balrat}. From this point, the strategy structure can only be used
as an ordering strategy, to be used by function {\tt SCOTCH\_\lbt
graph\lbt Order}, for instance.

See Section~\ref{sec-lib-format-strat-default} for a description of
the available flags. When any of the {\tt SCOTCH\_\lbt STRAT\lbt
LEVEL\lbt MIN} or {\tt SCOTCH\_\lbt STRAT\lbt LEVEL\lbt MAX} flags is
set, the {\tt levlnbr} parameter is taken into account.

\progret

{\tt SCOTCH\_stratGraphOrderBuild} returns $0$ if the strategy string
has been successfully set, and $1$ else.
\end{itemize}

\subsubsection{{\tt SCOTCH\_stratMeshOrder}}

\begin{itemize}
\progsyn

{\tt\begin{tabular}{l@{}ll}
int SCOTCH\_stratMeshOrder ( & SCOTCH\_Strat * & straptr, \\
                             & const char *    & string)
\end{tabular}}

{\tt\begin{tabular}{l@{}ll}
scotchfstratmeshorder ( & doubleprecision (*) & stradat, \\
                        & character (*)       & string,  \\
                        & integer             & ierr)
\end{tabular}}

\progdes

The {\tt SCOTCH\_stratMeshOrder} routine fills the strategy
structure pointed to by {\tt straptr} with the mesh ordering
strategy string pointed to by {\tt string}. From this point,
strategy {\tt strat} can only be used as a mesh ordering
strategy, to be used by function
{\tt SCOTCH\_\lbt mesh\lbt Order}, for instance.

When using the C interface, the array of characters pointed to by
{\tt string} must be null-terminated.

\progret

{\tt SCOTCH\_stratMeshOrder} returns $0$ if the strategy string
has been successfully set, and $1$ else.
\end{itemize}

\subsubsection{{\tt SCOTCH\_stratMeshOrderBuild}}

\begin{itemize}
\progsyn

{\tt\begin{tabular}{l@{}ll}
int SCOTCH\_stratMeshOrderBuild ( & SCOTCH\_Strat *   & straptr, \\
                                  & const SCOTCH\_Num & flagval, \\
                                  & const double      & balrat)
\end{tabular}}

{\tt\begin{tabular}{l@{}ll}
scotchfstratmeshorderbuild ( & doubleprecision (*) & stradat, \\
                             & integer*{\it num}   & flagval, \\
                             & doubleprecision     & balrat,  \\
                             & integer             & ierr)
\end{tabular}}

\progdes

The {\tt SCOTCH\_stratMeshOrderBuild} routine fills the strategy
structure pointed to by {\tt straptr} with a default ordering strategy
tuned according to the preference flags passed as {\tt flagval} and to
the desired nested dissection imbalance ratio {\tt balrat}. From this
point, the strategy structure can only be used as an ordering
strategy, to be used by function {\tt SCOTCH\_\lbt mesh\lbt Order},
for instance. See Section~\ref{sec-lib-format-strat-default} for a
description of the available flags.

\progret

{\tt SCOTCH\_stratMesdOrderBuild} returns $0$ if the strategy string
has been successfully set, and $1$ else.
\end{itemize}

\subsection{Geometry handling routines}
\label{sec-lib-geom}

Since the \scotch\ project is based on algorithms that rely
on topology data only, geometry data do not play an
important role in the \libscotch\ library. They are only
relevant to programs that display graphs, such as the
{\tt gout} program. However, since all routines that are
used by the programs of the \scotch\ distributions have
an interface in the \libscotch\ library, there exist
geometry handling routines in it, which manipulate
{\tt SCOTCH\_\lbt Geom} structures.

Apart from the routines that create, destroy or access
{\tt SCOTCH\_\lbt Geom} structures, all of the routines
in this section are input/output routines, which read or
write both {\tt SCOTCH\_\lbt Graph} and {\tt SCOTCH\_\lbt Geom}
structures. We have chosen to define the interface of the
geometry-handling routines such that they also handle
graph or mesh topology because some external file formats
mix these data, and that we wanted our routines to be able
to read their data on the fly from streams that can only be
read once, such as communication pipes. Having both aspects
taken into account in a single call makes the writing of
file conversion tools, such as {\tt gcv} and {\tt mcv},
very easy. When the file format from which to read or into
which to write mixes both sorts of data, the geometry file
pointer can be set to {\tt NULL}, as it will not be used.

\subsubsection{{\tt SCOTCH\_geomAlloc}}

\begin{itemize}
\progsyn

{\tt\begin{tabular}{l@{}l}
SCOTCH\_Geom * SCOTCH\_geomAlloc ( & void)
\end{tabular}}

\progdes

The {\tt SCOTCH\_geomAlloc} function allocates a memory area of a
size sufficient to store a {\tt SCOTCH\_\lbt Geom} structure. It is
the user's responsibility to free this memory when it is no longer
needed, using the {\tt SCOTCH\_\lbt mem\lbt Free} routine. The
allocated space must be initialized before use, by means of the
{\tt SCOTCH\_\lbt geom\lbt Init} routine.

\progret

{\tt SCOTCH\_geomAlloc} returns the pointer to the memory area if it
has been successfully allocated, and {\tt NULL} else.
\end{itemize}

\subsubsection{{\tt SCOTCH\_geomInit}}
\label{sec-lib-func-geominit}

\begin{itemize}
\progsyn

{\tt\begin{tabular}{l@{}ll}
int SCOTCH\_geomInit ( & SCOTCH\_Geom * & geomptr)
\end{tabular}}

{\tt\begin{tabular}{l@{}ll}
scotchfgeominit ( & doubleprecision (*) & geomdat, \\
                  & integer             & ierr)
\end{tabular}}

\progdes

The {\tt SCOTCH\_geomInit} function initializes a {\tt SCOTCH\_\lbt
Geom} structure so as to make it suitable for future operations. It
should be the first function to be called upon a {\tt SCOTCH\_\lbt
Geom} structure. When the geometrical data is no longer of use, call
function {\tt SCOTCH\_\lbt geom\lbt Exit} to free its internal
structures.

\progret

{\tt SCOTCH\_geomInit} returns $0$ if the geometrical structure has been
successfully initialized, and $1$ else.
\end{itemize}

\subsubsection{{\tt SCOTCH\_geomExit}}

\begin{itemize}
\progsyn

{\tt\begin{tabular}{l@{}ll}
void SCOTCH\_geomExit ( & SCOTCH\_Geom * & geomptr)
\end{tabular}}

{\tt\begin{tabular}{l@{}ll}
scotchfgeomexit ( & doubleprecision (*) & geomdat)
\end{tabular}}

\progdes

The {\tt SCOTCH\_geomExit} function frees the contents of a
{\tt SCOTCH\_\lbt Geom} structure previously initialized by
{\tt SCOTCH\_\lbt geomInit}. All subsequent calls to
{\tt SCOTCH\_\lbt *Geom*} routines other than {\tt SCOTCH\_\lbt
geomInit}, using this structure as parameter, may yield
unpredictable results.
\end{itemize}

\subsubsection{{\tt SCOTCH\_geomData}}
\label{sec-lib-func-geomdata}

\begin{itemize}
\progsyn

{\tt\begin{tabular}{l@{}ll}
void SCOTCH\_geomData ( & const SCOTCH\_Geom * & geomptr, \\
                        & SCOTCH\_Num *        & dimnptr, \\
                        & double **            & geomtab)
\end{tabular}}

{\tt\begin{tabular}{l@{}ll}
scotchfgeomdata ( & doubleprecision (*) & geomdat, \\
                  & doubleprecision (*) & indxtab, \\
                  & integer*{\it num}   & dimnnbr, \\
                  & integer*{\it idx}   & geomidx)
\end{tabular}}

\progdes

The {\tt SCOTCH\_geomData} routine is a multiple
accessor to the contents of {\tt SCOTCH\_Geom} structures.

{\tt dimnptr} is the pointer to a location that will hold the
number of dimensions of the graph vertex or mesh node vertex
coordinates, and will therefore be equal to $1$, $2$ or $3$.
{\tt geomtab} is the pointer to a location that will hold the
reference to the geometry coordinates, as defined in
section~\ref{sec-lib-format-geom}.

Any of these pointers can be set to {\tt NULL} on input if the
corresponding information is not needed. Else, the reference to a
dummy area can be provided, where all unwanted data will be written.

Since there are no pointers in Fortran, a specific mechanism is used
to allow users to access the coordinate array. The {\tt scotchf\lbt geom\lbt
data} routine is passed an integer array, the first element of which is used
as a base address from which all other array indices are
computed. Therefore, instead of returning a reference, the routine
returns an integer, which represents the starting index of the coordinate
array with respect to the base input array. For
instance, if some base array {\tt myarray\lbt (1)} is passed as
parameter {\tt indxtab}, then the first cell of array {\tt geomtab}
will be accessible as {\tt myarray\lbt (geomidx)}.
In order for this feature to behave properly, the {\tt indxtab}
array must be double-precision-aligned with the geometry array. This is
automatically enforced on most systems, but some care should be
taken on systems that allow one to access data that is not
double-aligned. On such systems, declaring the array after a
dummy {\tt double\lbt precision} array can coerce the compiler
into enforcing the proper alignment. Also, on 32\_64 architectures,
such indices can be larger than the size of a regular
{\tt INTEGER}. This is why the indices to be returned are defined by
means of a specific integer type. See
Section~\ref{sec-lib-inttypesize} for more information on this
issue.
\end{itemize}

\subsubsection{{\tt SCOTCH\_graphGeomLoadChac}}

\begin{itemize}
\progsyn

{\tt\begin{tabular}{l@{}ll}
int SCOTCH\_graphGeomLoadChac ( & SCOTCH\_Graph * & grafptr, \\
                                & SCOTCH\_Geom *  & geomptr, \\
                                & FILE *          & grafstream, \\
                                & FILE *          & geomstream, \\
                                & const char *    & string)
\end{tabular}}

{\tt\begin{tabular}{l@{}ll}
scotchfgraphgeomloadchac ( & doubleprecision (*) & grafdat,    \\
                           & doubleprecision (*) & geomdat,    \\
                           & integer             & graffildes, \\
                           & integer             & geomfildes, \\
                           & character (*)       & string)
\end{tabular}}

\progdes

The {\tt SCOTCH\_graphGeomLoadChac} routine fills the {\tt
SCOTCH\_\lbt Graph} structure pointed to by {\tt grafptr} with the
source graph description available from stream {\tt graf\lbt stream}
in the \chaco\ graph format~\cite{hele93c}. Since this graph format
does not handle geometry data, the {\tt geomptr} and
{\tt geom\lbt stream} fields are not used, as well as the {\tt string}
field.

Fortran users must use the {\tt PXFFILENO} or {\tt FNUM} functions to
obtain the number of the Unix file descriptor {\tt graf\lbt fildes}
associated with the logical unit of the graph file.

\progret

{\tt SCOTCH\_graphGeomLoadChac} returns $0$ if the graph structure has
been successfully allocated and filled with the data read, and $1$ else.
\end{itemize}

\subsubsection{{\tt SCOTCH\_graphGeomSaveChac}}

\begin{itemize}
\progsyn

{\tt\begin{tabular}{l@{}ll}
int SCOTCH\_graphGeomSaveChac ( & const SCOTCH\_Graph * & grafptr, \\
                                & const SCOTCH\_Geom *  & geomptr, \\
                                & FILE *                & grafstream, \\
                                & FILE *                & geomstream, \\
                                & const char *          & string)
\end{tabular}}

{\tt\begin{tabular}{l@{}ll}
scotchfgraphgeomsavechac ( & doubleprecision (*) & grafdat,    \\
                           & doubleprecision (*) & geomdat,    \\
                           & integer             & graffildes, \\
                           & integer             & geomfildes, \\
                           & character (*)       & string)
\end{tabular}}

\progdes

The {\tt SCOTCH\_graphGeomSaveChac} routine saves the contents
of the {\tt SCOTCH\_\lbt Graph} structure pointed to by {\tt grafptr}
to stream {\tt graf\lbt stream}, in the \chaco\ graph
format~\cite{hele93c}. Since this graph format does
not handle geometry data, the {\tt geomptr} and {\tt geom\lbt stream}
fields are not used, as well as the {\tt string} field.

Fortran users must use the {\tt PXFFILENO} or {\tt FNUM} functions to
obtain the number of the Unix file descriptor {\tt graf\lbt fildes}
associated with the logical unit of the graph file.

\progret

{\tt SCOTCH\_graphGeomSaveChac} returns $0$ if the graph structure
has been successfully written to {\tt graf\lbt stream}, and $1$ else.
\end{itemize}

\subsubsection{{\tt SCOTCH\_graphGeomLoadHabo}}

\begin{itemize}
\progsyn

{\tt\begin{tabular}{l@{}ll}
int SCOTCH\_graphGeomLoadHabo ( & SCOTCH\_Graph * & grafptr, \\
                                & SCOTCH\_Geom *  & geomptr, \\
                                & FILE *          & grafstream, \\
                                & FILE *          & geomstream, \\
                                & const char *    & string)
\end{tabular}}

{\tt\begin{tabular}{l@{}ll}
scotchfgraphgeomloadhabo ( & doubleprecision (*) & grafdat,    \\
                           & doubleprecision (*) & geomdat,    \\
                           & integer             & graffildes, \\
                           & integer             & geomfildes, \\
                           & character (*)       & string)
\end{tabular}}

\progdes

The {\tt SCOTCH\_graphGeomLoadHabo} routine fills the {\tt
SCOTCH\_\lbt Graph} structure pointed to by {\tt grafptr} with the
source graph description available from stream {\tt graf\lbt stream}
in the Harwell-Boeing square assembled matrix format~\cite{dugrle92}.
Since this graph format does not handle geometry data, the
{\tt geomptr} and {\tt geom\lbt stream} fields are not used. Since
multiple graph structures can be encoded sequentially within the same
file, the {\tt string} field contains the string representation of an
integer number that codes the rank of the graph to read within the
Harwell-Boeing file. It is equal to ``0'' in most cases.

Fortran users must use the {\tt PXFFILENO} or {\tt FNUM} functions to
obtain the number of the Unix file descriptor {\tt graf\lbt fildes}
associated with the logical unit of the graph file.

\progret

{\tt SCOTCH\_graphGeomLoadHabo} returns $0$ if the graph structure has
been successfully allocated and filled with the data read, and $1$ else.
\end{itemize}

\subsubsection{{\tt SCOTCH\_graphGeomLoadScot}}

\begin{itemize}
\progsyn

{\tt\begin{tabular}{l@{}ll}
int SCOTCH\_graphGeomLoadScot ( & SCOTCH\_Graph * & grafptr, \\
                                & SCOTCH\_Geom *  & geomptr, \\
                                & FILE *          & grafstream, \\
                                & FILE *          & geomstream, \\
                                & const char *    & string)
\end{tabular}}

{\tt\begin{tabular}{l@{}ll}
scotchfgraphgeomloadscot ( & doubleprecision (*) & grafdat, \\
                           & doubleprecision (*) & geomdat, \\
                           & integer             & graffildes, \\
                           & integer             & geomfildes, \\
                           & character (*)       & string)
\end{tabular}}

\progdes

The {\tt SCOTCH\_graphGeomLoadScot} routine fills the {\tt
SCOTCH\_\lbt Graph} and {\tt SCOTCH\_\lbt Geom} structures pointed to
by {\tt grafptr} and {\tt geomptr} with the source graph description
and geometry data available from streams {\tt graf\lbt stream} and
{\tt geom\lbt stream} in the \scotch\ graph and geometry formats
(see sections~\ref{sec-file-sgraph} and~\ref{sec-file-geom},
respectively). The {\tt string} field is not used.

Fortran users must use the {\tt PXFFILENO} or {\tt FNUM} functions to
obtain the numbers of the Unix file descriptors {\tt graf\lbt fildes}
and {\tt geom\lbt fildes} associated with the logical units of the
graph and geometry files.

\progret

{\tt SCOTCH\_graphGeomLoadScot} returns $0$ if the graph topology and
geometry have been successfully allocated and filled with the data
read, and $1$ else.
\end{itemize}

\subsubsection{{\tt SCOTCH\_graphGeomSaveScot}}

\begin{itemize}
\progsyn

{\tt\begin{tabular}{l@{}ll}
int SCOTCH\_graphGeomSaveScot ( & const SCOTCH\_Graph * & grafptr, \\
                                & const SCOTCH\_Geom *  & geomptr, \\
                                & FILE *                & grafstream, \\
                                & FILE *                & geomstream, \\
                                & const char *          & string)
\end{tabular}}

{\tt\begin{tabular}{l@{}ll}
scotchfgraphgeomsavescot ( & doubleprecision (*) & grafdat, \\
                           & doubleprecision (*) & geomdat, \\
                           & integer             & graffildes, \\
                           & integer             & geomfildes, \\
                           & character (*)       & string)
\end{tabular}}

\progdes

The {\tt SCOTCH\_graphGeomSaveScot} routine saves the contents
of the {\tt
SCOTCH\_\lbt Graph} and {\tt SCOTCH\_\lbt Geom} structures pointed to
by {\tt grafptr} and {\tt geomptr} to streams {\tt graf\lbt stream}
and {\tt geom\lbt stream}, in the \scotch\ graph and geometry
formats (see sections~\ref{sec-file-sgraph} and~\ref{sec-file-geom},
respectively). The {\tt string} field is not used.

Fortran users must use the {\tt PXFFILENO} or {\tt FNUM} functions to
obtain the numbers of the Unix file descriptors {\tt graf\lbt fildes}
and {\tt geom\lbt fildes} associated with the logical units of the
graph and geometry files.

\progret

{\tt SCOTCH\_graphGeomSaveScot} returns $0$ if the graph topology and
geometry have been successfully written to {\tt graf\lbt stream} and
{\tt geom\lbt stream}, and $1$ else.
\end{itemize}

\subsubsection{{\tt SCOTCH\_meshGeomLoadHabo}}

\begin{itemize}
\progsyn

{\tt\begin{tabular}{l@{}ll}
int SCOTCH\_meshGeomLoadHabo ( & SCOTCH\_Mesh * & meshptr, \\
                               & SCOTCH\_Geom * & geomptr, \\
                               & FILE *         & meshstream, \\
                               & FILE *         & geomstream, \\
                               & const char *   & string)
\end{tabular}}

{\tt\begin{tabular}{l@{}ll}
scotchfmeshgeomloadhabo ( & doubleprecision (*) & meshdat, \\
                          & doubleprecision (*) & geomdat, \\
                          & integer             & meshfildes, \\
                          & integer             & geomfildes, \\
                          & character (*)       & string)
\end{tabular}}

\progdes

The {\tt SCOTCH\_meshGeomLoadHabo} routine fills the {\tt
SCOTCH\_\lbt Mesh} structure pointed to by {\tt meshptr} with the
source mesh description available from stream {\tt mesh\lbt stream}
in the Harwell-Boeing square elemental matrix format~\cite{dugrle92}.
Since this mesh format does not handle geometry data, the
{\tt geomptr} and {\tt geom\lbt stream} fields are not used. Since
multiple mesh structures can be encoded sequentially within the same
file, the {\tt string} field contains the string representation of an
integer number that codes the rank of the mesh to read within the
Harwell-Boeing file. It is equal to ``0'' in most cases.

Fortran users must use the {\tt PXFFILENO} or {\tt FNUM} functions to
obtain the number of the Unix file descriptor {\tt mesh\lbt fildes}
associated with the logical unit of the mesh file.

\progret

{\tt SCOTCH\_meshGeomLoadHabo} returns $0$ if the mesh structure has
been successfully allocated and filled with the data read, and $1$ else.
\end{itemize}

\subsubsection{{\tt SCOTCH\_meshGeomLoadScot}}

\begin{itemize}
\progsyn

{\tt\begin{tabular}{l@{}ll}
int SCOTCH\_meshGeomLoadScot ( & SCOTCH\_Mesh * & meshptr, \\
                               & SCOTCH\_Geom * & geomptr, \\
                               & FILE *         & meshstream, \\
                               & FILE *         & geomstream, \\
                               & const char *   & string)
\end{tabular}}

{\tt\begin{tabular}{l@{}ll}
scotchfmeshgeomloadscot ( & doubleprecision (*) & meshdat, \\
                          & doubleprecision (*) & geomdat, \\
                          & integer             & meshfildes, \\
                          & integer             & geomfildes, \\
                          & character (*)       & string)
\end{tabular}}

\progdes

The {\tt SCOTCH\_meshGeomLoadScot} routine fills the {\tt
SCOTCH\_\lbt Mesh} and {\tt SCOTCH\_\lbt Geom} structures pointed to
by {\tt meshptr} and {\tt geomptr} with the source mesh description
and node geometry data available from streams {\tt mesh\lbt stream}
and {\tt geom\lbt stream} in the \scotch\ mesh and geometry formats
(see sections~\ref{sec-file-smesh} and~\ref{sec-file-geom},
respectively). The {\tt string} field is not used.

Fortran users must use the {\tt PXFFILENO} or {\tt FNUM} functions to
obtain the numbers of the Unix file descriptors {\tt mesh\lbt fildes}
and {\tt geom\lbt fildes} associated with the logical units of the
mesh and geometry files.

\progret

{\tt SCOTCH\_meshGeomLoadScot} returns $0$ if the mesh topology and
node geometry have been successfully allocated and filled with the
data read, and $1$ else.
\end{itemize}

\subsubsection{{\tt SCOTCH\_meshGeomSaveScot}}

\begin{itemize}
\progsyn

{\tt\begin{tabular}{l@{}ll}
int SCOTCH\_meshGeomSaveScot ( & const SCOTCH\_Mesh * & meshptr, \\
                               & const SCOTCH\_Geom * & geomptr, \\
                               & FILE *               & meshstream, \\
                               & FILE *               & geomstream, \\
                               & const char *         & string)
\end{tabular}}

{\tt\begin{tabular}{l@{}ll}
scotchfmeshgeomsavescot ( & doubleprecision (*) & meshdat, \\
                          & doubleprecision (*) & geomdat, \\
                          & integer             & meshfildes, \\
                          & integer             & geomfildes, \\
                          & character (*)       & string)
\end{tabular}}

\progdes

The {\tt SCOTCH\_meshGeomSaveScot} routine saves the contents
of the {\tt SCOTCH\_\lbt Mesh} and {\tt SCOTCH\_\lbt Geom}
structures pointed to by {\tt meshptr} and {\tt geomptr} to
streams {\tt mesh\lbt stream} and {\tt geom\lbt stream}, in
the \scotch\ mesh and geometry formats (see
sections~\ref{sec-file-smesh} and~\ref{sec-file-geom},
respectively). The {\tt string} field is not used.

Fortran users must use the {\tt PXFFILENO} or {\tt FNUM} functions to
obtain the numbers of the Unix file descriptors {\tt mesh\lbt fildes}
and {\tt geom\lbt fildes} associated with the logical units of the
mesh and geometry files.

\progret

{\tt SCOTCH\_meshGeomSaveScot} returns $0$ if the mesh topology and
node geometry have been successfully written to {\tt mesh\lbt stream}
and {\tt geom\lbt stream}, and $1$ else.
\end{itemize}

\subsection{Other data structure handling routines}
\label{sec-lib-other}

\subsubsection{{\tt SCOTCH\_mapAlloc}}

\begin{itemize}
\progsyn

{\tt\begin{tabular}{l@{}l}
SCOTCH\_Mapping * SCOTCH\_mapAlloc ( & void)
\end{tabular}}

\progdes

The {\tt SCOTCH\_mapAlloc} function allocates a memory area of a
size sufficient to store a {\tt SCOTCH\_\lbt Mapping} structure. It is
the user's responsibility to free this memory when it is no longer
needed, using the {\tt SCOTCH\_\lbt mem\lbt Free} routine.

\progret

{\tt SCOTCH\_mapAlloc} returns the pointer to the memory area if it
has been successfully allocated, and {\tt NULL} else.
\end{itemize}

\subsubsection{{\tt SCOTCH\_orderAlloc}}

\begin{itemize}
\progsyn

{\tt\begin{tabular}{l@{}l}
SCOTCH\_Ordering * SCOTCH\_orderAlloc ( & void)
\end{tabular}}

\progdes

The {\tt SCOTCH\_orderAlloc} function allocates a memory area of a
size sufficient to store a {\tt SCOTCH\_\lbt Ordering} structure. It is
the user's responsibility to free this memory when it is no longer
needed, using the {\tt SCOTCH\_\lbt mem\lbt Free} routine.

\progret

{\tt SCOTCH\_orderAlloc} returns the pointer to the memory area if it
has been successfully allocated, and {\tt NULL} else.
\end{itemize}

\subsection{Error handling routines}
\label{sec-lib-error}

The handling of errors that occur within library routines is
often difficult, because library routines should be able to issue
error messages that help the application programmer to find the
error, while being compatible with the way the application handles
its own errors.

To match these two requirements, all the error and warning messages
produced by the routines of the \libscotch\ library are issued using
the user-definable variable-length argument routines {\tt SCOTCH\_\lbt
error\lbt Print} and {\tt SCOTCH\_\lbt error\lbt PrintW}. Thus, one
can redirect these error messages to his own error handling routines,
and can choose if he wants his program to terminate on error or to
resume execution after the erroneous function has returned.

In order to free the user from the burden of writing a basic error
handler from scratch, the {\tt libscotcherr.a} library provides error
routines that print error messages on the standard error stream
{\tt stderr} and return control to the application. Application
programmers who want to take advantage of them have to add
{\tt -lscotcherr} to the list of arguments of the linker, after
the {\tt -lscotch} argument.

\subsubsection{{\tt SCOTCH\_errorPrint}}
\label{sec-lib-func-errorprint}

\begin{itemize}
\progsyn

{\tt\begin{tabular}{l@{}ll}
void SCOTCH\_errorPrint ( & const char * const & errstr, ...)
\end{tabular}}

\progdes

The {\tt SCOTCH\_errorPrint} function is designed to output a
variable-length argument error string to some stream.
\end{itemize}

\subsubsection{{\tt SCOTCH\_errorPrintW}}

\begin{itemize}
\progsyn

{\tt\begin{tabular}{l@{}ll}
void SCOTCH\_errorPrintW ( & const char * const & errstr, ...)
\end{tabular}}

\progdes

The {\tt SCOTCH\_errorPrintW} function is designed to output a
variable-length argument warning string to some stream.
\end{itemize}

\subsubsection{{\tt SCOTCH\_errorProg}}

\begin{itemize}
\progsyn

{\tt\begin{tabular}{l@{}ll}
void SCOTCH\_errorProg ( & const char * & progstr)
\end{tabular}}

\progdes

The {\tt SCOTCH\_errorProg} function is designed to be called at
the beginning of a program or of a portion of code to identify the
place where subsequent errors take place.
This routine is not reentrant, as it is only a minor help function. It
is defined in {\tt lib\lbt scotch\lbt err.a} and is used by the
standalone programs of the \scotch\ distribution.
\end{itemize}

\subsection{Miscellaneous routines}
\label{sec-lib-misc}

\subsubsection{{\tt SCOTCH\_memCur}}

\begin{itemize}
\progsyn

{\tt\begin{tabular}{l@{}l}
SCOTCH\_Idx SCOTCH\_memCur ( & void)
\end{tabular}}

{\tt\begin{tabular}{l@{}ll}
scotchfmemcur ( & integer*{\it idx} & memcur) \\

\end{tabular}}

\progdes

When \scotch\ is compiled with the {\tt COMMON\_\lbt MEMORY\_\lbt
TRACE} flag set, the {\tt SCOTCH\_memCur} routine returns the amount
of memory, in bytes, that is currently allocated by \scotch\ on the
current processing element, either by itself or on the behalf of the
user. Else, the routine returns {\tt -1}.

The returned figure does not account for the memory that has been
allocated by the user and made visible to \scotch\ by means of
routines such as {\tt SCOTCH\_\lbt dgraph\lbt Build} calls. This
memory is not under the control of \scotch, and it is the user's
responsibility to free it after calling the relevant
{\tt SCOTCH\_\lbt *\lbt Exit} routines.

Some third-party software used by \scotch, such as the strategy string
parser, may allocate some memory for internal use and never free it.
Consequently, there may be small discrepancies between memory
occupation figures returned by \scotch\ and those returned by
third-party tools. However, these discrepancies should not exceed a
few kilobytes.

While memory occupation is internally recorded in a variable of type
{\tt intptr\_\lbt t}, it is output as a {\tt SCOTCH\_\lbt Idx} for the
sake of interface homogeneity, especially for Fortran. It is therefore
the installer's responsibility to make sure that the support integer
type of {\tt SCOTCH\_\lbt Idx} is large enough to not overflow. See
section~\ref{sec-lib-inttypesize} for more information.
\end{itemize}

\subsubsection{{\tt SCOTCH\_memFree}}

\begin{itemize}
\progsyn

{\tt\begin{tabular}{l@{}ll}
void SCOTCH\_memFree ( & void * & dataptr)
\end{tabular}}

\progdes

The {\tt SCOTCH\_memFree} routine frees the memory space allocated
by routines such as {\tt SCOTCH\_\lbt graph\lbt Alloc},
{\tt SCOTCH\_\lbt mesh\lbt Alloc}, or
{\tt SCOTCH\_\lbt strat\lbt Alloc}.

The standard {\tt free} routine of the {\textsc libc} must not be
used for this purpose. Else, the allocated memory will not be
considered as properly released by memory accounting routines
{\tt SCOTCH\_\lbt mem\lbt Cur} and {\tt SCOTCH\_\lbt mem\lbt Max},
and segmentation errors would happen when the
{\tt COMMON\_\lbt MEMORY\_\lbt CHECK} compile flag is set.
\end{itemize}

\subsubsection{{\tt SCOTCH\_memMax}}

\begin{itemize}
\progsyn

{\tt\begin{tabular}{l@{}l}
SCOTCH\_Idx SCOTCH\_memMax ( & void)
\end{tabular}}

{\tt\begin{tabular}{l@{}ll}
scotchfmemmax ( & integer*{\it idx} & memcur) \\

\end{tabular}}

\progdes

When \scotch\ is compiled with the {\tt COMMON\_\lbt MEMORY\_\lbt
TRACE} flag set, the {\tt SCOTCH\_memMax} routine returns the maximum
amount of memory, in bytes, ever allocated by \scotch\ on the current
processing element, either by itself or on the behalf of the
user. Else, the routine returns {\tt -1}.

The returned figure does not account for the memory that has been
allocated by the user and made visible to \scotch\ by means of
routines such as {\tt SCOTCH\_\lbt dgraph\lbt Build} calls. This
memory is not under the control of \scotch, and it is the user's
responsibility to free it after calling the relevant
{\tt SCOTCH\_\lbt *\lbt Exit} routines.

Some third-party software used by \scotch, such as the strategy string
parser, may allocate some memory for internal use and never free it.
Consequently, there may be small discrepancies between memory
occupation figures returned by \scotch\ and those returned by
third-party tools. However, these discrepancies should not exceed a
few kilobytes.

While memory occupation is internally recorded in a variable of type
{\tt intptr\_\lbt t}, it is output as a {\tt SCOTCH\_\lbt Idx} for the
sake of interface homogeneity, especially for Fortran. It is therefore
the installer's responsibility to make sure that the support integer
type of {\tt SCOTCH\_\lbt Idx} is large enough to not overflow. See
section~\ref{sec-lib-inttypesize} for more information.
\end{itemize}

\subsubsection{{\tt SCOTCH\_numSizeof}}

\begin{itemize}
\progsyn

{\tt\begin{tabular}{l@{}l}
int SCOTCH\_numSizeof ( & void)
\end{tabular}}

{\tt\begin{tabular}{l@{}ll}
scotchfnumsizeof ( & integer & size )
\end{tabular}}

\progdes

The {\tt SCOTCH\_numSizeof} routine returns the size, in bytes, of a
{\tt SCOTCH\_\lbt Num}. This information is useful to export the
interface of the {\sc libScotch} to interpreted languages, without
access to the ``{\tt scotch.h}'' include file.
\end{itemize}

\subsubsection{{\tt SCOTCH\_randomReset}}

\begin{itemize}
\progsyn

{\tt\begin{tabular}{l@{}l}
void SCOTCH\_randomReset ( & void)
\end{tabular}}

{\tt\begin{tabular}{l@{}l}
scotchfrandomreset ( & )
\end{tabular}}

\progdes

The {\tt SCOTCH\_randomReset} routine resets the seed of the
pseudo-random generator used by the graph partitioning routines
of the \libscotch\ library. Two consecutive calls to
the same \libscotch\ partitioning or ordering routines, separated
by a call to {\tt SCOTCH\_\lbt random\lbt Reset}, will always yield
the same results.
\end{itemize}

\subsubsection{{\tt SCOTCH\_randomSeed}}

\begin{itemize}
\progsyn

{\tt\begin{tabular}{l@{}ll}
void SCOTCH\_randomSeed ( & SCOTCH\_Num & seedval)
\end{tabular}}

{\tt\begin{tabular}{l@{}ll}
scotchfrandomseed ( & integer*{\it num} & seedval )
\end{tabular}}

\progdes

The {\tt SCOTCH\_randomSeed} routine sets to {\tt seedval} the
seed of the pseudo-random generator used internally by several
algorithms of \scotch. All subsequent calls to {\tt SCOTCH\_\lbt
random\lbt Reset} will use this value to reset the pseudo-random
generator.

This routine needs only to be used by users willing to evaluate
the robustness and quality of partitioning algorithms with respect to
the variability of random seeds. Else, depending whether \scotch\ has
been compiled with any of the flags {\tt COMMON\_\lbt RANDOM\_\lbt
FIXED\_\lbt SEED} or {\tt SCOTCH\_\lbt DETERMINISTIC} set or not,
either the same pseudo-random seed will be always used, or a
process-dependent seed will be used, respectively.
\end{itemize}

\subsubsection{{\tt SCOTCH\_version}}

\begin{itemize}
\progsyn

{\tt\begin{tabular}{l@{}ll}
int SCOTCH\_version ( & int * const & versptr, \\
                      & int * const & relaptr, \\
                      & int * const & patcptr)
\end{tabular}}

{\tt\begin{tabular}{l@{}ll}
scotchfversion ( & integer & versval, \\
                 & integer & relaval, \\
                 & integer & patcval)
\end{tabular}}

\progdes

The {\tt SCOTCH\_version} routine writes the version, release and
patchlevel numbers of the \scotch\ library that is currently being
used, to integer values {\tt *versptr}, {\tt *relaptr} and {\tt
patcptr}, respectively. This routine is mainly useful for applications
willing to record runtime information, such as the library against
which they are dynamically linked.
\end{itemize}

\subsection{\metis\ compatibility library}
\label{sec-lib-metis}

The \metis\ compatibility library provides stubs which redirect some
calls to \metis\ routines to the corresponding \scotch\ counterparts.
In order to use this feature, the only thing to do is to re-link the
existing software with the {\tt lib\lbo scotch\lbo metis} library, and
eventually with the original \metis\ library if the software uses
\metis\ routines which do not need to have \scotch\ equivalents, such
as graph transformation routines.
In that latter case, the ``{\tt -lscotch\lbt metis}'' argument must be
placed before the ``{\tt -lmetis}'' one (and of course before the
``{\tt -lscotch}'' one too), so that routines that are redefined by
\scotch\ are chosen instead of their \metis\ counterpart. When no
other \metis\ routines than the ones redefined by \scotch\ are used,
the ``{\tt -lmetis}'' argument can be omitted. See
Section~\ref{sec-examples} for an example.

\subsubsection{{\tt METIS\_EdgeND}}

\begin{itemize}
\progsyn

{\tt\begin{tabular}{l@{}ll}
void METIS\_EdgeND ( & const SCOTCH\_Num * const & n, \\
                     & const SCOTCH\_Num * const & xadj, \\
                     & const SCOTCH\_Num * const & adjncy, \\
                     & const SCOTCH\_Num * const & numflag, \\
                     & const SCOTCH\_Num * const & options, \\
                     & SCOTCH\_Num * const       & perm, \\
                     & SCOTCH\_Num * const       & iperm)
\end{tabular}}

{\tt\begin{tabular}{l@{}ll}
metis\_edgend ( & integer*{\it num}     & n, \\
                & integer*{\it num} (*) & xadj, \\
                & integer*{\it num} (*) & adjncy, \\
                & integer*{\it num}     & numflag, \\
                & integer*{\it num} (*) & options, \\
                & integer*{\it num} (*) & perm, \\
                & integer*{\it num} (*) & iperm)
\end{tabular}}

\progdes

The {\tt METIS\_EdgeND} function performs a nested dissection ordering
of the graph passed as arrays {\tt xadj} and {\tt adjncy}, using
the default \scotch\ ordering strategy. The {\tt options} array is not
used. The {\tt perm} and {\tt iperm} arrays have the opposite meaning
as in \scotch: the \metis\ {\tt perm} array holds what is called
``inverse permutation'' in \scotch, while {\tt iperm} holds what is
called ``direct permutation'' in \scotch.

While \scotch\ has also both node and edge separation capabilities,
all of the three \metis\ stubs {\tt METIS\_\lbo EdgeND}, {\tt
METIS\_\lbo NodeND} and {\tt METIS\_\lbo NodeWND} call the same
\scotch\ routine, which uses the \scotch\ default ordering strategy
proved to be efficient in most cases.
\end{itemize}

\subsubsection{{\tt METIS\_NodeND}}

\begin{itemize}
\progsyn

{\tt\begin{tabular}{l@{}ll}
void METIS\_NodeND ( & const SCOTCH\_Num * const & n, \\
                     & const SCOTCH\_Num * const & xadj, \\
                     & const SCOTCH\_Num * const & adjncy, \\
                     & const SCOTCH\_Num * const & numflag, \\
                     & const SCOTCH\_Num * const & options, \\
                     & SCOTCH\_Num * const       & perm, \\
                     & SCOTCH\_Num * const       & iperm)
\end{tabular}}

{\tt\begin{tabular}{l@{}ll}
metis\_nodend ( & integer*{\it num}     & n, \\
                & integer*{\it num} (*) & xadj, \\
                & integer*{\it num} (*) & adjncy, \\
                & integer*{\it num}     & numflag, \\
                & integer*{\it num} (*) & options, \\
                & integer*{\it num} (*) & perm, \\
                & integer*{\it num} (*) & iperm)
\end{tabular}}

\progdes

The {\tt METIS\_NodeND} function performs a nested dissection ordering
of the graph passed as arrays {\tt xadj} and {\tt adjncy}, using
the default \scotch\ ordering strategy. The {\tt options} array is not
used. The {\tt perm} and {\tt iperm} arrays have the opposite meaning
as in \scotch: the \metis\ {\tt perm} array holds what is called
``inverse permutation'' in \scotch, while {\tt iperm} holds what is
called ``direct permutation'' in \scotch.

While \scotch\ has also both node and edge separation capabilities,
all of the three \metis\ stubs {\tt METIS\_\lbo EdgeND}, {\tt
METIS\_\lbo NodeND} and {\tt METIS\_\lbo NodeWND} call the same
\scotch\ routine, which uses the \scotch\ default ordering strategy
proved to be efficient in most cases.
\end{itemize}

\subsubsection{{\tt METIS\_NodeWND}}

\begin{itemize}
\progsyn

{\tt\begin{tabular}{l@{}ll}
void METIS\_NodeWND ( & const SCOTCH\_Num * const & n, \\
                      & const SCOTCH\_Num * const & xadj, \\
                      & const SCOTCH\_Num * const & adjncy, \\
                      & const SCOTCH\_Num * const & vwgt, \\
                      & const SCOTCH\_Num * const & numflag, \\
                      & const SCOTCH\_Num * const & options, \\
                      & SCOTCH\_Num * const       & perm, \\
                      & SCOTCH\_Num * const       & iperm)
\end{tabular}}

{\tt\begin{tabular}{l@{}ll}
metis\_nodwend ( & integer*{\it num}     & n, \\
                 & integer*{\it num} (*) & xadj, \\
                 & integer*{\it num} (*) & adjncy, \\
                 & integer*{\it num} (*) & vwgt, \\
                 & integer*{\it num}     & numflag, \\
                 & integer*{\it num} (*) & options, \\
                 & integer*{\it num} (*) & perm, \\
                 & integer*{\it num} (*) & iperm)
\end{tabular}}

\progdes

The {\tt METIS\_NodeWND} function performs a nested dissection
ordering of the graph passed as arrays {\tt xadj}, {\tt adjncy}
and {\tt vwgt}, using the default \scotch\ ordering strategy. The {\tt
options} array is not used. The {\tt perm} and {\tt iperm} arrays have
the opposite meaning as in \scotch: the \metis\ {\tt perm} array
holds what is called ``inverse permutation'' in \scotch, while {\tt
iperm} holds what is called ``direct permutation'' in \scotch.

While \scotch\ has also both node and edge separation capabilities,
all of the three \metis\ stubs {\tt METIS\_\lbo EdgeND}, {\tt
METIS\_\lbo NodeND} and {\tt METIS\_\lbo NodeWND} call the same
\scotch\ routine, which uses the \scotch\ default ordering strategy
proved to be efficient in most cases.
\end{itemize}

\subsubsection{{\tt METIS\_PartGraphKway}}

\begin{itemize}
\progsyn

{\tt\begin{tabular}{l@{}ll}
void METIS\_PartGraphKway ( & const SCOTCH\_Num * const & n, \\
                            & const SCOTCH\_Num * const & xadj, \\
                            & const SCOTCH\_Num * const & adjncy, \\
                            & const SCOTCH\_Num * const & vwgt, \\
                            & const SCOTCH\_Num * const & adjwgt, \\
                            & const SCOTCH\_Num * const & wgtflag, \\
                            & const SCOTCH\_Num * const & numflag, \\
                            & const SCOTCH\_Num * const & nparts, \\
                            & const SCOTCH\_Num * const & options, \\
                            & SCOTCH\_Num * const       & edgecut, \\
                            & SCOTCH\_Num * const       & part)
\end{tabular}}

{\tt\begin{tabular}{l@{}ll}
metis\_partgraphkway ( & integer*{\it num}     & n, \\
                       & integer*{\it num} (*) & xadj, \\
                       & integer*{\it num} (*) & adjncy, \\
                       & integer*{\it num} (*) & vwgt, \\
                       & integer*{\it num} (*) & adjwgt, \\
                       & integer*{\it num}     & wgtflag, \\
                       & integer*{\it num}     & numflag, \\
                       & integer*{\it num}     & nparts, \\
                       & integer*{\it num} (*) & options, \\
                       & integer*{\it num}     & edgecut, \\
                       & integer*{\it num} (*) & part)
\end{tabular}}

\progdes

The {\tt METIS\_PartGraphKway} function performs a mapping onto the
complete graph of the graph represented by arrays {\tt xadj}, {\tt
adjncy}, {\tt vwgt} and {\tt adjwgt}, using the default
\scotch\ mapping strategy. The {\tt options} array is not used. The
{\tt part} array has the same meaning as the {\tt parttab} array of
\scotch.

All of the three \metis\ stubs
{\tt METIS\_\lbo Part\lbo Graph\lbo Kway}, {\tt
METIS\_\lbo Part\lbo Graph\lbo Recursive} and {\tt METIS\_\lbo
Part\lbo Graph\lbo VKway} call the same \scotch\ routine, which uses
the \scotch\ default mapping strategy proved to be efficient in most
cases.
\end{itemize}

\subsubsection{{\tt METIS\_PartGraphRecursive}}

\begin{itemize}
\progsyn

{\tt\begin{tabular}{l@{}ll}
void METIS\_PartGraphRecursive ( & const SCOTCH\_Num * const & n, \\
                                 & const SCOTCH\_Num * const & xadj, \\
                                 & const SCOTCH\_Num * const & adjncy, \\
                                 & const SCOTCH\_Num * const & vwgt, \\
                                 & const SCOTCH\_Num * const & adjwgt, \\
                                 & const SCOTCH\_Num * const & wgtflag, \\
                                 & const SCOTCH\_Num * const & numflag, \\
                                 & const SCOTCH\_Num * const & nparts, \\
                                 & const SCOTCH\_Num * const & options, \\
                                 & SCOTCH\_Num * const       & edgecut, \\
                                 & SCOTCH\_Num * const       & part)
\end{tabular}}

{\tt\begin{tabular}{l@{}ll}
metis\_partgraphrecursive ( & integer*{\it num}     & n, \\
                            & integer*{\it num} (*) & xadj, \\
                            & integer*{\it num} (*) & adjncy, \\
                            & integer*{\it num} (*) & vwgt, \\
                            & integer*{\it num} (*) & adjwgt, \\
                            & integer*{\it num}     & wgtflag, \\
                            & integer*{\it num}     & numflag, \\
                            & integer*{\it num}     & nparts, \\
                            & integer*{\it num} (*) & options, \\
                            & integer*{\it num}     & edgecut, \\
                            & integer*{\it num} (*) & part)
\end{tabular}}

\progdes

The {\tt METIS\_PartGraphRecursive} function performs a mapping onto
the complete graph of the graph represented by arrays {\tt xadj}, {\tt
adjncy}, {\tt vwgt} and {\tt adjwgt}, using the default \scotch\
mapping strategy. The {\tt options} array is not used. The {\tt part}
array has the same meaning as the {\tt parttab} array of \scotch. To
date, the computation of the {\tt edgecut} field requires extra
processing, which increases running time to a small extent.

All of the three \metis\ stubs
{\tt METIS\_\lbo Part\lbo Graph\lbo Kway}, {\tt
METIS\_\lbo Part\lbo Graph\lbo Recursive} and {\tt METIS\_\lbo
Part\lbo Graph\lbo VKway} call the same \scotch\ routine, which uses
the \scotch\ default mapping strategy proved to be efficient in most
cases.
\end{itemize}

\subsubsection{{\tt METIS\_PartGraphVKway}}

\begin{itemize}
\progsyn

{\tt\begin{tabular}{l@{}ll}
void METIS\_PartGraphVKway ( & const SCOTCH\_Num * const & n, \\
                             & const SCOTCH\_Num * const & xadj, \\
                             & const SCOTCH\_Num * const & adjncy, \\
                             & const SCOTCH\_Num * const & vwgt, \\
                             & const SCOTCH\_Num * const & vsize, \\
                             & const SCOTCH\_Num * const & wgtflag, \\
                             & const SCOTCH\_Num * const & numflag, \\
                             & const SCOTCH\_Num * const & nparts, \\
                             & const SCOTCH\_Num * const & options, \\
                             & SCOTCH\_Num * const       & volume, \\
                             & SCOTCH\_Num * const       & part)
\end{tabular}}

{\tt\begin{tabular}{l@{}ll}
metis\_partgraphvkway ( & integer*{\it num}     & n, \\
                        & integer*{\it num} (*) & xadj, \\
                        & integer*{\it num} (*) & adjncy, \\
                        & integer*{\it num} (*) & vwgt, \\
                        & integer*{\it num} (*) & vsize, \\
                        & integer*{\it num}     & wgtflag, \\
                        & integer*{\it num}     & numflag, \\
                        & integer*{\it num}     & nparts, \\
                        & integer*{\it num} (*) & options, \\
                        & integer*{\it num}     & volume, \\
                        & integer*{\it num} (*) & part)
\end{tabular}}

\progdes

The {\tt METIS\_PartGraphVKway} function performs a mapping onto the
complete graph of the graph represented by arrays {\tt xadj}, {\tt
adjncy}, {\tt vwgt} and {\tt vsize}, using the default
\scotch\ mapping strategy. The {\tt options} array is not used. The
{\tt part} array has the same meaning as the {\tt parttab} array of
\scotch.

Since \scotch\ does not have methods for explicitely reducing the
communication volume according to the metric of {\tt METIS\_\lbo
Part\lbo Graph\lbo VKway}, this routine creates a temporary
edge weight array such that each edge $(u,v)$ receives a weight
equal to $mbox{\tt vsize}(u) + mbox{\tt vsize}(v)$. Consequently,
edges which are incident to highly communicating vertices will be
less likely to be cut. However, the communication volume value
returned by this routine is exactly the one which would be returned
by \metis\ with respect to the output partition. Users interested
in minimizing the exact communication volume should consider using
hypergraphs, implemented in \scotch\ as
meshes (see Section~\ref{sec-lib-format-mesh}).

All of the three \metis\ stubs
{\tt METIS\_\lbo Part\lbo Graph\lbo Kway}, {\tt
METIS\_\lbo Part\lbo Graph\lbo Recursive} and {\tt METIS\_\lbo
Part\lbo Graph\lbo VKway} call the same \scotch\ routine, which uses
the \scotch\ default mapping strategy proved to be efficient in most
cases.
\end{itemize}
                                  % Bibliotheque
%%%%%%%%%%%%%%%%%%%%%%%%%%%%%%%%%%%%
%                                  %
% Titre  : s_d.tex                 %
% Sujet  : Manuel de l'utilisateur %
%          du projet 'Scotch'      %
%          Distribution programmes %
% Auteur : Francois Pellegrini     %
%                                  %
%%%%%%%%%%%%%%%%%%%%%%%%%%%%%%%%%%%%

\section{Installation}
\label{sec-install}

Version {\sc \scotchver} of the \scotch\ software package is
distributed as free/libre software under the CeCILL-C free/libre
software license~\cite{cecill}, which is very similar to the GNU LGPL
license. Therefore, it is no longer distributed as a set of binaries,
but instead in the form of a source distribution, which can be
downloaded from the \scotch\ web page at
\url{http://www.labri.fr/~pelegrin/scotch/}~.
\\

All \scotch\ users are welcome to send an e-mail to the author so that
they can be added to the \scotch\ mailing list, and be automatically
informed of new releases and publications.
\\

The extraction process will create a {\tt scotch\_\scotchversub}
directory, containing several subdirectories and files. Please refer
to the files called {\tt LICENSE\_\lbt EN.txt} or
{\tt LICENCE\_\lbt FR.txt}, as well as file
{\tt INSTALL\_\lbt EN.txt}, to see under which conditions your
distribution of \scotch\ is licensed and how to install it.

\subsection{Thread issues}

To enable the use of POSIX threads in some routines, the {\tt
SCOTCH\_\lbt PTHREAD} flag must be set. If your MPI implementation is
not thread-safe, make sure this flag is not defined at compile time.

\subsection{File compression issues}

To enable on-the-fly compression and decompression of various formats,
the relevant flags must be defined. These flags are {\tt COMMON\_\lbt
FILE\_\lbt COMPRESS\_\lbt BZ2} for {\tt bzip2} (de)compression, {\tt
COMMON\_\lbt FILE\_\lbt COMPRESS\_\lbt GZ} for {\tt gzip}
(de)compression, and {\tt COMMON\_\lbt FILE\_\lbt COMPRESS\_\lbt LZMA}
for {\tt lzma} decompression. Note that the corresponding
development libraries must be installed on your system before compile
time, and that compressed file handling can take place only on systems
which support multi-threading or multi-processing. In the first case,
you must set the {\tt SCOTCH\_\lbt PTHREAD} flag in order to take
advantage of these features.

On Linux systems, the development libraries to install are {\tt
libbzip2\_1-\lbt devel} for the {\tt bzip2} format, {\tt zlib1-\lbt
devel} for the {\tt gzip} format, and {\tt liblzma0-\lbt devel} for
the {\tt lzma} format. The names of the libraries may vary according
to operating systems and library versions. Ask your system engineer in
case of trouble.

\subsection{Machine word size issues}
\label{sec-install-inttypesize}

The integer values handled by \scotch\ are based on the
{\tt SCOTCH\_\lbt Num} type, which equates by default to the {\tt int}
C type, corresponding to the {\tt INTEGER} Fortran type, both of which
being of machine word size. To coerce the length of the
{\tt SCOTCH\_\lbt Num} integer type to 32 or 64 bits, one can use the
``{\tt -DINTSIZE32}'' or ``{\tt -DINTSIZE64}'' flags, respectively, or
else use the ``{\tt -DINT=}'' definition, at compile time. For
instance, adding ``{\tt -DINT=long}'' to the {\tt CFLAGS} variable in
the {\tt Makefile.inc} file to be placed at the root of the source
tree will make all {\tt SCOTCH\_\lbt Num} integers become {\tt long} C
integers.

Whenever doing so, make sure to use integer types of equivalent length
to declare variables passed to \scotch\ routines from caller C and
Fortran procedures. Also, because of API conflicts, the
\metis\ compatibility library will not be usable. It is usually safer
and cleaner to tune your C and Fortran compilers to make them
interpret {\tt int} and {\tt INTEGER} types as 32 or 64 bit values,
than to use the aforementioned flags and coerce type lengths in your
own code.

Fortran users also have to take care of another size issue: since
there are no pointers in Fortran~77, the Fortran interface of some
routines converts pointers to be returned into integer indices with
respect to a given array (e.g. see
sections~\ref{sec-lib-func-graphdata},
\ref{sec-lib-func-meshdata}
and~\ref{sec-lib-func-geomdata}).
For 32\_64 architectures, such indices can be larger than the size of
a regular {\tt INTEGER}. This is why the indices to be returned are
defined by means of a specific integer type, {\tt SCOTCH\_Idx}. To
coerce the length of this index type to 32 or 64 bits, one can use the
``{\tt -DIDXSIZE32}'' or ``{\tt -DIDXSIZE64}'' flags, respectively, or
else use the ``{\tt -DIDX=}'' definition, at compile time. For
instance, adding ``{\tt -DIDX="long~long"}'' to the {\tt CFLAGS}
variable in the {\tt Makefile.inc} file to be placed at the root of
the source tree will equate all {\tt SCOTCH\_\lbt Idx} integers to
C {\tt long long} integers. By default, when the size of
{\tt SCOTCH\_\lbt Idx} is not explicitly defined, it is assumed to be
the same as the size of {\tt SCOTCH\_\lbt Num}.
                                  % Distribution
%%%%%%%%%%%%%%%%%%%%%%%%%%%%%%%%%%%%
%                                  %
% Titre  : s_e.tex                 %
% Sujet  : Manuel de l'utilisateur %
%          du projet 'Scotch'      %
%          Exemples d'utilisation  %
% Auteur : Francois Pellegrini     %
%                                  %
%%%%%%%%%%%%%%%%%%%%%%%%%%%%%%%%%%%%

\section{Examples}
\label{sec-examples}

This section contains chosen examples destined to show how the programs
of the \scotch\ project interoperate and can be combined.
It is supposed that the current directory is directory
``{\tt scotch\_\scotchver}'' of the \scotch\ distribution.
Character ``{\tt\bf \%}'' represents the shell prompt.
\begin{itemize}
\item
Partition source graph {\tt brol.grf} into $7$ parts, and save the
result to file {\tt /tmp/brol.map}.
\\

\noi
{\tt
{\bf\%} echo cmplt 7 > /tmp/k7.tgt\\
{\bf\%} gmap brol.grf /tmp/k7.tgt /tmp/brol.map
}
\spa

\noi
This can also be done in a single piped command:

{\tt {\bf\%} echo cmplt 7 | gmap brol.grf - /tmp/brol.map}
\spa

\noi
If compressed data handling is enabled, read the graph as a {\tt gzip}
compressed file, and output the mapping as a {\tt bzip2} file, on the fly:

{\tt {\bf\%} echo cmplt 7 | gmap brol.grf.gz - /tmp/brol.map.bz2}
\item
Partition source graph {\tt brol.grf} into two uneven parts of
respective weights $\frac{4}{11}$ and $\frac{7}{11}$, and save
the result to file {\tt /tmp/brol.map}.
\\

\noi
{\tt
{\bf\%} echo cmpltw 2 4 7 > /tmp/k2w.tgt\\
{\bf\%} gmap brol.grf /tmp/k2w.tgt /tmp/brol.map
}
\spa

\noi
This can also be done in a single piped command:

{\tt {\bf\%} echo cmpltw 2 4 7 | gmap brol.grf - /tmp/brol.map}
\spa

\noi
If compressed data handling is enabled, use {\tt gzip} compressed
streams on the fly:

{\tt {\bf\%} echo cmpltw 2 4 7 | gmap brol.grf.gz - /tmp/brol.map.gz}
\item
Map a 32 by 32 bidimensional grid source graph onto a 256-node hypercube, and
save the result to file {\tt /tmp/brol.map}.
\\

\noi
{\tt {\bf\%} gmk\_m2 32 32 | gmap - tgt/h8.tgt /tmp/brol.map}
\item
Build the {\sc Open Inventor} file {\tt graph.iv} that contains
the display of a source graph the source and geometry files of which
are named {\tt graph.grf} and {\tt graph.xyz}.
\\

\noi
{\tt {\bf\%} gout -Mn -Oi graph.grf graph.xyz - graph.iv}
\spa

\noi
Although no mapping data is required because of the ``{\tt -Mn}'' option,
note the presence of the dummy input mapping file name ``{\tt -}'', which is
needed to specify the output visualization file name.
\item
Given the source and geometry files {\tt graph.grf} and {\tt graph.xyz} of
a source graph, map the graph on a 8 by 8 bidimensional mesh and display
the mapping result on a color screen by means of the public-domain
{\tt ghostview} PostScript previewer.
\\

\noi
{\tt {\bf\%} gmap graph.grf tgt/m8x8.tgt | gout graph.grf graph.xyz '-Op\{c,f,l\}' | ghostview -}
\item
Build a 24-node Cube-Connected-Cycles graph target architecture which will be
frequently used. Then, map compressed source file {\tt graph.grf.gz} onto it,
and save the result to file {\tt /tmp/brol.map}.
\\

\noi
{\tt
{\bf\%} amk\_ccc 3 | acpl - /tmp/ccc3.tgt\\
{\bf\%} gunzip -c graph.grf.gz | gmap - /tmp/ccc3.tgt /tmp/brol.map
}
\spa

\noi
To speed up target architecture loading in the future, the
decomposition-defined target architecture is compiled by means of {\tt acpl}.
\item
Build an architecture graph which is the subgraph of the $8$-node de~Bruijn
graph restricted to vertices labeled $1$, $2$, $4$, $5$, $6$, map graph
{\tt graph.grf} onto it, and save the result to file {\tt /tmp/brol.map}.
\\

\noi
{\tt
{\bf\%} (gmk\_ub2 3; echo 5 1 2 4 5 6) | amk\_grf -L |
gmap graph.grf - /tmp/brol.map}
\spa

\noi
Note how the two input streams of program {\tt amk\_grf} (that is, the
de~Bruijn source graph and the five-elements vertex label list) are
concatenated into a single stream to be read from the standard input.
%% \item
%% Output the pattern of the adjacency matrix associated with graph
%% {\tt graph.grf.gz} to the encapsulated PostScript file {\tt graph.pattern.ps}.
%% \\

%% \noi
%% {\tt
%% {\bf\%} gunzip -c graph.grf.gz | gout - - - -Gn -Mn '-Om\{e\}' graph.pattern.ps}
%% \item
%% Output the pattern of the factored reordered matrix associated with graph
%% {\tt graph.grf} to the encapsulated PostScript file {\tt graph\_\lbt pattern.ps}.
%% \\

%% \noi
%% {\tt
%% {\bf\%} gord graph.grf -F- /dev/null | gout graph.grf - - -Gn -Mn '-Om\{e\}' graph\_\lbt pattern.ps}
\item
Compile and link the user application {\tt brol.c} with the \libscotch\ library,
using the default error handler.
\\

\noi
{\tt
{\bf\%} cc brol.c -o brol -lscotch -lscotcherr -lm}
\spa

\noi
Note that the mathematical library should also be included, after
all of the \scotch\ libraries.
\item
Recompile a program that used \metis\ so that it uses \scotch\ instead.
\\

\noi
{\tt
{\bf\%} cc brol.c -o brol -I\$\{metisdir\} -lscotchmetis -lscotch -lscotcherr -lmetis -lm}
\spa

\noi
Note that the ``{\tt -lscotch\lbt metis}'' option must be placed before the
``{\tt -lmetis}'' one, so that routines that are redefined by \scotch\ are
selected instead of their \metis\ counterpart. When no other
\metis\ routines than the ones redefined by \scotch\ are used, the
``{\tt -lmetis}'' option can be omitted. The ``{\tt -I\$\{metisdir\}}''
option may be necessary to provide the path to the original {\tt metis.h}
include file, which contains the prototypes of all of the \metis\ routines.

\end{itemize}
                                  % Relevant examples
%%%%%%%%%%%%%%%%%%%%%%%%%%%%%%%%%%%%
%                                  %
% Titre  : s_n.tex                 %
% Sujet  : Manuel de l'utilisateur %
%          du projet 'Scotch'      %
%          Codage de nouvelles     %
%          methodes                %
% Auteur : Francois Pellegrini     %
%                                  %
%%%%%%%%%%%%%%%%%%%%%%%%%%%%%%%%%%%%

\section{Adding new features to \scotch}
\label{sec-coding}

Since \scotch\ is free/libre software, users have the ability
to add new features to it. Moreover, as \scotch\ is intended to be a
testbed for new partitioning and ordering algorithms, it has been
developed in a very modular way, to ease the development and inclusion
of new partitioning and ordering methods to be called within
\scotch\ strategies.

All of the source code for partitioning and ordering methods for
graphs and meshes is located in the {\tt src/\lbt libscotch/} source
subdirectory. Source file names have a very regular pattern, based on
the internal data structures they handle.

\subsection{Graphs and meshes}

The basic structures in \scotch\ are the {\tt Graph} and {\tt Mesh}
structures, which model a simple symmetric graph the definition of
which is given in file {\tt graph.h}, and a simple mesh, in the form
of a bipartite graph, the definition of which is given in file {\tt
mesh.h}, respectively. From this structure are derived enriched graph
and mesh structures:
\begin{itemize}
\item
{\tt Bgraph}, in file {\tt bgraph.h}: graph with bipartition, that is,
edge separation, information attached to it;
\item
{\tt Kgraph}, in file {\tt kgraph.h}: graph with mapping information
attached to it;
\item
{\tt Hgraph}, in file {\tt hgraph.h}: graph with halo information
attached to it, for computing graph orderings;
\item
{\tt Vgraph}, in file {\tt vgraph.h}: graph with vertex bipartition
information attached to it;
\item
{\tt Hmesh}, in file {\tt hmesh.h}: mesh with halo information
attached to it, for computing mesh orderings;
\item
{\tt Vmesh}, in file {\tt vmesh.h}: graph with vertex bipartition
information attached to it.
\end{itemize}
As version {\sc \scotchver} of the \libscotch\ does not provide mesh
mapping capabilities, neither {\tt Bmesh} nor {\tt Kmesh} structures
have been defined to date, but this work is in progress, and these
features should be available in the upcoming releases.

All of the structures are in fact defined as {\tt typedef}ed types.

\subsection{Methods and partition data}

Methods are routines which take one of the above structures as input,
and update the fields of the given structure according to the
implemented algorithm. Initial methods will behave irrespective of the
former values of the structure (like graph growing methods, which
compute partitions from scratch), while refinement methods must be
provided an existing partition to improve.

In addition to the topological description of the underlying graph,
the working graph and mesh structures comprise variables describing
the current state of the vertex or edge partition. In all cases is
provided a partition array called {\tt parttax}, of size equal to the
number of graph vertices, which tells which part every vertex is
assigned to. Other variables comprise the communication load and the
load imbalance of the current cut, that is, all of the data necessary
to measure the quality of a partition. Some other data are also often
provided, such as the number of vertices in each part and the list of
frontier vertices. They are not relevant to measure the quality of
the partition, but to improve the speed of computations. They are used
for instance in the multilevel algorithms to compute incremental
updates of the current partition state, without having to recompute
these values from scratch by considering all of the graph vertices.
Implementers of new methods are highly encouraged to use these
variables to speed-up their computations, taking examples on typical
algorithms such as the multilevel or Fiduccia-Mattheyses ones.

\subsection{Adding a new method to \scotch}

We will assume in this section that the new method to add is a graph
separation method. The procedure explained below is exactly the same
for graph bipartitioning, graph mapping, graph ordering, mesh
separation, or mesh ordering methods.

Please proceed as explained below.
\begin{enumerate}
\item
Write the code of the method itself. First, choose a free two-letter
code to describe your method, say ``xy''. In the {\tt libscotch}
source directory, create files {\tt vgraph\_\lbt separate\_\lbt xy.c}
and {\tt vgraph\_\lbt separate\_\lbt xy.h}, basing on existing
files such as {\tt vgraph\_\lbt separate\_\lbt gg.c} and {\tt
vgraph\_\lbt separate\_\lbt gg.h}, for instance.

If the method is complex, it can be split across several other files,
which will be named {\tt vgraph\_\lbt separate\_\lbt xy\_\lbt first\lbt
module\lbt name.c}, {\tt vgraph\_\lbt separate\_\lbt xy\_\lbt second\lbt
module\lbt name.c}, eventually with matching header files.

If the method has parameters, create a structure called {\tt
Vgraph\lbt Separate\lbt Xy\lbt Param}, which contains fields of
types that can be handled by the strategy parser, such as the
{\tt INT} generic integer type (see below), or {\tt double}, for
instance.

The execution of your method should result in the setting or in the
updating of the {\tt Vgraph} structure that is passed to it. See its
definition in {\tt vgraph.h} and read several simple graph separation
methods, such as {\tt vgraph\_\lbt separate\_\lbt zr.c}, to figure out
what all of its parameters mean.

At the end of your method, always call, when the {\tt SCOTCH\_\lbt
DEBUG\_\lbt VGRAPH2} debug flag is set, the {\tt vgraph\lbt Check}
routine, to avoid the spreading of eventual bugs to other parts of
the \libscotch\ library.
\item
Add the method to the parser tables. The files to update are
{\tt vgraph\_\lbt separate\_\lbt st.c} and {\tt vgraph\_\lbt
separate\_\lbt st.h}, where ``{\tt st}'' stands for ``strategy''.

First, edit {\tt vgraph\_\lbt separate\_\lbt st.h}. In the {\tt
Vgraph\lbt Separate\lbt St\lbt Method\lbt Type} enumeration,
add a line for your new method {\tt VGRAPH\lbt SEPA\lbt ST\lbt
METH\lbt XY}. Then, edit {\tt vgraph\_\lbt separate\_\lbt st.c},
where all of the remaining actions take place.

In the top of the file, add a {\tt \#include} directive to include
{\tt vgraph\_\lbt separate\_\lbt xy.h}.

If the method has parameters, create a {\tt vgraph\lbt separate\lbt
default\lbt xy} C union, basing on an existing one, and fill it with
the default values of your method parameters.

In the {\tt vgraph\lbt separate\lbt st\lbt meth\lbt tab} method array,
add a line for the new method. To do so, choose a free single-letter
code that will be used to designate the new method in strategy strings.
If the method has parameters, the last field should be a pointer to
the default structure, else it should be set to {\tt NULL}.

If the method has parameters, update the {\tt vgraph\lbt separate\lbt
st\lbt para\lbt tab} parameter array. Add one data block per
parameter. The first field is the name of the method to which the
parameter applies, that is, {\tt VGRAPH\lbt SEPA\lbt ST\lbt
METH\lbt XY}. The second field is the type of the parameter, which can
be:
\begin{itemize}
\item
{\tt STRATPARAMCASE}: the support type is an {\tt int}. It receives
the index in the case string, which is provided as the last field of
the parameter line, of the given case character;
\item
{\tt STRATPARAMDOUBLE}: the support type is a {\tt double};
\item
{\tt STRATPARAMINT}: the support type is an {\tt INT}, which
is the generic integer type handled internally by \scotch. This type
has variable extent, depending on compilation flags,
as described in Section~\ref{sec-lib-inttypesize};
\item
{\tt STRATPARAMSTRING}: a (small) character string;
\item
{\tt STRATPARAMSTRAT}: strategy. For instance, the graph ordering
method by nested dissection takes a vertex partitioning strategy as
one of its parameters, to compute the vertex separators.
\end{itemize}
The fourth and fifth fields are the address of the location of the
default structure and the address of the parameter within this default
structure, respectively. From these two values can be computed at run
time the offset of the parameter within any instance of the parameter
structure, which is used to fill the actual structures in the parsed
strategy evaluation tree.
The value of the sixth parameter depends on the type of the
parameter. It should be {\tt NULL} for {\tt STRAT\lbt PARAM\lbt
DOUBLE} and {\tt STRAT\lbt PARAM\lbt INT} parameters, points to the
string of available case letters for {\tt STRAT\lbt PARAM\lbt CASE}
parameters, points to the target string buffer for {\tt STRAT\lbt
PARAM\lbt STRING} parameters, and points to the relevant method
parsing table for {\tt STRAT\lbt PARAM\lbt STRAT} parameters.
\item
Edit the makefile of the \libscotch\ source directory to enable the
compilation and linking of the method. Depending on \libscotch\
versions, this makefile is either called {\tt Makefile} or {\tt
make\_\lbt gen}.
\item
Compile in debug mode and experiment with your routine, by creating
strategies that contain its single-letter code.
\item
To change the default strategy string used by the \libscotch\ library,
update file {\tt library\_\lbt graph\_\lbt order.c}, since it is
the graph ordering routine which makes use of graph vertex separation
methods to compute separators for the nested dissection ordering method.
\end{enumerate}

\subsection{Licensing of new methods and of derived works}

According to the terms of the CeCILL-C license~\cite{cecill} under
which the \scotch\ software package is distributed, the works that are
carried out to improve and extend the \libscotch\ library must be
licensed under the same terms. Basically, it means that you will have
to distribute the sources of your new methods, along with the sources
of \scotch, to any recipient of your modified version of the
\libscotch, and that you grant these recipients the same rights of
update and redistribution as the ones that are given to you under the
terms of CeCILL-C. Please read it carefully to know what you can do
and cannot do with the \scotch\ distribution.
\\

You should have received a copy of the CeCILL-C
license along with the \scotch\ distribution; if not, please
browse the CeCILL website at
\url{http://www.cecill.info/licenses.en.html}.
                                  % Addition of a new method

%% Remerciements.

\section*{Credits}

I wish to thank all of the following people:
\begin{itemize}
\item
Patrick Amestoy collaborated to the design of the Halo Approximate
Minimum Degree algorithm~\cite{peroam99} that had been embedded into
\scotch\ {\sc 3.3}, and provided versions of his Approximate Minimum
Degree algorithm, available since version {\sc 3.2}, and of his
Halo Approximate Minimum Fill algorithm, available since version
{\sc 3.4}. He designed the mesh versions of the approximate
minimum degree and approximate minimum fill algorithms, which are
available since version {\sc 4.0};
\item
S\'ebastien Fourestier coded the mapping with fixed vertices,
remapping, and remapping with fixed vertices sequential routines that
are available since version {\sc 6.0};
\item
Jun-Ho Her coded the graph partitioning with overlap routines that
were introduced in the unpublished {\sc 5.2} release, and publicly
released in version {\sc 6.0};
\item
Alex Pothen kindly provided a version of his Multiple Minimum Degree
algorithm, which was embedded into \scotch\ from version {\sc 3.2} to
version {\sc 3.4};
\item
Luca Scarano, visiting Erasmus student from the {\it Universit\'a
degli Studi di Bologna}, coded the multilevel graph algorithm
in \scotch\ {\sc 3.1};
\item
Yves Secretan contributed to the MinGW32 port;
\item
David Sherman proofread version {\sc 3.2} of this manual.
\end{itemize}

%% Bibliographie.

\bibliographystyle{plain}
\bibliography{s}

\end{document}
