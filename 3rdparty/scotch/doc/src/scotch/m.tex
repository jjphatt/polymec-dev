%%%%%%%%%%%%%%%%%%%%%%%%%%%%%%%%%%%%
%                                  %
% Titre  : m.tex                   %
% Sujet  : Manuel de maintenance   %
%          de Scotch 4.0           %
%          Corps du document       %
% Auteur : Francois Pellegrini     %
%                                  %
%%%%%%%%%%%%%%%%%%%%%%%%%%%%%%%%%%%%

%% Formatage et pagination.

\documentclass{article}
\usepackage{a4}
\usepackage[dvips]{graphicx}
%\documentstyle[11pt,a4,fullpage,epsf]{article}
%\textwidth      16.0cm
%\oddsidemargin   -0.5cm
%\evensidemargin  -0.5cm
%\marginparwidth  0.0cm
%\marginparsep    0.0cm
%\marginparpush   0.0cm
%\topmargin        0.5cm
%\headheight      0.0cm
%\headsep         0.0cm
%\textheight     25.0cm
%\footheight      0.0cm
%\footskip        0.0cm

\sloppy                                          % Gestion des overfull hbox
\renewcommand{\baselinestretch}{1.05}            % Hauteur lignes x 1.05

\setcounter{secnumdepth}{3}                      % Sous-sous-sections numerotees
\setcounter{tocdepth}{3}                         % Sous-sous-sections dans la table

%% Macros et commandes utiles.

\makeatletter
\@definecounter{enumv}                           % 8 niveaux d'itemizations
\@definecounter{enumvi}
\@definecounter{enumvii}
\@definecounter{enumviii}
\def\itemize{\ifnum \@itemdepth >8 \@toodeep\else \advance\@itemdepth \@ne
\edef\@itemitem{labelitem\romannumeral\the\@itemdepth}%
\list{\csname\@itemitem\endcsname}{\def\makelabel##1{\hss\llap{##1}}}\fi}
\let\enditemize =\endlist

\def\@iteme[#1]{\if@noparitem \@donoparitem      % Item long pour options
  \else \if@inlabel \indent \par \fi
         \ifhmode \unskip\unskip \par \fi
         \if@newlist \if@nobreak \@nbitem \else
                        \addpenalty\@beginparpenalty
                        \addvspace\@topsep \addvspace{-\parskip}\fi
           \else \addpenalty\@itempenalty \addvspace\itemsep
          \fi
    \global\@inlabeltrue
\fi
\everypar{\global\@minipagefalse\global\@newlistfalse
          \if@inlabel\global\@inlabelfalse
             \setbox\@tempboxa\hbox{#1}\relax
             \hskip \itemindent \hskip -\parindent
             \hskip -\labelwidth \hskip -\labelsep
             \ifdim \wd\@tempboxa > \labelwidth
               \box\@tempboxa\hfil\break
             \else
               \hbox to\labelwidth{\box\@tempboxa\hfil}\relax
               \hskip \labelsep
             \fi
             \penalty\z@ \fi
          \everypar{}}\global\@nobreakfalse
\if@noitemarg \@noitemargfalse \if@nmbrlist \refstepcounter{\@listctr}\fi \fi
\ignorespaces}
\def\iteme{\@ifnextchar [{\@iteme}{\@noitemargtrue \@iteme[\@itemlabel]}}

\let\@Hxfloat\@xfloat
\def\@xfloat#1[{\@ifnextchar{H}{\@HHfloat{#1}[}{\@Hxfloat{#1}[}}
\def\@HHfloat#1[H]{%
\expandafter\let\csname end#1\endcsname\end@Hfloat
\vskip\intextsep\def\@captype{#1}\parindent\z@
\ignorespaces}
\def\end@Hfloat{\vskip \intextsep}
\makeatother

\newcommand{\bn}{\begin{displaymath}}            % Equations non-numerotees
\newcommand{\en}{\end{displaymath}}
\newcommand{\bq}{\begin{equation}}               % Equations numerotees
\newcommand{\eq}{\end{equation}}

\newcommand{\lbo}{\linebreak[0]}
\newcommand{\lbt}{\linebreak[2]}
\newcommand{\noi}{{\noindent}}                   % Pas d'indentation
\newcommand{\spa}{{\protect \vspace{\bigskipamount}}} % Espace vertical

\newcommand{\eg}{{\it e\@.g\@.\/\ }}             % e.g.
\newcommand{\ie}{{\it i\@.e\@.\/\ }}             % i.e.

\newcommand{\scotch}{{\sc Scotch}}               % "scotch"
\newcommand{\libscotch}{{\sc libScotch}}         % "libscotch"

%% Page de garde.

\begin{document}

\date{\today}

\title{\includegraphics{s_f_logo.ps}\\[1em]
       {\LARGE\bf \libscotch\ {\sc 4.0} Maintainer's Guide}\\[1em]}

\author{Fran\c cois Pellegrini\\
ScAlApplix project, INRIA Bordeaux Sud-Ouest\\
ENSEIRB \& LaBRI, URM CNRS 5800\\
Universit\'e Bordeaux~I\\
351 cours de la Lib\'eration, 33405 TALENCE, FRANCE\\
{\tt pelegrin@labri.fr}}

\maketitle

\begin{abstract}
This document briefly describes the internals of the \libscotch\
library, and mainly focuses on the way for contributors to add new
partitioning and ordering methods to it.
\end{abstract}

\clearpage

%% Table des matieres.

\tableofcontents

%% Corps du document.

%%%%%%%%%%%%%%%%%%%%%%%%%%%%%%%%%%%%
%                                  %
% Titre  : s_i.tex                 %
% Sujet  : Manuel de maintenance   %
%          du projet 'Scotch'      %
%          Introduction            %
% Auteur : Francois Pellegrini     %
%                                  %
%%%%%%%%%%%%%%%%%%%%%%%%%%%%%%%%%%%%

\section{Introduction}

This document is a starting point for the persons interested in using
\scotch\ as a testbed for their new partitioning methods, and/or
willing to contribute to it by making these methods available to the
rest of the scientific community.

Much information is missing. If you need specific information, please
send an e-mail, so that relevant additional information can be added
to this document.
                                  % Introduction
%%%%%%%%%%%%%%%%%%%%%%%%%%%%%%%%%%%%
%                                  %
% Titre  : m_s.tex                 %
% Sujet  : Manuel de maintenance   %
%          du projet 'Scotch'      %
%          Structure generale      %
% Auteur : Francois Pellegrini     %
%                                  %
%%%%%%%%%%%%%%%%%%%%%%%%%%%%%%%%%%%%

\section{General structure of the \libscotch\ library}
\label{sec-structure}

\subsection{Naming conventions}

All of the files of the \scotch\ project have been written using
strict coding conventions, to ease maintenance and further extension
by external contributors. Therefore, contributors are {\bf strongly}
invited to follow these coding conventions so as to ease the work of
their followers.

\subsubsection{Variables}

Variables are named by specialization, with prefixes and suffixes.

Common prefixes are:
\begin{itemize}
\item
act : active, wrt. source
\item
src : source, wrt. active
\item
tgt : target.
\item
coar : coarse, wrt. fine.
\item
fine : fine, wrt. coarse.
\item
mult : multinode, for coarsening.
\end{itemize}

Common radicals are:
\begin{itemize}
\item
vert : vertex.
\item
velo : vertex load.
\item
vnum : vertex number, used as index, for instance in {\tt vnumtab}).
\item
vlbl : vertex label (for interface).
\item
edge : edge.
\item
edlo : edge load.
\end{itemize}

Common suffices are:
\begin{itemize}
\item
{\tt nbr} : number of (instances of).
\item
{\tt num} : number of some instance of.
\item
{\tt val} : value of.
\item
{\tt sum} : sum of several values of.
\item
{\tt tab} : pointer to the first memory element of an array of, for
instance as returned by a {\tt mem\lbt Alloc} routine.
\item
{\tt tax} : (for ``{\it table access}'') pointer to the first element
of an array that is to be accessed by means of based indices. The
\libscotch\ library can accept data structures that come both from
FORTRAN, where array indices start at $1$, and C, where they start at
$0$. By accessing ``{\tt tax}'' arrays, where the {\tt tax} pointer is
equal to the {\tt tab} pointer minus the value of the indexing base,
this is completely transparent for coding, save for memory allocation
and freeing operations, which must operate on {\tt tab} pointers only.
\item
{\tt tnd} : pointer to the end of an array of (mostly used as a bound
in loops).
\item
{\tt ptr} : pointer to an instance of.
\end{itemize}

For instance, {\tt coarvertnum} is a variable that holds the number of
a coarse vertex, within some coarsening algorithm. It is also easy to
guess how to name a variable that holds a pointer to an active edge.

\subsubsection{Functions}

Routines that operate on data structures are named by specialization,
from the name of the structure type they apply to.

Common suffixes are:
\begin{itemize}
\item
{\tt Init} : initialization of a structure (object) passed as parameter.
\item
{\tt Free} : freeing of the external structures of the object, to save
space. The object may still be used after it is initialized again.
\item
{\tt Exit} : freeing of the internal structures of the object. The
object must not be passed to other routines after the {Exit} method
has been called.
\item
{\tt Copy} : make a fully operational, independent, copy of the object.
\item
{\tt Load} : load from stream.
\item
{\tt Save} : save to stream.
\item
{\tt View} : display internal structures and statistics, for debugging
purposes.
\item
{\tt Check} : check internal coherence, for debug.
\end{itemize}

\subsection{Structure of the library}

%\subsubsection{Modules}

All of the routines that comprise the \libscotch\ libraries are
grouped by the type of data structure onto which they apply and by
function, in an object-oriented manner. This organization is reflected
into the naming and contents of all of the source files.

The main modules of the \libscotch\ library are the following :
\begin{itemize}
\item
{\tt arch} : target architectures used by the static mapping methods.
\item
{\tt bgraph} : graph edge bipartitioning methods.
\item
{\tt graph} : basic graph handling methods.
\item
{\tt hgraph} : graph ordering methods. These are based on an extended
``halo'' graph structure, thus for the name.
\item
{\tt hmesh} : mesh ordering methods.
\item
{\tt kgraph} : k-way graph partitioning methods.
\item
{\tt library} : interface routines for the \libscotch\ library.
\item
{\tt mapping} : definition of the mapping structure.
\item
{\tt mesh} : basic mesh handling methods.
\item
{\tt order} : definition of the ordering structure.
\item
{\tt parser} : strategy parsing routines, based on a Lex-Yacc
parser.
\item
{\tt vgraph} : graph vertex bipartitioning methods.
\item
{\tt vmesh} : mesh node bipartitioning methods.
\end{itemize}

Each of the file names is prefixed by the name of the module, followed
by one or two words that describe the type of action performed by the
routines of the file.
For instance, in module {\tt bgraph} :
\begin{itemize}
\item
{\tt bgraph.h} is the header file that defines the {\tt bgraph}
structure,
\item
{\tt bgraph\_bipart\_fm.[ch]} are the files that contain the
Fiduccia-Mattheyses-like graph bipartitioning method,
\item
{\tt bgraph\_check.c} is the file that contains the consistency
checking routine for {\tt Bgraph} structures,
\end{itemize}
and so on. Every source file has a comments header briefly describing
the purpose of the routines it contains.

%This section lists the main structures used within \libscotch. Each of
%these is defined in a C header file of same name in the {\tt
%libscotch\_\lbt 4.x/\lbo src} directory. For instance, the {\tt Graph}
%structure is defined in file {\tt graph.h}, the {Vmesh} structure is
%defined in file {\tt vmesh.h}, and so on.
                                  % Structure de la libScotch
%%%%%%%%%%%%%%%%%%%%%%%%%%%%%%%%%%%%
%                                  %
% Titre  : m_m.tex                 %
% Sujet  : Manuel de maintenance   %
%          du projet 'Scotch'      %
%          Ajout d'une methode     %
% Auteur : Francois Pellegrini     %
%                                  %
%%%%%%%%%%%%%%%%%%%%%%%%%%%%%%%%%%%%

\section{Adding a method to the \libscotch\ library}
\label{sec-method}

The \libscotch\ has been carefully designed so as to allow external
contributors to add their new partitioning or ordering methods, and
to use \scotch\ as a testbed for them.

\subsection{What to add}

There are currently six types of methods which can be added:
\begin{itemize}
\item
k-way graph mapping methods, in module {\tt kgraph},
\item
graph bipartitioning methods by means of edge separators, in module
{\tt bgraph}, used by the mapping method by dual recursive
bipartitioning, implemented in {\tt kgraph\_\lbt map\_\lbt rb.[ch]},
\item
graph ordering methods, in module {\tt vmesh},
\item
graph separation methods by means of vertex separators, in module
{\tt vgraph}, used by the nested dissection ordering method
implemented in {\tt hgraph\_\lbt order\_\lbt nd.[ch]},
\item
mesh separation methods with node separators, in module
{\tt vmesh},
\item
mesh separation methods with vertex separators, in module
{\tt vmesh}, used by the nested dissection ordering method
implemented in {\tt hmesh\_\lbt order\_\lbt nd.[ch]}.
\end{itemize}
Every method of these six types operates on instances of augmented
graph structures that contain, in addition to the graph topology,
data related to the current state of the partition or of the
ordering. For instance, all of the graph bipartitioning methods
operate on an instance of a {\tt Bgraph}, defined in {\tt bgraph.h},
and which contains fields such as {\tt compload0}, the current load
sum of the vertices assigned to the first part, {\tt commload}, the
load sum of the cut edges, etc.

In order to understand better the meaning of each of the fields
used by some augmented graph or mesh structure, contributors can read
the code of the consistency checking routines, located in files ending
in {\tt \_check.c}\enspace, such as {\tt bgraph\_check.c} for a
{\tt Bgraph} structure. These routines are regularly called during the
execution of the debug version of \scotch\ to ease bug tracking. They
are time-consuming but proved very helpful in the development and
testing of new methods.

\subsection{Where to add}

Let us assume that you want to code a new graph separation
routine. Your routine will operate on a {\tt Vgraph} structure, and
thus will be stored in files called {\tt vgraph\_\lbt separate\_
xy\lbt .[ch]}, where {\tt xy} is a two-letter reminder of the name
of your algorithm. Look into the \libscotch\ source directory for
already used codenames, and pick a free one.
In case you have more that one single source file, use extended names,
such as {\tt vgraph\_\lbt separate\_\lbt xy\_\lbt subname\lbt
.[ch]}\enspace .

In order to ease your coding, copy the files of a simple and already
existing method and use them as a pattern for the interface of your
new method. Some methods have an optional parameter data structure,
others do not. Browse through all existing methods to find the one
that looks closest to what you want.

Some methods can be passed parameters at run time from the strategy
string parser. These parameters can be of fixed types only. These
types are:
\begin{itemize}
\item
an integer ({\tt int}) type,
\item
an floating-point ({\tt double}) type,
\item
an enumerated ({\tt char}) type : this type is used to make a
choice among a list of single character values, such as ``{\tt
yn}''. It is more readable than giving integer numerical values to
method option flags,
\item
a strategy (\scotch\ {\tt Strat} type) : a method can be passed a
sub-strategy of a given type, which can be run on an augmented graph
of the proper type. For instance, the nested dissection method in {\tt
hgraph\_\lbt order\_\lbt nd\lbt .c} uses a graph separation strategy
to compute its vertex separators.
\end{itemize}

\subsection{Declaring the new method to the parser}

Once the new method has been coded, its interface must be known to the
parser, so that it can be used in strategy strings. All of this is
done in the module strategy method files, the name of which always end
in {\tt \_st.[ch]}, that is, {\tt vgraph\_\lbt separate\_\lbt st.[ch]}
for the {\tt vgraph} module. Both files are to be updated.

In the header file {*\_st.h}, a new identifier must be created for the
new method in the {\tt St\lbt Method\lbt Type} enumeration type,
preferrably placed in alphabetical order.

In file {*\_st.c}, there are several places to update.
First, in the beginning of the module file, the header file of the new
method, {\tt vgraph\_\lbt separate\_\lbt xy\lbt .h} in this example,
must be added in alphabetical order to the list of included method
header files.

Then, if the new method has parameters, an instance of the method
parameter structure must be created, which will hold the default
values for the method. This is in fact a {\tt union} structure,
of the following form :
{\tt\begin{verbatim}
static union {
  VgraphSeparateXyParam     param;
  StratNodeMethodData       padding;
} vgraphseparatedefaultxy = { { ... } };
\end{verbatim}}
where the dots should be replaced by the list of default values of the
fields of the {\tt Vgraph\lbt Separate\lbt Xy\lbt Param} structure.
Note that the size of the {\tt Strat\lbt Node\lbt Method\lbt Data}
structure, which is used as a generic padding structure, must always
be greater than or equal to the size of each of the parameter
structures. If your new parameter structure is larger, you will have
to update the size of the {\tt Strat\lbt Node\lbt Method\lbt Data}
type in file {\tt parser.h}\enspace. The size of the {\tt Strat\lbt
Node\lbt Method\lbt Data} type does not depend directly on the size of
the parameter structures (as could have been done by making it an union
of all of them) so as to to reduce the dependencies between the files
of the library. In most cases, the default size is sufficient, and a
test is added in the beginning of all method routines to ensure it is
the case in practice.

Finally, the first two method tables must be filled accordingly. In the
first one, of type {\tt Strat\lbt Method\lbt Tab}, one must add a new
line linking the method identifier to the character code used to name
the method in strategy strings (which must be chosen among all of
the yet unused letters), the pointer to the routine, and the pointer
to the above default parameter structure if it exists (else, a {\tt
NULL} pointer must be given).
In the second one, of type {\tt Strat\lbt Param\lbt Tab}, one must add
one line per method parameter, giving the identifier of the method,
the type of the parameter, the name of the parameter in the strategy
string, the base address of the default parameter structure, the
actual address of the field in the parameter structure (both fields
are required because the relative offset of the field with respect to
the starting address of the structure cannot be computed at
compile-time), and an optional pointer that references either the
strategy table to be used to parse the strategy parameter (for
strategy parameters) or a string holding all of the values of the
character flags (for an enumerated type), this pointer being set to
{\tt NULL} for all of the other parameter types (integer and floating
point).

\subsection{Adding the new method to the makefile}

Of course, in order to be compiled, the new method must be added to
the {\tt makefile} of the {\tt libscotch} source directory. There are
several places to update.

First, you have to create the entry for the new method source files
themselves. The best way to proceed is to search for the one of an
already existing method, such as {\tt vgraph\_\lbt separate\_\lbt fm},
and copy it to the right neighboring place, preferrably following the
alphabetical order.

Then, you have to add the new header file to the dependency list of
the module strategy method, that is, {\tt vgraph\_\lbt separate\_\lbt
st} for graph separation methods. Here again, search for the
occurences of string {\tt vgraph\_\lbt separate\_\lbt fm} to see where
it is done.

Finally, add the new object file to the component list of the {\tt
libscotch} library file.

Once all of this is done, you can recompile \scotch\ and be able to
use your new method in strategy strings.
                                  % Ajout d'une methode

\end{document}
