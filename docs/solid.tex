\documentclass[12pt]{article}

\tolerance=750

%\input{../Common/fontset.tex}
%\input{../Common/preamble.tex}
%\input{../Common/macros.tex}

\usepackage{makeidx}
\usepackage{amsmath, amsthm, amssymb}
\usepackage{graphicx}
\usepackage{subfigure}
\usepackage{url} % For web pages
\usepackage{listings} % For source code with captions.
\lstset{numbers=left, stepnumber=2, frame=single, basicstyle=\footnotesize}

% Useful Commands
\newcommand{\labelSec}[1]{\label{sec:#1}}
\newcommand{\labelEq}[1]{\label{eq:#1}}
\newcommand{\labelFig}[1]{\label{fig:#1}}
\newcommand{\labelTab}[1]{\label{tab:#1}}
\newcommand{\labelApp}[1]{\label{app:#1}}
\newcommand{\labelListing}[1]{\label{listing:#1}}
\newcommand{\refEq}[1]{(\ref{eq:#1})}
\newcommand{\refFig}[1]{Figure \ref{fig:#1}}
\newcommand{\refSec}[1]{Section \ref{sec:#1}}
\newcommand{\refTab}[1]{Table \ref{tab:#1}}
\newcommand{\refApp}[1]{Appendix \ref{app:#1}}
\newcommand{\refListing}[1]{Listing \ref{listing:#1}}
\newcommand{\polymec}{\textsc{polymec} }
\newcommand{\Lame}{Lam\'e } 
\newcommand{\newEq}[2]{\begin{equation} \label{eq:#1} #2 \end{equation}}

% Mathy math.
\renewcommand{\vec}[1]{\mathbf{#1}}
\newcommand{\tens}[1]{\mathbf{#1}}
\newcommand{\ddx}[1]{\frac{\partial #1}{\partial x}}
\newcommand{\ddy}[1]{\frac{\partial #1}{\partial y}}
\newcommand{\ddz}[1]{\frac{\partial #1}{\partial z}}
\newcommand{\ddt}[1]{\frac{\partial #1}{\partial t}}
\newcommand{\DDt}[1]{\frac{\text{d}#1}{\text{d}t}}
\newcommand{\ddr}[1]{\frac{\partial #1}{\partial r}}
\newcommand{\diverg}[1]{\nabla\cdot#1}
\newcommand{\grad}[1]{\nabla#1}
\newcommand{\curl}[1]{\nabla \times #1}
\newcommand{\laplacian}[1]{\nabla^2 #1}
\newcommand{\half}{\small\frac{1}{2}}
\newcommand{\third}{\small\frac{1}{3}}
\newcommand{\dV}{~\mathrm{d}V}
\newcommand{\dS}{~\mathrm{d}S}
\newcommand{\stress}{\boldsymbol\sigma}
\newcommand{\strain}{\boldsymbol\epsilon}
\DeclareMathOperator{\trace}{Tr}
\DeclareMathOperator{\erf}{erf}
\DeclareMathOperator{\sgn}{sgn}

% Make pdflatex act like latex.
%\pdfadjustspacing=1

\title{Polymec's solid mechanics model}
\author{Jeffrey Johnson \\ 
jjphatt@gmail.com}

\makeindex % tell \index to actually write the .idx file

\begin{document}

\maketitle

%\input{disclaimer.tex} 

\section{Overview}\labelSec{Overview}

We discuss here the specification for a \polymec model that solves the equations 
of solid mechanics, with attention to fractures and geomechanical material models.
We are interested in the static approximation, in which a material responds instantaneously
to stresses. In this description, a material defines a domain $\Omega$ containing a 
field $\vec{u}(\vec{x}, t)$ describing the {\em displacement} of the point $\vec{x}$ from
its equilibrium position at time $t$. If $\stress(\vec{u})$ is the stress in the 
material resulting from the displacement $\vec{u}$, then the static equation of solid 
mechanics is written

\begin{equation}
\diverg{\stress(\vec{u})} = \vec{f} \labelEq{staticSolidMechEq}
\end{equation}

\noindent
where $\vec{f}$ is some force (per unit volume) within the material. The displacement 
field $\vec{u}$ is subject to some set of boundary conditions on the material's surface 
$\Gamma = \partial\Omega$.

The equation for calculating the stress tensor $\stress$ is given by the 
{\em stress-strain relations}, which form an algebraic relationship between 
$\stress$ and the {\em strain} tensor, which is defined as the antisymmetric 
portion of the gradient of the displacement field:

\begin{equation}
\strain = \half\left[\grad{\vec{u}} + (\vec{u})^T\right]
\end{equation}

\noindent
In ``linear" materials, the stress and strain tensors are related linearly through 
the (rank 4) {\em stiffness} tensor $\tens{C}$:

\begin{equation}
\sigma_{ij} = C_{ijkl} \epsilon_{kl}
\end{equation}

\noindent
Here, we have used index notation, since vector notation does not convey the significance 
of higher-rank tensors. In practice, many linear materials are characterized in terms 
of coefficients representing stiffness tensors with high degrees of symmetry, which eliminate
many of the components. For example, the stress-strain relations in a linear  isotropic medium 
are described by a {\em bulk modulus} $K$ and a {\em shear modulus} $\mu$:

\begin{equation}
\sigma_{ij} = K \delta_{ij}\epsilon_{kk} + 2\mu(\epsilon_{ij} - \third\delta_{ij}\epsilon_{kk})
\end{equation}

\noindent
Equivalently, such a material may described in terms of the {\em \Lame coefficients}, which 
consist of $\mu$ and the parameter $\lambda = K - 2\mu/3$. Finally, many isotropic materials are 
described in terms of Poisson's Ratio $\nu$ and Young's modulus $E$.

\subsection*{Boundary conditions}

There are two basic types of boundary conditions that can be applied to the 
displacement field:

\begin{enumerate}
 \item {\em Displacements}, in which the value of $\vec{u}$ itself is prescribed at the boundary:
 \begin{equation}
 \vec{u}(\vec{x}) = \vec{u}_D(\vec{x}), ~~\vec{x} \in \Gamma \labelEq{displacementBC}
 \end{equation}
 \item {\em Tractions}, in which the normal force is given at the boundary:
 \begin{equation}
 \vec{n}\cdot\stress\big|_\vec{x} = \tau, ~~\vec{x} \in \Gamma \labelEq{tractionBC}
 \end{equation}
 \noindent
\end{enumerate}

\subsection*{Model Equations}

Our \polymec model will treat solid mechanics by solving the following partial differential equation 
with a solution $\vec{u}(\vec{x}, t)$, defined on the domain $\Omega$ with a boundary 
$\Gamma = \partial\Omega = \bigcup(\Gamma_D, \Gamma_T)$, with the two $\Gamma$ regions 
corresponding to the two types of boundary conditions:

\begin{align}
\diverg\stress(\vec{u}) &= \vec{f}(\vec{u}, \vec{x}, t), ~~\vec{x} \in \Omega \labelEq{modelEq} \\
\vec{u}(\vec{x}, t) &= \vec{u}_D(\vec{x}, t), ~~\vec{x} \in \Gamma_D, \notag \\
\vec{n}\cdot\stress &= \tau(\vec{x}, t) ~~\vec{x} \in \Gamma_T \notag
\end{align}

\noindent
Here, we have added a parameter $t$ to represent a time in a simulation, which allows us to 
make use of time-dependent source functions and boundary conditions for various model problems. 

\section{Numerical Formulations}\labelSec{Numerics}

\polymec targets two different numerical formulations for its applications:

\begin{enumerate}
 \item Finite-volume methods on arbitrary polyhedral meshes (FV).
 \item Finite-volume-like methods on point clouds that use neighbor searches 
       to construct topological relationships in lieu of a mesh. The particular 
       family of methods we are concerned with here are known as {\em Finite Volume Particle Methods} (FVPM), 
       described in \cite{Hietel2000}, \cite{Teleaga2005}, and \cite{Nestor2009}.
\end{enumerate}

\noindent
We describe approaches for each of these formulations in this section.

\subsection{FV formulation}

\subsection{FVPM formulation}

\section{Requirements}\labelSec{Requirements}

\section{Design}\labelSec{Design}

% Bibliography
\bibliography{references}
\bibliographystyle{plain}

\end{document} 
